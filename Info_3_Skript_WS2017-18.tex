% Vorlesungsskript / -mitschrieb zu Informatik III - Theoretische Informatik, gehalten von Prof. Dr. Peter Thiemann im WS 2014/15
%    Copyright (C) 2016 Ralph Lesch
%
%    This program is free software: you can redistribute it and/or modify
%    it under the terms of the GNU General Public License as published by
%    the Free Software Foundation, either version 3 of the License, or
%    (at your option) any later version.
%
%    This program is distributed in the hope that it will be useful,
%    but WITHOUT ANY WARRANTY; without even the implied warranty of
%    MERCHANTABILITY or FITNESS FOR A PARTICULAR PURPOSE.  See the
%    GNU General Public License for more details.
%
%    You should have received a copy of the GNU General Public License
%    along with this program.  If not, see <http://www.gnu.org/licenses/>.

% Compiled with pdflatex -enable-write18 -synctex=1 -interaction=nonstopmode --shell-escape %.tex
%\RequirePackage[l2tabu,orthodox]{nag}
\documentclass[11pt,paper=a4,titlepage,headsepline,ngerman,listof=totoc]{scrartcl}
\usepackage[utf8]{inputenc}
\usepackage[T1]{fontenc}
\usepackage{lmodern}
\usepackage{textcomp}
\usepackage{babel}
\usepackage{needspace}
%
\usepackage{acronym}
\usepackage{amsmath,amsfonts,amssymb}
\usepackage{array}
\usepackage{booktabs} % Better rules: \toprule, \midrule, \bottomrule
\usepackage{calc}
\usepackage{cancel}
\usepackage[iso]{datetime} % Date format
\usepackage{enumitem}
\usepackage{float} % H - option for figure
\usepackage{graphicx}
\usepackage{listings} % Source printing & syntax highlighting
\usepackage{mathtools}
\usepackage{multirow}
\usepackage{placeins} % \FloatBarrier
\usepackage{stmaryrd} % Symbols like \llbracket for [[
\usepackage{tabu}

% \usepackage[perpage]{footmisc} % custom footnote markers
% \DefineFNsymbols{daggers}{{$\dagger$}{$\dagger\dagger$}{${\dagger}{\dagger}{\dagger}$}{${\dagger}{\dagger}{\dagger}{\dagger}$}{${\dagger}{\dagger}{\dagger}{\dagger}{\dagger}$}}
% \setfnsymbol{daggers}
% \renewcommand{\thefootnote}{\fnsymbol{footnote}}



% \usepackage[charter]{mathdesign}
% \KOMAoptions{DIV=calc}

% physics package, without redefinitions
\usepackage[log-declarations=false]{xparse} % Fix for invalid error file reference for errors in log file (because of parentheses in d() arguments).
\usepackage[notrig]{physics}
\let\div\divisionsymbol             % \div = tex: \div | physics: \divergence
\let\Real\Re \let\Re\real           % \Re = tex: \Re   | physics: \Re => \Real
\let\Imaginary\Im \let\Im\imaginary % \Im = tex: \Im   | physics \Im => \Imaginary

\usepackage[dvipsnames]{xcolor}
% \usepackage{ulem}  % emph = \underline
\newcommand\coloruline[2]{\colorlet{colorsave}{.}{\color{#1}\uline{{\color{colorsave}#2}}}}
%\usepackage{xhfill}

% References
\usepackage[bookmarksnumbered,colorlinks=false,linkbordercolor={0 0 0},pdfborder={0 0 0}]{hyperref}
%\usepackage{hypcap}
\usepackage[hypcap]{caption} % \captionof + hypcap = link at begin of figure etc.
\usepackage{subcaption} % subfigure
%\usepackage{lastpage}

% Custom packages and macros
%\usepackage{import} \subimport{..\}{latexmacros.tex}
% LaTeX Macros, shortcuts, frequently used code
% Version: 2015-11-15
%
%    Copyright (C) 2015 Ralph Lesch
%
%    This program is free software: you can redistribute it and/or modify
%    it under the terms of the GNU General Public License as published by
%    the Free Software Foundation, either version 3 of the License, or
%    (at your option) any later version.
%
%    This program is distributed in the hope that it will be useful,
%    but WITHOUT ANY WARRANTY; without even the implied warranty of
%    MERCHANTABILITY or FITNESS FOR A PARTICULAR PURPOSE.  See the
%    GNU General Public License for more details.
%
%    You should have received a copy of the GNU General Public License
%    along with this program.  If not, see <http://www.gnu.org/licenses/>.
%
% Include with: \usepackage{import} \subimport{../}{latexmacros.tex}

% == Default packages ==
%\usepackage{amsmath,amsfonts,amssymb} % Math packages.
%\usepackage{array} % Extended array and tabular environments.
%\usepackage{booktabs} % Better rules: \toprule, \midrule, \bottomrule
%\usepackage{enumitem} % Extended lists, with key=value options.
%\usepackage{mathtools} % amsmath extension.

% == Optional packages ==
%\usepackage[bookmarksnumbered,colorlinks=false,linkbordercolor={0 0 0},pdfborder={0 0 0}]{hyperref} % Links and bookmarks.
%\usepackage{hypcap} % Link to top of float instead of caption.

%\usepackage{float} % Put float HERE with [H].
%\usepackage{graphicx} % \includegraphics
%\usepackage{placeins} % \FloatBarrier

%\usepackage{stmaryrd} % \lightning for contrapositions or failed proofs

%\usepackage{scrlayer-scrpage} % Header and footer.
%\usepackage{lastpage} % \pageref{LastPage}
%\ofoot{\usekomafont{pagenumber}\pagemark/\pageref{LastPage}} % Site number.

% == Custom packages ==
\RequirePackage{arrowmacros} % Arrow macros, e.g. \=>

% == Number types ==
\RequirePackage{amssymb}
\newcommand*\N{\mathbb{N}} % Natural numbers
\newcommand*\Z{\mathbb{Z}} % Integers
\newcommand*\Q{\mathbb{Q}} % Rational numbers
\newcommand*\R{\mathbb{R}} % Real numbers
\newcommand*\C{\mathbb{C}} % Complex numbers
\DeclareMathAlphabet{\mathbbn}{U}{bbold}{m}{n} % bbold font for numbers.
\newcommand*{\1}{\mathbbn{1}} % Symbol for identity
\newcommand*{\0}{\mathbbn{0}} % Symbol for empty element

% == Column types for tables (tabular/array/tabu) ==
\newcolumntype{M}[1]{>{$}#1<{$}} % Math mode for column, e.g. M{c}
% {RL}: Correct aligned columns for e.g. 1.2 &\pm 0.2
\newcolumntype{R}{>{$}r<{{}$}} % = M{r}<{{}}
\newcolumntype{L}{@{}>{${}}l<{$}} % Empty math symbol {} for correct space.

% == Equation numbering ==
% (single, manually) for math environments - typically followed by \label{eq:}
% In display math mode
\newcommand*\numbereq{\refstepcounter{equation}\tag{\theequation}}
% Inline numbering (for $math mode$), s: * without \hfill
\DeclareDocumentCommand\numberinlineeq{s}{
	\IfBooleanTF#1{}{\hfill}\refstepcounter{equation}(\theequation)%
}

% == Equation labeling ==
% An underbrace upside down (overbrace with under the line), taking no horizontal space
% \underoverbrace{text}{subscript}
\newcommand\underoverbrace[2]{\underset{\mathclap{\overbrace{#1}}}{#2}}
% An overbrace upside down (underbrace over the line), taking no horizontal space
% \overunderbrace{text}{superscript}
\newcommand\overunderbrace[2]{\overset{\mathclap{\underbrace{#1}}}{#2}}

% == Shortcuts ==
\newcommand*\x{\times}
\newcommand*\zz{\ensuremath{\mathrm{Z\kern-.5em\raise-0.5ex\hbox{Z}}}} % "Zu zeigen" symbol

% == Short figure names ==
\RequirePackage{babel}
% Change figure name from Figure to Fig. for english
\addto\captionsenglish{\renewcommand*{\figurename}{Fig.}}%
\addto\extrasenglish{\renewcommand*{\figureautorefname}{Fig.}}%
% Change figure name from Abbildung to Abb. for german
\addto\captionsngerman{\renewcommand*{\figurename}{Abb.}}%
\addto\extrasngerman{\renewcommand*{\figureautorefname}{Abb.}}%

% proof theorem with amsthm qed symbol
% with ntheorem
%\PassOptionsToPackage{amsmath,hyperref,thmmarks}{ntheorem}
%\RequirePackage{ntheorem,thmtools}
%%\usepackage[amsmath,hyperref,thmmarks]{ntheorem}
%%\usepackage{thmtools}
%\newcommand{\openbox}{\leavevmode
%  \hbox to.77778em{%
%  \hfil\vrule
%  \vbox to.675em{\hrule width.6em\vfil\hrule}%
%  \vrule\hfil}}
%
%\AtBeginDocument{%
%\declaretheoremstyle[
%	headfont=\scshape,
%	bodyfont=\normalfont,
%	headpunct={:\ },
%	qed=\openbox
%]{proofstyle}
%\newtheorem*{proof}{\csname proofname\endcsname} % \proofname from babel for babel
%}

% with amsthm
%\RequirePackage{amsthm,thmtools}
%\declaretheoremstyle[
%	headfont=\bfseries,
%	notefont=\normalfont,
%	bodyfont=\normalfont,
%	headpunct={:\ },
%	qed=\openbox,
%	spacebelow=
%]{proofstyle}
%\let\proof=\relax

%\AtBeginDocument{%
%	\declaretheorem[style=proofstyle,name=\proofname,numbered=no]{proof}
%}


% == fixme ==
\usepackage{fixme}
% Register users for user commands
\FXRegisterAuthor{rl}{anrl}{RL} % Ralph Lesch  => \rlnote = \txnote[user=RL]
\FXRegisterAuthor{pt}{anpt}{\bfseries PT} % Peter Thiemann
\FXRegisterAuthor{date}{andate}{\color{black}Vorlesung} % \datenote = Lecture date
\FXRegisterAuthor{draft}{andraft}{\color{red}Entwurf für Vorlesung} % \draftnote = Lecture date
 
%
\fxsetup{
	status=draft,
	multiuser,
	theme=color,
	innerlayout={layout=marginnote}, % fxnotes also in displaymath...
%	targetface=\color{red},
%	marginface=\color{red},
%	inlineface=\color{red}
}
\fxloadlayouts{marginnote}
% For color theme/layout:
\definecolor{fxnote}{named}{Green}
\definecolor{fxwarning}{named}{Orange}
\definecolor{fxerror}{named}{red}
\definecolor{fxfatal}{named}{BrickRed}
\renewcommand*{\marginfont}{\color{red}}
%\renewcommand*{\fxnotename}[1]{\!} % Not named for noncolor theme
%\definecolor{fxtarget}{named}{red} % color for target with color layout/theme
\renewcommand\germanlistfixmename{Anmerkungsverzeichnis}
% Customisation
\makeatletter
% target color = note (type) color
\renewcommand\FXTargetLayoutColor[2]{\@fxuseface{target}\color{fx#1}#2}
% Inline: with braces []
\renewcommand*\FXLayoutInline[3]{%
    \@fxdocolon{#3}{\@fxuseface{inline}\color{fx#1}[\ignorespaces#3\@fxcolon#2]}%
}
% Color theme: marginnote like marginpar
\renewcommand*\FXLayoutMarginNote[3]{%
	\@fxdocolon{#3}%
	\marginnote[%
		\raggedleft\@fxuseface{margin}\color{fx#1}\ignorespaces#3\@fxcolon#2%
	]{%
		\raggedright\@fxuseface{margin}\color{fx#1}\ignorespaces#3\@fxcolon#2%
	}%
}
\makeatother
% ====

% == tikz ==
\usepackage{tikz}
\usepackage{tikz-qtree}
\usetikzlibrary{arrows.meta,automata,calc,decorations.pathmorphing,decorations.pathreplacing,graphs,positioning%
%,external
}
%\tikzexternalize
\tikzset{>=stealth,
	block/.style={% Nodes as blocks with aligned text, e.g. for Turing machines
		draw, rectangle,
		%minimum height=1em,
		minimum width=1.5em,
		outer sep=0pt,
		node distance=0pt,
		text height=2ex,
		text depth=.5ex,
		align=center
	},
	circle/.style={
		draw,
		shape=circle,
		minimum size=0.5cm,
		text=black, 
		text width=0.5cm,
		align=center
	}
}
% ====

% == Header and footer ==
\usepackage{scrlayer-scrpage}
\automark[subsection]{section}
\pagestyle{scrheadings}
\clearscrheadfoot
\setkomafont{pageheadfoot}{\normalfont\sffamily}
\ohead{\pagemark}
\ihead{\rightmark}
%\ofoot{\usekomafont{pagenumber}\pagemark/\pageref{LastPage}} % Site number

% == Style ==
\setkomafont{captionlabel}{\bfseries}
% Paragraph
\setlength{\parindent}{0pt}
\setlength{\parskip}{\medskipamount}

% == Environments ==
\usepackage{amsthm,thmtools}
% Fix for theoream key "restate" with name={[short name]}
\makeatletter
\kv@set@family@handler{restate phase 2}{%
%  \ifthmt@restatethis
%  \@xa\@xa\@xa\g@addto@macro\@xa\@xa\@xa\thmt@storedoptargs\@xa\@xa\@xa{\@xa\@xa\@xa,%
%    \@xa\kv@key\@xa=\kv@value}%
%  \fi
}
\makeatother
% German "continues" text
\renewcommand\thmcontinues[1]{%
	\ifcsname hyperref\endcsname%
		\hyperref[#1]{Fortsetzung}%
	\else%
		Fortsetzung%
	\fi%
	von S.\,\pageref{#1}%
}
% Theorem Styles
\declaretheoremstyle[
	headfont=\bfseries,%\scshape
	notefont=\normalfont,
	bodyfont=\normalfont,
	headpunct={:\ },
	qed={},
	spaceabove=\bigskipamount,
	spacebelow=\parskip
]{basic}
\declaretheoremstyle[
	headfont=\bfseries,%\scshape
	notefont=\normalfont,
	bodyfont=\normalfont,
	headpunct={:\ },
	qed={\ensuremath{\diamond}},
	spaceabove=\bigskipamount,
	spacebelow=\parskip
]{definition}
\declaretheoremstyle[
	headfont=\scshape,%\bfseries,
	notefont=\normalfont,
	bodyfont=\normalfont,
	headpunct={:\ },
	qed=\openbox,
	spaceabove=\parskip,
	spacebelow=\parskip
]{proofstyle}

%\theoremstyle{theorem}
\declaretheorem[style=basic,numbered=no,name=Bemerkung]{Bemerkung}
\declaretheorem[style=basic,numbered=no,name=Bem.]{Bem}
\declaretheorem[style=basic,numbered=no,name=Beobachtung]{Beobachtung}
\declaretheorem[style=basic,parent=section,name=Bsp.]{Bsp}
\declaretheorem[style=basic,name=Bsp.,numbered=no]{Bsp*}
\declaretheorem[style=definition,parent=section,name=Def.]{Def}
\declaretheorem[style=definition,name=Def.,numbered=no]{Def*}
\declaretheorem[style=definition,parent=Def,name=Def.]{subDef}
\declaretheorem[style=definition,numbered=no,name=Erinnerung]{Erinnerung}
\declaretheorem[style=basic,parent=section,name=Satz]{Satz}
\declaretheorem[style=basic,name=Satz,numbered=no]{Satz*}
\declaretheorem[style=basic,numbered=no,name=Lemma]{lemma*}
\declaretheorem[style=basic,sibling=Satz,name=Lemma]{lemma}
\declaretheorem[style=definition,sibling=Satz,name=Korollar]{Korollar}
\declaretheorem[style=definition,numbered=no,name=Korollar]{Korollar*}
\let\proof=\relax
\AtBeginDocument{% For \proofname as of babel.
	\declaretheorem[style=proofstyle,name=\proofname,numbered=no]{proof}
}

% cases environment with 2 arrows
% Source: https://tex.stackexchange.com/questions/89250/modify-case-equations-brace/89257#89257
% \splitlines[->] for arrow
\newcommand*{\splitlines}[1][]{
\begin{tikzpicture}[baseline=-0.5ex]
\draw[#1] (0,0) -- (0.4,0.25);
\draw[#1] (0,0) -- (0.4,-0.15);
\end{tikzpicture}
}
\newenvironment{casesarrows}[1][->]%
{\;\splitlines[#1]\;\begin{array}{@{}l@{}}}%
{\end{array}}


% == Custom commands ==

% Source: https://tex.stackexchange.com/questions/164506/how-to-get-a-curved-arrow-pointing-left-and-right/164511#164511
%\newcommand{\curvearrowleftright}{\scalebox{1.2}[2]{$\mathclap{\curvearrowleft}\mkern2.2mu\mathclap{\curvearrowright}$}}
\newcommand{\curvearrowleftright}{\mathrel{\curvearrowleft\mkern-2.7mu\mathllap{\curvearrowright}}}

% \dash of given length
\newcommand*{\xdash}[1]{\rule[0.5ex]{#1}{0.55pt}}

% \ruleplaceholder["]{width} = --"--
\newcommand*{\ruleplaceholder}[2][\text{\texttt{ " }}]{%
    \setlength{\dimen0}{(#2-\widthof{#1})/2}%
    \xdash{\dimen0}\text{#1}\xdash{\dimen0}}

% Math operators and abbreviations
\usepackage{forloop}
% Define \mcX = \mathcal{X}, with X = A-Z
% and \mbX = \mathbb{X}
\newcounter{ct}
\newcommand*\mc[1]{\mathcal{#1}}
\newcommand*\mb[1]{\mathbb{#1}}
\forloop{ct}{1}{\value{ct} < 27}{%
	\expandafter\edef\csname mc\Alph{ct}\endcsname{\noexpand\mathcal{\Alph{ct}}}%
	\expandafter\edef\csname mb\Alph{ct}\endcsname{\noexpand\mathbb{\Alph{ct}}}%
}
%
\newcommand*{\blank}{\raisebox{1pt}{\texttt{\char32}}}
\newcommand*\A{\mathcal{A}}
\newcommand*\Powerset{\mathcal{P}}
\newcommand*\dotcup{\mathrel{\dot\cup}}
\DeclareMathOperator\Konf{Konf}
\DeclareMathOperator\out{out}
\DeclareMathOperator\Sim{Sim}
\DeclareMathOperator\Instr{Instr}
\DeclareMathOperator\inc{inc}
\DeclareMathOperator\dec{dec}
\DeclareMathOperator\DTAPE{DTAPE}
\DeclareMathOperator\NTAPE{NTAPE}
\DeclareMathOperator\NTIME{NTIME}
\DeclareMathOperator\DTIME{DTIME}
\DeclareMathOperator\code{code}
\DeclareMathOperator\pos{pos}
\DeclareMathOperator\state{state}
\DeclareMathOperator\tape{tape}
\DeclareMathOperator\CLIQUE{CLIQUE}
\DeclareMathOperator\PREC{PREC} % Def. 8.2
\DeclareMathOperator\add{add} % Bsp. zu 8.2
\DeclareMathOperator\mult{mult} % Bsp. zu 8.2

\newcommand{\Eps}{\varepsilon}
\newcommand{\qedherefix}{\\[-3\baselineskip]} % \qedhere fix for eqnarray

% Faster compilation: one section / file.
%\includeonly{6-Berechenbarkeit}

%========================%
\begin{document}
\title{Vorlesung \\Informatik III -- Theoretische Informatik}
\subtitle{Formale Sprachen, Berechenbarkeit, Komplexitätstheorie}
\author{Matthias Heizmann\\\normalsize{Basierend auf einem Mitschrieb\ von Ralph Lesch\thanks{ralph.lesch@neptun.uni-freiburg.de}}\\\normalsize{\ der\ von\ Prof.\ Dr.\ Peter\ Thiemann\ im\ WS\, 2015/16 gehaltenen Vorlesung}}
\date{WS\,2017/18%\\2015-10-21 -- 2016-02-22
  \\
  \vspace{10ex}
\small\sffamily Zuletzt aktualisiert: \today}

\maketitle

\vspace{\baselineskip}
\tableofcontents

\section[Vorspann: Sprachen]{Vorspann: Sprachen\datenote{18.10.2017}}
\begin{Def}[name={[Alphabet $\Sigma$]}]
	Ein \emph{Alphabet} ist eine endliche Menge von \emph{Zeichen}.
\end{Def} % 1.1
Zeichen sind hier beliebige abstrakte Symbole.

\begin{Bsp*} für Alphabete die in dieser Vorlesung, im täglichem Umgang mit Computern oder in der Forschung an unserem Lehrstuhl eine Rolle spielen.
  \begin{itemize}
  \item $\{a,\dots,z\}$
  \item $\{0, 1\}$
  \item Die Menge aller ASCII Symbole
  \item Die Menge aller Statements eines Computerprogrammes
  \end{itemize}
\end{Bsp*}
Wir verwenden typischweise den griechischen Buchstaben $\Sigma$ als Namen für ein Alphabet und die lateinischen Buchstaben $a,b,c$ als Namen für Zeichen.
\footnote{Analoge Konventionen die sie möglicherweise aus der Schule kennen: Verwende $n,m$ für natürliche Zahlen. Verwende $\alpha, \beta$ für Winkel in Dreiecken. Verwende $A$ für Matrizen.}
Im Folgenden sei $\Sigma$ immer ein beliebiges Alphabet.
\footnote{Dieser Satz dient dazu, dass die Autoren dieses Skriptes nicht jede Definition mit ``Sei $\Sigma$ ein Alphabet beginnen müssen''.}

\begin{Def}[name={[Wort $w$ über $\Sigma$]}]\label{def:1.2}
  Die Menge $\Sigma^+$ der \emph{nicht leeren Wörter} über $\Sigma$ ist induktiv definiert:
  \begin{enumerate}
   \item Wenn $a\in\Sigma$ dann $a\in\Sigma^+$
   \item Wenn $a\in\Sigma$ und $w\in\Sigma^+$ dann auch $aw \in \Sigma^*$
  \end{enumerate}
  Wir verwenden immer den griechischen Buchstaben $\Eps$ um das \emph{leere Wort} zu bezeichnen.\footnote{Eine analoge Konvention die sie aus der Schule kennen: Verwende immer $\pi$ für die Kreiszahl}
  Die Menge $\Sigma^*$ der \emph{Wörter} über $\Sigma$ ist $\{\Eps\}\cup\Sigma^+$.
  
  Die \emph{Länge} eines Wortes, $|\cdot| : \Sigma^* \to \N$, ist induktiv definiert durch
  \begin{enumerate}
  \item $|\Eps| = 0$
  \item $|aw| = 1 + |w|$
  \end{enumerate}
\end{Def}
Wir verwenden typischerweise $u,v,w$ als Namen für Wörter.

\goodbreak

Ein Wort ist also immer eine endliche Folge von Zeichen.
\begin{Bsp*} für Wörter über $\Sigma=\{a,\dots,z\}$
  \begin{itemize}
  \item \verb/rambo/ (Länge $5$)
  \item \verb/pizza/, \verb/zipza/ (ungleich) 
  \item $\Eps$ (Länge $0$)
  \item \verb/rambopizza/
  \end{itemize}
\end{Bsp*}
Wörter lassen sich ,,verketten''/,,hintereinanderreihen''.
Die entsprechende Operation heißt \emph{Konkatenation}, geschrieben ,,$\cdot$'' (wie Multiplikation).
\begin{Def}[Konkatenation von Wörtern]
  Die \emph{Konkatenation}, $\mathord{\cdot} : \Sigma^* \times \Sigma^* \to \Sigma^*$, ist induktiv definiert durch:
  \begin{enumerate}
  \item $\Eps\cdot v = v$
  \item $(aw)\cdot v = a(w\cdot v)$
  \end{enumerate}
\end{Def}
\begin{Bsp*} ~
  \begin{itemize}
  \item \begin{align*}\mathtt{rambo}\cdot\mathtt{pizza} &=\mathtt{r(ambo}\cdot\mathtt{pizza})\\
                                                        &=\mathtt{ra(mbo}\cdot\mathtt{pizza})\\
                                                        &=\mathtt{ram(bo}\cdot\mathtt{pizza})\\
                                                        &=\mathtt{ramb(o}\cdot\mathtt{pizza})\\
                                                        &=\mathtt{ramb((\Eps\cdot o)}\cdot\mathtt{pizza})\\
                                                        &=\mathtt{rambopizza}
        \end{align*}
  \item $\mathtt{rambo} \cdot \Eps = \mathtt{rambo} = \Eps \cdot \mathtt{rambo}$ 
  \end{itemize}
\end{Bsp*}
Eigenschaften von "`$\cdot$"':
\begin{itemize}
\item Assoziativität
\item $\Eps$ ist neutrales Element
\item \emph{nicht} kommutativ
\end{itemize}
\begin{lemma}
  Für alle $w \in \Sigma^*$ gilt $w \cdot \Eps = w$.
\end{lemma}
\begin{proof}
  Per struktureller Induktion über das Wort $w$:
  \begin{description}
  \item[IA] $w = \Eps$. \\
      Es folgt aus Fall 1 der Definition von ,,$\cdot$'': $\Eps \cdot \Eps = \Eps$.
  \item[IV] $w' \cdot \Eps = w'$.
  \item[IS] $w = aw'$.
    \\
    Mit Fall 2 von ,,$\cdot$'' folgt $aw' \cdot \Eps = a (w' \cdot \Eps) \stackrel{\mathsf{IV}}{=} aw'$
  \end{description}
\end{proof}
\begin{proof}
  Alternativ, per Induktion über $n = |w|$:
  \begin{description}
  \item[IA] $n = 0$.

    Es folgt aus der Definition von ,,$|\cdot|$'' (und dem arithmetischen Fakt, dass $1 + x \not = 0$), dass $w = \Eps$.

    Es folgt aus Fall 1 der Definition von ,,$\cdot$'', dass $\Eps \cdot \Eps = \Eps$.
  \item[IV] Für alle $w'$ mit $|w'| = n'$ gilt $w' \cdot \Eps = w'$.
  \item[IS] $n = n' + 1$.

    Es folgt aus der Definition von ,,$|\cdot|$'' (und dem arithmetischen Fakt, dass $1 + x \not = 0$), dass $w = aw'$.

    Somit gilt $|aw'| = 1 + |w'| = 1 + n'$ und daher auch $|w'| = n'$.

    Mit Fall 2 von ,,$\cdot$'' folgt $aw' \cdot \Eps = a (w' \cdot \Eps) \stackrel{\mathsf{IV}}{=} aw'$
  \end{description}
\end{proof}
\begin{lemma}
  Für $v,w \in \Sigma^*$ gilt $|v\cdot w| = |v| + |w|$.
\end{lemma}
\begin{proof}
  Per Induktion über $v$.
  \begin{description}
  \item[IA] $v = \Eps$.

    Nach Def.\ von ,,$\cdot$'' gilt $|\Eps \cdot w| = |w|$.

    Mit Def.\ von ,,$|\cdot|$'', Fall 1 folgt $|w| = 0 + |w| = |\Eps| + |w|$
 \item[IV] $|v' \cdot w| = |v'| + |w|$.
 \item[IS] $v = av'$.

   Mit Def.\ von ,,$|\cdot|$'' und ,,$\cdot$'' folgt:
   \begin{displaymath}
   |av'\cdot w| = |a(v'\cdot w)| = 1 + |v' \cdot w| \stackrel{\mathsf{IV}}{=} 1 + |v'| + |w| = (1 + |v'|) + |w| = |v| + |w|
   \end{displaymath}
  
  \end{description}
\end{proof}

Der Konkatenationsoperator ,,$\cdot$'' wird oft weggelassen (ähnlich wie der Mulitplikationsoperator in der Arithmetik).
Ebenso können durch die Assoziativität Klammern weggelassen werden:
\begin{displaymath}
  w_1w_2w_3 \quad \text{ heißt also } \quad w_1\cdot w_2\cdot w_3 \quad = (w_1 \cdot w_2) \cdot w_3 = w_1 \cdot (w_2 \cdot w_3)
\end{displaymath}

Wörter lassen sich außerdem \emph{potenzieren}:
\begin{Def}
  Die \emph{Potenzierung} von Worten, $\cdot ^ \cdot: \Sigma^*\times \N \to \Sigma^*$, ist induktiv definiert durch
  \begin{enumerate}
  \item $w^0 = \Eps$ 
  \item $w^{n+1} = w \cdot w^n$
  \end{enumerate}
\end{Def}
\begin{Bsp*} ~
  \begin{itemize}
  \item $\mathtt{rambo}^1 = \mathtt{rambo}$
  \item $\mathtt{rambo}^0 = \Eps$
  \item $\mathtt{rambo}^3 = \mathtt{ramboramborambo}$
  \end{itemize}
\end{Bsp*}

\begin{Def}[name={[Sprache über $\Sigma$]}]
	Eine \emph{Sprache} über $\Sigma$ ist eine Menge $L\subseteq\Sigma^*$.
\end{Def}
\begin{Bsp*}~ 
  \begin{itemize}
  \item 
	$\{\mathtt{banane}, \mathtt{aprikose},\mathtt{orange},\dots\}$
  \item
    $\{\mathtt{rot},\mathtt{gelb},\texttt{grün}\}$
  \item
    $\{\mathtt{rambo}, \mathtt{pizza}, \Eps, \texttt{blümchen}\}$
  \item $\{\}$ (die ,,leere Sprache'')
  \item $\{ \Eps \}$
  \item $\Sigma^*$
  \end{itemize}
\end{Bsp*}
Sämtliche Mengenoperationen sind auch Sprachoperationen, insbesondere Schnitt ($L_1 \cap L_2$), Vereinigung ($L_1 \cup L_2$), Differenz ($L_1 \setminus L_2$) und Komplement ($L_1^{-1} = \Sigma^* \setminus L_1$).

Weitere Operationen auf Sprachen sind Konkatenation und Potenzierung, sowie der \emph{Kleene Abschluss}.
\begin{Def}[Konkatenation und Potenzierung von Sprachen] % 1.5
	Sei $U,V\subseteq \Sigma^*$ dann ist die \emph{Konkatenation} $U$ und $V$ definiert durch
	\[ U\cdot V = \{uv \mid u\in U, v\in V \} \]
  und die \emph{Potenzierung} von $U$ induktiv definiert durch
  \begin{enumerate}
  \item $U^0 = \{\Eps\}$
  \item $U^{n+1} = U \cdot U^{n}$
  \end{enumerate}
\end{Def}
\begin{Bsp*}~
  \begin{itemize}
  \item $\{\mathtt{rambo}, \mathtt{pizza}\} \cdot \{\mathtt{rot}, \mathtt{gelb}\} = \{\mathtt{ramborot}, \mathtt{pizzarot}, \mathtt{rambogelb}, \mathtt{pizzagelb}\}$
  \item $\{\mathtt{rambo}, \mathtt{pizza}\} \cdot \{\Eps, \mathtt{gelb}\} = \{\mathtt{rambo}, \mathtt{pizza}, \mathtt{rambogelb}, \mathtt{pizzagelb}\}$
  \item $\{\mathtt{rambo, \Eps}\}^3 = \{\mathtt{\Eps}, \mathtt{rambo}, \mathtt{ramborambo}, \mathtt{ramboramborambo}\}$
  \end{itemize}
\end{Bsp*}
Wie bei der Konkatenation von Wörtern lässt man den Konkatenationsoperator oft weg.
%
\begin{Def}[Kleene-Abschluss, Kleene-Stern]
	Sei $U\subseteq\Sigma^*$.
  Der \emph{Kleene-Abschluss} ist induktiv definiert als
  \begin{enumerate}
  \item 
    $U^* = \bigcup_{n\in\N} U^n \quad [\ni\Eps]$
  \item
		$U^+ = \bigcup_{n\ge1} U^n$
  \end{enumerate}
\end{Def}

%%% Local Variables:
%%% mode: latex
%%% TeX-master: "Info_3_Skript_WS2016-17"
%%% End:

\section[Reguläre Sprachen und endliche Automaten]{Reguläre Sprachen und endliche Automaten\datenote{20.10.17}}
Wie können wir potentiell unendlich große Mengen von Wörtern darstellen?
Eine Lösung für dieses Problem sahen wir bereits im vorherigen Kapitel, als wir die (unendlich große) Menge der binärcodierten Primzahlen mit Hilfe der folgenden Zeile darstellten.
\[
L_\mathsf{prim}=\{w\in\{0,1\}^*\mid w \text{ ist Bin\"arcodierung einer Primzahl}\}
\]
Ein weiteres Beispiel ist die folgende Zeile.
\[
L_\mathsf{even}=\{w\in\{0,1\}^*\mid \text{ die Anzahl der Einsen in $w$ ist gerade. }\}
\]
Ein häufig interessante Fragestellung für ein gegebenes Wort $w$ und eine Sprache $L$ ist: ``Ist $w$ in $L$ enthalten?'' (Also gilt $w\in L$?)
Wir nennen dieses Entscheidunsproblem das \emph{Wortproblem}. Eine konkrete Instanz des Wortproblems wäre z.B. $1100101\in L_1$? oder $1100101\in L_2$?

Die obige Darstellung der unendlichen Mengen $L_1$ und $L_2$ ist zwar sehr kompakt, 
wir können daraus aber nicht direkt ein Vorgehen zur Lösung des Wortproblems ableiten.
Wir müssen zunächst verstehen, was die Begriffe ``Binärcodierung'', ``Primzahl'' oder ``gerade Anzahl'' bedeuten und für $L_1$ und $L_2$ jeweils einen Algorithmus zur Entscheidung entwickeln.

In diesem Kapitel werden wir mit \emph{endlichen Automaten} einen weiteren Formalismus kennenlernen um (potentiell unendlich große) Mengen von Wörtern darzustellen. 
Ein Vorteil dieser Darstellung ist, dass es einen einheitlichen und effizienten Algorithmus für das Wortproblem gibt.
Wir werden aber auch sehen, dass sich nicht jede Sprache (z.B. $L_1$) mit Hilfe eines endlichen Automaten darstellen lässt.

\subsection{Endliche Automaten}
Wir beschreiben zunächst informell die Bestandteile eines endlichen Automaten:

\newcommand{\qinit}{{q^\mathsf{init}}}
\newcommand{\B}{\mathcal{B}}
\newcommand{\calN}{\mathcal{N}}
\newcommand{\calP}{\mathcal{P}}
\newcommand{\ecl}{\textsf{ecl}}
\newcommand{\reach}{\textsf{reach}}
\newcommand{\hide}[1]{}

\begin{description}
\item[Endliches Band] 
(read-only, jede Zelle enth"alt ein $a_i\in\Sigma$, der Inhalt des Bandes ist das \emph{Eingabewort}, bzw.\ die \emph{Eingabe})

\begin{figure}[H]\centering
        \begin{tikzpicture}
                \node (A) [block]{$a_0$};
                \node (B) [block,right=of A] {$\dots$};
                \node (C) [block,right=of B] {$a_n$};

    \node (L) [above=of A, node distance = 0.25cm] {Lesekopf};
    \draw[->] (L) -- (A);
        \end{tikzpicture}
        \caption{Endliches Band}
\end{figure}
\vspace{-1em}
\item[Lesekopf] ~\\
  \vspace{-\baselineskip}
  \begin{itemize}
        \item Der \emph{Lesekopf} zeigt auf ein Feld des Bandes, oder hinter das letzte Feld.
        \item Er bewegt sich feldweise nach rechts; andere Bewegungen (Vor- bzw.\ Zurückspulen) sind nicht möglich.
        \item Wenn er hinter das letzte Zeichen zeigt, \emph{stoppt} der Automat.
    Er muss sich nun ,,entscheiden'' ob er das Wort \emph{akzeptiert} oder nicht.
  \end{itemize}
\item[Zustände] $q$ aus \emph{endlicher} Zustandsmenge $Q$
\item[Startzustand] $\qinit \in Q$
\item[Akzeptierende Zustände] $F \subseteq Q$ 
\item[Transitionsfunktion] Im Zustand $q$ beim Lesen von $a$ gehe nach Zustand $\delta(q) = q'$.
\end{description}
Der endliche Automat akzeptiert eine Eingabe, falls er in einem akzeptierenden Zustand stoppt.

% Spiral - By Keven Law - originally posted to Flickr as What's your Colour???, CC BY-SA 2.0, https://commons.wikimedia.org/w/index.php?curid=6851868
% Bunch - By Mariajudit - Own work, CC BY-SA 4.0, https://commons.wikimedia.org/w/index.php?curid=48726001
% Almond - By Michelle Naherny - Own work, CC BY-SA 4.0, https://commons.wikimedia.org/w/index.php?curid=44361114
\needspace{5\baselineskip}
\begin{Bsp}Aufgabe: 
  \\
  \label{Bsp:3.1}
  ,,Erkenne alle Stapel von Macarons in denen höchstens ein grünes Macaron vorkommt.''
\begin{center}
  \includegraphics[scale=0.4]{macaron-stacks.png}~\footnote{
  \tiny Von links nach rechts: \\
By Mariajudit - Own work, CC BY-SA 4.0, https://commons.wikimedia.org/w/index.php?curid=48726001
  \\
By Michelle Naherny - Own work, CC BY-SA 4.0, https://commons.wikimedia.org/w/index.php?curid=44361114
  \\
By Keven Law - originally posted to Flickr as What's your Colour???, CC BY-SA 2.0, https://commons.wikimedia.org/w/index.php?curid=6851868
}
% Bunch - By Mariajudit - Own work, CC BY-SA 4.0, https://commons.wikimedia.org/w/index.php?curid=48726001
% Almond - By Michelle Naherny - Own work, CC BY-SA 4.0, https://commons.wikimedia.org/w/index.php?curid=44361114

\end{center}
Ein passendes Alphabet wäre $\Sigma = \{\texttt{grün} , \texttt{nicht-grün} \}$.
Wir definieren die folgenden Zustände.
(die Metapher hier ist: ,,wenn ich mehr als einen grünen Macaron esse wird mir übel, und das wäre nicht akzeptabel'')
\begin{center}
\begin{tabular}{cl}
  Zustand & Bedeutung \\
  \hline
  $q_0$& ,,alles gut'' \\
  $q_1$& ,,mir wird schon flau'' \\
  $q_2$& ,,mir ist übel''
\end{tabular}
\end{center}
Der Startzustand ist $q_0$.
Akzeptierende Zustände sind $q_0$ und $q_1$.
Die Transistionsfunktion $\delta$ ist
\begin{center}
\begin{tabular}{cccl}
  &\texttt{grün} & \texttt{nicht-grün} \\
  \hline
  $q_0$ & $q_1$ & $q_0$ & wechsle nach $q_1$ falls \texttt{grün}, ansonsten verweile \\
  $q_1$ & $q_2$ & $q_1$ & wechsle nach $q_2$ falls \texttt{grün}, ansonsten verweile \\
  $q_2$  & $q_2$ & $q_2$ & verweile, da es nichts mehr zu retten gibt
\end{tabular}
\end{center}
\end{Bsp}

% \begin{Bsp}
%       $L=\{w\in\{0,1\}^* \mid w \text{ enthält gerade Anzahl von 0 und gerade Anzahl von 1}\}$

%       \begin{minipage}[t]{.4\textwidth}\centering\vspace{0pt}
%           \captionsetup{type=figure}
%               \begin{tikzpicture}[circle/.style={
%                       shape=circle,
%                       minimum size=0.5cm,
%                       text=black, draw,
%                       text width=0.5cm,
%                       align=center}]
%                       \node (v1) at (-3.5,3.5) {};
%                       \node [circle,double] (v2) at (-2.5,3) {$q_{00}$};
%                       \node [circle] (v3) at (0.5,3) {$q_{01}$};
%                       \node [circle] (v4) at (0.5,0.5) {$q_{10}$};
%                       \node [circle] (v5) at (-2.5,0.5) {$q_{11}$};
%                       \draw [->] (v1) edge (v2);
%                       \draw [->] (v2) edge [bend left=15] node[auto] {1} (v3);
%                       \draw [->] (v3) edge [bend left=15] node[auto] {1} (v2);
%                       \draw [->] (v3) edge [bend left=15] node[auto] {0} (v4);
%                       \draw [->] (v4) edge [bend left=15] node[auto] {0} (v3);
%                       \draw [->] (v2) edge [bend left=15] node[auto] {0} (v5);
%                       \draw [->] (v5) edge [bend left=15] node[auto] {0} (v2);
%                       \draw [->] (v5) edge [bend left=15] node[auto] {1} (v4);
%                       \draw [->] (v4) edge [bend left=15] node[auto] {1} (v5);
%               \end{tikzpicture}
%               \captionof{figure}{Automat zu $L$}
%       \end{minipage}\begin{minipage}[t]{.55\textwidth}\vspace{0pt}
%       Graphische Darstellung $\hat=$ gerichteter Graph mit Knoten $Q$ und markierten Kanten gemäß $\delta$.\\
%       $Q=\{q_{00},q_{01},q_{10},q_{11}\}$\\
%       $q_{00}$ einziger akzeptierender Zustand ($F=\{q_{00}\}$)
%       \end{minipage}
        
%       \begin{tabular}{M{l}|M{l}|M{l}l @{\quad}l}
%               & 0 & 1 &\\ \cline{1-3}
%               q_{00} & q_{10} & q_{01} && gerade Anzahl von 0 und 1 gesehen\\
%               q_{01} & q_{11} & q_{00} && gerade \ruleplaceholder{\widthof{Anzahl von 0}}, ungerade Anzahl von 1 gesehen\\
%               q_{10} & q_{00} & q_{11} && ungerade \ruleplaceholder{\widthof{Anzahl von 0}}, gerade \ruleplaceholder{\widthof{Anzahl von 1 gesehen}} \\
%               q_{11} & q_{01} & q_{10} && ungerade \ruleplaceholder{\widthof{Anzahl von 0}}, ungerade \ruleplaceholder{\widthof{Anzahl von 1 gesehen}}
%       \end{tabular}
% \end{Bsp}
\begin{Def}[\acs*{DEA}]
        Ein \emph{\acf{DEA}}, (\acsu{DFA} $\hat=$ \acl{DFA}) ist ein 5-Tupel
        \[ \A= (\Sigma, Q,\delta,\qinit,F) \]
dabei ist
        \begin{itemize}
                \item $\Sigma$ ein Alphabet,
                \item $Q$ eine \emph{endliche} Menge deren Elemente wir \emph{Zustände} nennen,
                \item $\delta:Q\x\Sigma\->Q$ eine Funktion die wir \emph{Transitionsfunktion} nennen,
                \item $\qinit\in Q$ ein Zustand den wir \emph{Startzustand} nennen und
                \item $F\subseteq Q$ eine Teilmenge der Zustände deren Elemente wir \emph{akzeptierende} Zustände nennen.
        \end{itemize}
\end{Def}

DEAs lassen sich auch graphisch darstellen.
Dabei gibt man für den Automaten einen gerichteten Graphen an.
Die Knoten des Graphen sind die Zustände und mit Zeichen beschriftete Kanten zeigen, welchen Zustandsübergang die Transitionsfunktion für das nächste Zeichen erlaubt.
Der Startzustand ist mit einem unbeschrifteten Pfeil markiert und akzeptierende Zustände sind doppelt eingekreist.
Hier ist die graphische Darstellung von $A_{\mathtt{Macaron}}$ aus Beispiel \ref{Bsp:3.1}:

\begin{center}
\begin{tikzpicture}[node distance = 3cm]
  \node[state, accepting] (0) {$q_0$};
  \node[state, accepting, right of = 0] (1) {$q_1$};
  \node[state, right of = 1] (2) {$q_2$};

  \node[left of = 0, node distance = 1cm] (start){}; 
  \draw[->] (start) to (0);

  \draw[->] (0) to node[above] {\texttt{grün}} (1);
  \draw[->, loop above] (0) to node[above] {\texttt{nicht-grün}} (0);

  \draw[->] (1) to node[above] {\texttt{grün}} (2);
  \draw[->, loop above] (1) to node[above] {\texttt{nicht-grün}} (1);

  \draw[->, loop right] (2) to node[right]{\texttt{grün}, \texttt{nicht-grün}} (2);
\end{tikzpicture}
\end{center}

% DEAs charakterisieren die Sprachen durch die Menge an Wörtern, die sie akzeptieren.
Die folgenden beiden Definitionen erlauben uns mit Hilfe eines DEAs eine Sprache zu charakterisieren.
% \begin{Bsp}
%     Sei $M=(\Sigma, Q,\delta,q_0,F)$ ein DEA.
%     \begin{itemize}
%     \item Wenn $F=Q$, dann ist $L(M)=\Sigma^*$.
%     \item Wenn $F=\emptyset$, dann ist $L(M)=\emptyset$.
%     \end{itemize}
% \end{Bsp}
\begin{Def}[name={[Induktive erweiterung von $\delta$ auf Worte]}]\label{def:2.deltaschlange}
        Die \emph{induktive Erweiterung} von $\delta:Q\x\Sigma\->Q$ auf Worte $\tilde{\delta}: Q\x\Sigma^*\->Q$ ist (induktiv) definiert durch
  \begin{enumerate}
  \item $\tilde\delta(q,\Eps) =q$\ \ \ \  (Wortende erreicht)
  \item $\tilde\delta(q,aw)=\tilde\delta(\delta(q,a),w)$\ \ \ \ (Rest im Folgezustand verarbeiten)
  \end{enumerate}
\end{Def}
\begin{Def}[name={[Die durch einen \acs*{DEA} akzeptierte Sprache]}]\label{def:2.sprache}
        Sei $\A=(\Sigma, Q,\delta,\qinit,F)$. 
        Wir sagen ein Wort $w\in\Sigma^*$ wird von $\A$ \emph{akzeptiert} falls $\tilde\delta(\qinit,w)\in F$.
        Die \emph{von $\A$ akzeptierte Sprache}, geschrieben $L(\A)$, ist die Menge aller Wörter die von $\A$ akzeptiert werden. D.h., 
        \[ L(\A) = \{ w\in\Sigma^* \mid \tilde\delta(\qinit,w)\in F \}. \]
        Eine durch einen \ac{DEA} akzeptierte Sprache heißt \emph{regulär}.
\end{Def}

\begin{Bsp} 
\label{bsp:3.1} Der Automat für die Sprache 
$$L_\mathsf{even}=\{w\in\{0,1\}^*\mid \text{ die Anzahl der Einsen in $w$ ist gerade. }\}$$ 
aus der Einleitung dieses Kapitels hat die folgende graphische Repräsentation.
%         $L=\{w\in\{0,1\}^* \mid w \text{ enthält gerade Anzahl von 0 und gerade Anzahl von 1}\}$
  \begin{center}
                \begin{tikzpicture}[circle/.style={
                        shape=circle,
                        minimum size=0.5cm,
                        text=black, draw,
                        text width=0.5cm,
                        align=center}]
                        \node (v1) at (-3.5,3.5) {};
                        \node [circle,double] (v2) at (-2.5,3) {$q_{0}$};
                        \node [circle] (v3) at (0.5,3) {$q_{1}$};
                        \draw [->] (v1) edge (v2);
                        \draw [->] (v2) edge [bend left=15] node[auto] {1} (v3);
                        \draw [->] (v3) edge [bend left=15] node[auto] {1} (v2);
%                         \draw [->] (v1) edge [bend left=15] node[auto] {0} (v1);
%                         \draw [->] (v2) edge [bend left=15] node[auto] {0} (v1);
                        \draw [->] (v2) edge [loop below] node[auto] {0} (v2);
                        \draw [->] (v3) edge [loop below] node[auto] {0} (v3);
                \end{tikzpicture}
  \end{center}
\end{Bsp}



% Es folgen zwei Beispiele für reguläre Sprachen:
% \begin{Bsp} 
% \label{bsp:3.1}
%         $L=\{w\in\{0,1\}^* \mid w \text{ enthält gerade Anzahl von 0 und gerade Anzahl von 1}\}$
%   \begin{center}
%                 \begin{tikzpicture}[circle/.style={
%                         shape=circle,
%                         minimum size=0.5cm,
%                         text=black, draw,
%                         text width=0.5cm,
%                         align=center}]
%                         \node (v1) at (-3.5,3.5) {};
%                         \node [circle,double] (v2) at (-2.5,3) {$q_{00}$};
%                         \node [circle] (v3) at (0.5,3) {$q_{01}$};
%                         \node [circle] (v4) at (0.5,0.5) {$q_{10}$};
%                         \node [circle] (v5) at (-2.5,0.5) {$q_{11}$};
%                         \draw [->] (v1) edge (v2);
%                         \draw [->] (v2) edge [bend left=15] node[auto] {1} (v3);
%                         \draw [->] (v3) edge [bend left=15] node[auto] {1} (v2);
%                         \draw [->] (v3) edge [bend left=15] node[auto] {0} (v4);
%                         \draw [->] (v4) edge [bend left=15] node[auto] {0} (v3);
%                         \draw [->] (v2) edge [bend left=15] node[auto] {0} (v5);
%                         \draw [->] (v5) edge [bend left=15] node[auto] {0} (v2);
%                         \draw [->] (v5) edge [bend left=15] node[auto] {1} (v4);
%                         \draw [->] (v4) edge [bend left=15] node[auto] {1} (v5);
%                 \end{tikzpicture}
%   \end{center}
% \end{Bsp}
% \begin{Bsp}\label{bsp:3.2}
%         
%         Sei $A\ge 0$ nat. Zahl, $\Sigma=\{0,1,\dots,A\}$
%         \begin{equation*}
%                 L = \{ a_1\dots a_n \mid \exists J\subseteq \{1,\dots,n \},\ \sum_{i\in J} a_i = A \} \subseteq \Sigma^* 
%         \end{equation*}
%         D.h. gegeben eine Liste von Zahlen $\in\Sigma$.
%         Akzeptiere diejenigen Listen, für die eine Teilliste existiert, deren Summe genau $A$ ist.
%         \begin{align*}
%                 Q &=\Powerset\{0,1,\dots,A\} \\
%                 \delta(q,a) &= q \cup \{ x\in \{0,\dots,A\} \mid x-a \in q \} \\
%                 q_0 &=\{0\} \\
%                 F &= \{ q\in Q \mid A \in q \}
%         \end{align*}
%   $q \in Q$ bezeichnet die Menge an möglichen Summen $\le A$, die mit den bisher gelesenen Zeichen gebildet werden kann.
%   Die Transitionsfunktion $\delta$ fügt die Summen zum aktuellen Zustand hinzu, die sich durch addieren der aktuell gelesenen Ziffer zu den alten Möglichkeiten ergeben.
% \end{Bsp}
% \begin{Bsp}\label{bsp:3.3}
%         Beispiel f"ur eine nicht-regul"are Sprache.
%         \begin{equation*}
%                 L = \{ 0^n1^n \mid n\in\N \} 
%         \end{equation*}
%         erkennbar durch \ac{TM} die immer anhält, \emph{aber nicht} von einem \ac{DEA} [\emph{nicht} regulär] akzeptiert werden kann.
%         \begin{proof}
%                 Angenommen $L=L(M)$ für \ac{DEA} $M=(\Sigma, Q,q_0,\delta,F)$
%                 
%                 Beobachtung: $\exists m\neq n$, so dass $\tilde\delta(q_0,0^m)=\tilde\delta(q_0,0^n)=q'$ weil $Q$ endlich.
%                 \begin{itemize}
%                         \item Falls nun $\tilde\delta(q',1^m)\in F$, dann ist auch $\tilde\delta(q_0,0^n1^m)\in F$ und somit $0^n1^m\in L(M)$ mit $n\neq m\ \lightning$
%                         \item Falls $\tilde\delta(q',1^m)\notin F$, dann gilt auch $\tilde\delta(q_0,0^m1^m)\notin F$ und somit $0^m1^m \notin L$ $\lightning$
%                 \end{itemize}
%                 Also kann $M$ nicht existieren!
%         \end{proof}
% \end{Bsp}


Frage: Gegeben seien zwei reguläre Sprachen $L_1$, $L_2$ über einem gemeinsamen Alphabet $\Sigma$; ist dann auch die Vereinigung $L_1\cap L_2$ eine reguläre Sprache? Wir beantworten diese Frage mit dem folgenden Satz.

\begin{Satz}
  Reguläre Sprachen sind abgeschlossen unter der Schnittoperation. (D.h. Gegeben zwei reguläre Sprachen $L_1$, $L_2$ über $\Sigma$ dann ist auch der Schnitt $L_1\cap L_2$ eine reguläre Sprache.)
\end{Satz}

% Der folgende Beweis ist ein typi

\begin{proof}\footnote{Dieser erste Beweis ist außergewöhnlich detailiert. Im den folgenden Beweisen werden wir einfache Umformungen zusammenfassen.}
Da $L_1$ und $L_2$ regulär sind, gibt es zwei \ac{DEA}s $A_1=(\Sigma, Q_1,\delta_1,\qinit_1,F_1)$ und $A_2=(\Sigma, Q_2,\delta_2,\qinit_2,F_2)$ mit $L_1=L(A_1)$ und $L_2=L(A_2)$.
Wir konstruieren nun zunächst den \emph{Produktautomaten für Schnitt} $A_\cap=(\Sigma, Q_\cap,\delta_\cap,\qinit_\cap,F_\cap)$ wie folgt.
		\begin{align*}
			Q_\cap &= Q_1\x Q_2\\
			\delta_\cap((q_1,q_2),a) &= (\delta_1(q_1,a),\delta_2(q_2,a))\;, \quad \text{für alle } a\in\Sigma\\
			\qinit_\cap &= (\qinit_1,\qinit_2)\\
			F_\cap &= F_1\x F_2\\
		\end{align*}
Anschließend zeigen wir, dass $L(A_\cap)=L(A_1)\cap L(A_2)$ gilt. 
Hierfür zeigen wir zunächst via Induktion über die Länge von $w$, dass für alle $w\in\Sigma^*$, für alle $q_1\in Q_1$ und für alle $q_2\in Q_2$ die folgende Gleichung gilt.
$$\tilde{\delta}_\cap((q_1,q_2),w) = (\tilde{\delta}_1(q_1,w), \tilde{\delta}_2(q_2,w))$$
Der Induktionsanfang für $n=0$ folgt dabei direkt aus \autoref{def:2.deltaschlange}, da $\Eps$ das einzige Wort der Länge $0$ ist.
$$\tilde{\delta}_\cap((q_1,q_2),\Eps) = (q_1,q_2)$$
Den Induktionsschritt $n\rightsquigarrow n+1$ zeigen wir mit Hilfe der folgenden Umformungen, wobei $a\in\Sigma$ ein beliebiges Zeichen und $w\in\Sigma^n$ ein beliebiges Wort der Länge $n$ ist.
\begin{eqnarray*}
  \tilde{\delta}_\cap((q_1,q_2),aw) 
    & \stackrel{\text{\autoref{def:2.deltaschlange}}}{=} & \tilde{\delta}_\cap(\delta_\cap((q_1,q_2),a),w)\\
    & \stackrel{\text{def. }\delta_\cap}{=} & \tilde{\delta}_\cap (\big(\delta_1(q_1,a),\delta_2(q_2,a)\big),w)\\
    & \stackrel{\text{I.V.}}{=} & \big(\tilde{\delta}_1 (\delta_1(q_1,a),w), \tilde{\delta}_2(\delta_2(q_2,a),w)\big)\\
    & \stackrel{\autoref{def:2.deltaschlange}}{=} & \big(\tilde{\delta}_1(q_1,aw), \tilde{\delta}_2(q_2,aw)\big)
\end{eqnarray*}
Schließlich zeigen wir $L(A_\cap)=L(A_1)\cap L(A_2)$ mit Hilfe der folgenden Umformungen für ein beliebiges $w\in\Sigma^*$.
\begin{eqnarray*}
  w\in L(A_\cap)
  & \stackrel{\autoref{def:2.sprache}}{\mathsf{gdw}} & \tilde{\delta}_\cap(\qinit_\cap,w)\in F_\cap\\
  & \stackrel{\text{def. }\qinit_\cap}{\mathsf{gdw}} & \tilde{\delta}_\cap((\qinit_1,\qinit_2),w)\in F_\cap\\
  & \stackrel{}{\mathsf{gdw}} & (\tilde{\delta}_1(\qinit_1,w),\tilde{\delta}_2(\qinit_2,w))\in F_\cap\\
  & \stackrel{\text{def. }F_\cap}{\mathsf{gdw}} & \tilde{\delta}_1(\qinit_1,w)\in F_1 \text{ und } \tilde{\delta}_2(\qinit_2,w)\in F_2\\
  & \stackrel{\autoref{def:2.sprache}}{\mathsf{gdw}} & w\in L(A_1) \text{ und } w\in L(A_2)
\end{eqnarray*}
\end{proof}

\datenote{25.10.17}
Im folgenden Beispiel sei $A_1$ der \ac{DEA} über dem Alphabet $\Sigma=\{0,1\}$ dessen graphische Repräsentation nahezu mit $A_{\mathtt{Macaron}}$ identisch ist.

\begin{center}
\begin{tikzpicture}[node distance = 3cm]
  \node[state, accepting] (0) {$q_0$};
  \node[state, accepting, right of = 0] (1) {$q_1$};
  \node[state, right of = 1] (2) {$q_2$};

  \node[left of = 0, node distance = 1cm] (start){}; 
  \draw[->] (start) to (0);

  \draw[->] (0) to node[above] {$1$} (1);
  \draw[->, loop above] (0) to node[above] {$0$} (0);

  \draw[->] (1) to node[above] {$1$} (2);
  \draw[->, loop above] (1) to node[above] {$0$} (1);

  \draw[->, loop right] (2) to node[right]{$1$, $0$} (2);
\end{tikzpicture}
\end{center}

\goodbreak

\begin{Bsp}\label{bsp:2.schnitt}
Der Produktautomat für Schnitt von $A_1$ und $A_\mathsf{even}$ hat die folgende graphische Repräsentation, wobei wir um Platz zu sparen ``$q_{ij}$'' statt ``$(q_i, q_j)$'' schreiben.

\begin{center}
\begin{tikzpicture}[node distance = 3cm]
  \node[state, accepting] (00) {$q_{00}$};
  \node[state, accepting, right of = 00] (10) {$q_{10}$};
  \node[state, right of = 10] (20) {$q_{20}$};
  \node[state, below of = 00] (01) {$q_{01}$};
  \node[state, below of = 10] (11) {$q_{11}$};
  \node[state, below of = 20] (21) {$q_{21}$};

  \node[left of = 0, node distance = 1cm] (start){}; 
  \draw[->] (start) to (00);

  \draw[->, pos=0.3] (00) to node[above] {$1$} (11);
  \draw[->, pos=0.3] (01) to node[below] {$1$} (10);
  \draw[->, loop above] (00) to node[above] {$0$} (00);
  \draw[->, loop below] (01) to node[below] {$0$} (01);

  \draw[->, pos=0.3] (10) to node[above] {$1$} (21);
  \draw[->, pos=0.3] (11) to node[below] {$1$} (20);
  \draw[->, loop above] (10) to node[above] {$0$} (10);
  \draw[->, loop below] (11) to node[below] {$0$} (11);

  \draw[->,bend left] (20) to node[right] {$1$} (21);
  \draw[->,bend left] (21) to node[right] {$1$} (20);
  \draw[->, loop above] (20) to node[above]{$0$} (20);
  \draw[->, loop below] (21) to node[below]{$0$} (21);
\end{tikzpicture}
\end{center}

\end{Bsp}




\subsection{Minimierung endlicher Automaten}
%  \datenote{26.10.16}

% Betrachte den Automaten aus Beispiel \ref{bsp:3.2}.
% Sei $A =4$, $\Sigma = \{0, 1, 2, 3, 4\}$ mit Zustandsmenge $Q = \Powerset(\Sigma)$.
% D.h.\ unter anderem: $\{0, 1, 3\} \in Q$.
% Hier ist ein Ausschnitt aus dem Zustandsdiagramm:
% 
% \begin{center}
% \begin{tikzpicture}[node distance = 2cm]
%   \node (0) at (0, 0) {$\{0\}$};
%   \node[below of = 0] (03) {$\{0,3\}$};
%   \node[right of = 03] (02) {$\{0,2\}$};
%   \node[below of = 02] (0134) {$\{0,1,3,4\}$};
%   \node[right of = 0] (01) {$\{0,1\}$};
%   \node[right of = 01] (012) {$\{0,1,2\}$};
%   \node[below of = 012] (0123) {$\{0,1,2, 3\}$};
% 
%    \draw[->] (- 0.8, 0) to (0);
%   \draw[->] (0) to node[left] {$3$} (03);
%   \draw[->] (0) to node[above] {$2$} (02);
%   \draw[->] (0) to node[above] {$1$} (01);
%   \draw[->] (03) to node[left] {$1$} (0134);
%   \draw[->] (01) to node[above] {$1$} (012);
%   \draw[->] (02) to node[above] {$1$} (0123);
% \end{tikzpicture}
% \end{center}
% Es ist zu bemerken, dass manche Zustände von $Q$ nie erreicht werden können, z.B.\ $\emptyset$.
% Sei für die folgenden Überlegungen $M = (Q, \Sigma, \delta, \qinit, F)$ ein DEA.

Beobachtung: Der Zustand $q_{01}$ im Beispiel \ref{bsp:2.schnitt} scheint nutzlos. Wir charakterisieren diese ``Nutzlosigkeit'' formal wie folgt.

\begin{Def}
  Ein Zustand $q \in Q$ heißt \emph{erreichbar}, falls ein $w \in \Sigma^*$ existiert, so dass $\hat \delta(\qinit, w) = q$.
%   $M$ heißt \emph{reduziert}, falls alle Zustände erreichbar sind.
\end{Def}
% \begin{Satz}
%   Die Menge der erreichbaren Zustände kann in $O(|Q|*|\Sigma|)$ berechnet werden.
% \end{Satz}
% \begin{proof}~\\
%   \vspace{-\baselineskip}
%   \begin{itemize}
%   \item Fasse $A$ als Graphen auf.
%   \item Wende Tiefensuche an, markiere dabei alle besuchten Zustände.
%   \item Die markierten Zustände bilden die Menge der erreichbaren Zustände.
%   \end{itemize}
% \end{proof}
Bemerkung: Die Menge der erreichbaren Zustände kann mit dem folgenden Verfahren in $O(|Q|*|\Sigma|)$ berechnet werden.
  \begin{itemize}
  \item Fasse $A$ als Graph auf.
  \item Wende Tiefensuche an und markiere dabei alle besuchten Zustände.
  \item Die markierten Zustände bilden die Menge der erreichbaren Zustände.
  \end{itemize}


% \ldots (TODO: hier fehlt noch etwas)
% \datenote{28.10.16}

% \emph{Beobachtung:} Auch ein Automat mit lauter erreichbaren Zuständen muss nicht minimal sein.
% \begin{Bsp}\label{Bsp:3.4}\
%   \begin{center}
%                 \begin{tikzpicture}
%                         \node (start) at (-4,1.5) {};
%                         \node (q0) [circle,double] at (-3,1) {$\qinit$};
%                         \node (q1) [circle,double] at (-0.5,1) {$q_1$};
%                         \node (q2) [circle] at (2,1) {$q_2$};
%                         \node (q3) [circle] at (3.5,1) {$q_3$};
%                         \draw [->] (start) edge (q0);
%                         \draw [->] (q0) edge[loop above] node {0} (q0);
%                         \draw [->] (q0) edge node [auto] {1} (q1);
%                         \draw [->] (q1) edge[loop above] node {0} (q1);
%                         \draw [->] (q1) edge node [auto] {1} (q2);
%                         \draw [->] (q2) edge[loop above] node {1} (q2);
%                         \draw [->] (q2) edge[bend left] node [auto] {0} (q3);
%                         \draw [->] (q3) edge[bend left] node [auto] {1,0} (q2);
%                 \end{tikzpicture}\\
%               \end{center}
%         Dieser Automat erkennt die gleiche Sprache wie in \eqref{bsp:3.1} (,,höchstens eine 1''), hat nur erreichbare Zustände, aber mehr Zustände als in \eqref{bsp:3.1}.
%         
%         \emph{Beobachtung:} Aber $q_2$ und $q_3$ verhalten sich gleich in dem Sinn, dass
%         \[ \forall w: \tilde\delta(q_2,w) \notin F\text{ und }\tilde\delta(q_3,w)\notin F \]
% \end{Bsp}
%
%\stepcounter{Def}
%

\medskip

Beobachtung: Auch nach dem Entfernen der nicht erreichbaren Zustände $q_{01}$ und $q_{10}$ scheint der DEA aus \autoref{bsp:2.schnitt} unnötig groß. Das Verhalten des DEA in den Zuständen $q_{11}$, $q_{20}$ und $q_{21}$ ist sehr ähnlich. 
Wir charakterisieren diese ``Ähnlichkeit'' formal wie folgt.

\begin{Def}[name={[Äquivalenz von \acs*{DEA}-Zuständen]}] %\rlnote{Def.-Num. überprüfen}
  Wir nennen zwei Zustände $p,q\in Q$ eines \ac{DEA} \emph{äquivalent}, geschrieben $p\equiv q$, falls
  \begin{displaymath}
  \forall w\in\Sigma^*, \tilde\delta(p,w)\in F \text{ gdw } \tilde\delta(q,w)\in F
\end{displaymath}
\end{Def}
%\stepcounter{lemma}

\begin{Bsp}
Für \autoref{bsp:2.schnitt} gilt: 
Die Zustände $q_{11}$, $q_{20}$ und $q_{21}$ aus sind paarweise äquivalent,
die Zustände $q_{00}$ und $q_{10}$ sind äquivalent,
keine weiteren Zustandspaare sind äquivalent.

Geschrieben als Menge von Paaren sieht die Relation $\equiv\subset Q\times Q$ also wie folgt aus:
$$\{ \,
(q_{00}, q_{10}), (q_{10}, q_{01}), \,
(q_{11}, q_{20}), (q_{20}, q_{11}), \,
(q_{20}, q_{21}), (q_{21}, q_{20}), \,
(q_{21}, q_{11}), (q_{11}, q_{21}) \,
\}$$
\end{Bsp}


\subsubsection{Exkurs: Äquivalenzrelationen}
Sie haben Äquivalenzrelationen bereits in ``Mathematik II für Studierende der Informatik''\footnote{\url{http://home.mathematik.uni-freiburg.de/junker/ss17/matheII.html}} kennengelernt.
Dieser kurze Exkurs wiederholt die für unsere Vorlesung relevanten Definitionen.
Sie $X$ eine beliebige Menge. Eine \emph{Relation} $R$ über $X$ ist eine Teilmenge des Produktes $X\times X$ (d.h. $R\subseteq X\times X$).

Eine Relation $R\subseteq X\times X$ heißt
\begin{itemize}
 \item \emph{reflexiv}, wenn $\forall x\in X$\ $(x,x)\in R$, 
 \item \emph{symmetrisch}, wenn $\forall x, y\in X$\ $(x,y)\in R \;\; \Rightarrow\;\; (y,x)\in R$, 
 \item \emph{transitiv}, wenn $\forall x, y, z\in X$\ $(x,y)\in R \land (y,z)\in R\;\; \Rightarrow\;\; (x,z)\in R$.
\end{itemize}

\begin{Bsp} Im folgenden interessieren wir uns nur Relationen die alle drei Eigenschaften erfüllen, 
aber die folgenden Beispiele sollen helfen sich mit diesen Eigenschaften vertraut zu machen.

\begin{center}
\begin{tabular}{lccc}
& reflexiv & symmetrisch & transitiv\\ \hline
``gewinnt`` bei Schere, Stein, Papier & nein & nein & nein\\
$(\N, <)$ & nein & nein & ja\\
$(\N, \neq)$ & nein & ja & nein\\
die leere Relation & nein & ja & ja\\
$\{(a,b)\in\mathbb{Z}\times\mathbb{Z}\mid a-b\leq 3\}$ & ja & nein & nein\\
$(\N, \leq)$& ja & nein & ja\\
direkte genetische Verwandtschaft & ja & ja & nein\\
logische Äquivalenz von Formeln & ja & ja & ja\\
\end{tabular}
\end{center}
Bemerkung: Wir können kein Beispiel für eine nicht leere, symmetrische, transitive Relation finden die nicht Reflexiv ist.
Für nicht leere Relationen folgt Reflexivität bereits aus Symmetrie und Transitivität: 
$(a,b)\in R\stackrel{\text{sym}}{\Rightarrow} (b,a)\in R\stackrel{\text{trans}}{\Rightarrow} (a,a)\in R$.

\end{Bsp}

\begin{Def}
Eine Äquivalenzrelation $R$ ist eine Relation die reflexiv, symmetrisch und transitiv ist.

Für ein $x\in X$ nennen wir die Menge $\{y\in X \mid y \equiv x\}$ die \emph{Äquivalenzklasse} von $x$
und verwenden die Notation $[x]_R$ für diese Menge.
Wenn aus dem kontext klar ist, welche Relation gemeint ist dürfen wir das Subskript $\cdot_R$ auch weglassen und schreiben nur $[x]$.

Wenn wir eine Äquivalenzklase mit Hilfe der Notation $[x]_R$ beschreiben, nennen wir $x$ den \emph{Repräsentanten} dieser Äquivalenzklasse.

Wir nennen die Anzahl der Äquivalenzklassen von $R$ den \emph{Index} von $R$.
\end{Def}

Zwei Fakten (ohne Beweis).
\begin{description}
 \item[Fakt 1] Die Äquivalenzklassen von $R$ sind paarweise disjunkt.
 \item[Fakt 2] Die Vereinigung aller Äquivalenzklassen ist die Menge $X$.
\end{description}

Hiermit endet der Exkurs zu Äquivalenzrelationen und wir wollen mit diesem Wissen die oben definierte Relation $\equiv\subseteq Q\times Q$ genauer analysieren.

\begin{lemma}[name={[$\equiv$ ist Äquivalenzrelation]}] %\rlnote{Satz = Lemma-Nummer: 3.2 statt 3.1?}
        Die Relation ,,$\equiv$'' ist eine \emph{Äquivalenzrelation}.
\end{lemma}
\begin{proof}

  Die Relation $\equiv$ ist offensichtlich reflexiv.
  Die Symmetrie und Transitivität von $\equiv$ folgt aus der Transitivität und Symmetrie der logischen Interpretation von ,,genau dann wenn'' (gdw).
\end{proof}


\begin{Bsp} 
Für den \ac{DEA} aus \autoref{bsp:2.schnitt} hat die Relation $\equiv$ drei Äquivalenzklassen.
\footnote{Den Zustand $q_{11}$ als Repräsentanten für die dritte Äquivalenzklasse zu wählen ist eine völlig willkürliche Entschidung. 
Wir könnten genauso gut $q_{20}$ oder $q_{21}$ wählen.}
\begin{align*}
 [q_{00}] & = \{q_{00}, q_{10}\},\\
 [q_{01}] & = \{q_{01}\},\\
 [q_{11}] & = \{q_{11}, q_{20}, q_{21}\}
\end{align*}
\end{Bsp}





Idee: ``Verschmelze'' alle Zustände aus einer Äquivalenzklasse zu einem einzigen Zustand.
Bedenken: Bei einem \ac{DEA} hat jeder Zustand hat für jedes Zeichen einen Nachfolger. 
Wenn wir Zustände verschmelzen könnte es mehrere Nachfolger geben und das Resultat wäre kein wohldefinierter \ac{DEA} mehr.

Das folgende Lemma zeigt dass unsere Bedenken nicht gerechtfertigt sind. 
Sind zwei Zustände äquivalent so sind auch für jedes Zeichen ihre Nachfolger äquivalent.

% \draftnote{28.10.16}
% \eqref{bsp:3.1}:
% \begin{minipage}{.5\textwidth}
%     \captionsetup{type=figure}
%         \begin{tikzpicture}
%                 \node (start) at (-4,1.5) {};
%                 \node (q0) [circle,double] at (-3,1) {$\qinit$};
%                 \node (q1) [circle,double] at (-0.5,1) {$q_1$};
%                 \node (q2) [circle] at (2,1) {$q_2$};
%                 \draw [->] (start) edge (q0);
%                 \draw [->] (q0) edge[loop above] node {0} (q0);
%                 \draw [->] (q1) edge[loop above] node {0} (q1);
%                 \draw [->] (q2) edge[loop above] node {0,1} (q2);
%                 \draw [->] (q0) edge node [auto] {1} (q1);
%                 \draw [->] (q1) edge node [auto] {1} (q2);
%         \end{tikzpicture}
%         \captionof{figure}{Automat zu \eqref{bsp:3.1}}
% \end{minipage}

\begin{lemma}
\label{eqn:delta-wohldefiniert}
Für alle $p,q\in Q$ gilt:
        $$p \equiv q \quad\=>\quad \forall a\in\Sigma\;\; \delta(p,a)\equiv\delta(q,a)$$
\end{lemma}
\begin{proof}
\begin{eqnarray*}
        p \equiv q  
        & \text{gdw} & \forall w\in\Sigma^*: \tilde\delta(p,w)\in F \<=> \tilde\delta(q,w) \in F\\
        & \text{gdw} & (p\in F \<=> q \in F) \land \forall a\in \Sigma: \forall w\in\Sigma^*:
        \tilde\delta(p,aw)\in F \<=> \tilde\delta(q,aw)\in F\\
        &\text{impliziert} &  \forall a\in\Sigma: \forall w\in\Sigma^*: \tilde\delta(\delta(p,a),w)\in F \<=> \tilde\delta(\delta(q,a),w)\in F\\
        & \text{gdw} & \forall a\in\Sigma: \delta(p,a)\equiv\delta(q,a)
\end{eqnarray*}
\end{proof}
Wir formalisieren das ``Verschmelzen'' von Zuständen wie folgt.
\begin{Def}[name={[Äquivalenzklassenautomat]}]
        Der Äquivalenzklassenautomat $\A_\equiv=(Q_\equiv,\Sigma,\delta_\equiv,{\qinit}_\equiv,F_\equiv)$ zu einem DEA $\A=(Q,\Sigma,\delta,\qinit,F)$ ist bestimmt durch:
        \begin{align*}
                Q_\equiv &= \{[q]\mid q\in Q\} & \delta_\equiv([q],a) &= [\delta(q,a)]\\
                \qinit_\equiv &= [\qinit] & F_\equiv & =\{[q]\mid q\in F \}
        \end{align*}
\end{Def}

\begin{Bsp}
Der Äquivalenzklassenautomat $\A_\equiv$ zum \ac{DEA} aus \autoref{bsp:2.schnitt} hat das folgende Zustandsdiagramm.
 % \eqref{bsp:3.1}:
\begin{center}
   \captionsetup{type=figure}
        \begin{tikzpicture}
                \node (start) at (-1.5,1.5) {};
                \node (q0) [circle] at (-3,1) {$[q_{10}]$};
                \node (q1) [circle,double] at (-0.5,1) {$[q_{00}]$};
                \node (q2) [circle] at (2,1) {$[q_{00}]$};
                \draw [->] (start) edge (q1);
                \draw [->] (q0) edge[loop above] node {0} (q0);
                \draw [->] (q1) edge[loop above] node {0} (q1);
                \draw [->] (q2) edge[loop above] node {0,1} (q2);
                \draw [->] (q0) edge node [auto] {1} (q1);
                \draw [->] (q1) edge node [auto] {1} (q2);
        \end{tikzpicture}
 
\end{center}
\end{Bsp}



\begin{Satz}[name={[Äquivalenzklassenautomat ist wohldefiniert]}]
        Der Äquivalenzklassenautomat ist wohldefiniert und $L(\A_\equiv)=L(\A)$.
\end{Satz}


\begin{proof}\ 
        \begin{enumerate}
                \item Wohldefiniert: zu zeigen $\delta_\equiv([q],a) =[\delta(q,a)]$ ist nicht abhängig von der Wahl des Repräsentanten $q\in [q]$. Das folgt direkt aus \autoref{eqn:delta-wohldefiniert}.
                \item $L(\A)=L(\A_\equiv)$:
                Zunächst zeigen wir via Induktion über die Länge von $w$ dass für alle $w\in\Sigma^*$ und alle $q\in Q$ die folgende Äquivalenz.
                $\tilde\delta(q,w)\in F \<=> \tilde\delta_\equiv([q],w)\in F_\equiv$
                
                
                I.A. $(n=0,$ also $w=\Eps)$: $\tilde\delta(q,\Eps)=q\in F \<=> \tilde\delta_\equiv([q],\Eps)=[q]\in F_\equiv$\\
                I.S.: ($n\rightsquigarrow n+1$) \begin{align*}
                \tilde\delta(q,aw_\equiv)\in F &\<==> \tilde\delta(\delta(q,a),w_\equiv)\in F\\ &\xLeftrightarrow{I.V.} \tilde\delta_\equiv([\delta(q,a)],w_\equiv)\in F_\equiv\\
                &\<==> \tilde\delta_\equiv(\delta_\equiv([q],a),w_\equiv)\in F_\equiv\\
                &\<==> \tilde\delta_\equiv([q],a w_\equiv)\in F_\equiv
                \end{align*}
                Mir Hilfe dessen zeigen wir nun $L(\A)=L(\A_\equiv)$. Sei $w\in\Sigma^*$
                \begin{align*}
                 w\in L(\A)
                 & \<==> \tilde\delta(\qinit,w) \in F\\
                 & \<==> \tilde\delta_\equiv([\qinit],w) \in F_\equiv\\
                 & \<==> w\in L(\A_\equiv)
                \end{align*}
        \end{enumerate}
\end{proof}
In den Übungen werden wir ein Verfahren mit $O(|Q|^4\cdot |\Sigma|\log|Q|)$ Laufzeit zur Konstruktion des Äquivalenzklassenautomat kennenlernen. Es gibt aber auch schnellere Verfahren. Z.B. mit dem Algorithmus von Hopcroft kann $\A_\equiv$ in $O(|Q||\Sigma|\log|Q|)$ erzeugt werden.

Wir werden später (\-> Satz von Myhill-Nerode) sehen dass $\A_\equiv$ der kleinste DEA ist, der $L(\A)$ akzeptiert.

\begin{Def}[name={[Rechtskongruente Äquivalenzrelation]}]
        \datenote{27.10.17}
        Eine Äquivalenzrelation $R\subseteq\Sigma^*\x\Sigma^*$ heißt \emph{rechtskongruent}, falls
        \[ (u,v)\in R \quad\=>\quad \forall w\in\Sigma^*\;(u\cdot w,v\cdot w) \in R \]
\end{Def}
%\setcounter{Bsp}{5}
\begin{Bsp} %\rlnote{Bsp.-Num. überprüfen (3.6)}
  \label{Bsp:R_m}
        Für einen \ac{DEA} $\A$ definiere
        \[ R_\A = \{(u,v) \mid \tilde\delta(\qinit,u)=\tilde\delta(\qinit,v)\}. \]
        \begin{description}
                \item[Beobachtung 1] $R_\A$ ist Äquivalenzrelation.
                
                Folgt daraus dass ``='' eine Äquivalenzrelation ist.
                \item[Beobachtung 2] $R_\A$ ist rechtskongruent. 
                
                Beweis: In den Übungen
                \item[Beobachtung 3] Wir haben pro Zustand der von $\qinit$ erreichbar ist genau eine Äquivalenzklasse.
                Index von $R_\A$ ist also die Anzahl der erreichbaren Zustände.
        \end{description}
        
        \medskip
        
        Für den \ac{DEA} aus \autoref{bsp:2.schnitt} hat $R_\A$ die folgenden Äquivalenzklassen.
        \begin{align*}
        [\Eps] &= \{0^n\mid n\in \N\}\\
        [1] &= \{0^n10^m\mid n,m\in \N\}\\
        [11] &= \{w\mid \text{ Anzahl von Einsen in $w$ ist gerade und $\geq 2$}\}\\
        [111] &= \{w\mid \text{ Anzahl von Einsen in $w$ ist ungerade und $\geq 2$}\}\\
        \end{align*}
\end{Bsp}
\begin{Def}
        Für eine Sprache $L\subseteq \Sigma^*$ ist die \emph{Nerode Relation} wie folgt definiert.
        \[ R_L = \{(u,v) \mid \forall w\in\Sigma^*\; uw\in L \<=> vw\in L \} \]
\end{Def}
\begin{description}
 \item[Beobachtung 1] Die Nerode Relation ist eine Äquivalenzrelation.
 
 Folgt daraus dass ``$\Leftrightarrow$'' (Biimplikation, ``Genau dann wenn``) eine Äquivalenzrelation ist.
 \item[Beobachtung 2] Die Nerode Relation ist rechtskongruent.
                \begin{proof}
                Sei $(u,v)\in R_L$. Zeige $\forall w\in\Sigma^*$ , dass $(uw,vw)\in R_L$.
                Wir führen diesen Beweis via Induktion über die Länge von $w$.\\
                I.A. ($n=0$) Für $w=\Eps$ ist $ (u\Eps,v\Eps)=(u,v)\in R_L$.\\
%                 I.S. Sei Aussage gezeigt für alle Wörter $w'$ mit $|w'| \le k$.\\
                I.S. ($n\rightsquigarrow n+1$) Betrachte mit $w=w'a$ ein beliebiges Wort der Länge $n$.
                Nach Induktionsvoraussetzung ist dann auch $(uw', vw') \in R_L $.
                \begin{eqnarray*}
		(uw',vw')\in R_L 
		& \stackrel{\text{def } R_L}{\text{gdw}} & \forall z\in\Sigma^*, \quad uw'z\in L \<=> vw'z\in L\\
		& \stackrel{\text{zerlege } z=az'}{\text{impliziert}} & \forall a\in \Sigma, z'\in\Sigma^*: uw'az'\in L \<=> vw'az'\in L\\
		& \stackrel{\text{def } R_L}{\text{gdw}} & (uw'a,vw'a)\in R_L
	      \end{eqnarray*}
              \end{proof}
\end{description}

\begin{Bsp}
Sei $\Sigma=\{0,1\}$. Die Sprache $L=\{w\in\Sigma^*\mid \text{vorleztes Zeichen ist } 1\}$ hat die folgenden Äquivalenzklassen bezüglich der Nerode Relation.
        \begin{align*}
        [\Eps] &= \{w\mid w \text{ endet mit } 00 \} \cup \{\varepsilon, 0\}\\
        [1] &= \{w\mid w \text{ endet mit } 01 \} \cup \{1\}\\
        [10] &= \{w\mid w \text{ endet mit } 10 \}\\
        [11] &= \{w\mid w \text{ endet mit } 11 \}
        \end{align*}
\end{Bsp}

\begin{Bsp}
      Für ein beliebiges Alphabet $\Sigma$ gilt:
      \begin{enumerate}
       \item Die Sprache $L=\{\Eps\}$ hat genau zwei Äquivalenzklassen bezüglich der Nerode Relation.
      Eine Äquivalenzklasse ist $\{\Eps\}$ die andere ist $\Sigma^+$.
       \item Die Sprache $L=\{\}$ hat genau eine Äquivalenzklassen (nämlich $\Sigma^*$) bezüglich der Nerode Relation.
      \end{enumerate}
\end{Bsp}

\begin{Bsp}
Sei $\Sigma=\{0,1\}$. Die Sprache $L_\text{centered}=\{0^n10^n \mid n\in\N\}$ hat bezüglich der Nerode Relation die folgende Menge von Äquivalenzklassen:
$$\{ [w'] \mid w' \text{ ist Prefix eines Wortes } w\in L_\text{centered}\} \cup \{ [11] \}$$
Dabei gilt, dass für je zwei verschiedene $k\in\N$
auch die Äquivalenzklassen $[0^k1]$ verschieden sind. Somit gibt es unendlich viele Äquivalenzklassen.

Bemerkung: Die Äquivalenzklasse $[11]$ enthält alle Wörter die kein Präfix eines Wortes aus $L_\text{centered}$ sind.
 
\end{Bsp}


% Beispiel: Drei Sprachen $L_a, L_b$ und $L_c$ mit unterschiedlichem Index  (Anzahl an Äquivalenzklassen) bzgl. der Nerode Relation:
% \begin{alignat*}{2}
%         &\begin{rcases}
%         L_a=\{\Eps\} & [\Eps]\equiv[\Eps]\\
%         w,v\in\Sigma^*,\ w,v\ne\Eps & [w]=[v]
%         \end{rcases} &\ &\text{Index}= 2\\
%         &L_b = \varnothing, L_c= \Sigma^* &&\text{Index}= 1
% \end{alignat*}
% Die Menge der Äquivalenzklassen der Nerode Relation $R_{L_a}$ auf $L_a$ ist
% \[ L_a / R_{L_a} = \{ [\Eps], [w] \} \]

\begin{Satz}[Myhill und Nerode] % 3.4
        Die folgende Aussagen sind äquivalent:
        \begin{enumerate}
                \item\label{itm:Nerode1} $L\subseteq \Sigma^*$ wird von \ac{DEA} akzeptiert.
                \item\label{itm:Nerode2} $L$ ist Vereinigung von Äquivalenzklassen einer rechtskongruenten Äquivalenzrelation mit \emph{endlichem} Index.
                \item\label{itm:Nerode3} Die Nerode Relation $R_L$ hat \emph{endlichen} Index
        \end{enumerate}
\end{Satz}

% \datenote{2.11.16}
\begin{proof}
  Wir beweisen die paarweise Äquivalenz in drei Schritten:

  \begin{center}
    (1) \=> (2), \quad \mbox{(2) \=> (3)}\quad \text{und} \quad (3) \=> (1)
  \end{center}

        \paragraph{(1) \=> (2)}: Sei $\A$ ein \ac{DEA} mit
  \begin{displaymath}
    L(\A)=\{w \mid \tilde\delta(\qinit,w)\in F \} = \bigcup_{q \in F} \{ w \mid \tilde\delta(\qinit,w) = q\}
\end{displaymath}

Nun sind $\{ w \mid \tilde\delta(\qinit,w)=q \} = [q]_\A$ genau die Äquivalenzklassen der Relation $R_\A$ aus Bsp \ref{Bsp:R_m}, einer rechtskongruenten Äquivalenzrelation.
Der Index ist die Anzahl der erreichbaren Zustände und somit endlich: $\operatorname{\operatorname{Index}}(R_\A) \le |Q| < \infty$.
        
\paragraph{(2) \=> (3)} Sei $R$ rechtskongruente Äquivalenzrelation mit endlichem Index, so dass $L$ Vereinigung von $R$-Äquivalenzklassen
        
Es genügt zu zeigen, dass die Nerode Relation $R_L$ eine Obermenge von $R$ ist.\footnote{
Zur Erklärung: Falls $R \subseteq R_L$, dann $\operatorname{Index}(R) \ge \operatorname{Index}(R_L)$.
Intuitiv: Je mehr Elemente eine Äquivalenzrelation $R$ enthält, desto mehr Elemente sind bzgl. dieser Relation äquivalent, d.h. desto weniger unterschiedliche Klassen gibt es.
}
\begin{comment}
% Ausformulierter Beweis
Es folgt $u \in L \Leftrightarrow v \in L$, da $L$ aus Vereinigung von Äquivalenzklassen besteht. Da $R$ rechtskongruent ist, folgt für alle $w$:
\[ uw \in L \Leftrightarrow vw \in L \]
Das aber ist die Definition der Nerode-Relation, d.h. $(u, v) \in R_L$. Da wir $(u, v)$ beliebig gewählt haben, gilt somit: 
\[\text{\#Klassen($R_L$) $\leq$ \#Klassen($R$)}<\infty\]
\end{comment}
\begin{align*}
(u, v) \in R & \quad\Rightarrow\quad u \in L \Leftrightarrow v \in L \text{, \ \ \  da L Vereinigung von Äquivalenzklassen ist} \\
& \quad\Rightarrow\quad \forall w \in \Sigma^*\; uw \in L \Leftrightarrow vw \in L \text{, \ \ \ da $R$ rechtskongruent } \\
& \quad\Rightarrow\quad (u, v) \in R_L \text{, \ \ \  nach Definition der Nerode Relation}
\end{align*}
Es gilt also $R \subseteq R_L$ und somit $\operatorname{Index}(R_L) \leq \operatorname{Index}(R)<\infty$.


\paragraph{(3) \=> (1)} Gegeben $R_L$, konstruiere $\A=(Q,\Sigma,\delta,\qinit,F)$
    \begin{itemize}
    \item $Q = \{ [w]_{R_L} \mid w\in \Sigma^* \}$ \quad endlich, weil \textit{index}($R_L$) endlich
    \item $\delta([w],a) = [wa]$ wohldefiniert, da $R_L$ rechtskongruent
    \item $\qinit = [\Eps]$
    \item $F = \{ [w] \mid w\in L \}$
    \end{itemize}
    Wir wollen nun $L(\A)=L$ zeigen. Dafür beweisen wir zunächst via Induktion über die Länge von $w$ die folgende Eigenschaft.
        \begin{displaymath}
      \forall w\in\Sigma^* :
                        \forall v\in\Sigma^* : \tilde\delta([v],w) = [v\cdot w]
    \end{displaymath}
    \begin{description}
    \item[IA] $(w = \Eps)$: $\tilde\delta([v],\Eps) = [v]=[v\cdot\Eps]$
    \item [IS] Sei $w = aw'$ beliebiges Wort der Länge $n+1$.
      \begin{displaymath}
        \begin{array}{lcl}
        \tilde\delta([v],aw') &=& \tilde\delta(\delta([v],a),w') \\
                             &=& \tilde\delta([v\cdot a],w') \\
                             &\stackrel{\text{I.V.}}{=}&[va\cdot w'] \\
                             &=& [v\cdot \underbrace{aw'}_{=w}]
        \end{array}
      \end{displaymath}
    \end{description}

Nun zeigen wir $L(\A)=L$ wie folgt:
\begin{eqnarray*}
        w\in L(\A)
        & \text{gdw} & \tilde\delta([\Eps],w)\in F\\
        & \text{gdw} & [w]\in F, \quad \text{(via Induktion gezeigte Eigenschaft für $v=\Eps$)}\\
        & \text{gdw} & w\in L
\end{eqnarray*}
\end{proof}
%
\setcounter{Korollar}{4}
\begin{Korollar}\label{kor:2.minAutomat}\datenote{3.11.16}
        Der im Beweisschritt (3) \=> (1) konstruierte Automat $\A$ ist minimaler Automat für eine reguläre Sprache $L$.
\end{Korollar}
\begin{proof}
        Sei $\A'$ \emph{beliebiger} \ac{DEA} mit $L(\A')=L$.
        
        Aus ``\ref{itm:Nerode1} {\=>} \ref{itm:Nerode2}'' wissen wir, dass $\operatorname{index}(R_{\A'})\leq |Q'|$ gilt.
        
        Aus ``\ref{itm:Nerode2} \=> \ref{itm:Nerode3}'' wissen wir, dass $R_{\A'}\subseteq R_L$ und somit $\operatorname{index}(R_L) \leq \operatorname{index}(R_{\A'})$ gilt.
        
        In ``\ref{itm:Nerode3} \=> \ref{itm:Nerode1}'' definieren wir $A$ sodass $|Q|=\text{index}(R_L) \leq \text{index}(R_\A)\leq |Q'|$.
        
        Für beliebigen DEA $A'$ ist $|Q|$ also nie größer als $|Q'|$.
\end{proof}


\subsection{\acf{PL} für reguläre Sprachen} %\rlnote{subsection \# (3.3)?}
% \datenote{26.10.16 (Eingeschoben)}
Welche interessanten Eigenschaften haben reguläre Sprachen?

Notation: Sei $\operatorname{bin}:\{0,1\}^*\rightarrow \N$ die Dekodierung von Bitstings in natürliche Zahlen.

Z.B. $\operatorname{bin}(101) = 5$, $\operatorname{bin}(\Eps) = 0$

\begin{Bsp*} Betrachete den folgenden \ac{DEA}, der die Sprache der Binärcodierungen von durch drei Teilbaren Zahlen akzeptiert:
                $L = \{ w\in \{0,1\}^* \mid \operatorname{bin}(w)\equiv_3 0\}$
    \begin{center}
                \begin{tikzpicture}
                        \node (start) at (-4,1.5) {};
                        \node (q0) [circle, double] at (-3,1) {$q_0$};
                        \node (q1) [circle] at (-0.5,1) {$q_1$};
                        \node (q2) [circle] at (2,1) {$q_2$};
                        \draw [->] (start) edge (q0);
                        \draw [->] (q0) edge [loop above] node {0} (q0);
                        \draw [->] (q0) edge node [auto] {1} (q1);
                        \draw [->] (q1) edge [bend left] node [auto] {1} (q0);
                        \draw [->] (q1) edge node [auto] {0} (q2);
                        \draw [->] (q2) edge [bend left] node [auto] {0} (q1);
                        \draw [->] (q2) edge [loop above] node {1} (q2);
                \end{tikzpicture}
  \end{center}
  Beobachtungen:
  \begin{itemize}
  \item Es gilt offensichtlich, dass $11 \in L$
  \item Es gilt auch, dass $1 \underline{0 0} 1 \in L$.
  \item Der Automat hat eine Schleife bei $\tilde\delta({q_1,00}) = q_1$, die mehrfach ,,abgelaufen'' werden kann ohne die Akzeptanz zu beinflussen.
  \item Also gilt auch $100001 \in L$,
  \item und im Allgemeinen $\forall i\in\N: 1(00)^i1 \in L$
  \end{itemize}
\end{Bsp*}
Verdacht: Alle ``langen'' Wörter lassen sich in der Mitte ``aufpumpen''.
Wir formalisieren diesen Verdacht im folgenden Lemma.

\begin{lemma}[Pumping Lemma]\label{lem:pumping}
        Sei $L$ eine reguläre Sprache. Dann gilt:
        \begin{alignat*}{2}
                &\exists n\in\N,\ n>0:\quad \forall z\in L,\ |z|\geq n:\\
                &\exists u,v,w\in\Sigma^* :\\
                &z = uvw,\ |uv| \leq n,\ |v| \geq 1\\
                \text{und }& \forall i\in\N:\ uv^iw\in L
        \end{alignat*}
\end{lemma}
\vspace{-2em}
\begin{proof}
	Sei $\A=(\Sigma, Q,\delta,\qinit,F)$ ein beliebiger \ac{DEA} für $L$.
	Wähle $n=|Q|$ und $z\in L$ beliebig mit $|z|\geq n$.
	
	\vspace{-1em}
	
	Beim Lesen von $z$ durchläuft $\A$ genau $\overbrace{|z|+1}^{\geq n+1}$ Zustände und somit gibt es mindestens einen Zustand $q\in Q$ der mehrmals besucht wird (Schubfachprinzip).
	
	Wähle das $q$, dessen zweiter Besuch zuerst passiert.
	\begin{alignat*}{3}
		\text{Nun gilt: }\quad
		\exists u:\; && \tilde\delta(\qinit,u)&=q &\qquad& u\text{ Präfix von }z\\
		\exists v:\; && \tilde\delta(q,v)&=q && uv\text{ Präfix von }z\\
		\exists w:\; && \tilde\delta(q,w)&\in F && uvw=z\\
		&& |v| &\geq 1\\
		&& |uv| &\leq n && \text{da $q$ zwei mal besucht}
	\end{alignat*}
	\begin{alignat*}{2}
		\text{Es folgt für beliebiges $i\in\N$ :}\quad \tilde\delta(\qinit,uv^iw) &= \tilde\delta(q,v^iw)\\
		&= \tilde\delta(q,w) &\qquad& \text{denn }\forall i: \tilde\delta(q,v^i)=q\\
		&\in F \tag*{\qedhere}
	\end{alignat*}
\end{proof}

%
\begin{Bsp*}
        Die Sprache $L_\text{centered}=\{0^n10^n \mid n\in\N\}$ ist nicht regulär.
        
        Wir geben hierfür einen Widerspruchsbeweis mit Hilfe des Pumping Lemma \ac{PL}.
        
        Sei $n$ die Konstante aus dem \ac{PL}. Wähle $z=0^n10^n$. 
        (Gültige Wahl, da $|z|=2n+1\geq n$)\\
        Laut PL existieren $u$, $v$, $w$, sodass $z=uvw$ mit $|v|\geq 1, |uv|\leq n$ und $\forall i \in \N$ $uv^iw \in L$. 
        Nach Wahl von $z$ gilt nun
  \begin{itemize}
  \item $uv = 0^m$ mit $m\leq n$
  \item $v = 0^k$ mit $k\geq 1$
  \item $w = 0^{n-m}10^n$ 
  \end{itemize}
  Betrachte $uv^2w = 0^{m-k}0^{2k}0^{n-m}10^n = 0^{n+k}10^n \notin L$. Widerspruch!
  Somit ist $L$ nicht regulär.
  
  Zur Illustration:
        \begin{gather*}
                \underbrace{0\ \dots\dots\ 0}_{n} 1 \underbrace{0\ \dots\dots\ 0}_{n}\\
                |\!\ruleplaceholder[u]{\widthof{0\ \dots }} \!|\! \ruleplaceholder[v]{\widthof{\dots 0}}\!|% 
                \ruleplaceholder[w]{\widthof{\ \ \ $1\ \dots\dots\ 1$}}\!|
        \end{gather*}
\end{Bsp*}

% \begin{Bsp*}
%         $L=\{0^{x^2} \mid x\in\N\}$ ist nicht regulär.\\
%         Sei $n$ die Konstante aus dem \ac{PL}.
%         
%         Wähle $z=0^{n^2}$. Also $|z|=n^2\geq n$\\
%         Laut PL existieren $u$, $v$, $w$, sodass $z=uvw$ mit $|v|\geq 1, |uv|\leq n$ und $\forall i \in \N$ $uv^iw \in L$. Nach Wahl von $z$ gilt nun
%   \begin{itemize}
%   \item $uv = 0^m$ mit $m\leq n$
%   \item $v = 0^k$ mit $k\geq 1$
%   \item $w = 0^{n^2 - m}$ mit $k\geq 1$
%   \end{itemize}
%   Betrachte $uv^2w = uvvw = 0^{m}0^k0^{n^2-m} = 0^{n^2+k}$.
%   Da $n^2+k$ keine Quadratzahl sein kann ist $uv^2w \not \in L$, und somit ist $L$ nicht regulär.
%   Begründung: betrachte $(n+1)^2 - n^2 = n^2 + 2n + 1 - n^2 = 2n + 1$.
%   Aber $k \le m \le n \le 2n + 1$.
% \end{Bsp*}
% 
% \begin{Bsp*}
% $L_2 = \{0^p \mid p\text{ ist Primzahl}\}$ ist nicht regulär.
% 
% Sei $n$ Konst. aus dem \ac{PL}, $p$ Primzahl mit $p \geq n$.\\
% Wähle $z=0^p \in L_2$
% \begin{align*}
%         \ac{PL}:\ &z=uvw \text{ mit } |uv|\leq n &&,|v| \geq 1\\
%         &\curvearrowright |z|= p=a+b &&, a = |uw| \quad, b= |v|\\
%         &\curvearrowright |uv^iw| = a + ib &&, \text{w"ahle }i=p+1\\
%         &\curvearrowright |uv^{p+1}w| = a + (p+1)b & =& a + pb + b = p+pb \text{ keine Primzahl} \\
%         \text{Also } &uv^{p+1}w \notin L_2\\
%         &\curvearrowright L_2\text{ nicht regulär.}
% \end{align*}
% \end{Bsp*}


\subsection[\acf{NEA}]{Nichtdeterministischer endlicher Automat (NEA)}
Aufgabe: Konstruiere für eine natürliche Zahl $n\in\N$ einen DEA für die folgende Sprache.
	\begin{align*}
		L_n &= \{ w\in\{0,1\}^* \mid \text{das $n$-letzte Symbol von $w$ ist 1} \}
	\end{align*}
Naiver Lösungsversuch:
	\begin{figure}[H]\centering
		\begin{tikzpicture}[>=stealth, shorten >=1pt,
				node distance=1.5cm, on grid, initial text=,
				every state/.style={minimum size=0pt,inner sep=0pt}
			]
			\node[state,initial] (q0) {};
			\node[state] (q1) [right of=q0] {};
			\node        (q2) [right of=q1] {\dots};
			\node[state,accepting] (q3) [right of=q2] {};
			\path[->]
				(q0) edge [loop above] node [auto] {$\Sigma$} ()
				     edge              node [auto] {1}        (q1)
				(q1) edge              node [auto] {$\Sigma$} (q2)
				(q2) edge              node [auto] {$\Sigma$} (q3)
			;
			\draw [thick, decoration={brace, mirror, raise=.3cm, amplitude=10pt}, decorate]
			    (q1.west) -- (q3.east)
			    node [pos=0.5,anchor=north,yshift=-0.65cm] {n};
		\end{tikzpicture}
		\caption{Nichtdet. Automat für $L_n$}
		\label{fig:2.nletztesNea}
	\end{figure}
Problem: Diagramm beschreibt keinen DEA. Startzustand hat zwei ausgehende Kanten für $1$.

Untersuche Sprache mit Hilfe von Nerode Relation.
Beobachtung: Je zwei Wörter der Länge $n$ sind in unterschiedlichen Äquivalenzklassen.
Es gibt also mindestens $2^n$ Äquivalenzklassen und aus \autoref{kor:2.minAutomat} wissen wir dass ein minimaler DEA, der $L_n$ akzeptiert $2^n$ Zustände hat.

Idee: Definiere einen neue Art von Automaten bei dem ein Zustand pro Zeichen mehrere Nachfolger haben darf.

% \begin{Bsp*} Mustererkennung\\
% 	kommt ein String (konsistent) in einem anderen vor?
% 	
% 	Gegeben: festes Wort $w$.\\
% 	Gesucht: Sprache aller Worte, in denen $w$ als Teilwort vorkommt.
% 	\begin{align*}
% 		L &= \{ v\in\Sigma^* \mid \exists u,x\in\Sigma^*, v=uwx \}\\
% 		\Sigma &= \{a,b,c\}\\
% 		& \text{konkretes Beispiel:}\\
% 		w &= abac
% 	\end{align*}
% 	\begin{figure}[tp]
% 	\centering
% 		\begin{tikzpicture}[>=stealth, shorten >=1pt,
% 				node distance=2cm, on grid, initial text=,
% 				every state/.style={minimum size=0pt,inner sep=0pt}
% 			]
% 			\node[state,initial] (q0) {};
% 			\node[state] (q1) [right of=q0] {};
% 			\node[state] (q2) [right of=q1] {};
% 			\node[state] (q3) [right of=q2] {};
% 			\node[state,accepting] (q4) [right of=q3] {};
% 			\path[->]
% 				(q0) edge [loop above]    node [auto]  {$b,c$}    ()
% 				     edge                 node [auto]  {$a$}      (q1)
% 				(q1) edge [loop above]    node [auto]  {$a$}      ()
% 				     edge [bend left]     node [auto]  {$c$}      (q0)
% 				     edge                 node [auto]  {$b$}      (q2)
% 				(q2) edge [bend left=50]  node [auto]  {$b,c$}    (q0)
% 				     edge                 node [auto]  {$a$}      (q3)
% 				(q3) edge                 node [auto]  {$c$}      (q4)
% 				     edge [bend right=70] node [above] {$a$}      (q1)
% 				     edge [bend right=40] node [above] {$b$}      (q2)
% 				(q4) edge [loop right]    node [auto]  {$\Sigma$} ()
% 			;
% 		\end{tikzpicture}
% 	\caption{DEA für $L$}
% 	\label{fig:dfa-teilwort}
% 	\end{figure}
% 	\hyperref[fig:dfa-teilwort]{Abbildung~\ref*{fig:dfa-teilwort}} enthält einen \ac{DEA} für die Sprache $L$. Beobachtung: nicht-trivial zu konstruieren.
% 	\begin{figure}[tp]\centering
% 		\begin{tikzpicture}[>=stealth,shorten >=1pt,
% 				node distance=2cm,on grid,
% 				initial text=
% 			]
% 			\node[state,initial] (q0) {$q_0$};
% 			\node[state] (q1) [right of=q0] {$q_1$};
% 			\node[state] (q2) [right of=q1] {$q_2$};
% 			\node[state] (q3) [right of=q2] {$q_3$};
% 			\node[state,accepting] (q4) [right of=q3] {$q_4$};
% 			\path[->]
% 				(q0) edge [loop above] node [auto] {$\Sigma$} ()
% 				     edge              node [auto] {$a$} (q1)
% 				(q1) edge              node [auto] {$b$} (q2)
% 				(q2) edge              node [auto] {$a$} (q3)
% 				(q3) edge              node [auto] {$c$} (q4)
% 				(q4) edge [loop above] node [auto] {$\Sigma$} ()
% 			;
% 		\end{tikzpicture}
% 		\caption{Bsp.: Mustererkennung}
% 		\label{fig:nfa-teilwort}
% 	\end{figure}
% 	
% 	\hyperref[fig:nfa-teilwort]{Abbildung~\ref*{fig:nfa-teilwort}} enthält einen nicht-deterministischen endlichen Automat für die Sprache $L$. Idee: Ein Wort $w$ wird akzeptiert, falls es einen mit $w$ markierten Pfad von $q_0$ zu einen akzeptierenden Zustand gibt.
% 	\begin{figure}[tp]
% 	\centering
% 		\begin{tikzpicture}[>=stealth,shorten >=1pt,
% 				node distance=2cm,on grid
% 			]
% 			\node (q0) {$\{0\}$};
% 			\node (q1) [right of=q0] {$\{0,1\}$};
% 			\node (q2) [right of=q1] {$\{0,2\}$};
% 			\node (q3) [right of=q2] {$\{0,1,3\}$};
% 			\path[->]
% 				(q0) edge [loop below] node [auto] {$b,c$} ()
% 				     edge              node [auto] {$a$} (q1)
% 				(q1) edge [loop below] node [auto] {$a$} ()
% 				(q1) edge              node [auto] {$b$} (q2)
% 				(q2) edge              node [auto] {$a$} (q3)
% 			;
% 		\end{tikzpicture}
% 	\caption{Potenzmengenkonstruktion auf dem NEA}
% 	\label{fig:nfa-teilwort-powerset}
% 	\end{figure}
% 	
% 	\hyperref[fig:nfa-teilwort-powerset]{Abbildung~\ref{fig:nfa-teilwort-powerset}} zeigt (einen Ausschnitt) aus dem deterministischen Automaten, der schematisch aus dem \acsu{NEA} in \autoref{fig:nfa-teilwort} konstruiert werden kann. Idee: bei Schritt mit Symbol $a$ ist der \ac{NEA} gleichzeitig in allen Zuständen, die durch $a$ von (der Menge der) aktuellen Zustände erreichbar sind.
% 	
% 	Variante: erkenne \textbf{Subwort} $w=a_1,\dots,a_n$
% 	\[ L' = \{ v\in\Sigma^* \mid \exists x_0,\dots,x_n\in\Sigma^*, v=x_0a_1x_1a_2\dots a_nx_n \} \]
% 	Nicht det. Automat für $L'$ mit $(w=abac)$ ist sehr einfach. Der entsprechende deterministische Automat ist deutlich komplizierter. (selbst)
% 	\begin{figure}[H]\centering
% 		\begin{tikzpicture}[>=stealth, shorten >=1pt,
% 				node distance=2cm, on grid, initial text=,
% 				every state/.style={minimum size=0pt,inner sep=0pt}
% 			]
% 			\node[state,initial] (q0) {};
% 			\node[state] (q1) [right of=q0] {};
% 			\node[state] (q2) [right of=q1] {};
% 			\node[state] (q3) [right of=q2] {};
% 			\node[state,accepting] (q4) [right of=q3] {};
% 			\path[->]
% 				(q0) edge [loop above] node [auto] {$\Sigma$} ()
% 				     edge              node [auto] {$a$}      (q1)
% 				(q1) edge [loop below] node [auto] {$\Sigma$} ()
% 				     edge              node [auto] {$b$}      (q2)
% 				(q2) edge [loop below] node [auto] {$\Sigma$} ()
% 				     edge              node [auto] {$a$}      (q3)
% 				(q3) edge [loop below] node [auto] {$\Sigma$} ()
% 				     edge              node [auto] {$c$}      (q4)
% 				(q4) edge [loop right] node [auto] {$\Sigma$} ()
% 			;
% 		\end{tikzpicture}
% 		\caption{Nichtdet. Automat für $L'$}
% 	\end{figure}
% 	
% 	Weiteres Beispiel, bei dem der deterministische Automat beweisbar exponentiell größer ist.
% 	\begin{align*}
% 		L_n &= \{ w\in\{0,1\}^* \mid \text{das $n$-letzte Symbol von $w$ ist 1} \}
% 	\end{align*}
% 	\begin{figure}[H]\centering
% 		\begin{tikzpicture}[>=stealth, shorten >=1pt,
% 				node distance=1.5cm, on grid, initial text=,
% 				every state/.style={minimum size=0pt,inner sep=0pt}
% 			]
% 			\node[state,initial] (q0) {};
% 			\node[state] (q1) [right of=q0] {};
% 			\node        (q2) [right of=q1] {\dots};
% 			\node[state,accepting] (q3) [right of=q2] {};
% 			\path[->]
% 				(q0) edge [loop above] node [auto] {$\Sigma$} ()
% 				     edge              node [auto] {1}        (q1)
% 				(q1) edge              node [auto] {$\Sigma$} (q2)
% 				(q2) edge              node [auto] {$\Sigma$} (q3)
% 			;
% 			\draw [thick, decoration={brace, mirror, raise=.3cm, amplitude=10pt}, decorate]
% 			    (q1.west) -- (q3.east)
% 			    node [pos=0.5,anchor=north,yshift=-0.65cm] {n};
% 		\end{tikzpicture}
% 		\caption{Nichtdet. Automat für $L_n$}
% 	\end{figure}
% 	
% 	Der entsprechende deterministische Automat für $L_n$ hat $\sim 2^n$ Zustände.
% \end{Bsp*}

\goodbreak

\begin{Def}[\acs*{NEA}]
        Ein \emph{\acf{NEA}}, (\acsu{NFA} $\hat=$ \acl{NEA}) ist ein 5-Tupel
        \[ \calN= (\Sigma, Q,\delta,\qinit,F) \]
dabei ist
        \begin{itemize}
                \item $\Sigma$ ein Alphabet,
                \item $Q$ eine \emph{endliche} Menge deren Elemente wir \emph{Zustände} nennen,
                \item $\delta:Q\x\Sigma\->\mathcal{P}(Q)$ eine Funktion die wir \emph{Transitionsfunktion} nennen,
                \item $\qinit\in Q$ ein Zustand den wir \emph{Startzustand} nennen und
                \item $F\subseteq Q$ eine Teilmenge der Zustände deren Elemente wir \emph{akzeptierende} Zustände nennen.
        \end{itemize}
\end{Def}
Bemerkung: Die Definition des NEA unterscheidet sich vom DEA also nur in der Transitionsfunktion. Beim DEA ist der Bildbereich der Transitionsfunktion die Menge der Zustände $Q$. Hier ist der Bildbereich die Potenzmenge der Zustandsmenge $\mathcal{P}(Q)$.
Analog zu DEAs werden wir auch NEAs mit Hilfe eines Zustandsdiagramms beschrieben. Z.B. beschreibt \autoref{fig:2.nletztesNea} für jedes $n\in\N$ einen NEA für die Sprache $L_n$.

Im folgenden sei $\calN$ immer ein NEA.
\begin{Def}[name={[Lauf eines Automaten]}]
	Wir nennen eine Folge von Zuständen $q_0q_1\dots q_n$ einen \emph{Lauf von $\calN$ über $w=a_1\dots a_n$} wenn
	$q_i\in\delta(q_{i-1},a_i)$ für alle $i$ mit $1\leq i\leq n$.
	Wir nennen einen Lauf \emph{initial} falls $q_0=\qinit$.
	Wir nennen einen Lauf \emph{akzeptierend}, falls $q_n\in F$.
\end{Def}
\datenote{8.11.17}
\begin{Def}[name={[NEA zu DEA]}]
	Ein Wort $w\in\Sigma^*$ wir von $\calN$ \emph{akzeptiert}, falls $\calN$ einen initialen und akzeptierenden Lauf über $w$ hat.
	Die von $\calN$ akzeptierte Sprache ist die Menge der von $\calN$ akzeptierten Wörter. D.h.	
	$L(\calN)=\{ w\in\Sigma^* \mid \exists\text{ initialer, akzeptierender Lauf von $\calN$ über }w \}$
\end{Def}




\begin{Bsp}
\label{bsp:2.zweitletztes} Der NEA für die Sprache 
$$L_2=\{w\in\{0,1\}^*\mid \text{ das zweitletzte Zeichen von $w$ ist $1$ }\}$$ hat die folgende graphische Repräsentation.
  \begin{center}
                \begin{tikzpicture}[circle/.style={
                        shape=circle,
                        minimum size=0.5cm,
                        text=black, draw,
                        text width=0.5cm,
                        align=center}]
                        \node (vs) at (-1.0,1.0) {};
                        \node [circle] (v0) at (0,0) {$q_{0}$};
                        \node [circle] (v1) at (3,0) {$q_{1}$};
                        \node [circle, double] (v2) at (6,0) {$q_{2}$};
                        \draw [->] (vs) edge (v0);
                        \draw [->] (v0) edge [loop above] node[auto] {0,1} (v2);
                        \draw [->] (v0) edge [bend left=15] node[auto] {1} (v1);
                        \draw [->] (v1) edge [bend left=15] node[auto] {0,1} (v2);
                \end{tikzpicture}
  \end{center}
\end{Bsp}


Bemerkung: Die Frage ob ein gegebenes Wort $w$ akzeptiert wird (das ``Wortproblem'') lässt sich für NEAs nicht mehr so leicht beantworten wie wir es von DEAs gewohnt sind.
Ein sinvolles Vorgehen scheint jeden initialen Lauf zu betrachten, doch z.B. für das Wort $111$ hat obiger NEA bereits  drei verschiedene initiale Läufe: $q_0q_0q_0$, $q_0q_0q_1$, $q_0q_0q_2$.


Bemerkung: Die Definitionen von \ac{NEA} und \ac{DEA} in der Literatur sind nicht einheitlich. 
Es gibt äquivalente \ac{NEA} Definitionen die statt der Transitionsfunktion $\delta:Q\x\Sigma\->\mathcal{P}(Q)$ eine Transitionsrelation $\delta\subseteq Q\x\Sigma\x Q$ verwenden.
Es gibt alternative \ac{NEA} Definitionen die eine Menge von Startzuständen erlauben.
Alternativ könnte man auch zunächst den \ac{NEA} einführen und den \ac{DEA} als Spezialfall dessen definieren (Spezialfall: Bild von Transitionsfunktion ist einelementig für alle $q\in Q$ und $a\in\Sigma$)

Bemerkung: Zu jedem \ac{DEA} $\A= (\Sigma, Q,\delta,\qinit,F)$ gibt es einen \ac{NEA} der die gleiche Sprache akzeptiert.
Z.B. der \ac{NEA} $\calN= (\Sigma, Q,\delta_\mathsf{NEA},\qinit,F)$, mit $\delta_\mathsf{NEA}(q,a)=\{\delta(q,a)\}$ der sich von $\A$ nur in der Transitionsfunktion unterscheidet.


\begin{Satz}[Rabin und Scott]\label{satz:2.rabinscott}
	Zu jedem \ac{NEA} $\calN$ mit $n$ Zuständen gibt es einen \ac{DEA} $\A_\calP$ mit $2^n$ Zuständen, so dass $L(\A_\calP)=L(\calN)$.
\end{Satz}
  Zur Vorbereitung des Beweises machen wir zunächst die folgende Definition.

\begin{Def}[Potenzmengenautomat]
 Für gegebenen \ac{NEA}  $\calN= (\Sigma, Q,\delta,\qinit,F)$ ist der Potenzmengenautomat $\A_\calP$ wie folgt definiert
         \begin{align*}
                Q_\calP &= \mathcal{P}(Q)\\
                \delta_\calP(p,a) &= \bigcup_{q\in p} \delta(q,a)\\
                \qinit_\calP &= \{\qinit\}\\
                F_\calP &= \{ p\in Q_\calP \mid p\cap F\neq \varnothing \}\qedhere
        \end{align*}
\end{Def}
\begin{Bsp}
Der Potzenzmengenautomat für den NEA aus \autoref{bsp:2.zweitletztes} hat das folgende Zutandsdiagramm.
  \begin{center}
                \begin{tikzpicture}[circle/.style={
                        shape=circle,
                        minimum size=0.5cm,
                        text=black, draw,
                        text width=0.5cm,
                        align=center}]
                        \node (vs) at (-1.0,1.0) {};
                        \node [draw, rounded corners=3mm] (v0) at (0,0) {$\{q_{0}\}$};
                        \node [draw, rounded corners=3mm] (v1) at (2,0) {$\{q_{0}, q_1\}$};
                        \node [draw, rounded corners=3mm, double] (v2) at (5,1) {$\{q_0, q_{2}\}$};
                        \node [draw, rounded corners=3mm, double] (v3) at (5,-1) {$\{q_0, q_1, q_2\}$};
                        \node [draw, rounded corners=3mm, double] (v4) at (8,1) {$\{q_1, q_2\}$};
                        \node [draw, rounded corners=3mm] (v5) at (12,1) {$\{q_1\}$};
                        \node [draw, rounded corners=3mm, double] (v6) at (10,1) {$\{q_2\}$};
                        \node [draw, rounded corners=3mm, double] (v7) at (10,-1) {$\{\}$};
                        \draw [->] (vs) edge (v0);
                        \draw [->] (v0) edge [loop above] node[auto] {0} (v2);
                        \draw [->] (v0) edge [bend left=0] node[below] {1} (v1);
                        \draw [->] (v1) edge [bend left=15] node[auto] {0} (v2);
                        \draw [->] (v1) edge [bend left=0] node[below] {1} (v3);
                        \draw [->] (v2) edge [bend right=30] node[above] {0} (v0);
                        \draw [->] (v2) edge [bend left=15] node[auto] {1} (v1);
                        \draw [->] (v3) edge [bend left=0] node[right] {0} (v2);
                        \draw [->] (v3) edge [loop right] node[auto] {1} (v3);
                        \draw [->] (v4) edge [] node[auto] {0,1} (v6);
                        \draw [->] (v5) edge [] node[auto] {0,1} (v6);
                        \draw [->] (v6) edge [] node[auto] {0,1} (v7);
                        \draw [->] (v7) edge [loop right] node[auto] {0,1} (v7);
                \end{tikzpicture}
  \end{center}
Die vier Zustände auf der rechten Seite sind nicht erreichbar.
\end{Bsp}


% \draftnote{4.11.16}
\begin{proof}[von \autoref{satz:2.rabinscott}]
        Zeige $L(\A_\calP)=L(\calN)$. 
        Dafür beweisen wir zunächst via Induktion über die Länge von $w$, die folgende Eigenschaft:
        
        $\forall w\in\Sigma^*\; \forall p\in Q_\calP\backslash\{\{\}\}\; \forall q\in Q$
        $$q\in\tilde\delta_\calP(p,w) \Leftrightarrow \exists \underbrace{q_0, q_1,\ldots q_n}_{\text{Lauf}}\in Q, \text{ sodass }  n=|w|, q_0\in p \text{ und } q_n=q$$
        I.A. $(n=0,$ also $w=\Eps)$: Gilt trivialerweise da $p\neq\{\}$\\
        I.S. ($n\rightsquigarrow n+1$): Seit $w=aw'$ beliebiges Wort der Länge $n+1$.
        \begin{eqnarray*}
                q\in\tilde\delta_\calP(p,w) 
                & \Leftrightarrow & q\in\tilde\delta_\calP(\delta_\calP(p,a),w')\\
                & \stackrel{IV}{\Leftrightarrow} & \exists\underbrace{q_1q_2\ldots q_{n+1}}_{\text{Lauf}}\in Q, \text{ sodass }  n=|w'|, q_1\in \delta_\calP(p,a) \text{ und } q_{n+1}=q\\
                & \Leftrightarrow & \exists\underbrace{q_1q_2\ldots q_{n+1}}_{\text{Lauf}}\in Q, \text{ sodass }  n=|w'|, \exists q_0\in p, q_1\in\delta(q_0,a) \text{ und } q_{n+1}=q\\
                & \Leftrightarrow & \exists\underbrace{q_0q_1q_2\ldots q_{n+1}}_{\text{Lauf}}\in Q,\text{ sodass }  n+1=|w|, q_0\in p \text{ und } q_{n+1}=q
        \end{eqnarray*}
        Mit Hilfe dieser Eigenschaft zeigen wir nun die Gleichheit $L(\A_\calP)=L(\calN)$
        \begin{align*}
         w\in L(\A_\calP) 
         &\<=> \tilde\delta_\calP(\qinit_\calP,w)\in F_\calP\\
         &\<=> \exists q_f\in \tilde\delta_\calP(\qinit_\calP,w)\cap F\\
         &\<=> \exists \underbrace{q_0, q_1,\ldots q_n}_{\text{Lauf}}\in Q, \text{ sodass }  n=|w|, q_0\in \qinit \text{ und } q_n\in F\\
         &\<=> \exists \text{ initialer, akzeptierender Lauf von $\calN$ über $w$} \\
         &\<=> w\in L(\calN)
        \end{align*}
\end{proof}

Bemerkung: Es gelten also die folgenden Äquivalenzen.
$$ L \text{ regulär} 
\quad \stackrel{\autoref{def:2.sprache}}{\Longleftrightarrow} \quad L = L(\A) \text{ für einen \ac{DEA} }\A
\quad \Longleftrightarrow \quad L = L(\calN) \text{ für einen \ac{NEA} }\calN
$$
Bemerkung: \ac{NEA}s sind eine exponentiell kompaktere Repräsentation von regulären Sprachen im folgenden Sinne:
\begin{enumerate}
 \item Es gibt mit $L_n$ ($n$-letztes Zeichen) eine Menge von Sprachen die sich mit Hilfe eines $n+1$ Zustands \ac{NEA} darstellen lassen, aber bei denen ein minimaler \ac{DEA} mindestens $2^n$ Zustände hat. (Siehe Übungsblatt 3, Aufgabe 2)
 \item Zu jedem \ac{NEA} mit $n$ Zuständen gibt es einen \ac{DEA} mit $2^n$ Zuständen der die gleiche Sprache akzeptiert. (\autoref{satz:2.rabinscott})
 \item Zu jedem \ac{DEA} mit $n$ Zuständen gibt es einen \ac{NEA} mit $n$ Zuständen der die gleiche Sprache akzeptiert.
\end{enumerate}

\subsubsection{$\Eps$-Übergänge}\label{sec:2.EpsNea}
In diesem Unterkapitel führen wir mit dem $\Eps$-NEA ein weiteres Automatenmodell ein. 
Wir wollen $\Eps$-NEAs zunächst durch die folgende Fragestellung und anschließende Diskussion motivieren.

Frage: Gegeben zwei reguläre Sprachen $L_1, L_2$ ist dann auch die Konkatenation $L_1\cdot L_2$ eine reguläre Sprache?

Idee: Geben DEA $\A_1$ mit $L(\A_1)=L_1$ und DEA $\A_2$ mit $L(\A_2)=L_2$,
konstruiere NEA für $L_1\cdot L_2$ durch ``Hintereinanderschalten'' von $\A_1$ und $\A_2$;
immer wenn wir in einem akzeptierenden Zustand von $\A_1$ sind erlauben wir in $\A_2$ zu ``wechseln''.

Erste, naive (und inkorrekte) Umsetzung dieser Idee: 
Verschmelze akzeptierende Zustände von $\A_1$ mit dem Startzustand von $\A_2$
Wir betrachten die folgenden Automaten um zu sehen, dass diese Umsetzung nicht zielführend ist.
\definecolor{auto1}{RGB}{200,200,255}
\definecolor{auto2}{RGB}{255,100,100}
\begin{Bsp}\label{bsp:2.naiveconcat}
Links: DEA $\A_1$, der Automat aus \autoref{bsp:2.zweitletztes} eingeschränkt auf die erreichbaren Zustände. Rechts: DEA $\A_2$ dessen Sprache $\{w\in\{0,1\}^*\mid \text{Anzahl $1$ in $w$ ist ungerade}\}$ ist.
  \begin{center}
                \begin{tikzpicture}[circle/.style={
                        shape=circle,
                        minimum size=0.5cm,
                        text=black, draw,
                        text width=0.5cm,
                        align=center}]
                        \node (vs) at (-1.0,1.0) {};
                        \node [circle, fill=auto1] (v0) at (0,0) {$p_0$};
                        \node [circle, fill=auto1] (v1) at (2,0) {$p_1$};
                        \node [circle, fill=auto1, double] (v2) at (5,1) {$p_2$};
                        \node [circle, fill=auto1, double] (v3) at (5,-1) {$p_3$};
                        \draw [->] (vs) edge (v0);
                        \draw [->] (v0) edge [loop above] node[auto] {0} (v2);
                        \draw [->] (v0) edge [bend left=0] node[below] {1} (v1);
                        \draw [->] (v1) edge [bend left=15] node[auto] {0} (v2);
                        \draw [->] (v1) edge [bend left=0] node[below] {1} (v3);
                        \draw [->] (v2) edge [bend right=30] node[above] {0} (v0);
                        \draw [->] (v2) edge [bend left=15] node[auto] {1} (v1);
                        \draw [->] (v3) edge [bend left=0] node[right] {0} (v2);
                        \draw [->] (v3) edge [loop right] node[auto] {1} (v3);
                \end{tikzpicture}
                \begin{tikzpicture}[circle/.style={
                        shape=circle,
                        minimum size=0.5cm,
                        text=black, draw,
                        text width=0.5cm,
                        align=center}]
                        \node (ss) at (-1.0,1.0) {};
                        \node [circle, fill=auto2] (s0) at (0,0) {$s_0$};
                        \node [circle, fill=auto2, accepting] (s1) at (3,0) {$s_1$};
                        \draw [->] (ss) edge (s0);
                        \draw [->] (s0) edge [loop above] node[auto] {0} (s0);
                        \draw [->] (s1) edge [loop above] node[auto] {0} (s1);
                        \draw [->] (s0) edge [bend left=15] node[auto] {1} (s1);
                        \draw [->] (s1) edge [bend left=15] node[auto] {1} (s0);
                \end{tikzpicture}
  \end{center}
Unten: NEA $\calN_\mathsf{naiv}$, Resultat einer einer naiven und inkorrekten Konstuktion für die Konkatenation.
  \begin{center}
                \begin{tikzpicture}[circle/.style={
                        shape=circle,
                        minimum size=0.5cm,
                        text=black, draw,
                        text width=0.5cm,
                        align=center}]
                        \node (vs) at (-1.0,1.0) {};
                        \node [circle, fill=auto1] (v0) at (0,0) {$p_0$};
                        \node [circle, fill=auto1] (v1) at (2,0) {$p_1$};
                        \node [circle, fill=auto1!50!auto2] (v2) at (5,1.5) {$p_2s_0$};
                        \node [circle, fill=auto1!50!auto2] (v3) at (5,-1.5) {$p_3s_0$};
                        \draw [->] (vs) edge (v0);
                        \draw [->] (v0) edge [loop above] node[auto] {0} (v2);
                        \draw [->] (v0) edge [bend left=0] node[below] {1} (v1);
                        \draw [->] (v1) edge [bend left=15] node[auto] {0} (v2);
                        \draw [->] (v1) edge [bend left=0] node[below] {1} (v3);
                        \draw [->] (v2) edge [bend right=30] node[above] {0} (v0);
                        \draw [->] (v2) edge [bend left=15] node[auto] {1} (v1);
                        \draw [->] (v3) edge [bend left=0] node[right] {0} (v2);
                        \draw [->] (v3) edge [loop below] node[auto] {1,0} (v3);
                        \node [circle, fill=auto2, accepting] (va) at (8,0) {$s_1$};
                        \draw [->] (va) edge [loop above] node[auto] {0} (va);
                        \draw [->] (v2) edge [bend left=15] node[auto] {1} (va);
                        \draw [->] (va) edge [bend left=0] node[auto] {1} (v2);
                        \draw [->] (v3) edge [bend left=0] node[auto] {1} (va);
                        \draw [->] (va) edge [bend left=15] node[auto] {1} (v3);
                        \draw [->] (v2) edge [loop above] node[auto] {0} (v2);
                \end{tikzpicture}
  \end{center}
Dieser NEA akzeptiert nun auch das Wort $w=11011$. Allerdings ist $w$ nicht in der Konkatenation $L(\A_1)\cdot L(\A_2)$ denn es gibt keine Zerlegung $w=w_1\cdot w_2$ bei der sowohl der Prefix $w_1$ von $\A_1$ als auch der Suffix $w_2$ von $\A_2$ akzeptiert wird.
\end{Bsp}


Das ``Verschmelzen'' von $p_2$ (bzw. $p_3$) mit $s_0$ war also keine gute Idee.
Was uns aber helfen würde wäre ein Zustandsübergang der uns erlaubt von Zustand $p_2$ (bzw. $p_3$) in den Zustand $s_0$ zu gehen ohne dabei ein Zeichens zu Lesen.

Wir nennen solch einen Zustandsübergang $\Eps$-Übergang und definieren einen Automaten der solche Zustandsübergänge haben kann wie folgt.

\begin{Def}[$\Eps$-NEA]\label{def2.EpsNea}
        Ein \emph{nichtdeterministischer endlicher Automat mit $\Eps$-Übergängen} ist ein 5-Tupel
        \[ \B= (\Sigma, Q,\delta,\qinit,F) \]
wobei $\Sigma$, $Q$, $\qinit$, $F$ wie bei NEAs (bzw. DEAs) definiert sind und die Transitionsfunkion den folgenden Typ hat.
$$ \delta:Q\x(\Sigma\cup\{\Eps\})\->\mathcal{P}(Q)$$
\end{Def}

\begin{Bsp}\label{bsp2.EpsNeaEx} Zustandsdiagramm eines $\Eps$-NEA über dem Alphabet $\Sigma=\{a,b\}$.
\datenote{10.11.17}
   \begin{center}
                \begin{tikzpicture}[circle/.style={
                        shape=circle,
                        minimum size=0.5cm,
                        text=black, draw,
                        text width=0.5cm,
                        align=center}]
                        \node (vs) at (-1.0,1.0) {};
                        \node [circle] (q0) at (0,0) {$q_0$};
                        \node [circle] (q2) at (4,-1) {$q_2$};
                        \node [circle] (q1) at (2,1) {$q_1$};
                        \node [circle, double] (q3) at (6,0) {$q_3$};
                        \draw [->] (vs) edge (q0);
                        \draw [->] (q1) edge [loop above] node[auto] {$b$} (q1);
                        \draw [->] (q0) edge [bend left=0] node[auto] {$\Eps$} (q2);
                        \draw [->] (q2) edge [bend left=0] node[auto] {$a$} (q3);
                        \draw [->] (q0) edge [bend left=0] node[auto] {$\Eps$} (q1);
                        \draw [->] (q1) edge [bend left=0] node[auto] {$\Eps$} (q3);
                        \draw [->] (q3) edge [bend left=30] node[auto] {$\Eps$} (q2);
                \end{tikzpicture}
   \end{center}
\end{Bsp}

Wie bei den bisher definierten Automaten wollen wir mit Hilfe eines $\Eps$-NEAs eine Sprache definieren.
Wir benötigen dafür zunächst zwei weitere Definitionen.

Der $\Eps$-Abschluss ist eine Abbildung die jedem Zustand $q$ die Menge der Zustände zuordnet, die von $q$ über $\Eps$-Übergänge erreichbar sind.
Wir definieren diese Abbildung formal wie folgt. Dabei verwenden wir den Abbildungsnamen $\ecl$ um an den englischen Begriff ``$\Eps$ closure'' zu erinnern.
\begin{Def}
 Der \emph{$\Eps$-Abschluss} $\ecl:Q\rightarrow\calP(Q)$ ist die kleinste Abbildung, die für alle $q,q',q''\in Q$ die folgenden Eigenschaften erfüllt:
 \begin{eqnarray*}
  & q\in\ecl(q) &\\
  & q'\in\ecl(q) \text{ und } q''\in\delta(q',\Eps) \quad\Rightarrow\quad q''\in\ecl(q) & \qedhere
 \end{eqnarray*}
\end{Def}
Offensichtlich kann immer eine endliche explizite Repräsentation von $\ecl$ berechnet werden: Starte in jedem Zustand einmal und folge mit Breitensuche allen $\Eps$-Kanten im Zustandsdiagramm.

Für den $\Eps$-NEA aus \autoref{bsp2.EpsNeaEx} sieht $\ecl$ wie folgt aus:
$$
\begin{array}{c||c|c|c|c|c|c}
   q  & q_0 & q_1 & q_2 & q_3\\ \hline
& & & &\\[-.5em]
\ecl(q) & \{q_0, q_1, q_2, q_3\} & \{q_1, q_2, q_3\} & \{q_2, q_3\} & \{q_2, q_3\}
\end{array}
$$

Als nächstes definieren wir eine dreistellige Relation die uns für je zwei Zustände sagt für welche Wörter den Automaten vom ersten Zustand in den zweiten Zustand überführen. Der Name der Relation ``$\reach$'' soll dabei an des englische Wort ``reachability'' erinnern.

\begin{Def}
 Die \emph{Erreichbarkeitsrelation} $\reach\subseteq Q\times\Sigma^*\times Q$ ist die kleinste Relation, die für alle $q,q',q'',q'''\in Q$ und für alle $w\in\Sigma^*$ die folgenden Eigenschaften erfüllt:
  \begin{eqnarray*}
& q'\in\ecl(q) \quad\Rightarrow\quad (q,\Eps,q')\in\reach &\\
& q'\in\ecl(q), q''\in\delta(q',a) \text{ und } (q'',w,q''')\in\reach \Rightarrow (q,aw,q''')\in\reach & \qedhere
 \end{eqnarray*}
\end{Def}

Für den $\Eps$-NEA aus \autoref{bsp2.EpsNeaEx} sieht $\reach$ wie folgt aus:
$$
\begin{array}{c||c|c|c|c|c|c}
   \reach  & q_0 & q_1 & q_2 & q_3\\ \hline
q_0 & \{\Eps\} & \{b\}^* & \{b\}^*\cdot\{a\}^* & \{b\}^*\cdot\{a\}^* \\[2mm]
q_1 & \{\}     & \{b\}^* & \{b\}^*\cdot\{a\}^* & \{b\}^*\cdot\{a\}^* \\[2mm]
q_2 & \{\}     & \{\}    & \{a\}^*             & \{a\}\cdot\{a\}^*\\[2mm]
q_3 & \{\}     & \{\}    & \{a\}^*             & \{a\}\cdot\{a\}^*\cup\{\Eps\}
\end{array}
$$

\begin{Def}\label{def:2.EpsNeaSprache}
 Ein Wort $w\in\Sigma^*$ wird von $\B$ \emph{akzeptiert}, wenn $(\qinit,w,q_f)\in\reach$ für ein $q_f\in F$.
 Die von $\B$ akzeptierte Sprache $L(\B)$ ist die Menge der von $\B$ akzeptierten Wörter. D.h.	
	$L(\B)=\{ w\in\Sigma^* \mid \exists q\in F: (\qinit, w, q)\in\reach \}$
\end{Def}

Offensichtlich gibt es zu jedem \ac{NEA} $\calN$ einen $\Eps$-NEA $\B$ der die gleiche Sprache akzeptiert.
Die Konstuktion ist dabei sehr einfach: Erweitere die Transitionsfunkion um $\delta(q,\Eps)=\{\}$ für alle $q\in Q$.
Für die Sprachgleichheit zeigen wir via Induktion über die Länge von $w$, dass für alle Wörter und für alle Zustände
$$\exists \text{ Lauf } q_0q_1\ldots q_n \text{ von $\calN$ über } w \quad\Leftrightarrow\quad (q_0,w,q_n)\in\reach$$
Der folgende Satz zeigt uns dass auch die umgekehrte Richtung gilt.

\begin{Satz}\label{satz:2.EpsElim}
    Zu jedem $\Eps$-NEA $\B$ gibt es einen \ac{NEA} $\calN$, so dass $L(\calN)=L(\B)$.
\end{Satz}
Zur Vorbereitung des Beweises machen wir zunächst die folgende Definition.

\begin{Def}[$\Eps$-freier Automat]
 Für gegebenen $Eps$-\ac{NEA}  $\B= (\Sigma, Q,\delta,\qinit,F)$ definieren wir den NEA $\calN= (\Sigma, Q,\delta_\calN,\qinit,F_\calN)$ mit
        $$\delta_\calN(q,a)=\bigcup\limits_{q'\in\ecl(q)}\{q'''\mid \exists q'' : q''\in\delta(q',a) \text{ und } q'''\in\ecl(p'')\}$$
        $$F_\calN=\{q\in Q\mid \exists q_f\in F: q_f\in\ecl(q)\}$$
und nennen diesen den \emph{$\Eps$-freien Automat} von $\B$.
\end{Def}
\begin{Bsp}
Der $\Eps$-freie Automat für den $\Eps$-NEA aus \autoref{bsp2.EpsNeaEx} hat das folgende Zustandsdiagramm.
   \begin{center}
                \begin{tikzpicture}[circle/.style={
                        shape=circle,
                        minimum size=0.5cm,
                        text=black, draw,
                        text width=0.5cm,
                        align=center}]
                        \node (vs) at (-1.0,1.0) {};
                        \node [circle] (q0) at (0,0) {$q_0$};
                        \node [circle] (q2) at (4,-1) {$q_2$};
                        \node [circle] (q1) at (2,1) {$q_1$};
                        \node [circle, double] (q3) at (6,0) {$q_3$};
                        \draw [->] (vs) edge (q0);
                        \draw [->] (q1) edge [loop above] node[auto] {$b$} (q1);
                        \draw [->] (q0) edge [bend left=0, pos=0.2] node[below] {$a,b$} (q2);
                        \draw [->] (q2) edge [bend left=0] node[auto] {$a$} (q3);
                        \draw [->] (q0) edge [bend left=0] node[auto] {$b$} (q1);
                        \draw [->] (q1) edge [bend left=0] node[auto] {$a,b$} (q3);
                        \draw [->] (q3) edge [bend left=30] node[auto] {$a$} (q2);
                        \draw [->] (q0) edge [bend left=0] node[auto] {$a,b$} (q3);
                        \draw [->] (q1) edge [bend right=90, pos=0.6] node[below] {$a,b$} (q2);
                        \draw [->] (q3) edge [loop right] node[auto] {$a$} (q3);
                        \draw [->] (q2) edge [loop below] node[auto] {$a$} (q2);
                \end{tikzpicture}
   \end{center}
\end{Bsp}



\begin{proof}[von \autoref{satz:2.EpsElim}: $\Eps$-Eliminierung]
        Zeige $L(\calN)=L(\B)$. Dabei verwenden wir die folgende Eigenschaft, die wir via Induktion über die Länge von $w$ in den Übungen zeigen werden.
        
        $\forall w\in\Sigma^*\; \forall q, q'\in Q$
        $$(q,w,q')\in\reach \Leftrightarrow \exists \underbrace{q_0, q_1,\ldots q_n}_{\text{Lauf}}\in Q, \text{ sodass }  n=|w|, q_0=q \text{ und } q_n=q'$$
%         I.A. $(n=0,$ also $w=\Eps)$: Gilt trivialerweise da $(q,\Eps,q)\in\reach$ für alle $q$\\
%         I.S. ($n\rightsquigarrow n+1$): Seit $w=aw'$ beliebiges Wort der Länge $n+1$.
%         \begin{eqnarray*}
%                 (q,w,q')\in\reach\\
%                 & \Leftrightarrow & (q,w,q')\in\reach\\
%                 & \stackrel{IV}{\Leftrightarrow} & \exists\underbrace{q_1q_2\ldots q_n}_{\text{Lauf}}\in Q, \text{ sodass }  n=|w'|, q_1\in \delta_\calP(p,a) \text{ und } q_n=q\\
%                 & \Leftrightarrow & \exists\underbrace{q_1q_2\ldots q_n}_{\text{Lauf}}\in Q, \text{ sodass }  n=|w'|, \exists q_0\in p, \delta(q_0,a)=q_1 \text{ und } q_n=q\\
%                 & \Leftrightarrow & \exists\underbrace{q_0q_1q_2\ldots q_n}_{\text{Lauf}}\in Q,\text{ sodass }  n+1=|w|, q_0\in p \text{ und } q_n=q
%         \end{eqnarray*}

        \begin{align*}
         w\in L(\B) 
         &\stackrel{\autoref{def:2.sprache}}{\<=>} \exists q_f\in F: (\qinit, w, q_f)\in\reach\\
         &\<=> \exists q_f\in F: \underbrace{q_0, q_1,\ldots q_n}_{\text{Lauf}}\in Q, \text{ sodass }  n=|w|, q_0=\qinit \text{ und } q_n=q_f\\
         &\stackrel{\autoref{def:2.EpsNeaSprache}}{\<=>} L(\calN)\\
        \end{align*}
\end{proof}


\medskip

Mit Hilfe der $\Eps$-NEAs greifen wir nun die zu Beginn von Abschnitt~\ref{sec:2.EpsNea} aufgeworfene Fragestellung wieder auf und zeigen dass für je zwei reguläre Sprachen auch die Konkatenation regulär ist.

Wir geben hierfür zunächst eine Konstruktion an.

\begin{Def}
Gegeben zwei $\Eps$-NEA $\B_i= (\Sigma, Q_i,\delta_i,\qinit_i,F_i)$ definieren wir den \emph{$\Eps$-NEA für Konkatenation} $\B= (\Sigma, Q,\delta,\qinit,F)$ wie folgt.
                \begin{align*}
                Q &= Q_1 \overset.\cup Q_2\\
                \delta(q,a) &=
                                \begin{cases}
                                        \delta_1(q,a) & q\in Q_1\land (q\notin F_1 \lor a\neq \Eps)\\
                                        \delta_1(q,a)\cup\{\qinit_2\} & q\in F_1\land a=\Eps\\
                                        \delta_2(q,a) & q\in Q_2
                                \end{cases}\\
                \qinit & = \qinit_1\\
                F &= F_2 \qedhere
                \end{align*}
\end{Def}

\begin{Bsp} Der $\Eps$-NEA für Konkatenation für die beiden Automaten aus \autoref{bsp:2.naiveconcat} hat das folgende Zustandsdiagramm.
   \begin{center}
                \begin{tikzpicture}[circle/.style={
                        shape=circle,
                        minimum size=0.5cm,
                        text=black, draw,
                        text width=0.5cm,
                        align=center}]
                        \node (vs) at (-1.0,1.0) {};
                        \node [circle, fill=auto1] (v0) at (0,0) {$p_0$};
                        \node [circle, fill=auto1] (v1) at (2,0) {$p_1$};
                        \node [circle, fill=auto1, double] (v2) at (5,1) {$p_2$};
                        \node [circle, fill=auto1, double] (v3) at (5,-1) {$p_3$};
                        \draw [->] (vs) edge (v0);
                        \draw [->] (v0) edge [loop above] node[auto] {0} (v2);
                        \draw [->] (v0) edge [bend left=0] node[below] {1} (v1);
                        \draw [->] (v1) edge [bend left=15] node[auto] {0} (v2);
                        \draw [->] (v1) edge [bend left=0] node[below] {1} (v3);
                        \draw [->] (v2) edge [bend right=30] node[above] {0} (v0);
                        \draw [->] (v2) edge [bend left=15] node[auto] {1} (v1);
                        \draw [->] (v3) edge [bend left=0] node[right] {0} (v2);
                        \draw [->] (v3) edge [loop below] node[auto] {1} (v3);
                        \node [circle, fill=auto2] (s0) at (8,0) {$s_0$};
                        \node [circle, fill=auto2, accepting] (s1) at (11,0) {$s_1$};
                        \draw [->] (s0) edge [loop above] node[auto] {0} (s0);
                        \draw [->] (s1) edge [loop above] node[auto] {0} (s1);
                        \draw [->] (s0) edge [bend left=15] node[auto] {1} (s1);
                        \draw [->] (s1) edge [bend left=15] node[auto] {1} (s0);
                        
                        \draw [->] (v2) edge [bend left=0] node[auto] {$\Eps$} (s0);
                        \draw [->] (v3) edge [bend left=0] node[auto] {$\Eps$} (s0);
                \end{tikzpicture}
  \end{center}
\end{Bsp}


\begin{lemma}
Die vom $\Eps$-NEA für Konkatenation akzeptierte Sprache ist $L(\B_1)\cdot L(\B_2)$
\end{lemma}
\begin{proof}
 Zeige via Induktion über die Länge von $w_1$ dass $\forall w_1,w_2\in\Sigma^* \forall q_1\in Q_1, \forall q_1'\in F, \forall q_2'\in Q_2$ die folgende Eigenschaft gilt.
 $$
 (q_1,w_1,q_1')\in\reach \text{ und } (\qinit,w_2,q_2')\in\reach \quad\Leftrightarrow\quad (q_1,w,q_2')\in\reach
 $$
\end{proof}





\subsection{Abschlusseigenschaften}
\datenote{15.11.17}
% \begin{Def}[name={[Abgeschlossenheit von $\mathcal{L}$]}]
%         Eine Menge $\mathcal{L}\subseteq \mathcal{P}(\Sigma^*)$ von Sprachen heißt \emph{abgeschlossen} unter Operation \\
%         $f:\mathcal{P}(\Sigma^*)^n \-> \mathcal{P}(\Sigma^*)$ falls $\forall L_1,\dots, L_n\in \mathcal{L} : f(L_1,\dots, L_n)\in \mathcal{L}$.
% \end{Def}

\begin{Def}[name={[Abgeschlossenheit von $\mathcal{L}$]}]
        Eine Menge $X$ heißt \emph{abgeschlossen} unter Operation $f:X^n \-> X$ falls $\forall x_1,\dots, x_n\in X : f(x_1,\dots, x_n)\in X$.
\end{Def}
Z.B. sind die natürlichen Zahlen abgeschlossen unter Addition, aber nicht abgeschlossen unter Subtraktion.


\begin{lemma}
 Die Menge $REG$ der regulären Sprachen ist abgeschlossen unter Komplement.
\end{lemma}

\begin{proof}
 
\end{proof}

\begin{lemma}
 Die Menge $REG$ der regulären Sprache ist abgeschlossen unter dem Stern Operator
\end{lemma}

\begin{proof}
         \begin{align*}
                Q &= Q_1\overset.\cup \{q_0\}\\
                \delta(q,a) &=
                        \begin{cases}
                                \delta_1(q,a) & q\in Q_1\setminus F_1\\
                                \delta_1(q,a)\cup\delta_1(q_{01},a) & q\in F_1\\
                                \delta_1(q_{01},a) & q=q_0
                        \end{cases}\\
                F &= \{q_0\}\cup F_1\\
                \dots\ L(\A) &= L(\A_1)^*
        \end{align*}
\end{proof}




\hide{
\begin{Satz}[name={[Abgeschlossenheit von $REG$]}]\label{satz:3.8}
        Die Menge $REG$ der regulären Sprachen ist abgeschlossen unter $\cup$ (Vereinigung), $\cap$ (Durchschnitt), $\overline{\phantom{X}}$ (Komplement), Produkt (Konkatenation), Stern. Beispielsweise ist für $L_1, L_2$ reguläre Sprachen also auch $L_1 \cup L_2$, wie auch $L_1 \cap L_2$ etc. wieder eine reguläre Sprache.
\end{Satz}
\begin{proof}
	Sei $\A_i:=(Q_i,\Sigma,\delta_i,q_{0i},F_i)$, $\quad i=1,2$ \acs{NEA}s
	\begin{itemize}
	\item $\cup:$ Def $\A$ durch (vgl.\ Abb.~\ref{fig:reg-closure-union}):
		\begin{align*}
			Q &= Q_1\overset.\cup Q_2\overset.\cup\{q_0\}\\
			\delta(q,a) &=
                \begin{cases}
                    \delta_1(q,a) & q\in Q_1\\
                    \delta_2(q,a) & q\in Q_2\\
                    \delta_1(q,a)\cup\delta_2(q,a) & q=q_0
                        \end{cases}\\
                        F &= F_1\overset.\cup F_2\overset.\cup (q_{01}\in F_1\lor q_{02}\in F_2) \rhd \{q_0\}
                \end{align*}
                Zeige $L(\A)=L(\A_1)\cup L(\A_2)$ (Aufgabe zum Eigenstudium: Betrachte die Läufe).
                \begin{figure}[tp]\centering
                \begin{tikzpicture}[>=stealth]
                \node (q0) at (0,0) {$q_0$};
                
                \node (q01) at (1.5,1.5) {$q_{01}$};
                \node (y) at (2.5,2) {$\bullet$};
                \node (x1) at (2.5,1) {$\bullet$};
                \node (qf1) at (4,1.5) {$q_{f_1}$};
                
                \node (q02) at (1.5,-1.5) {$q_{02}$};
                \node (x2) at (2.5,-1) {$\bullet$};
                \node (l) at (2.5,-2) {$\bullet$};
                \node (qf2) at (4,-1.5) {$q_{f_2}$};
                
                \path [->] (q0) edge [bend left=45] (y)
                                edge (x1)
                                edge (x2)
                                edge [bend right=45] (l)
                          (q01) edge (y)
                                edge (x1)
                          (q02) edge (x2)
                                edge (l)
%                ;
%                \path (q01) edge [bend left=80] (qf1)
%                                    edge [bend right=80] (qf1)
                ;\draw (2.75,-1.5) ellipse (1.75 and 1.35);
                \node at (5,2) {$A_1$}
                ;
                \draw (2.75,1.5) ellipse (1.75 and 1.35);
                \node at (5,-1) {$A_2$};
            \end{tikzpicture}
            \caption{\acs{NEA} f"ur Vereinigung}
            \label{fig:reg-closure-union}
        \end{figure}
        \item $\cap:$ 
        Annahme: Seien $\A_1$ und $\A_2$ zwei DEAs. Konstruiere nun  den \emph{Produktautomaten} $\A$, für den gilt (Aufgabe zum Eigenstudium!): $L(\A) = L(\A_1) \cap L(\A_2)$.
		\begin{align*}
			Q &= Q_1\x Q_2\\
			\delta((q_1,q_2),a) &= (\delta_1(q_1,a),\delta_2(q_2,a))\\
			q_0 &= (q_{01},q_{02})\\
			F &= F_1\x F_2\\
		\end{align*}
	\item Komplement: Ang. $\A_1$ ist \ac{DEA}, der $L$ erkennt. Ersetze $F_1$ durch $Q_1\setminus F_1$ und erhalte einen DEA $\A_1'$, der genau das Komplement von $L$ erkennt. 
%
%
        \item Produkt: Seien $L_1$, $L_2$ regulär.\\
                Zeige $L_1\cdot L_2$ regulär.
                \begin{align*}
                        Q &= Q_1 \overset.\cup Q_2\\
                        \delta(q,a) &=
                                \begin{cases}
                                        \delta_1(q,a) & q\in Q_1\setminus F_1\\
                                        \delta_1(q,a)\cup\delta_2(q_{02},a) & q\in F_1\\
                                        \delta_2(q,a) & q\in Q_2
                                \end{cases}\\
                q_0& = q_{01}\\
                F &= F_2\cup(q_{02}\in F_2) \rhd F_1
                \end{align*}
                Zeige $L(\A) = L(\A_1)\cdot L(A_2)$
        \item Stern
        \begin{align*}
                Q &= Q_1\overset.\cup \{q_0\}\\
                \delta(q,a) &=
                        \begin{cases}
                                \delta_1(q,a) & q\in Q_1\setminus F_1\\
                                \delta_1(q,a)\cup\delta_1(q_{01},a) & q\in F_1\\
                                \delta_1(q_{01},a) & q=q_0
                        \end{cases}\\
                F &= \{q_0\}\cup F_1\\
                \dots\ L(\A) &= L(\A_1)^*
        \end{align*}
        \end{itemize}
\end{proof}
%

\subsection{Reguläre Ausdrücke}
\draftnote{9.11.16}
\begin{Def}[name={[RE($\Sigma$)]}]
        Die Menge $RE(\Sigma)$ der \emph{regulären Ausdrücke über $\Sigma$} ist induktiv definiert durch:
        \begin{itemize}
        \item $\0\in RE(\Sigma)$
        \item $\1\in RE(\Sigma)$
        \item $\forall a\in\Sigma$, $a\in RE(\Sigma)$
        \item falls $r,s\in RE(\Sigma)$
                \begin{itemize}[label=\textbullet]
                \item $r+s\in RE(\Sigma)$
                \item $r\cdot s\in RE(\Sigma)$
                \item $r^*\in RE(\Sigma)$
                \end{itemize}
        \end{itemize}
\end{Def}
\begin{Def}[name={[Semantik eines regulären Ausdrucks]}]
        Die Semantik eines regulären Ausdrucks $\llbracket\cdot \rrbracket : RE(\Sigma) \-> \mathcal{P}(\Sigma^*)$ ist induktiv definiert durch
        \begin{align*}
                \llbracket \0 \rrbracket &= \varnothing\\
                \llbracket \1 \rrbracket &= \{\Eps\}\\
                \llbracket a \rrbracket &= \{a\} \quad a\in\Sigma\\
                \llbracket r+s \rrbracket &= \llbracket r\rrbracket \cup \llbracket s\rrbracket\\
                \llbracket r\cdot s \rrbracket &= \llbracket r\rrbracket \cdot \llbracket s\rrbracket\\
                \llbracket r^* \rrbracket &= \llbracket r\rrbracket^* \qedhere
        \end{align*}
\end{Def}
%
    \begin{Bsp*} Mustererkennung
        \begin{itemize}
        \item Akzeptiere alle Wörter, die $abac$ enthalten: 
        \begin{align*} & \Sigma^* abac \Sigma^* \\
                       \Leftrightarrow & (a_1 + a_2 + ...)^* abac (a_1 + a_2 + ...)^* \forall a_i \in \Sigma
          \end{align*}
        \item Sei $\Sigma=\{0,1\}$. Die Sprache aller Wörter, deren $n$-letztes Symbol = 1, ist nicht regulär, denn:
        \begin{gather*}
        (0+1)^*1\underbrace{(0+1)\dots(0+1)}_{n-1}\\
        \xcancel{0^n1^n}\notin RE(\Sigma)
        \end{gather*}
        \item Ein regulärer Ausdrück für alle Wörter $\omega \in (0, 1)^*$, s.d. $\omega \mod 3=0$:
        \[ (0+1(01^*0)^*1)^* \]
        
    \captionsetup{type=figure}
    \begin{tikzpicture}[>=stealth, shorten >=1pt, on grid, node distance=2cm, initial text=]
        \node[state, initial, accepting] (q0) {$q_0$};
        \node[state] (q1) [right=of q0] {$q_1$};
        \node[state] (q2) [right=of q1] {$q_2$};
        \path [->]
            (q0) edge[loop below] node[auto] {0} ()
                 edge[bend left]  node[auto] {1} (q1)
            (q1) edge[bend left]  node[auto] {0} (q2)
                 edge[bend left]  node[auto] {1} (q0)
            (q2) edge[loop right] node[auto] {1} ()
                 edge[bend left]  node[auto] {0} (q1)
        ;
        
        \draw [->,decorate,decoration=snake] ($(q1.south) - (0,.5cm)$) -- ++(0,-1cm);
        
        \node [state, initial, accepting] (q0) [below=3cm of q0] {$q_0$};
        \node [state] (q1) [right=of q0] {$q_1$};
        \path [->]
            (q0) edge[loop below] node[auto] {0} ()
                 edge[bend left]  node[auto] {1} (q1)
            (q1) edge[loop right] node[auto] {$01^*0$} ()
                 edge[bend left]  node[auto] {1} (q0)
        ;
        
        \node [state, initial, accepting] (q0) [below=2.5cm of q0] {$q_0$};
        \path [->]
            (q0) edge[loop below] node[auto] {0} ()
                 edge[loop right] node[auto] {$1(01^*0)^*1$} ()
        ;
    \end{tikzpicture}
    \captionof{figure}{Informell vom Automaten zum regul"aren Ausdruck f"ur mod 3}
\end{itemize}
    \end{Bsp*}
\begin{Satz}[Kleene]
$L$ ist regulär \<=> $L$ ist Sprache eines regulären Ausdrucks.
\end{Satz}

\begin{proof}[Kleene, $\<=$]
Betrachte zu einem regulärem Ausdruck $r\in RE(\Sigma)$ die durch diesen erzeugte Sprache $L=\llbracket r \rrbracket$. Zeige per Induktion über $r$, dass $ \forall r\in RE(\Sigma) $ gilt: $ \llbracket r \rrbracket$ ist regulär.
    	\begin{description}[font=\normalfont]
      \item[I.A.:] \hfill
        \vspace{-\baselineskip}
        \begin{itemize}
        \item $r = \0$, $\llbracket r \rrbracket = \emptyset$ ist regulär
        \item $r = \1$, $\llbracket r \rrbracket = \{ \Eps \}$ ist regulär
        \item $r = a$, $\llbracket r \rrbracket = \{ a \}$ ist regulär \\
          (\tikz[>=stealth, shorten >=1pt, initial text=,
    					on grid, baseline=-.6ex,
    					every state/.style={minimum size=0pt,inner sep=0pt}
    				]{
    				\node [state,initial] (a) {}; \node [state,accepting] (b) [right=of a] {};
    				\path [->] (a) edge node [auto] {$a$} (b);
    			}
    			 \quad\acs{NEA}) 
        \end{itemize}
        \item[I.V.:] Für $i \in \{1, 2\}$ gilt: $\llbracket r_i \rrbracket$ ist regulär.
    		\item[I.S.:] \hfill
          \vspace{-\baselineskip}
          \begin{itemize}
          \item $r = r_1 + r_2$, $\llbracket r \rrbracket = \llbracket r_1 \rrbracket \cup \llbracket r_2 \rrbracket$ ist regulär nach \autoref{satz:3.8}
          \item $r = r_1 \cdot r_2$, $\llbracket r \rrbracket = \llbracket r_1 \rrbracket \cdot \llbracket r_2 \rrbracket$ ist regulär nach \autoref{satz:3.8}
          \item $r = r_1^*$, $\llbracket r \rrbracket = \llbracket r_1 \rrbracket^*$ ist regulär nach \autoref{satz:3.8}
          \end{itemize}
		\end{description}
\end{proof}

Zum Beweis (bzw.\ zur Konstruktion) der Richtung ,,\=>'' benötigen wir eine Rechenregel zum Lösen von Gleichungen zwischen regulären Sprachen und Ausdrücken:
\begin{lemma}[Ardens Lemma]\label{lem:arden}\ \\
        Sei die lineare Gleichung $X=A\cdot X+B$ über $A, B, X\subseteq \Sigma^*$ gegeben. Dann ist $X=A^*B$ eine Lösung. Falls $\Eps \notin A$ ist diese Lösung ausserdem eindeutig.
\end{lemma}

\begin{proof}
Sei $ X= AX+B$ mit $\Eps\notin A$. Zeige, dass $A^*B \subseteq X$:
        \begin{alignat*}{3}
                && A^*B &= AA^*B + B = (\1+AA^*)B = B+A(A^*B) \quad\checkmark\\
                \shortintertext{Angenommen $A^*B \subsetneq X$, d.h. $\exists w\in X$ mit $w\notin A^*B$, davon sei $w$ das kürzeste.}
                \exists n\geq 1: && X &= \underbrace{A^nX}_{\ni w} + \underbrace{A^{n-1}B+\dots +AB+B}_{\not\ni w}\\
                \curvearrowright && w &= u_1\dots u_n w'\text{ mit } u_1,\dots,u_n\in A\text{ und } w'\in X\\
                \curvearrowright && |w'| &< |w|\\
                \text{Falls}&& w'&\in A^*B \curvearrowright w\in A^nA^*B\subseteq A^*B \quad \lightning\\
                \text{Also} && w'&\notin A^*B \quad\lightning\text{ gegen Minimalität von }w\\
                \curvearrowright && X&\subseteq A^*B\\
                \-> && X &= A^*B \tag*{\qedhere}
        \end{alignat*}
\end{proof}

\begin{Korollar}
  Ardens Lemma lässt sich auch für reguläre Ausdrücke formulieren: Seien $r_X,r_A,r_B$ reguläre Ausdrücke mit $\Eps \not \in \llbracket r_A \rrbracket$ für die die folgende Gleichung gilt:
  \begin{displaymath}
    \llbracket r_x \rrbracket = \llbracket r_A \cdot r_x + r_B \rrbracket
  \end{displaymath}
  Dann ist 
  \begin{displaymath}
    r_x := r_A^*r_B
  \end{displaymath}
  eine eindeutige Lösung für $r_x$, die die Gleichung erfüllt.
\end{Korollar}

\datenote{11.11.16}

\begin{proof}[Kleene, \=>]
  Sei $L=L(\A)$ für einen \ac{DEA} $\A=({Q},\Sigma,\delta,q_0,F)$ mit $Q = \{q_0,q_1,\dots,q_n\}$.
        Definiere $L_i=\{ w \mid \tilde\delta(q_i,w)\in F \}$ als die Sprache der Worte, die von Zustand $q_i$ aus in einen akzeptierenden Zustand führen.
  Insbesondere gilt $L_0 = L$.

  Wir leiten nun ein lineares Gleichungssystem zwischen den Sprachen $L_i$ her.
  Zunächst teilen wir $Li$ in den Teil der (potentiell) das leere Wort enthält und den, der die nicht-leeren Wörter enthält.
  \begin{align*}
    L_i & =  \{ w \mid \tilde\delta(q_i,w)\in F \} \\
        & =  \{ \Eps \mid \tilde\delta(q_i, \Eps) \in F \} \cup \{a w' \mid a \in \Sigma, w' \in \Sigma^*, \tilde\delta(\delta(q_i, a), w') \in F\} \\
        & =  \{ \Eps \mid q_i \in F \} \cup \{a w' \mid a \in \Sigma, w' \in \Sigma^*, \tilde\delta(\delta(q_i, a), w') \in F\}
    \end{align*}
    Die nicht-leeren Wörter hängen von den Sprachen der Folgezustände ab:
    \begin{align*}
      \{a w' \mid a \in \Sigma, w' \in \Sigma^*, \tilde\delta(\delta(q_i, a), w') \in F\} & =
         \bigcup_{a \in \Sigma} \{a\} \{w' \mid a \in \Sigma, w' \in \Sigma^*, \tilde\delta(\delta(q_i, a), w') \in F\}
      \\
      & = \bigcup_{a \in \Sigma} \{a\} (L_j \text{ wobei  $q_j = \delta(q_i, a)$}) 
    \end{align*}
    Anstatt die Vereinigung über die Transitionen $a$ und Folgezustandssprachen $L_j$ zu bilden, lassen sich die nicht-leeren Worte von $L_i$ auch als Vereinigung über aller Zustände mit entsprechend gewählten \emph{Koeffizienten} formulieren; die Zustände die keine Folgezustände sind, habe den Koeffizienten $\emptyset$.
    \begin{align*}
      \bigcup_{a \in \Sigma} \{a\} (L_j \text{ wobei  $q_j = \delta(q_i, a)$}) & = \bigcup_{j = 0}^{n} \underbrace{\{a \mid a \in \Sigma, \delta(q_i, a) = q_j\}}_{A_{ij} \not \ni \Eps} L_j
    \end{align*}
    Man beachte das die Koeffizienten $A_{ij}$ nie das leere Wort enthalten.
    Nach diesen Überlegungen ergibt sich also für die Sprachen $L_i$ das lineare Gleichungssystem
    \begin{displaymath}
      L_i = \{ \Eps \mid q_i \in F \} \cup \bigcup_{j = 0}^{n} \underbrace{\{a \mid a \in \Sigma, \delta(q_i, a) = q_j\}}_{A_{ij} \not \ni \Eps} L_j
    \end{displaymath}
    Diese Gleichungen lassen sich analog als Gleichungen von regulären Ausdrücken $r_i$ formulieren:
    \begin{align*}
      r_i &= N(q_i) + \sum_{j = 0}^n R_{ij} r_j
    \end{align*}
    wobei $\llbracket r_i \rrbracket = L_i$ und $R_{ij } = \sum \{a \mid a \in \Sigma, \delta(q_i, a) = q_j\}$ mit $\Eps \not \in \llbracket R_{ij} \rrbracket$ und
    \begin{displaymath}
      N(q_i) =
      \begin{cases}
        \1 & q_i\in F\\
        \0 & q_i\notin F
      \end{cases}
    \end{displaymath}
    Dieses Gleichungssystem lässt sich, wie lineare Gleichungssysteme in der Arithmetik, mit dem \emph{Substituitionsverfahren} lösen (,,auflösen nach einer Variablen und einsetzen'').
    Dazu werden Ardens Lemma und weitere Rechenregeln für Sprachen, wie Distributivität, verwendet.
    Wir beginnen mit Gleichung $r_n$:
    \begin{align*}
      r_n &= N(q_n) + \sum_{j = 0}^n R_{nj} r_j \\
          &= \underbrace{N(q_n) + \left (\sum_{j = 0}^{n-1} R_{nj} r_j \right)}_{r_B} + \underbrace{R_{nn}}_{r_A} r_n
    \end{align*}
    Wie oben angedeutet ist nach dem Herausziehen des $n$ten Summenglieds Ardens Lemma anwendbar (merke, $\Eps \not \in \llbracket R_{nn} \rrbracket$), und wir setzen:
    \begin{displaymath}
      r_n := R_{nn}^*\left(N(q_n) + \sum_{j = 0}^{n-1} R_{nj} r_j  \right)
    \end{displaymath}
    Dieses Ergebnis in $r_0,\ldots,r_{n-1}$ eingesetzt ergibt:
    \begin{align*}
      r_i &= N(q_i) + \left (\sum_{j = 0}^{n-1} R_{nj} r_j \right) + R_{in}R_{nn}^*\left(N(q_n) + \sum_{j = 0}^{n-1} R_{nj} r_j  \right) \\
      \intertext{(Ausmultiplizieren von $R_{in}R_{nn}^*$)}
          &= N(q_i) + \left (\sum_{j = 0}^{n-1} R_{nj} r_j \right) + R_{in}R_{nn}^*N(q_n) + \sum_{j = 0}^{n-1} R_{in}R_{nn}^*R_{nj} r_j  \\
      \intertext{(Zusammenlegen der Summen und Ausklammern von $r_j$)}
          &= N(q_i) + R_{in}R_{nn}^*N(q_n) + \sum_{j = 0}^{n-1} (R_{nj} + R_{in}R_{nn}^*R_{nj}) r_j  \\
    \end{align*}
  Nach diesen Umformungen ergeben sich $\Eps$-freie Koeffizienten $R_{nj} + R_{in}R_{nn}^*R_{nj}$ für $r_j$ und wir mit dem Auflösen von $n-1$ analog zu $n$ fortfahren.
  Am Ende erhalten wir einen regulären Ausdruck als Lösung für $r_0$.
  Per Konstruktion haben wir immer noch $\llbracket r_0 \rrbracket = L_0 = L$.
  \end{proof}
    

\begin{Bsp*} Ein Beispiel für die Konvertierung eines \ac{DEA} in einen reg. Ausdruck.
	\begin{figure}[H]\centering
		\begin{tikzpicture}[>=stealth, shorten >=1pt, on grid, node distance=2cm, initial text=]
			\node[state, initial, accepting] (q0) {$q_0$};
			\node[state] (q1) [right=of q0] {$q_1$};
			\node[state] (q2) [right=of q1] {$q_2$};
			\path [->]
			    (q0) edge[loop above] node[auto] {0} ()
			         edge[bend left]  node[auto] {1} (q1)
			    (q1) edge[bend left]  node[auto] {0} (q2)
			         edge[bend left]  node[auto] {1} (q0)
			    (q2) edge[loop right] node[auto] {1} ()
			         edge[bend left]  node[auto] {0} (q1)
			;
		\end{tikzpicture}
		\caption{\ac{DEA} "`modulo 3"'}
	\end{figure}
	lineares Gleichungssystem mit 3 Unbekannten.
	\begin{align*}
		L_0 &= \1 + 0\cdot L_0 + 1\cdot L_1\\
		L_1 &= 1\cdot L_0 + 0\cdot L_2\\
		L_2 &= \underbrace{0\cdot L_1}_B + \underbrace{1}_A\cdot L_2\\
		\shortintertext{\nameref{lem:arden} auf $q_2$:}
		L_2 &= 1^*\cdot 0\cdot L_1\\
		\shortintertext{Einsetzen in $q_1$}
		L_1 &= \underbrace{1\cdot L_0}_B+\underbrace{0\cdot 1^*\cdot 0}_A\cdot L_1\\
		\shortintertext{\nameref{lem:arden} auf $q_1$:}
		L_1 &= (01^*0)^*\cdot 1\cdot L_0\\
		\shortintertext{Einsetzen:}
		L_0 &= \1 + 0\cdot L_0+1\cdot (01^*0)^*\cdot 1\cdot L_0\\
		&= \1+(0+1\cdot (01^*0)^*\cdot 1)\cdot L_0\\
		\shortintertext{\nameref{lem:arden} auf $q_0$:}
		L_0 &= (0+1\cdot (01^*0)^*\cdot 1)^*
	\end{align*}
\end{Bsp*}
%

\draftnote{11.11.16}
\subsection{Entscheidungsprobleme}
Ein Problem ist \emph{entscheidbar} wenn es sich durch eine binäre Antwort (ja/nein) lösen lässt und es einen Algorithmus gibt, der für alle Instanzen des Problem die korrekte Antwort liefert.
\begin{Satz}[name={[Wortproblem]}]\label{satz:wortproblem}
        Das Wortproblem ist für reguläre Sprachen entscheidbar.
        
        D.h. Falls $L$ reg. Sprache und $w\in\Sigma^*$, dann ex. Algorithmus, der entscheidet, ob $w\in L$.
\end{Satz}
\begin{proof}
	$L$ sei durch \ac{DEA} gegeben.\\
	Berechnung von $\tilde\delta(q_0,w)$ entspricht Durchlauf durch Graph des \ac{DEA} + Test ob erreichter Zustand $\in F$ in Zeit $O(n)\ ,\ n=|w|$.
\end{proof}

\begin{Satz}[name={[Leerheitsproblem]}]\label{satz:leerheitsproblem}
        Das \emph{Leerheitsproblem} ist für reg. Sprachen entscheidbar.
        Falls $L$ reg. Sprache, dann existiert ein Algorithmus, der entscheidet, ob $L=\varnothing$.
\end{Satz}
\begin{proof}
	Sei $\A$ \ac{DEA} für $L$.\\
	Setze Tiefensuche auf den Graphen von $\A$ an. Start bei $q_0$.\\
	Falls die Suche einem akzeptierenden Zustand findet: Nein.\\
	Ansonsten: Ja: $L=\varnothing$\\
	Zeit: $O(|\Sigma||Q|)$
\end{proof}

\begin{Satz}[name={[Endlichkeitsproblem]}]\label{satz:endlichkeitsproblem}
        Das Endlichkeitsproblem für reg. Sprachen ist entscheidbar.
\end{Satz}
\begin{proof}
	Falls $L$ durch $r\in RE(\Sigma)$ gegeben.\\
	$r$ enthält keinen $^* \=> \llbracket r \rrbracket$ endlich.\\
	Zeit: $O(|r|)$
	
	[Reicht nicht, liefert nur eine Richtung]
	
	Falls $L$ durch \ac{DEA} $\A$ gegeben.
	
	\begin{tikzpicture}[>=stealth]
		\node (q0) {$q_0$};
		\node (q1)  [right=.75cm of q0]{$\cdot$};
		\node (q2) [right=.48cm of q1] {$\times$};
		\node (q3) [below=.4cm of q2, xshift=.4cm, inner sep=0pt] {$\times$};
		\draw[->] (q0) edge (q1)
			(q2) ++(-.1cm,-.3cm) -- (q3);
		\draw[->] (q1.south) arc (-155:155:.5cm);
	\end{tikzpicture}
	
	\paragraph*{Oder:} mit \nameref{lem:pumping}.\\
	Sei $L$ regulär und $n$ die Konstante aus dem \ac{PL}.\\
	$L$ unendlich \<=> $\exists w\in L : n\leq |w| <2n$
	\begin{description}[font=\normalfont,labelwidth=\widthof{"'\<="':},leftmargin=!]
	\item["'\<="':] $w$ erfüllt Voraussetzung des \ac{PL}, also $w=uvx$ mit $|uv|\leq n$ und $|v|\geq 1$.\\
		Nach \ac{PL}: $\forall i\in\N$, $uv^ix\in L$, also $L$ unendlich.
	\item["'\=>"'] \emph{Angenommen} $L$ unendlich, aber $\forall w\in L : |w|<n$ oder $|w|\geq 2 n$\\
	Sei $w\in L$ minimal gewählt, so dass $|w|\geq 2n$.
	
	$w$ erfüllt Voraussetzung vom \ac{PL}, also $w=xyz$ mit $|xy|\leq n$ und $|y|\geq 1$\\
	also $\forall i\in\N: xy^iz\in L$ insbes. $i=0: xz\in L$ mit $|xz|<|w|$.
	
	Zwei Möglichkeiten:
	\begin{enumerate}[label=(\alph*)]
	\item $|xz|\geq 2n\ \lightning$ Minimalität von $w$
	\item $|xz|<2n$
		\begin{align*}
			|xz|+|y| &= |w|\geq 2n\text{ mit } 1\leq|y|\leq n\\ %TODO warum gilt 1\leq|y|\leq n ??
			\Rightarrow |xz| &= |w|-|y|\geq 2n-n=n \\
			\Rightarrow |xz| & \geq n \wedge |xz| < 2n \quad\lightning\text{zur Annahme}
		\end{align*}
		Also $\exists w\in L$ mit $n\leq|w|<2n$ \qedhere
	\end{enumerate}
	\end{description}
\end{proof}

\begin{Satz}[name={[Schnittproblem]}]\label{satz:schnittproblem}
        Das \emph{Schnittproblem} ist für REG entscheidbar.\\
        D.h. $L_1,L_2$ reguläre Sprachen. Ist $L_1\cap L_2 = \varnothing$?
\end{Satz}
\begin{proof}
        Nach Satz \ref{satz:schnittproblem} ist $L_1\cap L_2$ regulär. $L_1\cap L_2=\varnothing$ entscheidbar nach \autoref{satz:leerheitsproblem}.
\end{proof}

\begin{Satz}[name={[Äquivalenzproblem]}]\label{satz:äquivalenzproblem}
	Das \emph{Äquivalenzproblem} ist für REG entscheidbar.\\
	D.h. gegeben \ac{DEA}s für $L_1$ und $L_2$, $\A_1$ und $\A_2$
	\[ L_1 = L(\A_1) = L(\A_2) = L_2\ ? \qquad \framebox{Inklusionsproblem}\]
\end{Satz}
\vspace{-2em}
\begin{proof}
        \begin{alignat*}{3}
                L_1\cap \overline{L}_2 &= \varnothing &\quad&\<=>\quad & L_1 &\subseteq L_2\\
                (L_1\cap\overline{L}_2)\cup(L_2\cap \overline{L}_1) &= \varnothing &&\<=> & L_1 &= L_2 \tag*{\qedhere}
        \end{alignat*}
\end{proof}

\begin{Satz}[name={[Inklusionsproblem]}] Äquivalenzproblem \<=> Inklusionsproblem (für REG)
\end{Satz}
\begin{proof}
\begin{itemize}
\item  $=$ entspricht $\subseteq\land\supseteq$
\item $L_1 \subseteq L_2$ genau dann, wenn $L_1 \cup L_2 = L_2$; REG ist abgeschlossen unter Vereinigung
\end{itemize}
\end{proof}


Anwendungsbeispiel f"ur regul"are Sprachen.

$N$ -- liest vom Netz\\
$R$ -- liest lokalen Speicher (ggf. vertrauliche Info)\\
$W$ -- postet auf FB

Programm:

\begin{tabular}{M{l}@{}M{l}@{}M{l}}
        p = N | R | W &| \text{ if }&* \text{ then } p_1\\
        &&\phantom{*}\text{ else }p_2\\
        &|\text{ while }&*\text{ do }p\\
        \mathrlap{
                \begin{rcases}
                        \text{while }*\text{ do }N;\\
                        R;\\
                        \text{if }*\text{ then }N\text{ else }W
                \end{rcases} N^*R\cdot (N+W)
        }
\end{tabular}

Sicherheitspolitik: nach Lesen von lokalem Speicher kein Posten auf FB  $\overline{\Sigma^*R\Sigma^*W\Sigma^*}$

Programm erfüllt Sicherheitspolitik nicht, denn
\[
        \underset{\underset{\displaystyle N^*RW}{\rotatebox[origin=c]{-90}{$\supseteq$}}}{N^*\cdot R(N+W)} \not\subseteq \overline{\Sigma^*R\Sigma^*W\Sigma^*}
\]
}

%%% Local Variables:
%%% mode: latex
%%% TeX-master: "Info_3_Skript_WS2016-17"
%%% End:

% \section{Grammatiken und kontextfreie Sprachen}

\newcommand{\chz}{CH0}
\newcommand{\cho}{CH1}
\newcommand{\cht}{CH2}
\newcommand{\chr}{CH3}

% \datenote{18.11.16}
% Im folgenden Kapitel wechseln wir den Standpunkt von Spracherkennung auf \emph{Spracherzeugung}.
% Das Werkzeug sind hierbei sogenannte \emph{Phasenstrukturgrammatiken}, oder kurz, \emph{Grammatiken}.
% Bei Grammatiken gibt es neben dem Alphabet weitere Symbole, sogenannte \emph{Nichtterminalsymbole} oder \emph{Variablen}, und ein Regelsystem mit dem Wörter, die Nichtterminalsymbole enthalten, geändert werden können.
\begin{Def}
	Eine \emph{Grammatik} ist ein 4-Tupel $(\Sigma,N,P,S)$ mit folgenden Komponenten:
	\begin{itemize}
	\item $\Sigma$ ist ein Alphabet, dessen Elemente wir in diesem Kontext auch \emph{Terminalsymbole} nennen.
	\item $N$ ist eine endliche Menge, deren Elemente wir \emph{Nichtterminalsymbole} oder \emph{Variablen} nennen.
	\item $P\subseteq (N\cup\Sigma)^*N(N\cup\Sigma)^* \x (N\cup\Sigma)^*$ ist eine endliche Relation, 
	deren Elemente wir \emph{Regeln} oder \emph{Produktionen} nennen.
	\item $S\in N$ ist ein Nichtterminalsymbol, das wir \emph{Startsymbol} nennen.
	\qedhere
	\end{itemize}
\end{Def}

\begin{Bsp}\label{bsp:3.sameNumber}
  $\mathcal{G} = (\Sigma, N, P, S)$ mit\footnote{$P$ ist eine "`normale"' binäre Relation, 
  doch wir verwenden statt "`$(x,y)\in P$"' meist "`$x\-> y$"', also einen Pfeil und Infix-Notation, um die Lesbarkeit zu erhöhen.}
	\begin{align*}
		\Sigma &= \{ 0, 1 \}\\
		N &= \{S\}\\
		P &= \begin{aligned}[t]
      \{ S & \to 1S0S \\
        S & \to 0S1S \\
        S & \to \Eps
      \}
        \end{aligned}
      \qedhere
	\end{align*}
\end{Bsp}
% 
%   $G$ erzeugt geklammerte arithmetische Ausdrücke über der Konstante $\mathtt{a}$, wie zum Beispiel das Wort $\mathtt{(a*(a+a))}$.
%   Eine Grammatik erzeugt Wörter durch \emph{Ableitungen}, die wir im Folgenden definieren.
%   
\begin{Def}[Ableitungsrelation, Ableitung, Sprache einer Grammatik]
  Sei $\mathcal{G} =(\Sigma,N,P,S)$ eine Grammatik.
	Die \emph{Ableitungsrelation}\footnote{Analog zur Relation $P$ verwenden wir auch für die Ableitungsrelation wieder Infix-Notation, also $\alpha\vdash\beta$ statt $(\alpha,\beta)\in\;\vdash$.}
	zu $\mathcal{G}$ ist 
  \begin{displaymath}
    \cdot \vdash_\mathcal{G} \cdot \subseteq (N\cup\Sigma)^*\x(N\cup\Sigma)^*
  \end{displaymath}
  mit \ \ $\alpha \vdash_\mathcal{G} \beta$\ \  gdw\ \  $\alpha = \gamma_1\alpha'\gamma_2$,\ \  $\beta = \gamma_1\beta'\gamma_2$\ \  und \ \ $\alpha' \to \beta' \in P$

  Eine Folge $\alpha = \alpha_0,\ldots,\alpha_n = \beta$ heißt \emph{Ableitung von $\beta$ aus $\alpha$ in $n$ Schritten}, geschrieben $\alpha \stackrel{n}{\vdash}_\mathcal{G} \beta$, gdw $\alpha_i \vdash_\mathcal{G} \alpha_{i+1}$ für $0 \le i < n$.
  Jedes solche $\alpha_i$ heißt \emph{Satzform von $\mathcal{G}$}.

  Die \emph{Ableitung von $\beta$ aus $\alpha$}, geschrieben $\alpha \stackrel{*}{\vdash}_\mathcal{G} \beta$, existiert gdw ein $n \in \N$ existiert, sodass $\alpha \stackrel{n}{\vdash}_\mathcal{G} \beta$.
  Damit ist "`$\stackrel{*}{\vdash}_\mathcal{G}$"' die reflexive, transitive Hülle von "`$\vdash_\mathcal{G}$"'.

  Ein Wort $w\in\Sigma^*$ wird von $\mathcal{G}$ \emph{erzeugt}, wenn $S \stackrel{*}{\vdash}_{\mathcal{G}} w$ gilt.
	Die von $\mathcal{G}$ \emph{erzeugte Sprache} ist definiert als:
	\[ L(\mathcal{G}) = \{w\in\Sigma^* \mid S \stackrel{*}{\vdash}_{\mathcal{G}} w \}  \]
	
	Wir nennen zwei Grammatiken \emph{äquivalent}, wenn sie die gleiche Sprache erzeugen.
\end{Def}





\begin{Bsp}Wir betrachten nochmal die Grammatik aus \autoref{bsp:3.sameNumber}.\datenote{28.11.18}
Es gilt: $1001\in L(\mathcal{G})$.
$$ S 
\vdash_\mathcal{G} 1S0S
\vdash_\mathcal{G} 10S
\vdash_\mathcal{G} 100S1S
\vdash_\mathcal{G} 100S1
\vdash_\mathcal{G} 1001
$$
% \datenote{22.11.17}
Außerdem gilt: $L(\mathcal{G})$ ist die Sprache der Wörter über $\{0,1\}$, die gleich viele Nullen wie Einsen haben:
  \begin{displaymath}
    L = \{ w \in \Sigma^* \mid \#_0(w) = \#_1(w)\}
  \end{displaymath}
  Die Funktion $\#_a(w)$ berechnet hierbei die Anzahl der Vorkommen von $a \in \{0, 1\}$ in $w$.

  Dass $L(\mathcal{G}) \subseteq L$, lässt sich via Induktion über die Länge der Ableitung von $S \stackrel{*}{\vdash}_\mathcal{G} w$ zeigen.
  Der Beweis wird als Übung dem Leser überlassen.

  Wir zeigen $L \subseteq L(\mathcal{G})$.
  Dazu zeigen wir "`Wenn $w \in L$, dann $w \in L(\mathcal{G})$"' via Induktion über die Länge von $w$.
  Hierzu definieren wir noch die Hilfsfunktion $d : \Sigma^* \to \N$:
  \begin{align*}
    d(\Eps) &= 0 \\
    d(1w) &= d(w) + 1 \\
    d(0w) &= d(w) - 1
  \end{align*}
  Per Induktion über $|w|$ mit $w \in \Sigma^*$ lässt sich leicht zeigen, dass $L = \{w \in \Sigma^* \mid d(w) = 0\}$ und $d(v \cdot w) = d(v) + d(w)$.
  
Wir zeigen nun via Induktion über $n$ die folgende Eigenschaft:
$$\forall n' \leq n: \forall w \in \Sigma^*: \text{ falls } |w| = n' \text{ und } w \in L, \text{ dann } w \in L(\mathcal{G})$$

\begin{description}[font=\normalfont]
\item[I.A.:] $n = 0$: $w = \Eps$. Es gilt $\Eps \in L$, da $\#_0(\Eps) = \#_1(\Eps) = 0$.
% \item[IV] $\forall n' < n: \forall w' \in \Sigma^*: falls |w| = n' \text{ und } w \in L \text{ dann } w \in L(\mathcal{G})$
\item[I.S.:] $n \rightsquigarrow n+1$: $|w| = n > 1$, $w = aw'$, $a \in \{0,1\}$.
  Beachte: $|w|$ ist gerade für alle $w \in L$.

  Betrachte $a = 0$ (der Fall für $a = 1$ funktioniert analog).

  Da $0 = d(w) = d(0w') = d(w') - 1$, ist $d(w') = 1$.
  
  \medskip

  Wir zeigen zunächst, dass wir $w'$ in $w_11w_2$ mit $d(w_1) = 0$ und $d(w_2) = 0$ zerlegen können:
%   \begin{itemize}
%   \item[] 

    Sei $w' = a_1 \ldots a_n$.
    Betrachte die Folge $d_0,\ldots,d_n$ mit $d_0 = 0$ und $d_i = d(a_1\ldots a_i)$ für $1 \le i \leq n$.
    Wähle $0 \le i < n$ maximal, sodass für alle $0 \le j \le i$ gilt: $d_j < 1$.%
    \footnote{Wir wissen, dass $i < n$, da $d_n = d(w') = 1$.}

    Da $i$ maximal ist, folgt $d_{i+1} \ge 1$.
    Da $d_{j+1} - d_j \le 1$ für alle $0 \le j \le i$, folgt $d_{i+1} - d_i \le 1$ und damit auch $d_i = 0$, $d_{i+1} = 1$ und $a_{i+1} = 1$.
    Setze $w_1 = a_1\ldots a_{i}$.

    Es gilt also $w' = a_1\ldots a_{i}a_{i+1}w_2 = w_1w_2$ mit $d(w_1) = 0$.

    Da $d(v \cdot w) = d(v) + d(w)$ und $d(w') = d(w_11w_2) = 1$, folgt $d(w_2) = 0$.
%   \end{itemize}
  \bigskip
  \goodbreak

  Da $|w_1| < n$ und $|w_2| < n$, folgt nach I.V., dass $S \stackrel{*}{\vdash}_{\mathcal{G}} w_1$ und $S \stackrel{*}{\vdash}_{\mathcal{G}} w_2$.

  Es folgt mit den Regeln $S \vdash_{\mathcal{G}} 0S1S \stackrel{*}{\vdash}_{\mathcal{G}} 0w_11S \stackrel{*}{\vdash}_{\mathcal{G}} 0w_11w_2$.
  \qedhere
\end{description}
\end{Bsp}

\begin{Bsp}\label{bsp:3.anbncn}
	$\mathcal{G}=(\Sigma,N,P,S)$ mit
	\begin{align*}
		\Sigma &= \{a,b,c\}\\
		N &= \{S,B,C\}\\
		P &= 
		\begin{aligned}[t]
			 \{ &S \-> aSBC,\ S\->aBC,\ CB\-> BC,\ aB\-> ab,
			bB\-> bb, bC\-> bc, cC\-> cc \}
		\end{aligned}
% 		L(\mathcal{G}) &=  \{ a^nb^nc^n \mid n\geq 1 \}\\
% 		S &\=> aBC \=> abC \=> abc \qquad S\=>a\emph{S}BC \=> aa\emph{S}BCBC\\
% 		S &\=> aSBC \xRightarrow{n-2} n \=> a^{n-1}S(BC)^{n-1} \=> a^n(BC)^n = a^nBCBC \dots \=> a^nBBCCBC\\
% 		&\=> a^nB^nC^n \xRightarrow* a^nb^nc^n
	\end{align*}
Es gilt z.B. $aaabbbccc\in L(\mathcal{G})$.

Außerdem gilt $L(\mathcal{G}) =  \{ a^nb^nc^n \mid n\geq 1 \}$. (Ohne Beweis)
\end{Bsp}

\bigskip

Die Chomsky-Hierarchie teilt die Grammatiken in vier Typen unterschiedlicher Mächtigkeit ein.
\begin{Def}[Chomsky-Hierarchie]\
	\begin{itemize}
	\item Jede Grammatik ist eine \emph{Typ-0-Grammatik}.
	\item Eine Grammatik ist \emph{Typ-1} oder
          \emph{kontextsensitiv}, falls alle Regeln expansiv sind,
          d.h., für alle Regeln $\alpha\->\beta\in P$ ist
          $|\alpha|\leq |\beta|$. Ausnahme: falls $S$ nicht in einer
          rechten Regelseite auftritt, dann ist $S\->\Eps$ erlaubt. 
	\item Eine Grammatik heißt \emph{Typ-2} oder \emph{kontextfrei}, falls alle Regeln die Form $A\->\alpha$ mit $A\in N$ und $\alpha\in(N\cup\Sigma)^*$ haben.
	\item Eine Grammatik heißt \emph{Typ-3} oder \emph{regulär}, falls alle Regeln die folgende Form haben:
	\begin{alignat*}{2}
		&A\->w &\quad w&\in\Sigma^*\\
		\text{oder}\quad &A\->aB& a&\in\Sigma,\ B\in N
	\end{alignat*}
	\end{itemize}
	Eine Sprache heißt Typ-$i$-Sprache, falls es eine Typ-$i$-Grammatik für sie gibt.
\end{Def}

\begin{Beobachtung}\label{beob:3.ChomskyHierarchie}
	Jede Typ-$(i+1)$-Sprache ist auch eine Typ-$i$-Sprache.
\end{Beobachtung}
\begin{itemize}
 \item Jede Typ-3-Grammatik ist eine Typ-2-Grammatik.
 \item Wir werden im folgenden Unterkapitel zeigen, dass jede Typ-2-Grammatik in eine äquivalente $\Eps$-freie\footnote{Auf keiner rechten Seite steht $\Eps$; Ausnahme: Startsymbol $S$; vgl. Elimination von $\Eps$-Produktionen.} Typ-2-Grammatik transformiert werden kann.
 Diese erfüllt dann auch alle Bedingungen einer Typ-1-Grammatik.
 \item Jede Typ-1-Grammatik ist auch eine Typ-0-Grammatik.
\end{itemize}

Im Lauf der Vorlesung werden wir die folgende Aussage zeigen:\\
Sei $CHi$ die Menge der Typ-$i$ Sprachen; dann gilt: 
$CH3 \subsetneq CH2 \subsetneq CH1 \subsetneq CH0$.




\subsection{Kontextfreie Sprachen}\datenote{30.11.2018}
Die Sprache aus \autoref{bsp:3.sameNumber} ist kontextfrei. Weitere Beispiele sind:

\begin{Bsp}\label{bsp:3.ArithExpr}
Arithmetische Ausdrücke ohne Klammern: 
$\mathcal{G}=(\{E\}, \{a, \mathnormal{+}, \mathnormal{*}\}, P, E)$ mit
\begin{displaymath}
 P = \{ E \->a \mid E+E \mid E*E\}.
 \footnote{Notation: Wenn wir mehrere Regeln mit gleicher linker Seite wie 
 z.B. "`$A \-> \alpha$"' und "`$A \-> \beta$"' haben, dann dürfen wir diese auch mit Hilfe eines senkrechten Striches als "`$A \-> \alpha\mid\beta$"' schreiben.}
 \qedhere
\end{displaymath}
\end{Bsp}

\begin{Bsp}
Arithmetische Ausdrücke mit Klammern: 
$\mathcal{G}=(\{E\}, \{a, \mathnormal{+}, \mathnormal{*}, \mathnormal{(}, \mathnormal{)}\}, P, E)$ mit
  \begin{displaymath}
    P = \{ E \->a \mid (E+E) \mid (E*E)\}.
  \qedhere
  \end{displaymath}
\end{Bsp}
% 
\newcommand{\boxedt}[1]{\tikz[baseline]{\node[shape=rectangle,draw=lightgray,fill=lightgray!50,inner sep=0pt] at (0,.64ex){\hspace{.3em}\texttt{\strut#1}\hspace{.3em}\strut};}}
% 
\begin{Bsp}
Syntax von Programmiersprachen. 
Wir zeigen hier nur exemplarisch Produktionen für einige typische Bestandteile einer Programmiersprache.\footnote{
Sie finden am Ende der Java Language Specification \url{https://docs.oracle.com/javase/specs/} auf ca.\ 25 Seiten die kontextfreie Grammatik für diese Programmiersprache.
}
Wörter in spitzen Klammern sind hier Nichtterminalsymbole.
Terminalsymbole sind grau unterlegt.
	\begin{center}
		\begin{tabular}[t]{r@{ }l}
			<Stmt> \-> & <Var> \boxedt{=} <Exp>\\
			|\, & <Stmt>\,\boxedt{;}\,<Stmt>\\
			|\, & \boxedt{if}\,\boxedt{(}<Exp>\boxedt{)} <Stmt> \boxedt{else} <Stmt>\\
			|\, & \boxedt{while}\,\boxedt{(}<Exp>\boxedt{)} <Stmt>
		\qedhere
		\end{tabular}
	\end{center}
\end{Bsp}

% \begin{Bsp}
%  \item Palindrome über $\{a,b\}$
% 	\[ S\-> aSa \mid bSb \mid a \mid b \mid \Eps \]
% \end{Bsp}

%Abk.: \aclu{CFL}: \acs{CFL}\\
%\phantom{Abk.:} \aclu{CFG}: \acs{CFG}


Für den arithmetischen Ausdruck ohne Klammern $a+a$ gibt es in der Grammatik aus \autoref{bsp:3.ArithExpr} zwei verschiedene Ableitungen:
\begin{itemize}
 \item $E\vdash_\mathcal{G} E+E\vdash_\mathcal{G} a+E\vdash_\mathcal{G} a+a$
 \item $E\vdash_\mathcal{G} E+E\vdash_\mathcal{G} E+a\vdash_\mathcal{G} a+a$
\end{itemize}
Beide Ableitungen unterscheiden sich nur durch die Reihenfolge, in der Variablen ersetzt werden.
Die folgende Definition erlaubt es uns, beide Ableitungen durch das gleiche Objekt,
den sogenannten Ableitungsbaum, darzustellen.


\newcommand{\T}{\mathcal{T}}

\begin{Def}[name={[Ableitungsbaum]}] Sei $\mathcal{G} = (\Sigma, N, P, S)$ eine kontextfreie Grammatik.
  Wir definieren die Menge der \emph{in $A$ beginnenden Ableitungsbäume von $\mathcal{G}$}, $\operatorname{Abl}(A)$, für $A\in N$ als Menge von beschrifteten, geordneten Bäumen induktiv wie folgt:
  
  Falls $\pi = A \to w_0A_1w_1\ldots A_nw_n \in P$ mit $A_i \in N$, $w_0, w_i \in \Sigma^*$, $1 \le i \le n$ und $\T_i \in \operatorname{Abl}(A_i)$, dann ist
    \begin{center}
			\tikz[baseline=0cm]{\Tree [.$\pi$ $\T_1$ \edge[draw=none]; {$\dots$} $\T_n$ ]} $\in \operatorname{Abl}(A)$
    \end{center}
    Da es manchmal aufwendig ist, Bäume zu zeichnen, verwenden wir alternativ die folgende Notation: $\pi(\T_1, \ldots, \T_n)$

    Das \emph{abgeleitete Wort zu einem $\T \in \operatorname{Abl}(A)$}, $Y(\T)$\footnote{Wir verwenden $Y(\T)$, um an das englische Wort "`yield"' zu erinnern.}, ist definiert durch
    \begin{displaymath}
      Y\left(\tikz[baseline=-0.4cm]{\Tree [.$\pi$ $\T_1$ \edge[draw=none]; {$\dots$} $\T_n$ ]} \right) = w_0Y(\T_1)w_1\ldots Y(\T_n)w_n
    \end{displaymath}
    wobei $\pi = A \to w_0A_1w_1\ldots A_nw_n \in P$.
\end{Def}

\begin{Bemerkung}
Nach dieser Definition gibt es also insbesonere für jede Regel $\pi$ bei der nur Terminalsymbole auf der rechten Seite vorkommen einen Ableitungsbaum der die einelementige Knotenmenge $\{\pi\}$ und einer leeren Menge von Kanten hat.
\end{Bemerkung}



\begin{Bsp}\label{bsp:Ableitungsbaum} 
Für die Grammatik aus \autoref{bsp:3.ArithExpr} sind die folgenden drei Beispiele in $E$ beginnende Ableitungsbäume.

    \tikz[baseline=0cm]{\Tree [ .$E\to a$ ]}
    \hfill
    \tikz[baseline=0cm]{\Tree [ .$E\to E+E$ $E\to a$ [ .$E\to E*E$ $E\to a$ $E\to a$ ] ]}
    \hfill
    \tikz[baseline=0cm]{\Tree [ .$E\to E*E$ [ .$E\to E+E$ $E\to a$ $E\to a$ ] $E\to a$ ]}
\end{Bsp}

\begin{lemma}
  Sei $\mathcal{G} = (\Sigma, N, P, S)$ eine kontextfreie Grammatik.
  \begin{displaymath}
    w \in L(\mathcal{G})  \quad\text{ gdw }\quad \exists \;\T \in \operatorname{Abl}(S) \text{ mit } Y(\T) = w
    \qedhere
  \end{displaymath}
\end{lemma}
(Ohne Beweis.)

% \begin{proof}[$\Longrightarrow$]
%   Wir zeigen via vollständiger Induktion über die Länge der Ableitung $n$:
%   \begin{displaymath}
%    \forall n \in \N: \forall w \in \Sigma^*: \forall A \in N: A \stackrel{n}{\==>} w \text{ dann } \exists \T \in \operatorname{Abl}(\mathcal{G}, A) \text{ mit } Y(\T) = w
% \end{displaymath}
%    \begin{description}
%    \item[IA] $n=0$ (nichts zu tun, da $A \stackrel{n}{\==>} \Eps$ unmöglich)
%    \item[IS] Sei $n > 0$: $A \stackrel{n}{\==>}$ hat die Form
%      \begin{displaymath}
%        A {\==>} w_0A_1w_1\ldots A_nw_n \stackrel{n-1}{\==>} w
%      \end{displaymath}
%      Also ist $w = w_0v_1w_1\ldots v_nw_n$ mit $A_i \stackrel{n_i}{\==>} v_i$ für $1 \le i \le n$ und $n_i < n$.\footnote{Dieser Beweisschritt ist nicht ganz präzise\ldots die Ableitungen der $v_i$ sind nicht unbedingt Teil der gesamten Ableitung $A \stackrel{n}{\Longrightarrow} w$.
%      Man kann aber beweisen dass eine alternative Ableitung existieren muss, bei der dies der Fall ist.}
%    Nach IV ergibt sich
%    \begin{displaymath}
%      \exists \T_i \in \operatorname{Abl}(G, A_i) \text{ mit } Y(\T_i) = v_i
%    \end{displaymath}
%    Dann ist $\T = \pi(\T_1,\ldots, \T_n)$ mit $\pi = A \to w_0A_1w_1\ldots A_nw_n \in \operatorname{Abl}(\mathcal{G}, A)$ und $Y(\T) = w_0Y(\T_1)w_1\ldots Y(\T_n)w_n= w_0v_1w_1\ldots v_nw_n$.
%    \end{description}
% \end{proof}
% 
% \begin{proof}["`rechts nach links"']
%   Zu zeigen:
%   \begin{displaymath}
%     \forall n \in \N: \forall \T \in \operatorname{Abl}(\mathcal{G}, A): \text{falls } Y(\T) = w \text{ dann } A \stackrel{*}{\==>} w
%   \end{displaymath}
%   \begin{description}
%   \item[IS]
% \footnote{ Der Induktionsanfang ist bei Ableitungsbäumen ein Spezialfall des Induktionsschritts (Bäume ohne Kinder bzw.\ Regeln, die nur Terminale ableiten) und wird daher nicht gesondert aufgeführt }
%     Falls für $n \ge 0$
%     \begin{itemize}
%     \item  $\T = \pi(\T_1, \ldots, \T_n)$,
%     \item $\pi = w_0A_1w_1\ldots A_nw_n$,
%     \item $\T_i \in \operatorname{Abl}(\mathcal{G}, A_i)$,
%     \item $Y(\T) = w_0Y(\T_1)w_1\ldots Y(\T_n)w_n$
%     \end{itemize}
%     dann gilt nach IV:\footnote{Strengenommen müsste die Folgerung, die hier mit "`\ldots"' abgekürzt ist noch via Induktion über Ableitungen gezeigt werden} 
%     \begin{align*}
%       A &\==> w_0A_1\ldots A_nw_n \\
%         &\stackrel{*}{\==>} w_0Y(\T_1)\ldots A_nw_n \\
%         & \dots \\
%         & \stackrel{*}{\==>} w_0Y(\T_1)\ldots Y(\T_n)w_n \\
%         &= w
%     \end{align*}
%   \end{description}
%   
% \end{proof}

\begin{Def}[name={[Eindeutigkeit von \acs*{CFG} und \acs*{CFL}]}]\
	\begin{itemize}
% 	\item Sei CFG die Menge der kontextfreien Grammatiken.
  \item Eine Grammatik $\mathcal{G}$ heißt \emph{eindeutig}, falls es für jedes Wort $\in L(\mathcal{G})$ genau einen Ableitungsbaum gibt.
	\item Eine kontextfreie Sprache $L$ heißt \emph{eindeutig}, falls es eine eindeutige kontextfreie Grammatik gibt, die $L$ erzeugt.
	\qedhere
	\end{itemize}
\end{Def}
% \begin{Bsp}
% 	\begin{align*}
% 		L &= \{ a^ib^jc^k \mid i=j\text{ oder }j=k \}\\
% 		S &\-> AC \mid DB\\
% 		A &\-> aAb \mid \Eps & D &\->aD \mid \Eps\\
% 		C &\->cC \mid \Eps & B &\-> bBc \mid \Eps \quad[\text{für $L$ gibt es keine eindeutige \ac{CFG}}]
% 	\end{align*}
% 	Werte der Form $a^nb^nc^n$ haben zwei Ableitungen $\curvearrowright$ Grammatik ist nicht eindeutig.
% \end{Bsp}

Die Grammatik aus \autoref{bsp:3.ArithExpr} ist also nicht eindeutig, denn sowohl für den zweiten als auch für den dritten Ableitungsbaum aus \autoref{bsp:Ableitungsbaum} ist das abgeleitete Wort $a+a*a$.

Im Folgenden verwenden wir auch die Abkürzung \acsu{CFG} für "`kontextfreie Grammatik"' (engl.\ context-free grammar).



\subsection{Die Chomsky-Normalform für kontextfreie Sprachen}\label{sec:cnf}
% Notwendig für das Pumping Lemma kontextfreier Sprachen und für einen effizienten Algorithmus für das \nameref{satz:wortproblem}.
Wir lernen in diesem Unterkapitel eine spezielle Form von kontextfreien Grammatiken (Chomsky-Normalform) kennen, für die gilt:
\begin{itemize}
 \item Das Wortproblem ($w\in L(\mathcal{G})$?) lässt sich mit Hilfe eines einfachen Algorithmus entscheiden.
 \item Jede kontextfreie Grammatik lässt sich "`effizient"' in diese Normalform transformieren.
\end{itemize}

Wir stellen im Folgenden vier Transformationen (\textsc{Sep, Bin, Del, Unit}) 
vor und wollen dabei jeweils den Zeitaufwand und die Größe der resultierenden \ac{CFG} analysieren.

Dafür definieren wir die Größe einer Grammatik wie folgt.
\begin{displaymath}
  |\mathcal{G}| = \sum_{A \in N}\sum_{A \to \alpha \in P} |A\alpha|
\end{displaymath}
Die Größe ist also genau die Anzahl an Zeichen aus $\Sigma\cup N$, die man benötigt, um die Regeln der Grammatik aufzuschreiben.

\begin{Def}
  Eine \ac{CFG} heißt \emph{separiert}, wenn jede Regel eine der folgenden beiden Formen hat.
  \begin{alignat*}{3}
   A &\-> A_1\dots A_n && \quad \text{für } A\in N, A_i\in N,n\geq 0\\[1mm]
   A &\-> a && \quad \text{für } A\in N, a\in\Sigma
  \qedherefixalignat
  \end{alignat*}
\end{Def}

\begin{lemma}[\textsc{Sep}]
  Zu jeder \ac{CFG} gibt es eine äquivalente separierte \ac{CFG}.
\end{lemma}
\begin{proof}
  Sei $\mathcal{G} = (\Sigma, N, P, S)$ eine \ac{CFG}.
  Konstruiere $\mathcal{G}' = (\Sigma, N', P', S)$ mit
  \begin{itemize}
  \item $N' = N \overset.\cup \{Y_a \mid a \in \Sigma \}$\footnote{"`$\overset.\cup$"' steht für "`disjunkte Vereinigung"'}
  \item $P' = \{Y_a \to a \mid a \in \Sigma \} \cup \{A \to \beta[a \to Y_a] \mid A \to \beta \in P \}$.
  
    Die Schreibweise $\beta[a\to Y_a]$ bedeutet hier, dass alle $a \in \Sigma$, die in $\beta$ vorkommen, durch $Y_a$ ersetzt werden.
  \end{itemize}
  Offenbar gilt $L(\mathcal{G}) = L(\mathcal{G'})$ und $\mathcal{G}'$ ist separiert (ohne Beweis).
\end{proof}
\begin{Bemerkung}
\textsc{Sep} berechnet in $O(|\mathcal{G}|^2)$ eine Grammatik der Größe $O(|\mathcal{G}|)$.
\begin{itemize}
	\item Laufe für jedes Terminalsymbol einmal über alle Regeln.
	
	\item Für jedes neue Nichtterminalsymbol fügen wir eine Regel der Größe 2 hinzu.
	\qedhere
\end{itemize}
\end{Bemerkung}



\begin{lemma}[\textsc{Bin}]
  Zu jeder \ac{CFG} gibt es eine äquivalente \ac{CFG}, bei der für alle Regeln $A \to \alpha$ gilt, dass $|\alpha| \le 2$.
\end{lemma}
\begin{proof}
  Ersetze jede Regel der Form 
  \begin{displaymath}
    A \to X_1X_2\ldots X_n, \quad n\geq 3
  \end{displaymath}
  mit $X_i \in N \cup \Sigma$, $1 \le i \le n$,
  durch die Regeln 
  \begin{align*}
    A &\to X_1\langle  X_2\ldots X_n \rangle \\
    \langle  X_2\ldots X_n \rangle & \to X_2\langle  X_3\ldots X_n \rangle \\
    \vdots \\
    \langle  X_{n-1}X_n \rangle & \to X_{n-1}X_n
  \end{align*}
  Dabei sind  $\langle  X_2\ldots X_n \rangle$, \ldots, $\langle  X_{n-1}\ldots X_n \rangle$ neue Nichtterminalsymbole.
\end{proof}
\begin{Bemerkung}
\textsc{Bin} berechnet in $O(|\mathcal{G}|)$ eine Grammatik der Größe $O(|\mathcal{G}|)$.
\begin{itemize}
	\item Laufe einmal über alle Regeln.
	
	\item Für jede Regel der Größe $n+1$, $n\geq 3$ fügen wir höchstens $n-1$ neue Regeln der Größe 3 hinzu.
	\qedhere
\end{itemize}
\end{Bemerkung}


Wir definieren die Menge der Nichtterminalsymbole, aus denen das leere Wort abgeleitet werden kann, formal wie folgt:
\begin{Def}
  Sei $\mathcal{G} = (\Sigma, N, P, S)$ eine \ac{CFG}.
  Definiere die Menge $\operatorname{Nullable}(\mathcal{G}) \subseteq N$ als
  \begin{displaymath}
    \operatorname{Nullable}(\mathcal{G}) = \{ A \in N \mid A
    \stackrel{*}{\vdash}_\mathcal{G} \Eps \}
    \text{.} \qedhere
  \end{displaymath}
\end{Def}

\begin{Satz}\label{satz:3.nullable}\datenote{5.12.2018}
  Es gibt einen Algorithmus, der $\operatorname{Nullable}(\mathcal{G})$ in $O(|\mathcal{G}|^3)$ berechnet.
\end{Satz}
\begin{proof}
  Definiere $M_i$ als die Menge der Nichtterminalsymbole, aus denen sich
  $\Eps$ mit einem Ableitungsbaum der Höhe $< i$ ableiten lässt:\footnote{
  Im Folgenden ist mit $*$ bei $M_i^*$ und $M^*$ der Kleene-Stern gemeint.} 
  \begin{align*}
    M_0 &= \emptyset \\
    M_{i+1} &= M_i \cup \{ A \mid A \to \alpha \in P \text{ und }
              \alpha \in M_i^* \}
  \end{align*}
  Es gilt für alle $i \in \N$, dass $M_i \subseteq N$.
  Da $|N|$ endlich ist, existiert ein $m \in \N$, sodass
  \begin{displaymath}
    M_m = M_{m+1} = \bigcup_{i \in \N} M_i
  \end{displaymath}
  Wir können $\bigcup_{i \in \N} M_i$ mit dem folgenden Verfahren in $O(|\mathcal{G}|^3)$ berechnen.
  
  \begin{center}
  \begin{minipage}{7cm}
  \begin{lstlisting}[
  mathescape,
  columns=fullflexible,
  basicstyle=\fontfamily{lmvtt}\selectfont,
]
M = {}
done = $false$
while (not done):
    done = $true$
    M$_{old}$ = M
    foreach $A \to a \in P$:
         if ($ A \not \in M \wedge a \in M_{old}^*$):
             $M = M \cup \{A\}$
             done = $false$
return M
\end{lstlisting}
\end{minipage}
\end{center}

  Wir zeigen nun, dass $\operatorname{Nullable}(\mathcal{G}) = \bigcup_{i \in \N} M_i$.

  \medskip
  
  "`$\supseteq$"'\quad
  Wir zeigen via Induktion über die Höhe des Ableitungsbaums $i$, dass 
  $$\forall i \in \N: M_i \subseteq \operatorname{Nullable}(\mathcal{G})$$
  \begin{description}
  \item[IA] $i = 0$: $M_0 = \emptyset \subseteq \operatorname{Nullable}(\mathcal{G})$
  \item[IS] $i \rightsquigarrow i+1$

    Wenn $A \in M_{i+1}$, dann
    \begin{itemize}
    \item gilt entweder
      $A \in M_i \subseteq \operatorname{Nullable}(\mathcal{G})$ nach
      IV oder
    \item 
      es existiert $A \to A_1\ldots A_n \in P$ mit $A_j \in M_i$ für
      $1 \le j \le n$.
      Nach IV existieren Ableitungen
      \begin{displaymath}
        A_j \stackrel{*}{\vdash}\mathcal{G} \Eps
      \end{displaymath}
      sodass auch eine Ableitung von $A$ nach $\Eps$ existiert:
      \begin{displaymath}
        A  \vdash_\mathcal{G} A_1 \ldots A_n \stackrel{*}{\vdash}\mathcal{G} \underbrace{\Eps \ldots \Eps}_{n \text{ Mal}} = \Eps
      \end{displaymath}
      Also gilt $A \in \operatorname{Nullable}(\mathcal{G})$.
    \end{itemize}
  \end{description}

  \medskip
  
  "`$\subseteq$"'\quad
  Wenn $A \in \operatorname{Nullable}(\mathcal{G})$, dann existiert $m
  \in \N$, sodass $A \stackrel{*}{\vdash}\mathcal{G} \Eps$ mit einem
  Ableitungsbaum der Höhe $m$.
  Wir zeigen via Induktion über $m$, dass $A \in M_{m+1} \subseteq \bigcup_{i \in \N} M_i$.
  \begin{description}
  \item[IA] $m = 1$. Die Ableitung ist $A \vdash_\mathcal{G} \Eps$, sodass $A\in M_1$.
  \item[IS] $m > 1$.
    Die Wurzel des Ableitungsbaum muss mit $A \to A_1\ldots A_n$ ($n>0$)
    beschriftet sein und die Kinder sind jeweils Ableitungsbäume für
    $\Eps$ in  $Abl(A_i)$ der Höhe $m_i \le m-1$, wobei $1 \le i \le n$.
    
    Es gilt nun nach IV, dass $A_i \in M_{m_i} \subseteq M_{m-1}$.
    Somit ist $A_i \in M_{m-1}$ und damit, nach Definition, $A \in M_m$.
    \qedhere
  \end{description}
\end{proof}

\begin{Korollar}
  Man kann zu einer \ac{CFG} $\mathcal{G}$ berechnen, ob $\Eps \in L(\mathcal{G})$.
\end{Korollar}
\begin{proof}
  Prüfe, ob $S \in \operatorname{Nullable}(\mathcal{G})$.
\end{proof}
\begin{lemma}[\textsc{Del}]
  Zu jeder \ac{CFG} $\mathcal{G}=(\Sigma, N, P, S)$ gibt es eine äquivalente \ac{CFG} $\mathcal{G'} =(\Sigma, N', P', S')$, bei der die einzige $\Eps$-Regel $S' \to \Eps$ ist (falls $\Eps \in L(\mathcal{G})$) und bei der $S'$ auf keiner rechten Seite einer Regel vorkommt.
\end{lemma}
\begin{proof}
  \hfill
  \begin{enumerate}
  \item Erweitere $\mathcal{G}$ um ein neues Startsymbol $S' \notin N$ mit $S' \to S$ als neue Regel.
    Dieser Schritt stellt sicher, dass $S'$ auf keiner rechten Regelseite vorkommt.
  \item Wende erst \textsc{Sep}, dann \textsc{Bin} an.
    Nun hat jede rechte Regelseite die Form $a \in \Sigma$ oder $\Eps$ oder $A$ oder $BC$.
  \item Für alle Regeln $A \to BC \in P$:
    \begin{itemize}
    \item Falls $B \in \operatorname{Nullable}(\mathcal{G})$ füge $A \to C$ hinzu.
    \item Falls $C \in \operatorname{Nullable}(\mathcal{G})$ füge $A \to B$ hinzu.
    \end{itemize}
  \item Falls $S \in \operatorname{Nullable}(\mathcal{G})$, füge $S' \to \Eps$ hinzu
  \item Entferne alle Regeln $A \to \Eps$ für $A \neq S'$.
  \qedhere
  \end{enumerate}
\end{proof}
\begin{Bemerkung}
\textsc{Del} kann in $O(|\mathcal{G}|^3)$ berechnet werden und die Größe der resultierenden Grammatik ist $O(|\mathcal{G}|)$.
\begin{itemize}
	\item Die höchsten Kosten entstehen beim Berechnen von Nullable.
	
	\item In Schritt 3 verdoppelt sich die Anzahl der Regeln höchstens.
	\qedhere
\end{itemize}
\end{Bemerkung}

\begin{Def}
  Sei $\mathcal{G} = (\Sigma, N, P, S)$ eine Grammatik.
  Eine \emph{Kettenregel} von $\mathcal{G}$ ist eine Regel in $P$ der Form $A \to B$, wobei $A,B \in N$.
\end{Def}

\datenote{7.12.2018}
\begin{lemma}[\textsc{Unit}]
Es gibt einen Algorithmus der für eine \ac{CFG} $\mathcal{G}$ in $O(|\mathcal{G}|^3)$ eien äquivalente \ac{CFG} $\mathcal{G}'$ ohne Kettenregeln mit $|\mathcal{G}'|\in O(|\mathcal{G}|^2)$ berechnet.
%   Zu jeder \ac{CFG} $\mathcal{G} = (\Sigma, N, P, S)$ gibt es eine äquivalente \ac{CFG} $\mathcal{G}' = (N, \Sigma, P', S)$ ohne Kettenregeln.
\end{lemma}
\begin{proof} (Beweisskizze)
	\begin{itemize}
	\item Zur Äquivalenz:
	
	Wir zeigen dass jeden Schritt der die \ac{CFG} ändert eine äquivalente \ac{CFG} liefert.
	Dazu zeigen wir dass es für jedes Wort dass in der alten \ac{CFG} abgeleitet werden konnte auch in der neuen \ac{CFG} abgeleitet werden kann und umgekehrt.
	Für Schritt~\ref{alg:unit-step-chain} können wir -- es macht den Beweis etwas einfacher -- zeigen dass Äquivalenz sogar für jede Iteration der mittleren for-Schleife gilt.
	
	\item Zur Kettenregelfreiheit:
	
	In Schritt~\ref{alg:unit-step-DelCycle} wird mindestens ein Kettenregelzykel entfernt und da eine Grammatik nur endlich viele Regeln hat wird nach spätestens $|\mathcal{G}|$ Anwendungen von Schritt~\ref{alg:unit-step-DelCycle} der Schritt~\ref{alg:unit-step-dag}, in dem alle Zykeln entfernt sind, erreicht.
	
	In der mittleren for-Schleife von Schritt~\ref{alg:unit-step-chain} gilt aufgrund der topologischen Sortierung immer $i<k$.
	Die äußere Schleife hat die Invariante, dass am Ende jedes Durchlaufs für $n \ge i \ge j$ keine Kettenregel $A_i \to \ldots$ in $P'$ existiert.
	Beim Verlassen der Schleife ist $j = 1$ und es existieren überhaupt keine Kettenregeln mehr.
	
	\item Zu Laufzeit und Größe des Resultates:
	
		Vor Schritt~\ref{alg:unit-step-chain} wird die Grammatik nie vergrößert (wichtig sowohl für Laufzeit als auch für Größe!) und die Kosten übersteigen $O(|\mathcal{G}|^3)$ nicht.
		Die Anzahl der Schleifendurchläufe ist jeweils durch $|\mathcal{G}|$ beschränkt.
		Seien $B \to \alpha_1, \ldots, B \to \alpha_n$ die alle neuen Regeln die für eine Variable $B\in N$ in Schritt~\ref{alg:unit-step-chain} hinzugefügt werden.
		Dann gilt, dass die Summe der Größe der $\alpha_i$ die Größe der Grammatik $|\mathcal{G}|$ nicht übersteigt. 
		Somit ist die Größe des Resultates durch $|\mathcal{G}|^2$ beschränkt.
		\qedhere
	\end{itemize}
\end{proof}

\begin{figure}
\small
\framebox[\textwidth]{
\parbox{0.95\textwidth}{
\begin{description}
 \item[Eingabe:] \ac{CFG} $\mathcal{G} = (\Sigma, N, P, S)$
 \item[Ausgabe:] \ac{CFG} $\mathcal{G'} = (\Sigma, N', P', S)$ die Äquivalent zu $\mathcal{G}$ ist aber keine Kettenregeln enthält
\end{description}

  \begin{enumerate}
  \item Setze $N'=N$ und $P' = P$
  \item Betrachte den gerichteten Graphen $G$ dessen Knotenmenge $N$ und Kantenmenge $\{(A, B) \mid A \to B \in P' \}$ ist.
  (Es gibt also genau eine Kante pro Kettenregel.)
  \label{alg:unit-step-dg}
  \item \label{alg:unit-step-DelCycle} Suche, z.B. mittels Tiefensuche, einen Zyklus in $G$.
    Wenn kein Zyklus gefunden wurde, fahre mit Schritt \ref{alg:unit-step-dag} fort.
  \item Wurde ein Zyklus $A_1 \to A_2 \to \ldots \to A_k \to A_1$ gefunden, dann sei o.B.d.A. $A_1=S$ falls $S$ im Zyklus enthalten. Ersetze nun in $P'$ alle Vorkommen von $A_j$ mit $j > 1$ durch $A_1$ (auf linker \emph{und} rechter Regelseite).
    Entferne dann alle Regeln der Form $A \to A$.
    Fahre fort mit Schritt \ref{alg:unit-step-dg}.
  \item \label{alg:unit-step-dag}
    
    Sortiere die Knoten des nun azyklischen Graphen\footnotemark $G$ topologisch\footnotemark als $A_1, \ldots, A_n$, sodass $A_n$ auf keiner linken Seite einer Kettenregel vorkommt.
  \item $\mathbf{for}~j = n-1, \ldots, 1$ \label{alg:unit-step-chain}
    \begin{itemize}
    \item[] $\mathbf{for}~\mathbf{each}~A_j \to A_k \in P'$
      \begin{itemize}
                \item[] entferne $A_j \to A_k$ aus $P'$
                
                $\mathbf{for}~\mathbf{each}~A_k \to \alpha \in P'$
          \begin{itemize}
          \item[] füge $A_j \to \alpha$ zu $P'$
          \end{itemize}
      \end{itemize}
    \end{itemize}

  \end{enumerate}
  }
}

  \caption{Algorithmus \textsc{Unit}}
  \end{figure}
\footnotetext{\ac{DAG} (engl.\ \emph{directed acyclic graph}).}

\footnotetext{Topologies sortieren ist ein aus der Informatik 2 Vorlesung bekannter Begriff und bezeichnet das Erweitern einer gegebenen partiellen Ordnung zu einer totalen Ordnung.}

\begin{Bsp}
	Sei $\Sigma=\{0,1\}$. Betrachte und die \ac{CFG} $\mathcal{G} = (\Sigma, \{S,A,B,C,D,E\}, P, S)$
	deren Regeln $P$ wie folgt definiert sind.
	\begin{align*}
		S &\-> A\\
		A &\-> B \mid 0A1B \mid C\\
		B &\-> S \mid 1B0A \mid D\\
		C &\-> E\\
		D &\-> E\\
		E &\-> \varepsilon\\
	\end{align*}
	
	Nach der erstmaligen Anwendung von Schritt~\ref{alg:unit-step-dg} erhalten wir den folgenden gerichteten Graphen.
	\begin{center}
	\begin{tikzpicture}
		\node (s) at (-1,0) {S};
		\node (a) at (0,0.6) {A};
		\node (b) at (0,-0.6) {B};
		\node (c) at (1,0.6) {C};
		\node (d) at (1,-0.6) {D};
		\node (e) at (2,0) {E};
		\draw[->, bend left] (s) to (a);
		\draw[->, bend left] (a) to (b);
		\draw[->, bend left] (b) to (s);
		\draw[->] (a) to (c);
		\draw[->] (b) to (d);
		\draw[->] (c) to (e);
		\draw[->] (d) to (e);
	\end{tikzpicture}
	\end{center}
	In Schritt~\ref{alg:unit-step-dg} wählen wir den Zykel $S \-> A \-> B \-> S$ und erhalten für $P'$ die folgenden den Regeln.
	\begin{align*}
		S &\-> C\mid D\ \mid 0S1S \mid 1S0S\\
		C &\-> E\\
		D &\-> E\\
		E &\-> \varepsilon\\
	\end{align*}
	Nach dem zweiten Anwenden von Schritt~\ref{alg:unit-step-dg} erhalten wir den folgenden gerichteten Graphen und wählen in Schritt~\ref{alg:unit-step-dag} die durch die Zahlen angedeutete topologische Sortierung.
	\begin{center}
	\begin{tikzpicture}
		\node (s) at (0,0) {$^1$ S};
		\node (c) at (2,0.6) {$^2$ C};
		\node (d) at (2,-0.6) {$^3$ D};
		\node (e) at (4,0) {$^4$ E};
		\draw[->] (s) to (c);
		\draw[->] (s) to (d);
		\draw[->] (c) to (e);
		\draw[->] (d) to (e);
	\end{tikzpicture}
	\end{center}
	Am Ende von Schritt~\ref{alg:unit-step-chain} sieht die Regelmenge $P'$ wie folgt aus\footnote{
		Das Beispiel zeigt dass das Resultat des Algorithmus völlig unnötige Regeln enthalten kann.
	}.
	\begin{align*}
		S &\-> \varepsilon \mid 0S1S \mid 1S0S\\
		C &\-> \varepsilon\\
		D &\-> \varepsilon\\
		E &\-> \varepsilon\\
	\end{align*}

\end{Bsp}



\begin{Def}[name={[\acs*{CFG} in \acs*{CNF}]}]
	Eine \ac{CFG} $\mathcal{G} = (\Sigma, N, P, S)$ ist in \ac{CNF}, falls jede
  Regel die Form $A\->a$, $A\->BC$, oder $S \-> \Eps$ hat, wobei $A,B,C\in N$, $a\in\Sigma$.
  Falls $S \to \Eps \in P$, dann darf $S$ auf keiner rechten Seite einer Regel vorkommen.
\end{Def}
\begin{Satz}\label{satz:CFGtoCNF}
Es gibt einen Algorithmus der für eine \ac{CFG} $\mathcal{G}$ in $O(|\mathcal{G}|^3)$ Rechenschritten eine äquivalente \ac{CFG} $\mathcal{G}'$ in CNF mit $|\mathcal{G}'|\in O(|\mathcal{G}|^2)$ berechnet.
\end{Satz}

\begin{proof} (Beweisskizze)
	Ein Algorithmus der diese Eigenschaften erfüllt ist das Hintereinanderschalten von \textsc{Sep}, \textsc{Bin}, \textsc{Del} und \textsc{Unit}.\footnote{Es genügt sogar nur \textsc{Del} und dann \textsc{Unit} anzuwenden, da \textsc{Del} schon \textsc{Sep} und \textsc{Bin} ausführt.}
		\begin{itemize}
	\item Zur Korrektheit:
	
	Wir zeigen dass jedes dieser vier Verfahren die vom vorherigen Verfahren etablierte Eigenschaft nicht zerstört und dass die vier etablierten Eigenschaften für die \ac{CNF} hinreichend sind.
	
	\item Zu Laufzeit und Größe des Resultates:
	
	Die ersten drei Verfahren erhöhen die Größe der Grammatik nur linear und die Laufzeiten sind durch $O(|\mathcal{G}|^3)$ beschränkt. 
	Für \textsc{Unit} sind die in diesem Satz genannten Kosten eine obere Schranke. 
	\qedhere
	\end{itemize}
\end{proof}
	

\begin{lemma}
 Die Menge der Typ-2-Sprachen ist eine Teilmenge der Typ-1-Sprachen.
\end{lemma}
\begin{proof}
 Zu jeder Typ-2-Grammatik existiert eine äquivalente $\Eps$-freie Typ-2-Grammatik.
 Diese ist nach Definition auch eine Typ-1-Grammatik.
\end{proof}



% \begin{Beobachtung}
%   Ist $\mathcal{G} = (\Sigma, N, P, S)$ in \ac{CNF}, dann lassen sich für die Ableitungsbäume $\T \in \operatorname{Abl}(S)$ folgende Eigenschaften feststellen:
%   \begin{enumerate}
%   \item $\T$ ist ein Binärbaum. 
%   \item Falls $Y(\T) = w$, so ist die Anzahl der Blätter des Baumes $|w|$.
%   \qedhere
%   \end{enumerate}
% \end{Beobachtung}



\newcommand{\expl}[1]{{\color{blue} #1}}

\subsection{Wortproblem und Leerheitsproblem für kontextfreie Sprachen}
\begin{Satz}[name={[Wortproblem für \acs*{CFL} entscheidbar]}]
	Es gibt einen Algorithmus, für gegebene \ac{CFG} $\mathcal{G}$ in \ac{CNF} und gegebenes Wort $w$
	in $O(|w|^3\cdot |\mathcal{G}|)$ Schritten und mit $O(|w|^2\cdot |\mathcal{G}|)$ Speicherplatz
	entscheidet ob $w\in L(\mathcal{G})$ gilt.
\end{Satz}



\begin{proof}[\acsu{CYK}-Algorithmus] Sei $w = a_1\dots a_n\in\Sigma^*$.
	% Der \ac{CYK}-Algorithmus berechnet bei Eingabe von $\mathcal{G}$ und $w$, ob $w\in L(\mathcal{G})$.
	Der \ac{CYK}-Algorithmus aus \autoref{fig:AlgCYK} erfüllt diese Eigenschaften.
	Wir wollten zunächst die Idee des Algorithmus erklären.
	
	Definiere für $1\leq i < j\leq n$:
	\[
	M_{ij} = \{ A \in N \mid A \stackrel{*}{\vdash}_\mathcal{G} a_i\dots a_j \}\label{eqn:CykMatrixEintragDeriv}\tag{EntryDeriv}
	\]
	\expl{Wir definieren also Mengen von Nichtterminalsymbolen die sich schön als obere Dreiecksmatrix darstellen lassen.
		Die Matrix enthält einen Eintrag pro Teilwort von $w$, genauer der Eintrag $M_{ij}$ enthält das Nichtterminalsymbol $A$ 
		genau dann wenn wir von $A$ das Teilwort vom $i$-ten Zeichen bis zum $j$-ten Zeichen ableiten können.
		Zum Lösen des eigentlichen Problemes hilft dieses Vorgehen zunächst nichts.
		Urspünglich wollte wir wissen ob wir das Wort $W$ aus de Startsymbol ableiten können.
		Nun müssen wir diese Frage für alle Teilwörter und alle Nichtterminalsymbole beantworten.
	}
	
	Offensichtlich gilt: $w\in L(\mathcal{G})$ genau dann, wenn $S \stackrel{*}{\vdash}_\mathcal{G} w$ genau dann, wenn $S \in M_{1n}$.
	\expl{
		Zum Lösen des eigentlichen Problemes hilft uns das betrachten der Matrix also zunächst nicht.
		Urspünglich wollte wir wissen ob wir das Wort $W$ aus de Startsymbol ableiten können.
		Nun müssen wir diese Frage für alle Teilwörter und alle Nichtterminalsymbole beantworten.
	}
	\begin{align*}
		\intertext{Es gilt für $i=j$}
		M_{ii} &= \{A \mid A\-> a_i \in P\} \\
		\intertext{
			\expl{Die Einträge auf der Hauptdiagonalen hängen also nur von Regeln der Form $A \to a$ 
				ab und wir können diese Einträge jeweils durch einmaliges Iterieren über alle Regeln bestimmen.}
		}
		\intertext{Es gilt für $i<j$}
		M_{ij} &= \{ A \mid A\-> BC \in P \text{ und } BC \stackrel{*}{\vdash}_{\mathcal{G}} a_i\dots a_j \}\\
		&= \{ A \mid A\-> BC\in P \text{ und } \exists k \in\{i, \ldots j-1\} \text{ sodass }
		\begin{aligned}[t]
			B &\stackrel{*}{\vdash}_{\mathcal{G}} a_i\dots a_k \text{ und } C \stackrel{*}{\vdash}_{\mathcal{G}} a_{k+1}\dots a_j \}
		\end{aligned}\\
		&= \{ A \mid A\-> BC\in P \text{ und } \exists k \in\{i, \ldots j-1\} \text{ sodass } B\in M_{ik} \text{ und } C \in M_{(k+1)j}\}
		% &= \bigcup_{k=i}^{j-1} \{ A \mid A \-> BC \land 
		% 	\begin{aligned}[t]
		% 		B &\xRightarrow{*} a_i\dots a_k \land{}\\
		% 		C &\xRightarrow{*} a_{k+1}\dots a_j \}
		% 	\end{aligned}\\
		% &= \bigcup_{k=i}^{j-1} \{ A \mid A \-> BC \land B \in M_{i,k} \land C \in M_{k+1,j} \}
	\end{align*}
	
	\expl{
		Wir können also die Berechnung der Matrixeinträge die nicht auf der Hauptdiagonalen liegen 
		auf die Existenz bestimmter Regel der Form $A\-> BC$ und Inhalte bestimmter anderer Matrixeinträge zurückführen.
	}
	
	Beobachtung: Der Wert von $M_{ij}$ hängt nur von $M_{i'j'}$ mit $i'>i, j'\leq j$ oder $i'\geq i,j'<j$ ab.
	Wir können die $M_{ij}$ also mit dem Verfahren aus \autoref{fig:AlgCYK} berechnen.
	
	\expl{
		Es gibt also insbesonere keine zyklischen Abhängigkeiten und bei geeigneter Iterationsreihenfolge 
		über alle Matrixeinträge kann jeder Eintrag direkt mit Hilfe der Einträge aus vorherigen
		Iterationen bestimmt werden.
	}
	% \medskip
	
	Skizze der verbleibenden Beweisschritte.
	
	Zur Korrektheit: Zeige via einer verschachtelten Induktion über Zeilen und Spalten, 
	dass am Ende der zweiten for-Schleife für alle $M_{i'j'}$ mit $i'\in\{i,\ldots, n-1\}$ und $j'\in\{i+1,\ldots, j\}$
	die oben definierte Gleichung (\ref{eqn:CykMatrixEintragDeriv}) gilt.
	
	Zur Laufzeitabschätzung: Die Iterationen jeder for-Schleife sind durch die Länge des Wortes $|w|$ beschänkt. 
	In der innersten Schleife genügt es ein mal über die Regeln der Grammatik zu iterieren.
	
	Zum Speicherplatz: Wir benötigen weniger als $|w|^2$ Matrixeinträge 
	und jeder Eintrag enthält höchtens alle Variablen die in Regeln der Grammatik vorkommen.
	\qedhere
\end{proof}


\begin{figure}
	\small
	% \framebox[\textwidth]{
	% \parbox{0.95\textwidth}{
	\begin{description}
		\item[Eingaben:] \ac{CFG} $\mathcal{G} = (\Sigma, N, P, S)$ in CNF und Wort $a_1\ldots a_n\in\Sigma^*$
		\item[Ausgabe:] ``Ja'' falls $a_1\ldots a_n\in L(\mathcal{G})$, ``Nein'' sonst.
	\end{description}
	
	Sei $M$  eine $n\x n\text{-Matrix mit (initial) } M_{ij}=\varnothing \text{ für alle } i, j$  // $O(n^2)$
	
	\begin{center}
		\begin{minipage}{12cm}
			\begin{lstlisting}[mathescape,morekeywords={for,do,return},morecomment={[l]{//}}]
for i=1 .. n do  // $O(|\mathcal{G}|)\cdot O(n)$
  $M_{ii}=\{A \mid A \-> a_i \}$
for i=n-1 .. 1 do
  for j=i+1 .. n do
    for k=i .. j-1 do  // $O(|\mathcal{G}|)\cdot O(n^3)$
      $M_{ij}=M_{ij}\cup \{ A \mid A \-> BC, B\in M_{ik}, C\in M_{(k+1)j} \}$
return $S\in M_{1n}$
			\end{lstlisting}
			\qedherefixlstlisting
		\end{minipage}
	\end{center}
	%   }
	% }
	
	\caption{CYK Algorithmus}
	\label{fig:AlgCYK}
\end{figure}


\begin{Bsp}
	Sei $\Sigma=\{a,b,c\}$.\
	Betrachte das Wort $w=aaabbbcc$ und die \ac{CFG} $\mathcal{G} = (\Sigma, \{S,X,Y\}, P, S)$\footnote{
		Es gilt $L(\mathcal{G})=\{a^nb^nc^m \mid n,m\geq 1\}$ 
		aber das bestimmen der erzeugten Sprache ist für das Entscheiden des Wortproblemes 
		natürlich nicht erforderlich.}
	deren Regeln $P$ wie folgt definiert sind.
	\begin{align*}
		S &\-> XY\\
		X &\-> ab \mid aXb\\
		Y &\-> c \mid cY
	\end{align*}
	Damit wir den \ac{CYK} Algorithmus anwenden können transformieren wir $\mathcal{G}$ in eine äquivalente Grammatik in CNF.
	Z.B. in die \ac{CFG} $\mathcal{G'} = (\Sigma, \{S, X, Y, Z, A, B, C\}, P', S)$, deren Regeln $P'$ wie folgt definiert ist.
	\begin{align*}
		S &\-> XY\\
		X &\-> AB \mid AZ\\
		Z &\-> XB\\
		Y &\-> CY | c\\
		A &\-> a, B \-> b, C \-> c
	\end{align*}
	Wir wenden den CYK Algorithmus an und erhalten die folgende Matrix.
	

	
	\begin{minipage}{0.6\textwidth}
	\begin{center}
		\begin{tabular}[t]{M{c}|*8{M{c}|}}
			%       w = & a & a & a & b & b & b & c & c \\
			\cline{2-9}
			% 			M =
			& A               & \bullet & \bullet & \bullet & \bullet & X ~\text{\tiny 6}& \emph{S} ~\text{\tiny 7}& S~\text{\tiny 8}       \\
			\cline{2-9}
			\multicolumn{1}{r@{}}{a} & & A       & \bullet & \bullet & {X} ~\text{\tiny 4}& {Z} ~\text{\tiny 5}& \bullet & \bullet \\
			\cline{3-9}
			\multicolumn{1}{c}{} & \multicolumn{1}{r@{}}{a} &    & A       & X ~\text{\tiny 2}      & Z     ~\text{\tiny 3}  & \bullet & \bullet & \bullet \\
			\cline{4-9}
			\multicolumn{2}{c}{} & \multicolumn{1}{r@{}}{a} &         & B       & \bullet & \bullet & \bullet & \bullet \\
			\cline{5-9}
			\multicolumn{3}{c}{} & \multicolumn{1}{r@{}}{b} &              & B       & \bullet & \bullet & \bullet \\
			\cline{6-9}
			\multicolumn{4}{c}{} & \multicolumn{1}{r@{}}{b} &                   & B & \bullet & \bullet \\
			\cline{7-9}
			\multicolumn{5}{c}{} & \multicolumn{1}{r@{}}{b} &                        & C,Y     & Y ~\text{\tiny 1}      \\
			\cline{8-9}
			\multicolumn{6}{c}{} & \multicolumn{1}{r@{}}{c} &                              & C,Y     \\
			\cline{9-9}
			\multicolumn{7}{c}{} & \multicolumn{1}{r@{}}{c} & \multicolumn{1}{c}{}
		\end{tabular}
	\end{center}
	\end{minipage}
	\begin{minipage}{0.5\textwidth}
	\end{minipage}
	\begin{minipage}{0.35\textwidth}
		Zellen, die mit "`${\bullet}$"' markiert sind, sind leer.
		Die Reihenfolge, in der die nicht-diagonalen, nicht leeren Zellen eingetragen wurden, ist mit kleinen Zahlen markiert.
		Beispielsweise wurde $M_{35} = \{Z\}$ mit Zahl "`{\scriptsize 3}"' \emph{nach} $M_{78} = \{Y\}$ mit Zahl "`{\scriptsize 1}"' eingetragen.
	\end{minipage}

	Da $S \in M_{1n}$, gilt $w \in L(\mathcal{G}')$.
\end{Bsp}


\begin{Korollar}
	Es gibt einen Algorithmus, für gegebene \ac{CFG} $\mathcal{G}$ und gegebenes Wort $w$
	in $O(|w|^3\cdot |\mathcal{G}|^2)$ Schritten und mit $O(|w|^2\cdot |\mathcal{G}|^2)$ Speicherplatz
	entscheidet ob $w\in L(\mathcal{G})$ gilt.
\end{Korollar}

\begin{proof}
	Aus \autoref{satz:CFGtoCNF} wissen wir dass eine \ac{CFG} in \ac{CNF} 
	konvertieren können sodass deren Größe höchstens quadratisch wächst.
	Anschließend wenden wir den CYK Algorithmus auf das Resultat an.
\end{proof}

\expl{
Um \datenote{12.12.2018} möglichst schnell zum Unterkapitel über Typ-3-Sprachen zu kommen wurde das restliche Unterkapitel in der Freitagsvorlesung übersprungen.
}

\begin{Satz}[name={[Entscheidbarkeit des Leerheitsproblems für kontextfreie Sprachen]}]
  \label{thm:cfl-decidable-emptyness}
    Es gibt einen Algorithmus, der das Leerheitsproblem%
    \footnote{Gefragt ist, ob $L(\mathcal{G}) = \{ w\in\Sigma^* \mid S\overset{*}{\vdash} w \} \overset{?}{=} \varnothing$.}
    für kontextfreie Grammatiken in $O(|\mathcal{G}|^2)$ entscheidet.
\end{Satz}

\begin{Def}
	Sei $\mathcal{G} = (\Sigma, N, P, S)$ eine beliebige Grammatik. 
	Wir nennen ein Nichtterminalsymbol $A\in N$ co-erreichbar falls es ein Wort $w\in\Sigma$ existiert, 
	sodass $A \stackrel{*}{\vdash}_\mathcal{G} w$ gilt.
	\expl{Ein Symbol ist also co-erreichbar genau dann wenn wir daraus irgendein Wort ableiten können.}
\end{Def}


\begin{proof}
    Sei $\mathcal{G} = (\Sigma, N, P, S)$ eine \ac{CFG} für.
    Die $M$ die Menge der co-erreichbaren Nichtterminalsymbole.
    %
    $$M = \{ A \mid A \stackrel{*}{\vdash}_\mathcal{G} w, w\in \Sigma^* \}$$
    %
    Offensichtlich gilt: $L =\varnothing \<=> S\notin M$.
    
    Wir können $M$ induktiv berechnen. Wir definieren hierfür:
	\begin{align*}
		M_0 &= \{ A \mid A \-> w\in P, w\in \Sigma^* \}\\
		M_{i+1} &= M_i\cup \{ A \mid A \-> \alpha\in P, \alpha\in(\Sigma\cup M_i)^* \}
	\end{align*}
    Offensichtlich gilt für alle $i\in\N$ $M_i\subseteq M_{i+}$ und $M_i\subseteq M$.
    Da $M\subseteq N$ endlich ist, gibt es ein $n$, sodass $M_n = M_{n+1}$ (hier ohne Beweis).
    
    Wir können daher $M$ analog zu $\operatorname{Nullable}(\mathcal{G})$ (siehe \autoref{satz:3.nullable}) in $O(|\mathcal{G}|)$ berechnen.
\end{proof}
\begin{Bem}
        Wenn wir in einer Grammatik $\mathcal{G}$ ein nicht co-erreichbares Nichtterminalsymbol (und alle zugehörigen Regeln) entfernen, ändert sich $L(\mathcal{G})$ nicht.
        Wir können eine gegebene Grammatik also optimieren, indem wir alle nicht co-erreichbaren Nichtterminalsymbole entfernen.
\end{Bem}
% \begin{Satz}[name={[Entscheidbarkeit des Endlichkeitsproblem für \acs*{CFL}]}]
% 	Das Endlichkeitsproblem für kontextfreie Sprachen ist entscheidbar.
% \end{Satz}
% \begin{proof}
% 	Mit \ac{PL} analog zum Endlichkeitsproblem für reguläre Sprachen (Prüfe, ob $\exists w \in L$ mit $n < |w| \le 2n$).
% 	Die Grundidee ist: $L$ ist endlich genau dann, wenn es keinen (echt wachsenden) Zyklus gibt, der zu einer Wortbildung beitragen kann.
% \end{proof}


\subsection{Einschub: Typ-3-Sprachen}
\begin{Satz}[name={[Typ-3-Sprache ist regulär]}]
	$L$ ist regulär \<=> $L$ ist Typ-3-Sprache.
\end{Satz}
\begin{proof}
	\begin{description}[font=\normalfont,labelwidth=\widthof{"'\=>"':},leftmargin=!]
	\item["`\=>"'] Sei $\A =(\Sigma, Q,\delta,\qinit,F)$ \ac{DEA} für $L$.\\
		Konstruiere Typ-3-Grammatik $\mathcal{G}=(\Sigma,N,P,S)$ mit
		\begin{align*}
		 N &= Q\\
		 S &= \qinit\\
		 P &= \{ q\rightarrow aq'\mid q,q'\in Q, a\in\Sigma, \delta(q,a)=q'\}\cup \{ q\rightarrow\Eps\mid q\in F\}
		\end{align*}
		Es gilt $L(\mathcal{G})=L(\A)$. (ohne Beweis)
	\item["`\<="'] Sei $\mathcal{G}=(\Sigma,N,P,S)$ Typ-3-Grammatik.
	Dann erhalten wir durch Anwenden von \textsc{Bin} eine äquivalente Grammatik, in der jede Regel eine der folgenden Formen hat:
	  \begin{alignat*}{3}
	 A &\-> \Eps && \quad \text{für } A\in N\\[0.5mm]
   A &\-> a && \quad \text{für } A\in N, a\in\Sigma\\[0.5mm]
   A &\-> a_1a_2 && \quad \text{für } A\in N, a_1,a_2\in\Sigma\\[0.5mm]
   A &\-> aB && \quad \text{für } A\in N, a\in\Sigma, B\in N
  \end{alignat*}
  Wir machen nun die folgenden Modifikationen.
  \begin{itemize}
   \item Wir fügen eine frische Variable $Y_{\Eps}$ und die Regel $Y_{\Eps}\rightarrow\Eps$ hinzu.
   \item Wir löschen jede Regel der zweiten Form und führen stattdessen die Regel $A \-> aY_{\Eps}$ ein.
  \item Wir löschen jede Regel der dritten Form und führen stattdessen eine frische Variable $Y_{a_2}$ und die Regeln $A \-> a_1Y_{a_2}$ und $Y_{a_2} \-> a_2Y_\Eps$ ein.
  \end{itemize}
	Die resultierende Grammatik $\mathcal{G}=(\Sigma,N',P',S)$ ist äquivalent zu $\mathcal{G}$. (Ohne Beweis)

	Wir konstruieren nun den folgenden NEA. $\calN=(\Sigma,Q,\delta,\qinit,F)$
	\begin{alignat*}{2}
		Q &=N'\\
		B\in\delta(A,a) & \text{ \  gdw \ } A\rightarrow aB \in P'\\
		\qinit &=S\\
		F &= \{ A\in N'\mid A\rightarrow\Eps\in P'\}
	\end{alignat*}
		Es gilt $L(\mathcal{G})=L(\A)$. (ohne Beweis) \qedhere

	\end{description}
\end{proof}


\begin{lemma}
	$\chr \subsetneq \cht$ (d.h., die regulären Sprachen sind eine \emph{echte} Teilmenge der kontextfreien Sprachen).
\end{lemma}
\begin{proof}
	Betrachte $L_\text{centered}=\{0^n10^n \mid n\in\N \}\subseteq\{0,1\}^*$\\
	Sowohl aus \autoref{bsp:2.centeredMyhillNerode} als auch aus \autoref{bsp:2.centeredPL} ist bekannt, 
	dass $L_\text{centered}$ nicht regulär ist.
	Es gibt aber eine Typ-2-Grammatik für $L_\text{centered}$:
	\[ \mathcal{G} = (\{0,1\}, \{S\}, \{S\->1, S\->0S0 \}, S) \]
	(ohne Beweis)
\end{proof}





\hide{


\subsection{Abschlusseigenschaften für kontextfreie Sprachen}


\begin{Satz} % 4.7
  \label{thm:cfl-closed-reg-intersect}
	Die Menge \acs{CFL} ist abgeschlossen unter $\cup, \cdot, ^*$, jedoch \emph{nicht} unter $\cap, \overline{\phantom{A}}$.
\end{Satz}
\begin{proof}
	\begin{align*}
		\mathcal{G}_i :&= (N_i, \Sigma,P_i,S_i) \quad i=1,2\\
		\text{"`}\cup\text{"'}: N &= N_1\dotcup N_2\dotcup\{S\}\\
		P &= \{S\->S_1, S\->S_2 \} \cup P_1\cup P_2\\
		\text{"`}\cdot\text{"'}:: N &= N_1\dotcup N_2\dotcup\{S\}\\
		P &= \{ S\->S_1S_2 \} \cup P_1\cup P_2\\
		\text{"`}^*\text{"'}: N &= N_1 \dotcup \{S\}\\
		P &= \{ S\->\Eps, S\-> S_1S \} \dotcup P_1
	\end{align*}
	\begin{itemize}
	\item für $n\geq 1$
		\[ \underbrace{\{a^nb^nc^n\}}_{\notin \acs{CFL}} = \underbrace{\{a^nb^nc^m \mid n,m\geq 1 \}}_{\in \acs{CFL}} \cap \underbrace{\{a^mb^nc^n \mid m,n\geq 1 \}}_{\in \acs{CFL}} \]
		Also \ac{CFL} nicht abgeschlossen unter $\cap$.
	\item Angenommen CFL wäre abgeschlossen unter $\overline{\phantom{X}}$\\
		Falls $L_1,L_2\in \acs*{CFL}$, dann ist $\overline{L}_1,\overline{L}_2\in\acs*{CFL}$ nach Annahme.\\
		$\curvearrowright \overline{L}_1\cup \overline{L}_2\in\acs*{CFL}$ wegen Teil "'$\cup$"'.\\
		$\curvearrowright \overline{\overline{L}_1\cup \overline{L}_2} = L_1\cap L_2\in\acs*{CFL}\ \lightning$ zu Teil "'$\cap$"'. \qedhere
	\end{itemize}
\end{proof}








\datenote{14.12.16}
\begin{Satz} % 4.8
	Die Menge \ac{CFL} ist abgeschlossen unter "`$\cap$ mit regulären Sprachen REG"'.
  Das heißt, für alle $L \in \mathrm{CFL}$ und $R \in \mathrm{REG}$ gilt $L \cap R \in \mathrm{CFL}$.
\end{Satz}
\begin{proof}
  Sei $\mathcal{G} = (\Sigma, N, P, S)$ kontextfreie Grammatik in \ac{CNF} mit $L(\mathcal{G}) = L$.

  Sei $M = (Q, \Sigma, q, \delta, q_0, F)$ NEA mit $L(M) = R$.

  Definiere $\mathcal{G}' = (N', \Sigma, P', S)$ durch
  \begin{itemize}
  \item[] $N' = Q \times N \times Q \cup \{S\}$
  \item[] $P' =
    \begin{aligned}[t]
      &\{ (p, A, q) \to a \mid A \to a \in P \text{ und } \delta(p,a) \ni q \} \\
      &\cup \{ (p, A, q) \to (p,B,q')(q',C,q) \mid A \to BC \in P \text{ und } p,q,q' \in Q\} \\
      &\cup \{ S \to (q_0, S, q) \mid q \in F \}
    \end{aligned}
    $
  \end{itemize}
  Durch die $S \to \ldots$ Regeln erzeugt $\mathcal{G}'$ offensichtlich genau die Wörter, die durch die Nichtterminalsymbole $(q_0, S, q)$ für $q \in F$ erzeugt werden.
  Es bleibt zu zeigen, dass ein Nichtterminal genau die Wörter aus $L$ erzeugt die gleichzeitig einen (akzeptierenden) Lauf $q_0\ldots q$ von $M$ erlauben.
  Wir zeigen eine Verallgemeinerung: für alle $p,q \in Q$ und $A \in N$ gilt
  \begin{align*}
    &(p,A,q) \stackrel{*}{\Longrightarrow}_{\mathcal{G}'} w \\
    \text{ gdw } & A \stackrel{*}{\Longrightarrow}_{\mathcal{G}} w \text{ und es existiert ein Lauf } p\ldots q \text{ von $M$ auf $w$}
  \end{align*}
  \begin{itemize}
  \item Richtung "`links nach rechts"': (siehe Übung)
  \item Richtung "`rechts nach links"':

    Per Induktion über den Ableitungsbaum $\mathcal{A} = \pi(\mathcal{A}_1,\ldots,\mathcal{A}_n) \in \operatorname{Abl}(\mathcal{G}, A)$ mit $Y(\mathcal{A}) = w$.
    \begin{description}
      \item[IV] $\forall 1 \le i \le n:$ wenn $w_i = Y(\mathcal{A}_i)$ mit $\mathcal{A}_i \in \operatorname{Abl}(\mathcal{G}, A_i)$ und es existiert ein Lauf  $p_i\ldots q_i$  von M auf $w_i$, dann $(p_i, A_i, q_i) \stackrel{*}{\Longrightarrow}_{\mathcal{G}'} w_i$.
      \item[IS] \hfill
        \begin{itemize}
        \item $\pi = A \to a$, $a \in \Sigma$.
          Es gilt $w = a$ und $pq$ ist Lauf auf $a$.
          Folglich ist $\delta(p, a) \ni q$. Damit ist $(p,A,q) \to a \in P'$, woraus direkt folgt, dass $(p,A,q) \Longrightarrow_{\mathcal{G}'} a$.
        \item $\pi = A \to BC$, $\mathcal{A} = \pi(\mathcal{A}_1, \mathcal{A}_2)$, $Y(\mathcal{A}_1) = w_1$, $Y(\mathcal{A}_2) = w_2$, \mbox{$\mathcal{A}_1 \in \operatorname{Abl}(\mathcal{G}, B)$} und $\mathcal{A}_2 \in \operatorname{Abl}(\mathcal{G}, C)$.

          Es gilt $w = w_1w_2$ und $\zeta = p\ldots q$ ist Lauf auf $w_1w_2$.
          Es existiert also $q'$ sodass $\zeta = p\ldots q' \ldots q$ und $p\ldots q'$ ist Lauf auf $w_1$ und $q' \ldots q$ ist Lauf auf $w_2$.

          Es folgt nach IV, dass
          \begin{displaymath}
            (p, B, q') \stackrel{*}{\Longrightarrow}_{\mathcal{G}'} Y(\mathcal{A}_1) = w_1 \text{ und } (q', C, q) \stackrel{*}{\Longrightarrow}_{\mathcal{G}'} Y(\mathcal{A}_2) = w_2.
          \end{displaymath}
          Nun ist $(p,A,q) \to (p,B,q')(q',C,q) \in P'$ nach Konstruktion, also ist folgende Ableitung möglich:
          \begin{displaymath}
            (p,A,q) \Longrightarrow_{\mathcal{G}'} (p,B,q')(q',C,q) \stackrel{*}{\Longrightarrow}_{\mathcal{G}'} w_1(q',C,q) \stackrel{*}{\Longrightarrow}_{\mathcal{G}'} w_1w_2 = w
          \end{displaymath}
        \end{itemize}
    \end{description}
  \end{itemize}
  \end{proof}
%
%Zuletzt: $\acsu{CFL}\land\acsu{REG}=\ac{REG}$
\begin{Satz}[name={[$L\subseteq R$ entscheidbar]}] %\rlwarning{Satz 4.9}
	Sei $L\in\ac{CFL}$ und $R\in\textrm{REG}$. Es ist entscheidbar, ob $L\subseteq R$.
\end{Satz}
\begin{proof}
  Es gilt $L\subseteq R \text{ gdw } L \cap\overline{R} = \emptyset$.

  Da $R \in \mathrm{REG}$ ist durch die Abschlusseigenschaften regulärer Sprachen auch $\overline{R} \in \mathrm{REG}$.
  Nach Satz \ref{thm:cfl-closed-reg-intersect} ist $L \cap\overline{R} \in \mathrm{CFL}$.
  Nach Satz \ref{thm:cfl-decidable-emptyness} ist $L \cap\overline{R} = \emptyset$ daher entscheidbar
\end{proof}


\draftnote{25.11.16}


\datenote{30.11.16}


\datenote{2.12.16}


\subsection{Das Pumping Lemma für kontextfreie Sprachen}
\begin{Satz}[Pumping lemma für \acs*{CFL}, uvwxy Lemma]
\label{satz:PL für CFL}
%\rlnote{Satz \# 4.2?}
	Sei $L\in {CFL}$. Dann $\exists n>0$, sodass $\forall z\in L$ mit $|z|\geq n\ \exists u,v,w,x,y$ sodass $z=uvwxy$ mit
	\begin{itemize}
	\item $|vwx|\leq n$
	\item $|vx|\geq 1$
	\item $\forall i \in \N : uv^iwx^iy\in L$
	\end{itemize}
\end{Satz}
\begin{proof}
	Sei $\mathcal{G}= (\Sigma,N,P,S)$ in \ac{CNF} mit
    $L(\mathcal{G}) = L$. Wähle die Pumping-Konstante $n=2^{|N|}$.
    
     Betrachte den Ableitungsbaum $\mathcal{A} \in \operatorname{Abl}(\mathcal{G}, S)$ von $z$ mit $|z|\geq n$. 
    \footnote{Intuition: Da $\mathcal{G}$ in \ac{CNF}, ist Ableitungsbaum $\mathcal{A}$ ein Binärbaum. 
        Mit $|N|$ 
    verschiedenen Nichtterminalsymbolen kann man 
    also maximal Wörter der Länge $2^{|N|}$ ableiten, wenn man keine Ableitung doppelt nutzen möchte. 
    Leitet man ein längeres Wort ab, so muss man mind. eine Ableitung doppelt nutzen.
     Dann kann man sie aber auch gleich $i$-fach nutzen ($v^i$ und $x^i$), und ist immernoch in der Sprache.}
  $\mathcal{A}$ ist ein Binärbaum mit $|z|\geq n = 2^{|N|}$ Blättern. In einem solchen Binärbaum existiert ein Pfad $\zeta$ der Länge $\geq |N|$.
	Auf $\zeta$ liegen $\geq |N|+1$ \acfp{NT}, es muss also mindestens ein Nichtterminal $A$ mehrmals vorkommen.
	
  Folge $\zeta$ vom Blatt Richtung Wurzel und bis sich das ein $A$ das erste Mal wiederholt. Das geschieht nach $\le |N|$ Schritten.
  Nun teile $z = uvwxy$ wie hier skizziert:
  
			\begin{tikzpicture}[>=stealth,
				widebox/.style={draw, minimum width=.75cm, minimum height=.35cm, inner sep=0pt}
				,dot/.style={inner sep=0pt,outer sep=0pt,label={center:\scalebox{.75}{\textbullet}}}
				]
				
				\node[dot,label={above right:$S$}] (v1) at (0,-1) {};
				\node[dot] (v2) at (-0.25,-1.6) {};
				\node[dot,] (v3) at (0,-2) {};
				\node[dot] (v4) at (-0.5,-2.5) {};
				\node[dot,label={above right:$A$}] (v5) at (0,-3) {};
				\node[widebox] (v6) at (-1.5,-3.5) {$u$};
				\node[widebox] (v7) at (-.75,-3.5) {$v$};
				\node[widebox] (v8) at (0,-3.5) {$w$};
				\node[widebox] (v9) at (.75,-3.5) {$x$};
				\node[widebox] (v10) at (1.5,-3.5) {$y$};
				
				\draw  (v1) edge (v2);
				\draw  (v2) edge (v3);
				\draw  (v3) edge (v4);
				\draw  (v4) edge (v7.north west);
				\draw  (v4) edge (v5);
				\draw  (v3) edge (v10.north west);
				\draw  (v5) edge (v8.north west);
				\draw  (v5) edge (v8.north east);
				\draw  (v1) edge (v6.north west);
				\draw  (v1) edge (v10.north east);
				\node[align=left, text height=3em,inner sep=2pt] at (2.1,-1.5) {$A\rightarrow BC$~\small{ist Binärbaum}} edge[->,shorten >=.1cm] (v3);
				\node[inner sep=0pt] at (2.75,-3.5) {$A\rightarrow a$} edge[->] (v10);
				\node (v11) at (0,-4) {$z$};
				\draw[semithick]
				let \p1 = (v6.west), \p2 = (v11), \p3 = (v10.east) in 
				 (\x1,\y2) edge[{Bar[]<}-] (v11) (v11) edge[-{>Bar[]}] (\x3,\y2);
			\end{tikzpicture}
	
      Nun gilt
      \begin{itemize}
      \item $|vx|\geq 1$, da $\zeta$ entweder durch $B$ oder durch $C$ läuft. Nehme also an, $\zeta$ verläuft durch $B$.
        Somit muss $C \stackrel{*}{\Longrightarrow} x$. In \ac{CNF} ist $C \stackrel{*}{\Longrightarrow} \Eps$ nicht möglich und somit ist $|x| \ge 1$.
        Der Fall, das $\zeta$ durch $C$ verläuft, ist analog.
      \item $|vwx| \le 2^{|N|} = n$ TODO
      \end{itemize}
		\begin{figure}[H]\centering
		\begin{subfigure}[b]{.29\linewidth}\centering
			\begin{tikzpicture}[node distance=.75cm, on grid,
					every node/.style={
						execute at begin node=$,
						execute at end node=$
					}
				]
				\node (0) {};
				\node (A1) [below=of 0] {A};
				\node (A2) [below=of A1] {A};
				\node (A3) [below=of A2] {A};
				
				\newlength\Offset \setlength\Offset{.75cm}
				\coordinate(p1) at ($(A2.south) - (0,\Offset)$);
				%
				\coordinate (A2l) at ($(p1) - (\Offset,0)$);
				\coordinate (A1l) at ($(A2l) - (\Offset,0)$);
				\coordinate (0l) at ($(A1l) - (\Offset,0)$);
				%
				\coordinate (A2r) at ($(p1) + (\Offset,0)$);
				\coordinate (A1r) at ($(A2r) + (\Offset,0)$);
				\coordinate (0r) at ($(A1r) + (\Offset,0)$) ;
				
				\coordinate (A3l) at ($(A3.south) - (\Offset,\Offset)$);
				\coordinate (A2l2) at ($(A3l) - (\Offset,0)$);
				%
				\coordinate (A3r) at ($(A3.south) + (\Offset,-\Offset)$);
				\coordinate (A2r2) at ($(A3r) + (\Offset,0)$);
				
				\node (dots) [below=\Offset of A3] {\vdots};
				
				\node (A4) [below=\Offset of dots] {A};
				\coordinate (A4l) at ($(A3l) - (0,2*\Offset)$);
				\coordinate (A4r) at ($(A3r) - (0,2*\Offset)$);
				
				\draw (0.south) edge (0l.center) edge (0r.center)
					(A1.south) edge (A1l.center) edge (A1r.center)
					(A2.south) edge (A2l2.center) edge (A2r2.center)
					(A3.south) edge (A3l.center) edge (A3r.center)
					(A4.south) edge (A4l.center) edge (A4r.center)
				;
				\draw (0l.center) edge node[below] {u} (A1l.center)
					(A1l.center)  edge node[below] {v} (A2l.center)
					(A2r.center)  edge node[below] {x} (A1r.center)
					(A1r.center)  edge node[below] {y} (0r.center)
					(A2l2.center) edge node[below] {v} (A3l.center)
					(A3r.center)  edge node[below] {x} (A2r2.center)
					(A4l.center)  edge node[below] {w} (A4r.center)
				;
			\end{tikzpicture}
			\caption{$uv^iwx^iy\in L$}
		\end{subfigure}
		\begin{subfigure}[b]{.28\linewidth}\centering
			\begin{tikzpicture}[node distance=.5cm, on grid,
					every node/.style={
						execute at begin node=$,
						execute at end node=$
					},
					widebox/.style={draw, minimum width=.75cm, minimum height=.35cm, inner sep=0pt}
				]
				\node (S) {S};
				\node[inner sep=2pt] (A) [below=of S.south, xshift=.2cm] {A};
				\coordinate (Ap) at (A.south west);
				\node[widebox] (w) [below=of Ap] {w};
				\node[widebox] (u) at ($(Ap)!2!(w.north west)+.5*(-.75,-.3)$) {u};
				\node[widebox] (y) at ($(Ap)!2!(w.north east)+.5*(.75,-.3)$) {y};
				
				\draw (S.south) edge (u.north west) edge (y.north east)
					(Ap) edge (u.north east) edge (y.north west) edge (w.north)
					($(S.south) - (.25,.6)$) edge (S.south) edge (Ap)
				;
			\end{tikzpicture}
			\caption{$\curvearrowright uwy\in L$}
		\end{subfigure}
		\begin{subfigure}[b]{.33\linewidth}\centering
			\begin{tikzpicture}[>=stealth,
				widebox/.style={draw, minimum width=.75cm, minimum height=.35cm, inner sep=0pt}
				,dot/.style={inner sep=0pt,outer sep=0pt,label={center:\scalebox{.75}{\textbullet}}}
				]
				
				\node[dot,label={above right:$S$}] (v1) at (0,-1) {};
				\node[dot] (v2) at (-0.25,-1.6) {};
				\node[dot,label={above right:$A$}] (v3) at (0,-2) {};
				\node[dot] (v4) at (-0.5,-2.5) {};
				\node[dot,label={above right:$A$}] (v5) at (0,-3) {};
				\node[widebox] (v6) at (-1.5,-3.5) {$u$};
				\node[widebox] (v7) at (-.75,-3.5) {$v$};
				\node[widebox] (v8) at (0,-3.5) {$w$};
				\node[widebox] (v9) at (.75,-3.5) {$x$};
				\node[widebox] (v10) at (1.5,-3.5) {$y$};
				
				\draw  (v1) edge (v2);
				\draw  (v2) edge (v3);
				\draw  (v3) edge (v4);
				\draw  (v4) edge (v7.north west);
				\draw  (v4) edge (v5);
				\draw  (v3) edge (v10.north west);
				\draw  (v5) edge (v8.north west);
				\draw  (v5) edge (v8.north east);
				\draw  (v1) edge (v6.north west);
				\draw  (v1) edge (v10.north east);
				\node[align=left, text height=3em,inner sep=2pt] at (2.1,-1.5) {$A\rightarrow BC$\small{ist Binärbaum}} edge[->,shorten >=.1cm] (v3);
				\node[inner sep=0pt] at (2.75,-3.5) {$A\rightarrow a$} edge[->] (v10);
				\node (v11) at (0,-4) {$z$};
				\draw[semithick]
				let \p1 = (v6.west), \p2 = (v11), \p3 = (v10.east) in 
				 (\x1,\y2) edge[{Bar[]<}-] (v11) (v11) edge[-{>Bar[]}] (\x3,\y2);
			\end{tikzpicture}
			\caption{$\exists$ Pfad mit Länge $\geq k$}
		\end{subfigure}
		\caption{Schema zu \autoref{satz:PL für CFL}}
	\end{figure}\vspace{-2em}\qedhere
\end{proof}

\begin{lemma} %\rlnote{Lemma \# 4.3?}
	$\mathcal{L}_2 \subsetneq \mathcal{L}_1$
\end{lemma}
\begin{proof}
	Sei $L=\{L=\{a^nb^nc^n \mid n\geq 1\}$.\\
	$L$ ist nicht kontextfrei. Verwende \ac{PL}. Angenommen $L$ sei kontextfrei.
	Sei dann $n$ die Konstante aus dem \ac{PL}.
	
	Wähle $z=a^nb^nc^n$ und somit $|z| = 3n \ge n$.
  Nach PL ist $z = uvwxy$ mit $|vx| \ge 1$ und $|vwx| \le n$.

  Durch $|vwx| \le n$ ergeben sich folgende Möglichkeiten:
  \begin{itemize}
  \item $vwx = a^j$: 
    Für $i=0$ ist $vwx=w$ mit $|w| < j$.
    Anzahl der $a$ stimmt nicht mehr mit $n$ überein.
  \item $vwx = a^kb^j$:
    \begin{itemize}
    \item Falls $v = a^{k'}$, $x = b^{j'}$: Für $i=0$ stimmt die Anzahl der $a$ nicht mehr mit $n$ überein.
    \item Falls $v$ ein Gemisch aus $a$ und $b$ enthält.
      Für $i\ge 2$ würde eine Folge $a^n$ durch eine $b$-Folge unterbrochen.
    \item Falls $x$ ein Gemisch aus $a$ und $b$ enthält.
      Für $i \ge2$ würde eine Folge $a^n$ durch eine $b$-Folge unterbrochen.
    \end{itemize}
  \item $vwx = b^j$: analog Fall $a^j$
  \item $vwx = b^kc^j$: analog Fall $a^kb^j$
  \item $vwx = c^j$: analog Fall $a^j$
  \end{itemize}
  In jeder der Möglichkeiten lässt sich durch Pumpen ein Wort $w \not \in L$ finden.
  Daher kann $L$ nicht kontextfrei sein.
\end{proof}

\begin{Bsp} Die Sprache
  \begin{displaymath}
    L = \{ww \mid w \in \{a,b\}^*\}
  \end{displaymath}
  ist kontextsensitiv aber nicht kontextfrei.

  Sei $n$ die Konstante aus dem PL

  Betrachte $z = a^nb^na^nb^n \in L$ mit $|z| = 4n \ge n$.
  Nach PL ist $z = uvwxy$ mit $|vx| \ge 1$ und $|vwx| \le n$.

  Es ergeben sich folgende Möglichkeiten:
  \begin{itemize}
  \item $vwx = a^j$, $j \le n$.
    Für $i=0$ ist $vwx=w$ mit $|w| < j$.
    Anzahl der $a$ stimmt nicht mehr mit $n$ überein.
  \item $vwx = a^kb^j$, $k+j \le n$.

    \begin{itemize}
    \item Falls $v = a^{k'}$, $x = b^{j'}$: Für $i=0$ stimmt die Anzahl der $a$ nicht mehr mit $n$ überein.
    \item Falls $v$ ein Gemisch aus $a$ und b enthält.
      Für $i>2$ würde eine Folge $a^n$ durch eine $b$-Folge unterbrochen.
    \item Falls $x$ ein Gemisch aus $a$ und b enthält.
      Für $i>2$ würde eine Folge $a^n$ durch eine $b$-Folge unterbrochen.
    \end{itemize}
  \item $vwx = b^j$ analog Fall $a^j$
  \item $vwx = b^ka^j$ analog Fall $a^kb^j$
  \end{itemize}
\end{Bsp}

\draftnote{9.12.16}

}



%%% Local Variables:
%%% mode: latex
%%% TeX-master: "Info_3_Skript_WS2016-17"
%%% End:

% \section[Kellerautomaten (\acs*{PDA})]{Kellerautomaten \quad\normalfont\normalsize \acf{PDA}}
\newcommand{\ConfRel}{\rhd}
\newcommand{\Zinit}{Z^\mathsf{init}}
\newcommand{\K}{\mathcal{K}}

{\color{red} TODO: Dieser rote Teil wird noch überarbeitet.

Kellerautomat $\approx$ Endlicher Automat + Kellerspeicher von unbeschränkter Größe (Stack, push down)
\paragraph*{Neu:}
\begin{itemize}
        \item Bei jedem Schritt darf der \ac{PDA} das oberste Kellersymbol inspizieren und durch ein beliebiges Kellerwort ersetzen (das neue Kellerpräfix).
        \item Der \ac{PDA} darf auf dem Keller rechnen, ohne in der Eingabe weiterzulesen ($\Eps$-Transition oder Spontantransition).
\end{itemize}
\begin{Bsp}
\label{bsp:pda-wwr}
        \begin{align}
                \Sigma &= \{0,1\} &\qquad&\text{Eingabealphabet} \notag\\
                \Gamma &= \{0,1,\bot\} &&\text{Kelleralphabet} \notag\\
                Q &= \{q_0,q_1\} &&\bot\,\hat=\,\text{Kellerbodensymbol} \notag\\
                \delta(q_0,a,Z) &= \{(q_0,aZ) \}&& a\in\{0,1\},Z\in\Gamma\\
                \delta(q_0,\Eps,Z) &= \{(q_1,Z)\} \label{eq:5.2}\\
                \delta(q_1,a,a) &= \{(q_1,\Eps)\}\\
                \delta(q_1,\Eps,\bot) &= \{(q_1,\Eps)\}
        \end{align}
  Die Übergangsfunktion $\delta$ bildet den aktuellen Zustand, das aktuelle Eingabesymbol und das aktuell oberste Kellersymbol auf Paare von Folgezustand und neuem Kellerpräfix ab.
  In diesem Beispiel ist die zurückgegebene Menge von Paaren in allen Fällen einelementig.
  Durch die $\Eps$-Transitionen ist der \ac{PDA} aber trotzdem nichtdeterministisch.

  Bei der graphischen Darstellung werden die Transitionen mit Tripeln $(a;Z;\gamma)$ beschriftet, wobei $a \in \Sigma$ das Eingabesymbol, $Z \in \Gamma$ das oberste Kellersymbol und $\gamma \in \Gamma^*$ der neue Kellerpräfix ist:
  \begin{center}
  \begin{tikzpicture}[node distance = 3cm]
    \node[state] (0) at (0,0) {$q_0$};
    \node[node distance = 1cm, left of = 0] (start) {};
    \node[state, right of = 0] (1) {$q_1$};

    \draw[->] (start) to (0);
    \draw[->, loop above] (0) to node{$a;Z;aZ$} (0);
    \draw[->] (0) to node[auto]{$\Eps;Z;Z$} (1);
    \draw[->, loop above] (1) to node[auto]{$a;a;\Eps$} (1);
    \draw[->, loop right] (1) to node[auto]{$\Eps;\bot;\Eps$} (1);
  \end{tikzpicture}
\end{center}
wobei hier $a \in \Sigma$ und $Z \in \Gamma$ gilt.

Der Automat beginnt im Startzustand $\qinit$ mit einem Kellerspeicher, der nur das Kellerbodensymbol $\Zinit$ enthält.
Er akzeptiert ein Wort, wenn er alle Eingabesymbole gelesen hat und den Keller komplett leeren konnte.
Anders als bei endlichen Automaten gibt es \emph{keine} akzeptierenden Zustände.

Die akzeptierte Sprache ist hier $L=\{ww^R \mid w\in\{0,1\}^*\}$.
\end{Bsp}
}

\begin{Def}[name={[NPDA]}]
        Ein \ac{NPDA}\footnote{Englisch: Nondeterministic Pushdown Automaton} ist ein 6-Tupel 
        $$(\Sigma,Q,\Gamma,\qinit,\Zinit,\delta).$$
        Dabei ist
        \begin{itemize}
		\item $\Sigma$ ein Alphabet, das wir auch \emph{Eingabealphabet} nennen,
                \item $Q$ eine endliche Menge, deren Elemente wir \emph{Zustände} nennen,
                \item $\Gamma$ ein Alphabet, das wir auch \emph{Kelleralphabet} nennen,
                \item $\qinit\in Q$ ein Zustand, den wir Startzustand nennen,
                \item $\Zinit\in\Gamma$ ein Zeichen, das wir \emph{Kellerbodensymbol} nennen,
                \item $\delta: Q\x(\Sigma\cup\{\Eps\})\x\Gamma \-> \mathcal{P}(Q\x\Gamma^*)$ eine Funktion, die wir \emph{Transitionsfunktion} nennen. \qedhere
        \end{itemize}
\end{Def}
Im Weiteren sei $\K=(\Sigma,Q,\Gamma,\qinit,\Zinit,\delta)$ ein \ac{NPDA}.
\begin{Def}[name={[Menge der Konfigurationen eines \acs*{NPDA}]}]
        Die Menge der \emph{Konfigurationen} von $\K$ ist $\Konf(\K) = Q\x\Sigma^*\x\Gamma^*$.\\
        Die \emph{Schrittrelation} von $\K$
  \begin{displaymath}
    \mathop{\ConfRel} \subseteq \Konf(\K) \times \Konf(\K) 
  \end{displaymath}
  ist definiert durch
        \begin{align*}
                (q,aw,Z\gamma) &\ConfRel (q',w,\beta\gamma) &&\text{falls }\delta(q,a,Z)\ni(q',\beta)\\
                (q,w,Z\gamma) &\ConfRel (q',w,\beta\gamma) &&\text{falls }\delta(q,\Eps,Z)\ni(q',\beta).
        \end{align*}

  Wir schreiben $(q,w,\gamma) \ConfRel^n (q',w',\gamma')$, wenn $\K$ in $n \in \mathbb{N}$ Schritten von Konfiguration $(q,w,\gamma)$ in Konfiguration $(q',w',\gamma')$ gelangt.

  Wir schreiben ${\ConfRel^*}$ für die reflexive, transitive Hülle von ${\ConfRel}$.
  Falls $(q,w,\gamma) \ConfRel^* (q',w',\gamma')$, so existiert also $n \in \mathbb{N}$, sodass $(q,w,\gamma) \ConfRel^n (q',w',\gamma')$.
  
        Die von $\K$ \emph{akzeptierte Sprache} ist
  \begin{displaymath}
                L(\K) = \{ w\in\Sigma^* \mid \exists q\in Q: (\qinit,w,\Zinit) \ConfRel^{\!\!*} (q,\Eps,\Eps) \}
  \end{displaymath}
  Wir nennen eine Konfiguration der Form $(q,\Eps,\Eps)$ mit $q\in Q$ eine \emph{akzeptierende Konfiguration}.
\end{Def}

\begin{Bsp*}
\datenote{6.12.2017}
  Die folgenden Schritte von $\K$ aus Beispiel \ref{bsp:pda-wwr} zeigen, dass $w = 0110 \in L(\K)$:
  \begin{displaymath}
  \begin{array}{r@{\ }ll}
    & (\qinit, 0110, \bot) \\
    \ConfRel & (\qinit, 110, 0\bot)  &\text{(,,pushen'' des Eingabesymbols $0$)}\\
    \ConfRel & (\qinit, 10, 10\bot)  &\text{(,,pushen'' des Eingabesymbols $1$)}\\
    \ConfRel & (q_1, 10, 10\bot)  &\text{($\Eps$-Übergang von $\qinit$ nach $q_1$)} \\
    \ConfRel & (q_1, 0, 0\bot)  &\text{(,,poppen'' des Eingabesymbols $1$)}\\
    \ConfRel & (q_1,\Eps, \bot) &\text{(,,poppen'' des Eingabesymbols $0$)}\\
    \ConfRel & (q_1, \Eps, \Eps) &\text{($\Eps$-Übergang zum Entfernen von $\bot$)}
    \qedhere
  \end{array}
\end{displaymath}
\end{Bsp*}

\begin{lemma}\label{lem:4.cfgToNpda}
 Zu jeder \ac{CFG} $\mathcal{G}$ gibt es einen \ac{NPDA} $\mathcal{K}$, sodass $L(\mathcal{K})=L(\mathcal{G})$.
\end{lemma}

  Zum Führen des Beweises benötigen wir das folgende Lemma.

\begin{lemma}[Mehr Keller -- mehr Möglichkeiten.]\label{lem:4.mehrKeller}
Für alle $q,q' \in Q,\enspace w \in \Sigma^*,\enspace Z \in \Gamma$ und $n \in \N$ gilt:
  \begin{displaymath}
    \text{Wenn } (q,w,Z) \ConfRel^n (q', \Eps, \Eps), \text{dann } \forall v \in \Sigma^*, \gamma \in \Gamma^*: (q, wv, Z\gamma) \ConfRel^n (q', v, \gamma).
    \qedhere
  \end{displaymath}
\end{lemma}
\begin{proof}
Übungsaufgabe.
\end{proof}


  
\begin{proof}[von \autoref{lem:4.cfgToNpda}]
Sei $\mathcal{G} = (\Sigma,N,P,S)$ eine kontextfreie Grammatik für $L$ in \ac{CNF}.
                Definiere einen \ac{NPDA} $\mathcal{K} = (\Sigma,Q,\Gamma,\qinit,\Zinit,\delta)$ durch:
                \begin{itemize}
                        \item $Q = \{\qinit\}$ 
                        \item $\Gamma = \Sigma\overset.\cup N$
                        \item $\Zinit = S$
                        \item $\delta(\qinit,a,a) = \{(\qinit,\Eps)\} $ für $ a\in\Sigma$
                        \item $\delta(\qinit,\Eps,A) = \{(\qinit,\alpha)\}$ für $A\to\alpha\in P$
                \end{itemize}
Wir zeigen nun, dass $L(\K) = L(\mathcal{G})$.
\begin{itemize}
 \item Beweisrichtung ``$\supseteq$''
 
 Sei $w\in L(\mathcal{G})$, dann gilt $S\stackrel{*}{\vdash} w$.
 Wir wollen zeigen dass $(\qinit, w, S) \ConfRel^* (\qinit, \Eps, \Eps)$ folgt und beweisen dafür via vollständige Induktion über die Länge der Ableitung die folgende stärkere Eigenschaft.
     \begin{displaymath}
      \forall A \in N: \text{ wenn } A \stackrel{*}{\vdash} w, \text{dann } (\qinit, w, A) \ConfRel^* (\qinit, \Eps, \Eps)
    \end{displaymath}
  I.A.: $n=1$: Es gibt nur zwei Arten von Regeln, die für Ableitungen der Länge 1 in Frage kommen:
  \begin{itemize}
        \item $S \to \Eps$.
        Per Konstruktion gilt $(\qinit, \Eps) \in \delta(\qinit, \Eps, S)$ und somit $(\qinit, \Eps, S) \ConfRel (\qinit, \Eps, \Eps)$.
      \item $A \to a$, $a \in \Sigma$.
        Per Konstruktion gilt $(\qinit, a) \in \delta(\qinit, \Eps, A)$ und $(\qinit, \Eps) \in \delta(\qinit, a, a)$.

        Somit gilt $(\qinit, a, A) \ConfRel (\qinit, a, a) \ConfRel (\qinit, \Eps, \Eps)$.
   \end{itemize}
   
   I.S.: $n\rightsquigarrow n+1$: In diesem Fall verwendet der erste Ableitungsschritt eine Regel der Form $\pi = A \to BC$ mit $B,C \in N$.
   Wir betrachten den zugehörigen Ableitungsbaum $\mathcal{T} = \pi(\mathcal{T}_1, \mathcal{T}_2)$, $\mathcal{T}_1 \in \operatorname{Abl}(B)$, $\mathcal{T}_2 \in \operatorname{Abl}(C)$, $w = Y(\mathcal{T}_1)Y(\mathcal{T}_2)$.

        Per Konstruktion gilt $(\qinit, BC) \in \delta(\qinit, \Eps, A)$. 

        Ferner gilt per I.V., dass $(\qinit, Y(\mathcal{T}_1), B) \ConfRel^* (\qinit, \Eps, \Eps)$ und $(\qinit, Y(\mathcal{T}_2), C) \ConfRel^* (\qinit, \Eps, \Eps)$.

        Mit \autoref{lem:4.mehrKeller} folgt schließlich:
        \begin{align*}
          (\qinit, w, A) = (\qinit, Y(\mathcal{T}_1)Y(\mathcal{T}_2), A) &\ConfRel (\qinit, Y(\mathcal{T}_1)Y(\mathcal{T}_2), BC) \\
          &\ConfRel^* (\qinit, Y(\mathcal{T}_2), C) \\
          &\ConfRel^* (\qinit, \Eps, \Eps).
        \end{align*}
 \item Beweisrichtung  ``$\subseteq$''
 
 Sei $w\in L(\K)$, dann gilt $(\qinit, w, S) \ConfRel^* (\qinit, \Eps, \Eps)$.
 Wir wollen $S \stackrel{*}{\vdash} w$ folgern und beweisen dafür via vollständige Induktion über die Anzahl der Berechnungsschritte $n$ die folgende Aussage:
     \begin{displaymath}
      \forall n \in \mathbb{N}: \forall w \in \Sigma^*, \alpha \in \Gamma^*: \text{ wenn } (\qinit, w, \alpha) \ConfRel^n (\qinit, \Eps, \Eps), \text{dann } \alpha \stackrel{*}{\vdash} w
    \end{displaymath}
    
    \begin{description}
    \item[I.A.] $n = 0$.
      $w = \Eps$, $\alpha = \Eps$: Es gilt $\Eps \stackrel{*}{\vdash} \Eps$.

  \item[I.S.] $n > 0$, $\alpha = Z\alpha'$: Es gilt
    $(\qinit,w,Z\alpha') \ConfRel (\qinit, w', \beta\alpha') \ConfRel^{n-1} (\qinit, \Eps, \Eps)$, \linebreak
    $\delta(\qinit, x, Z) \ni (\qinit, \beta)$, $x \in \Sigma \cup \{\varepsilon\}$.

    Es gibt zwei Fälle für $Z$:
    \begin{itemize}
    \item $Z = a$ für $a \in \Sigma$.
%
      Es folgt $\beta = \Eps$, $w = aw'$ und $x = a$.

      Per I.V.\ gilt $\alpha' \stackrel{*}{\vdash} w'$ und somit auch $\alpha = a\alpha' \stackrel{*}{\vdash}aw' = w$.
    \item $Z = A$ mit $A \to \beta \in P$.
%
      Es folgt $w = w'$ und $x = \varepsilon$.

      Per I.V.\ gilt $\beta\alpha' \stackrel{*}{\vdash} w'$ und somit auch $A\alpha \vdash \beta\alpha' \stackrel{*}{\vdash} w' = w$.
      \qedhere
    \end{itemize}
    \end{description}
\end{itemize}
\end{proof}

Bei einem \ac{NPDA} nehmen wir in jedem Schritt ein Zeichen vom Kellerspeicher herunter,
aber dürfen beliebig viele Symbole auf den Kellerspeicher zurücklegen.
Was wäre, wenn die Höhe des Kellers pro Schritt höchstens um eins größer werden könnte?
Verlören wir dadurch nur Komfort oder könnten wir dadurch weniger Sprachen akzeptieren?

Das folgende Lemma beantwortet diese Frage.
\begin{lemma}\label{lem:4.limitStackIncrease}
        Zu jedem \ac{NPDA} gibt es einen äquivalenten \ac{NPDA}, bei dem
        für alle $q,q\in Q, Z\in\Gamma, \gamma\in\Gamma^*, x\in\Sigma\cup\{\Eps\}$
        gilt: Falls $(q',\gamma)\in\delta(q,x,Z)$, dann $|\gamma| \le 2$.
\end{lemma}
\begin{proof}
        Sei $(q',\gamma)\in\delta(q,x,Z)$ mit $\gamma = Z_n\dots Z_1$ für $n>2$:
        \begin{itemize}
        \item   neue Zustände $q_2\dots q_{n-1}$
        \item Ersetze $(q',\gamma)$ durch $(q_2, Z_2Z_1)$
        \item Definiere $\delta(q_i, \Eps, Z_i) = \{ (q_{i+1}, Z_{i+1}Z_i) \}$, für $2\le i < n-1$
        \item Definiere $\delta(q_{n-1}, \Eps, Z_{n-1}) = \{ (q', Z_nZ_{n-1}) \}$
        \end{itemize}
        Wiederhole bis alle Transitionen die gewünschte Form haben.
        Die Sprache des \ac{NPDA} ändert sich durch diese Transformation nicht (ohne Beweis).\qedhere
\end{proof}


\begin{lemma}\label{lem:4.npdaToCfg}
\datenote{8.12.2017}
 Zu jedem \ac{NPDA} $\mathcal{K}$ gibt es eine \ac{CFG} $\mathcal{G}$, sodass $L(\mathcal{K})=L(\mathcal{G})$.
\end{lemma}

\begin{proof}
 Ohne Einschränkung (\autoref{lem:4.limitStackIncrease}) sei $\mathcal{K} = (Q, \Sigma, \Gamma, \qinit, \Zinit, \delta)$ ein \ac{NPDA} mit $|\gamma| \le 2$ für alle $(q', \gamma) \in \delta(q, x, Z), q, q' \in Q, x \in \Sigma \cup \{\Eps\}, Z \in \Gamma$.

    Definiere $\mathcal{G} = (\Sigma, N, P, S)$ mit $N = (Q \times \Gamma \times Q) \cup \{S\}$ und $P=P_S\cup P_0\cup P_1\cup P_2$ mit:
    \begin{align*}
    P_S &= \{S \to [\qinit, \Zinit, q']\mid q'\in Q\}, \\[1mm]
    %
    P_0 &= \{ [q, Z, q'] \to x \mid (q', \Eps)\in \delta(q, x, Z), x \in \Sigma \cup \{\Eps\} \}, \\[1mm]
    %
    P_1 &= \{[q, Z, q'] \to x[q'', Z', q'] \mid (q'', Z')\in \delta(q, x, Z), x \in \Sigma \cup \{\Eps\}\}, \\[1mm]
    %
    P_2 &= \{[q, Z, q'] \to x[q_1,Z_1,q_2][q_2, Z_2, q'] \mid (q_1, Z_1Z_2)\in\delta(q, x, Z), q_2 \in Q, x \in \Sigma \cup \{\Eps\} \}
    \end{align*}
    
    Dabei unterscheiden $P_0$, $P_1$ und $P_2$ die Fälle, dass der Stack kleiner wird ($P_0$), gleich hoch bleibt ($P_1$) oder größer wird ($P_2$).
    
    Die dieser Konstruktion zugrunde liegende Idee wird durch die folgende Eigenschaft beschrieben:
      \begin{equation}\label{eq:npdaToCfgProp}
        [q,Z,q']\stackrel{n}{\vdash} w \text{ gdw } (q,w,Z) \ConfRel^n (q',\Eps,\Eps)
      \end{equation}
    In Worten: Wir können vom Nichtterminalsymbol $[q,Z,q']$ das Wort $w$ genau dann in $n$ Schritten ableiten,
    wenn der \ac{NPDA} die Konfiguration $(q,w,Z)$ in $n$ Schritten in eine akzeptierende Konfiguration mit Zustand $q'$ überführen kann.
    
    Beispiel:
    
    \ac{NPDA} für $L_\text{centered3}:=\{0^n10^n \mid n\in\N, n > 0 \}\subseteq\{0,1\}^*$:
    
\newcommand{\el}[3]{{\color{red!70!black}#1};{\color{blue!70!black}#2};{\color{blue!70!black}#3}}
    \begin{center}
    \begin{tikzpicture}
 
    \node[circle,draw] (q0) at (-1.5,0) {$q_0$};
    \node[circle,draw] (q1) at (1.5,0) {$q_1$};
    
    \draw[->] (-2.5,0.1) to  (q0) ;
    
    \draw[->] (q0) edge node[auto] {\el{1}{X}{X}} (q1) ;
    
    \draw[->,in=100, out=120, looseness=22, pos=0.50] (q0) edge node[auto] {\el{0}{X}{XX}} (q0) ;
    \draw[->,in=260, out=240, looseness=22, pos=0.50] (q0) edge node[left] {\el{0}{\#}{X\#}} (q0) ;
    
    \draw[->,in=50, out=70, looseness=22, pos=0.50] (q1) edge node[right] {\el{0}{X}{$\varepsilon$}} (q1) ;
    \draw[->,in=-70, out=-50, looseness=22, pos=0.45] (q1) edge node[right] {\el{$\varepsilon$}{\#}{$\varepsilon$}} (q1) ;
    \end{tikzpicture}
    \end{center}
    
    $010$ wird akzeptiert, weil
        $(q_0, 010, \#)\ConfRel (q_0, 10, X\#)\ConfRel (q_1, 0, X\#)\ConfRel (q_1, \Eps, \#)\ConfRel (q_1, \Eps, \Eps)$.
    
    \begin{center}
    \vspace*{-1.5\baselineskip}
    \begin{minipage}[t]{55mm}
    \begin{align*}
    P_S = \{ 
    & S \to [q_0, \#, q_0]\\
    & S \to [q_0, \#, q_1] 
    \}\\
    \\
    % \end{align*}
    % \begin{align*}
    P_0 = \{ 
    & [q_1, X, q_1] \to 0\\
    & [q_1, \#, q_1] \to \Eps
    \}\\
    \\
    % \end{align*}
    % \begin{align*}
    P_1 = \{ 
    & [q_0, X, q_0] \to 1[q_1, X, q_0]\\
    & [q_0, X, q_1] \to 1[q_1, X, q_1]
    \}
    \end{align*}
    \end{minipage}
    \begin{minipage}[t]{70mm}
    \begin{align*}
    P_2 = \{ 
    & [q_0, \#,q_0] \to 0[q_0, X, q_0] [q_0 \#, q_0]\\
    & [q_0, \#,q_0] \to 0[q_0, X, q_1] [q_1 \#, q_0]\\
    & [q_0, \#,q_1] \to 0[q_0, X, q_0] [q_0 \#, q_1]\\
    & [q_0, \#,q_1] \to 0[q_0, X, q_1] [q_1 \#, q_1]\\
    % 
    & [q_0, X,q_0] \to 0[q_0, X, q_0] [q_0 X, q_0]\\
    & [q_0, X,q_0] \to 0[q_0, X, q_1] [q_1 X, q_0]\\
    & [q_0, X,q_1] \to 0[q_0, X, q_0] [q_0 X, q_1]\\
    & [q_0, X,q_1] \to 0[q_0, X, q_1] [q_1 X, q_1]
    \}
    \end{align*}
    \end{minipage}
    \end{center}
    
    Ableitungsbaum für $010$:
    
    \begin{center}
    \vspace*{-\baselineskip}
    \begin{tikzpicture}
 
    \node[] (root) at (0,0) {$S \to [q_0, \#, q_1]$};
    
    \node[] (m) at (0, -1.5) {$[q_0, \#,q_1] \to 0[q_0, X, q_1] [q_1, \#, q_1]$};
    
    \node[] (ml) at (-2, -3) {$[q_0, X, q_1] \to 1[q_1, X, q_1]$};
    \node[] (mr) at (2, -3) {$[q_1, \#, q_1] \to \Eps$};
    
    \node[] (mlm) at (-2, -4.5) {$[q_1, X, q_1] \to 0$};
    
    
    
    \draw[] (root) to (m) ;
    \draw[] (m) to (ml) ;
    \draw[] (m) to (mr) ;
    \draw[] (ml) to (mlm) ;
    
    \end{tikzpicture}
    \end{center}
    
    Offensichtlich folgt aus dieser Eigenschaft $L(\mathcal{G}) = L(\mathcal{K})$, da wir vom Startsymbol $S$ genau die Nichtterminalsymbole der Form
    $[\qinit, \Zinit, q']$ ableiten können und der \ac{NPDA} genau die Wörter $w$ akzeptiert, für die  $(\qinit, w, \Zinit)$ in eine akzeptierende Konfiguration überführt werden kann.
    
    Es bleibt zu zeigen, dass die Eigenschaft \eqref{eq:npdaToCfgProp} gilt.
    Für den Fall $n=1$ folgt die Eigenschaft direkt aus den $P_0$-Regeln (und der Tatsache, dass keine der anderen Regeln ein einzelnes Symbol oder $\Eps$ ableiten kann).
    
    \begin{itemize}
     \item Beweisrichtung ``$\Rightarrow$''
     Wir zeigen diese Richtung via vollständige Induktion über die Anzahl der Ableitungsschritte.
     \begin{description}
      \item[I.A.] $n=1$: Folgt wie oben erwähnt aus den $P_0$-Regeln.
      \item[I.S.] $n\rightsquigarrow n+1$:
      Sei $[q,Z,q']\vdash\alpha\stackrel{n}{\vdash} w$ eine Ableitung für $w$. Für den ersten Ableitungsschritt kommen zwei Fälle in Frage:
      \begin{itemize}
      \item Fall 1: $\alpha$ hat die Form $x[q'', Z', q']$ ($[q,Z,q']\rightarrow\alpha \in P_1$).
      
	  Es folgt $w = xw'$ für ein $w'$ mit $[q'', Z', q']\stackrel{n}{\vdash} w'$.
	  
	  Nach I.V.\ gilt nun, dass $(q'', w', Z') \ConfRel^n (q',\Eps,\Eps)$.
	  
	  Nach Konstruktion gilt $\delta(q, x, Z) \ni (q'', Z'), q'' \in Q, x \in \Sigma \cup \{\Eps\}$ und somit
	  $(q, w, Z) \ConfRel (q'', w', Z') \ConfRel^n (q',\Eps,\Eps)$.
      \item Fall 2: $\alpha$ hat die Form $x[q_1,Z_1,q_2][q_2, Z_2, q']$ ($[q,Z,q']\rightarrow\alpha \in P_2$).
      
      Es gibt also einen Ableitungsbaum $\mathcal{T} = \pi(\mathcal{T}_1,\mathcal{T}_2)$, mit $\mathcal{T}_1 \in \operatorname{Abl}([q_1, Z_1, q_2])$ und $\mathcal{T}_2 \in \operatorname{Abl}([q_2, Z_2, q'])$, sodass $w = xY(\mathcal{T}_1)Y(\mathcal{T}_2)$.
      
      Es gilt also $[q_1, Z_1, q_2]\stackrel{k_1}{\vdash} Y(\mathcal{T}_1)$ und 
      $[q_2, Z_2, q']\stackrel{k_2}{\vdash} Y(\mathcal{T}_2)$ für $k_1,k_2\in\N$, sodass $k_1+k_2=n$.
      
          Per I.V.\ folgt
          \begin{displaymath}
            (q_1, Y(\mathcal{T}_1), Z_1) \ConfRel^{k_1} (q_2, \Eps, \Eps).
          \end{displaymath}
          Mit \autoref{lem:4.mehrKeller} folgt auch
          \begin{displaymath}
            (q_1, Y(\mathcal{T}_1)Y(\mathcal{T}_2), Z_1Z_2) \ConfRel^{k_1} (q_2, Y(\mathcal{T}_2), Z_2).
          \end{displaymath}
          

      Auch hier können wir wieder I.V.\ anwenden und erhalten:
          \begin{displaymath}
            (q_2, Y(\mathcal{T}_2), Z_2) \ConfRel^{k_2} (q', \Eps, \Eps)
          \end{displaymath}
          
                    Somit gilt:
          \begin{displaymath}
            (q, \underbrace{xY(\mathcal{T}_1)Y(\mathcal{T}_2)}_{w}, Z) \ConfRel (q_1, Y(\mathcal{T}_1)Y(\mathcal{T}_2), Z_1Z_2) \ConfRel^{k_1} (q_2, Y(\mathcal{T}_2), Z_2) \ConfRel^{k_2} (q', \Eps, \Eps)
          \end{displaymath}
      \end{itemize}
      
     \end{description}

    \end{itemize}
    
    \item Beweisrichtung ``$\Leftarrow$''
     Wir zeigen diese Richtung via vollständige Induktion über die Anzahl der Berechnungsschritte des \ac{NPDA}.
     \begin{description}
      \item[I.A.] $n=1$: Folgt wie oben erwähnt aus den $P_0$-Regeln.
      \item[I.S.] $n\rightsquigarrow n+1$:
      
      Sei $(q,w,Z) \ConfRel^{n+1} (q',\Eps,\Eps)$ eine Berechnung für $w$. Da wir annehmen, 
      dass der \ac{NPDA} in einem Schritt höchstens zwei Symbole auf den Stack legen darf, kommen für den ersten Berechnungsschritt nur zwei Fälle in Frage:
      \begin{itemize}
      \item Fall 1: $(q,w,Z)\ConfRel (q'', w',Z')$ mit $w=xw'$ für $x\in\Sigma\cup\{\Eps\}$
      
      Aus $(q'', w',Z')\ConfRel^n (q',\Eps,\Eps)$ folgern wir mit I.V.\
      $[q'', Z',q']\vdash^n w'$.
      Nach Konstruktion können wir $[q,Z,q']\vdash x [q'', Z',q']$ ableiten.

      
      \item Fall 2: $(q,w,Z)\ConfRel (q_1, w',Z_1Z_2)$ mit $w=xw'$ für $x\in\Sigma\cup\{\Eps\}$

            Aus $(q_1, w',Z_1Z_2)\ConfRel^n (q',\Eps,\Eps)$ folgern wir:
      $\exists k_1,k_2\in\N, \; q_2\in Q, \; w_1,w_2\in \Sigma^*$, sodass gilt:
      \begin{enumerate}
      \item $k_1+k_2=n$
      \item $w'=w_1w_2$
      \item $(q_1, w',Z_1Z_2)\ConfRel^{k_1} (q_2, w_2,Z_2)$
      \item $(q_2, w_2,Z_2)\ConfRel^{k_2} (q',\Eps,\Eps)$
      \end{enumerate}
      
      Aus 3.\ folgern wir $(q_1, w_1,Z_1)\ConfRel^{k_1} (q_2, \Eps,\Eps)$ und mit I.V.\
      $[q_1, Z_1, q_2] \vdash^{k_1} w_1$.
      
      Aus 4.\ folgern wir mit I.V.\ $[q_2, Z_2, q'] \vdash^{k_2} w_2$.
      
      Es gibt also einen Ableitungsbaum $\mathcal{T} = \pi(\mathcal{T}_1,\mathcal{T}_2)$, mit $\mathcal{T}_1 \in \operatorname{Abl}([q_1, Z_1, q_2])$ und $\mathcal{T}_2 \in \operatorname{Abl}([q_2, Z_2, q'])$, sodass $w = xY(\mathcal{T}_1)Y(\mathcal{T}_2)$.
      
      Somit gibt es auch eine Ableitung
      \begin{equation*}
      [q, Z, q'] \vdash x[q_1,Z_1,q_2][q_2, Z_2, q'] \vdash^{k_1+k_2} xw_1w_2. \qedhere
      \end{equation*}
      \end{itemize}
     \end{description}
\end{proof}








\hide{

\begin{Satz}\label{satz:5.1}
  \begin{align*}
                L \in \ac{CFL}  \text{ gdw } L =L(M) \text{ für einen \ac{NPDA} $M$}
  \end{align*}
\end{Satz}
\datenote{21.12.16}
\begin{proof}\hfill
        \begin{itemize}
        \item \ac{CFG} zu \ac{NPDA}:


  \item \ac{NPDA} zu \ac{CFG}:

    Zunächst zeigen wir, dass es genügt, \ac{NPDA}s zu betrachten, die bei jeder Transition Wörter der maximalen Länge $2$ auf den Keller schreiben:



\end{itemize}
  
\end{proof}

\begin{Def}[name={[DPDA]}]
        Ein \ac{DPDA} ist ein Tupel $(\underbrace{Q,\Sigma,\Gamma,\qinit,\Zinit}_{\text{wie gehabt}},\delta,F)$
        \vspace{-1em}
        \begin{itemize}
        \item $F\subseteq Q$ akzeptierende Zustände
        \item $\delta: Q\x (\Sigma\cup\{\Eps\})\x \Gamma \-> \mathcal{P}(Q\x \Gamma^*)$ wobei für alle $q\in Q,a\in\Sigma,Z\in\Gamma$ gelten muss, dass
        $|\delta(q,a,Z)| + |\delta(q,\Eps,Z)| \leq 1$
        \item Die Schrittrelation ,,$\ConfRel$'' ist definiert wie bei \ac{NPDA}s.
        \item $L(M) = \{w\in\Sigma^* \mid (\qinit,w,\Zinit) \ConfRel^* (q',\Eps,\gamma) \land q'\in F \}$ \qedhere
        \end{itemize}
\end{Def}

\begin{lemma}[name={[\acs*{DPDA}, der gesamte Eingabe verarbeitet]}]
        \label{lem:DPDA ges. Eingabe}
        Zu jedem \ac{DPDA} $M$ gibt es einen äquivalenten \ac{DPDA} $M'$, der
        jede Eingabe bis zum Ende liest.
\end{lemma}
\draftnote{23.12.16}
\begin{proof}

  Sei $M = (Q, \Sigma, \Gamma, \qinit, \Zinit, \delta, F)$. 
  Zwei Möglichkeiten, warum $M$ nicht die gesamte Eingabe verarbeiten:
  Die Transitionsrelation ist nicht total \footnote{d.h. es gibt zwei Konfigurationen $a$, $b$
  so dass $a \not \Rightarrow^* b$} oder der Automat bleibt bei
  leerem Keller stecken.

  Abhilfe: Führe einen Senkzustand ein, auf dem $M'$ weiterrechnet, wenn $M$ nicht mehr weiterrechnen kann. Definiere dazu $M' = (Q', \Sigma, \Gamma', {\qinit}', {\Zinit}', \delta', F')$ mit
  \begin{align*}
  Q' &= Q \cup \{{\qinit}', q_s\} \text{ neue Zustände: ${\qinit}'$ neuer Startzustand und $q_s$ Senkzustand} \\
  F' &= F \\
  \Gamma' &= \Gamma \cup \{{\Zinit}'\} \text{ neues Kellerbodensymbol}
  \end{align*}
  und Transitionsfunktion $\delta'$ gegeben durch
  \[\delta' ({\qinit}', \varepsilon, {\Zinit}') = \{ (\qinit, \Zinit{\Zinit}') \} \]
  \begin{itemize}
  \item Transition totalisieren: Für alle $q\in Q$, $Z \in \Gamma$\\
    \begin{align*}
      \delta' (q, \varepsilon, Z) &=
      \begin{cases}
        \delta (q, \varepsilon, Z) & \text{falls
        }\ne\emptyset\vee\exists a\in\Sigma, \delta (q, a, Z)\ne \emptyset \\
        \{ (q_s, Z) \} & \text{sonst}
      \end{cases} \\
      \delta' (q, a, Z) &=
      \begin{cases}
        \delta (q, a, Z) & \text{falls }\ne\emptyset\vee\delta (q,
        \varepsilon, Z) \ne \emptyset \\
        \{ (q_s, Z) \} & \text{sonst}
      \end{cases}
    \end{align*}
    Intuition: Wenn $\delta(q, a, Z) = \emptyset$ und ein $\epsilon$-Übergang ebenfalls ausscheidet, kann das Wort von $M$ nicht weiter abgearbeitet werden. $M'$ geht nun in Senkzustand $\qinit$ und arbeitet dort weiter. Analog für $\delta(q, \epsilon, Z)$.
  \item Kellerunterlauf: $M$ hat $\Zinit$ abgeräumt, so dass ${\Zinit}'$
    sichtbar geworden ist. Füge eine $\epsilon$-Transition in
     Senkzustand $q_s$ hinzu, indem man für alle $q\in Q$ definiere:\\
    \[\delta' (q, \varepsilon, {\Zinit}') = \{ (q_s, {\Zinit}') \} \]
  In $q_s$ kann der Automat nun für alle $a\in\Sigma$, $Z\in \Gamma'$ abarbeiten:\\
    \[\delta' (q_s, a, Z) = \{ (q_s, Z) \}\]   %TODO warum Z und nicht {\Zinit}' ?!
    Bemerke, dass $\qinit \not \in F'$, da ein Kellerunterlauf
    das Akzeptieren des eingelesen Wortes ausschliesst.
  \end{itemize}

  Es verbleibt zu zeigen, dass $\forall w\in \Sigma^*$ $\exists q\in Q'$
  $\gamma\in\Gamma'^*$ so dass $({\qinit}', w, {\Zinit}') \ConfRel^* (q,
  \varepsilon, \gamma)$. (Induktion über $w$) Weiter ist zu zeigen,
  dass $L (M) = L (M')$.
\end{proof}

\begin{Satz}[name={[Abgeschlossenheit der deterministischen \acs*{CFL}]}]
        Die deterministischen \ac{CFL} sind unter Komplement abgeschlossen.
\end{Satz}
\begin{proof}
  Sei $L=L(M)$ für \ac{DPDA} $M$. Nach \autoref{lem:DPDA ges. Eingabe} liest $M$ die komplette Eingabe. 

  Konstruiere \ac{DPDA} $M'$, s.d. $ L(M') = \overline{L}$. 
  Definiere dazu $Q'= q \times \{0,1,2\}$. Bedeutung des zusätzlichen ,,Flags'' $\in \{0,1,2\}$:
  \begin{itemize}
  \item $0$: seit Lesen des letzten Symbols $\epsilon \vee a \in \Sigma$ wurde kein akzeptierender Zustand durchlaufen
  \item $1$: seit Lesen des letzten Symbols wurde mindestens ein
    akzeptierender Zustand durchlaufen 
  \item $2$: markiert einen akzeptierenden Zustand in $M'$.
  \end{itemize}
        
  $F'=F\x \{2\}$

  Definiere Hilfsfunktion $\mathit{final}:Q\to \{0,1\}$ durch 
  \begin{displaymath}
    \mathit{final} (q) =
    \begin{cases}
      0 & q\notin F \\ 1 & q \in F
    \end{cases}
  \end{displaymath}
  \begin{align*}
    {\qinit}' &= (\qinit,\mathit{final} (\qinit))
  \end{align*}
  Für alle $q \in Q$, $Z\in\Gamma$.
  
  Falls $\delta(q,\Eps,Z) = \{(q',\gamma)\}$, dann
  \begin{align*}
    \delta'((q,0),\Eps,Z) &= ((q',\mathit{final} (q')),\gamma)
    \\
    \delta'((q,1),\Eps,Z) &= ((q',1),\gamma)
  \end{align*}
  Falls $\delta(q,a,Z) = \{(q',\gamma)\}$, dann
  \begin{align*}
    \delta'((q,0), \Eps, Z) &= \{ ((q,2), Z) \} \\
    \delta'((q,2), a, Z) &=((q',\mathit{final} (q')), \gamma)
    \\
    \delta'((q,1), a, Z) &=
    ((q',\mathit{final} (q')), \gamma)
    \\ \tag*{\qedhere}    %TODO Intuition?
  \end{align*}
\end{proof}
\begin{Satz}
    Die deterministischen \ac{CFL} sind \textbf{nicht} unter Vereinigung und Durchschnitt abgeschlossen.
\end{Satz}
\begin{proof}
    Betrachte
    \begin{align*}
        L_1 &= \{ a^nb^nc^m \mid n, m \ge 1 \} \\
        L_2 &= \{ a^mb^nc^n \mid n, m \ge 1 \}
    \end{align*}
    Sowohl $L_1$ als auch $L_2$ sind DCFL, aber $L_1 \cap L_2 = \{ a^nb^nc^n \mid n \ge 1\}$ ist nicht kontextfrei.
    
    DCFL ist nicht abgeschlossen unter Vereinigung. Angenommen doch: Seien $U, V$ DCFL. Dann sind auch $\overline{U}$ und $\overline{V}$ DCFL. Bei Abschluss unter Vereinigung wäre $\overline{U} \cup \overline{V}$ eine DCFL und somit auch $\overline{\overline{U} \cup \overline{V}} = U \cap V$, ein Widerspruch gegen den ersten Teil.
\end{proof}
\begin{Satz}
    DCFL ist abgeschlossen unter Schnitt mit REG.
\end{Satz}
\begin{proof}
    Sei $L$ DCFL und $R$ regulär.
    Konstruiere das Produkt aus einem DPDA für $L$ und einem DFA für $R$.
    Offenbar ist das Ergebnis ein DPDA, der $L\cap R$ akzeptiert. \footnote{
    Intution des Beweis: Konstruiere Produktautomaten, der akzeptiert gdw. der DFA und NDPA akzeptieren. Idee: Definiere Tupel $(q_1, q_2)$, DPDA arbeitet auf $q_1$, DFA arbeitet auf $q_2$. Beispielsweise werden $\epsilon$-Übergänge des DPDA nur auf der ,,DPDA-Seite'' d.h. auf $q_1$ abgearbeitet, ohne Beeinflussung von $q_2$.
    }

    $L = L (M_1)$ mit $M_1 = (Q_1, \Sigma, \Gamma, q_{01}, \Zinit,
    \delta_1, F_1)$ \ac{DPDA}

    $R = L (M_2)$ mit $M_2 = (Q_2, \Sigma, q_{02}, \delta_2, F_2)$
    \ac{DFA}

    Konstruiere $M'= (Q, \Sigma, \Gamma, \qinit, \Zinit, \delta, F)$ mit
    \begin{itemize}
    \item $Q = Q_1 \times Q_2$
    \item $\qinit = (q_{01}, q_{02})$
    \item $F = F_1  \times F_2$
    \item Falls $\delta_1 (q_1, \varepsilon, Z) = \{(q_1', \gamma)\}$, dann gilt für alle $q_2\in Q_2$: 
      \[ \delta ((q_1, q_2), \varepsilon, Z)
      = \{((q_1', q_2), \gamma)\} \text{.     (Ansonsten leer)} \]
    \item Falls $\delta_1 (q_1, a, Z) = \{(q_1', \gamma)\}$, dann gilt für alle $q_2\in Q_2$: 
      \[ \delta ((q_1, q_2), a, Z) =
      \{((q_1', \delta_2 (q_2, a)), \gamma)\} \text{   (Ansonsten leer)} \]
    \end{itemize}
    Zu zeigen ist noch $L (M') = L (M_1) \cap L (M_2)$.
\end{proof}
\begin{Satz}
    Sei $L$ DCFL und $R$ regulär.
    Es ist entscheidbar, ob $R=L$, $R\subseteq L$ und $L=\Sigma^*$.
\end{Satz}
\begin{proof}
    Es gilt $R\subseteq L$ gdw.\ $R \cap \overline{L} = \emptyset$.
    
    Weiter ist $R = L$ gdw.\ $R\subseteq L$ und $L \subseteq R$. Für den zweiten Teil betrachte $L\cap \overline{R}$.
    
    Für kontextfreie Sprachen ist $L\ne \emptyset$ entscheidbar, also betrachte $L=\Sigma^*$ gdw.\ $\overline{L}=\emptyset$.
\end{proof}

\begin{Satz} \textbf{DPDA Äquivalenzproblem}
    Seien $L_1, L_2$ DCFL. Dann ist $L_1 = L_2$ entscheidbar.
\end{Satz}

\begin{proof}
    Siehe Senizergues (2000) und Stirling (2001).
\end{proof}

Wir betrachten zum Ende des Kapitels noch eine praktische Fragestellung: Wie sieht man einer CFL an, dass sie von einem DPDA akzeptierbar ist?

Sei $\mathcal{G} = (\Sigma, N, P, S)$.
Wir können nach Satz \ref{satz:5.1} einen \ac{PDA} für $\mathcal{G}$ konstruieren, mit
\begin{align*}
  \delta(q, a, a) &= \{(q, \Eps)\} \quad a \in \Sigma \\
  \delta(q, \Eps, A) &= \{(q, \beta)  \mid A \to \beta \in P\}
\end{align*}
Die Transitionen für Eingabezeichen $a \in \Sigma$ sind deterministisch.
Die $\Eps$-Transitionen sind es nicht unbedingt.

Die Idee ist nun, den Automaten mit einem Symbol Lookahead \footnote{Das Problem ist, dass ein deterministischer \ac{PDA} nicht ,,alle mögl. Regeln gleichzeitig ausprobieren'' kann, so wie ein NDPA. Möchte der Automat ein Eingabesymbol $a$ matchen und kann aktuell mehrere Produktionen anwenden, müsste er in die Zukunft sehen können, um diejenige zu wählen, die das gewünschte $a$ an erster Position erzeugt. Diesen ,,Blick in die Zukunft'' gewähren die $first(A)$ Mengen: Sie geben für jedes Nichtterminal $A$ an, welche Terminale $a \in \Sigma$ als Präfix von $yield(A)$ auftreten können.} 
erweitern. Dieses Symbol wird jeweils im Zustand des Automaten
gespeichert. Der \ac{PDA} für $L (\mathcal{G})$ mit Lookahead hat nun
folgende Komponenten:
\begin{itemize}
\item $Q = \Sigma\cup \{\varepsilon, \$\}$
\item $\qinit = [\Eps]$ (zu Beginn ist der Lookahead leer)
\item $F = [\$]$  (das Symbol $\$$ markiert das Ende der Eingabe)
\end{itemize}
und die Transitionsfunktion
\begin{align*}
  \delta ([\Eps], a, Z) & = \{([a], Z) \} & & \text{lade Lookahead} \\
  \delta ([a], \Eps, a) & = \{([\Eps], \varepsilon) \} && \text{match}
  \\
  \delta ([a], \Eps, A) & = \{([a], \beta) \mid A\to\beta\in P, a \in
  \mathit{first} (\beta) \} && \text{select}
\end{align*}
Der Automat startet in der Konfiguration $([\Eps], w\$, S\$)$.

\textbf{Beispiel}: Sei ein \ac{CFG} gegeben mit den Produktionen $S \to ( S )$ $ |$ $ a$. Für die Eingabe $(a)$ ergibt sich folgende Abarbeitung:
\begin{align*}
              & ([\Eps], (a), S\$ )   && \text{Lade Lookahead ,('}\\
  \Rightarrow & ([(], (a), S\$ )   && \text{Select: Da ,(' $\in first($ $(a)$ $ )$, wende $S \to (a)$ an}\\
  \Rightarrow & ([(], (a), (S)\$ ) && \text{Match ,('}\\
  \Rightarrow & ([\Eps], a), S)\$) \\
  \Rightarrow & ...
\end{align*}

\begin{Def}
  Sei $\mathcal{G} = (\Sigma, N, P, S)$ \ac{CFG} und $\beta \in
  (N\cup\Sigma)^*$
  \begin{align*}
    \mathit{first} (\beta) &= \{ a \in \Sigma \mid \exists w\in
    \Sigma^* , \beta \=>^* aw \} \cup \{\Eps \mid \beta \=>^* \Eps \}
  \end{align*}
\end{Def}

Spezifikation von $\mathit{first}(\beta)$
\begin{align*}
  \mathit{first} (\Eps) & = \{ \Eps \} \\
  \mathit{first} (a\beta) & = \{ a \} \\
  \mathit{first} (A\beta) & =
  \begin{cases}
    \mathit{first} (A) & \Eps \notin \mathit{first (A)} \\
    \mathit{first} (A) \setminus \{\Eps\} \cup \mathit{first} (\beta)
    & A\=>^* \Eps
  \end{cases} \\
  \mathit{first} (A) &= \bigcup  \{ \mathit{first} (\beta) \mid A \to
  \beta \in P \}
\end{align*}

\textbf{Algorithmus} zur Berechnung von $\mathit{first} (A)$ für alle $A\in N$

Sei $FI[A] \subseteq N$ ein Feld indiziert mit Nichtterminalsymbolen.

\begin{itemize}
\item[] For each $A \in N$: $FI[A] \gets \emptyset$
\item[] Repeat
  \begin{itemize}
  \item[] For each $A\in N$: $FI'[A] \gets FI[A]$
  \item[] For each $A \to \beta \in P$
    \begin{itemize}
    \item[] $FI[A] \gets FI[A] \cup \mathit{first}_{FI} (\beta)$
    \end{itemize}
  \end{itemize}
\item[] Until $\forall A\in N$: $FI'[A] = FI[A]$
\end{itemize}

Dabei ist
\begin{align*}
  \mathit{first}_{FI} (\Eps) & = \{ \Eps \} \\
  \mathit{first}_{FI} (a\beta) & = \{ a \} \\
  \mathit{first}_{FI} (A\beta) & =
  \begin{cases}
    FI[A] & \Eps \notin FI[A] \\
    FI[A] \setminus \{\Eps\} \cup \mathit{first}_{FI} (\beta)
    & \Eps\in FI[A]
  \end{cases} 
\end{align*}

\textbf{Beispiel}: Betrachte eine  Grammatik arithmetische Ausdrücke
mit Startsymbol $S$ und $N = \{E, T, F\}$, $\Sigma = \{a, -, *\}$ und Produktionen
\begin{align*}
  E & \to T E' & E' & \to -T E' \mid \varepsilon \\
  T & \to F T' & T' & \to *F T' \mid \varepsilon \\
  F & \to a
\end{align*}

Tabelle der Werte von $FI[A]$ wobei Zeile $i$ den Werten in $FI$ nach
dem $i$-ten Schleifendurchlauf entspricht.

\begin{displaymath}
  \begin{array}{c||c|c|c|c|c|}
    FI & E & E' & T & T' & F \\\hline
    0  & \emptyset & \emptyset & \emptyset & \emptyset & \emptyset \\
    1  & \emptyset & \{ -, \Eps \} & \emptyset & \{ *, \Eps \} & , \{a\} \\
    2  & \emptyset & \{ -, \Eps \} & \{a\} & \{ *, \Eps \} & , \{a\} \\
    3  & \{a\} & \{ -, \Eps \} & \{a\} & \{ *, \Eps \} & , \{a\} \\
    4  & \{a\} & \{ -, \Eps \} & \{a\} & \{ *, \Eps \} & , \{a\} \\
  \end{array}
\end{displaymath}

Ergebnis nach vier Durchläufen. 

Anmerkung: first ist nicht die vollständige Lösung des Problems. Für
den Lookahead müssen auch noch die Symbole betrachtet werden, die
\emph{nach} einem bestimmten Nichtterminal auftreten können. Mehr dazu
in Vorlesung Compilerbau. 
}
%%% Local Variables:
%%% mode: latex
%%% TeX-master: "Info_3_Skript_WS2016-17"
%%% End:(\dots)
% \datenote{20.12.17}
\section{Turingmaschinen}
\newcommand{\M}{\mathcal{M}}
\newcommand{\godel}[1]{\ulcorner #1 \urcorner}
\newcommand{\D}{\mathcal{D}}
\newcommand{\mblank}{\text{\blank}}

\hide{
%\section{\acf{TM}}
1930er Jahre\\
Suche nach formalem Modell für maschinelle Berechenbarkeit
\begin{description}
	\item[Alan Turing:] (1912-1954) Turingmaschine 1936
	\item[Alonzo Church:] Lambdakalkül 1936
        \item[Emil Post:]  Postband 1936
	\item[Kleene, Sturgis:] partiell rekursive Funktionen
	\item[Chomsky:] Typ-0-Grammatiken 1956
\end{description}
\emph{Alan Turing:}\begin{minipage}[t]{.8\textwidth}
\begin{itemize}[parsep=0pt]
	\item Informatik, Logik
	\item Kryptographie (Enigma Entschlüsselung, Sprachverschlüsselung)
	\item KI (Turing-Test)
\end{itemize}\end{minipage}

außerdem: Turing-Award
}

{
\color{red}
TODO ganz kurze Einführung

\begin{itemize}
 \item Weiteres Maschinenmodell
 \item Maximale Eskalation alles was berechnet werden kann
 \item zahlen dafür hohen Preis: Nichttriviale Endlosschleifen
 \item werden sehen: Entspricht Typ-0 Grammatiken
\end{itemize}
}

\subsection{Turingmaschine \normalfont(informell)}

Turingmaschine ist eine Erweiterung endlicher Automaten mit den folgenden Zusätzen.
\begin{itemize}
 \item Die Eingabe steht auf einem Band das rechts und links der Eingabe unendlich viele Zellen hat.
 
 \begin{figure}[H]\centering
	\begin{tikzpicture}[every node/.style={block}]
		\node (1) {\blank};
		\node (2) [right=of 1] {b};
		\node (3) [right=of 2] {a};
		\node (4) [right=of 3] {n};
		\node (5) [right=of 4] {a};
		\node (6) [right=of 5] {n};
		\node (7) [right=of 6] {e};
		\node (8) [right=of 7] {\blank};
		\node (9) [right=of 8] {\blank};
		\node (10) [right=of 9] {\blank};
		\node (11) [right=of 10] {\blank};
		\node (last) [right=of 11] {\blank};
		
		\node (Kopf) [below=1.5em of 2, draw=none, text height=.5em] {Kopf}
		edge [->, shorten >=.5ex, semithick] (2);
		
		\node (TB) [below=.5em of 9, draw=none, text height=.5em,  anchor=north west] {Turingband};
		
		\draw (8.south) -- ($(8.south)-(0,1em)$) -- ($(TB.north west)-(0,.5em)$);
		
		% Open begin and end.
		\draw (1.north west) -- ++(-1cm,0) (1.south west)
		-- ++ (-1cm,0) (last.north east)
		-- ++ ( 1cm,0) (last.south east)
		-- ++ ( 1cm,0);
	\end{tikzpicture}
	\vspace{-1em}
	\caption{Turingband}
% 	\framebox{q} = Zustand
\end{figure}
 
 \item Der Kopf der Maschine kann nicht nur Zeichen lesen, sondern auch Zeichen schreiben.
 Der Kopf steht wie bei endlichen Automaten immer genau auf einer Zelle und liest in jedem Rechenschritt das Zeichen welches auf dieser Zelle geschrieben steht.
 Im Gegensatz zu endlichen Automaten kann der Kopf aber auch Zeichen schreiben und er tut dies in jedem Rechenschritt. 
 Dabei wird das aktuelle Zeichen der Zelle durch ein neues Zeichen ersetzt.
 Neues und altes Zeichen dürfen aber auch identisch sein.
 \item Wir haben nicht nur ein Eingabealphabet $\Sigma$, sondern auch ein Bandalphabet $\Gamma$.
 Jedes Zelle des Bandes ist mit einem Zeichen des Bandalphabets beschriftet.
 Wir haben in unserem Bandalphabet ein spezielles Zeichen $\blank\in\Gamma$ das wir Blank nennen und welches ein Feld als ,,leer'' markiert.
 Zu beginn sind alle Zellen auf denen keine Eingabe steht mit den Blank Zeichen beschriftet.
 \item Der Kopf der Maschine kann nicht nur nach rechts bewegt werden, er darf auch nach links bewegt werden oder seine Position beibehalten.
\end{itemize}

 Wie bei endlichen Automaten steht der Kopf zu Beginn auf dem ersten (am weitesten Links stehenden) Zeichen der Eingabe.
 
 \smallskip
 
 Wir geben die Transitionsfunktion einer Turingmaschine mit Hilfe einer Tabelle, der sogenannten Turingtabelle an.
 
 \begin{tabu}{>{\bfseries}X[.27]X[.62]}
% 	Turingtabelle\newline\normalfont
	\begin{tabular}{|*5{c|}}
		q & a & q' & a' & r \\\hline
		&&&&\\
		&&&&
	\end{tabular}
	& 	Die Bedeutung eines Tabelleneintrages ist dabei: Wenn \ac{TM} in Zustand $q$ und Kopf liest
        gerade Symbol $a\in\Gamma$ dann wechsle in Zustand $q'$,
        schreibe $a'$ (über altes $a$) und bewege den Kopf gemäß
        $d\in\{L,R,N\}$ 
\end{tabu}\

Eine Turingmaschine arbeitet schrittweise und in jedem Rechnschritt wir eine Zeile der Turingtabelle angewendet.
Bei nichtdeterministischen Turingmaschinen kann es Situationen geben bei denen mehrere Einträge der Turingtabelle angewendet werden können,
bei deterministischen Turingmaschinen kann immer höchstes ein Tabelleneintrag angewendet werden.
In beiden Fällen (deterministisch/nichtdeterministisch) darf es auch den Fall geben dass ein Eintrag angewendet werden kann,
in solch einem Fall hält die Turingmaschine an.
Nachdem die Turingmaschine angehalten hat, prüfen wir ob der aktuelle Zustand ein akzeptierender Zustand ist. Falls ja wird die Eingabe akzeptiert, falls nein wird die Eingabe verworfen.

Wurde eine Eingabe nicht akzeptiert kann dies also zwei Ursachen haben
\begin{enumerate}
 \item die Turingmaschine war nach dem Anhalten nicht in einem akzeptierenden Zustand
 \item die Turingmaschine hat niemals angehalten.
\end{enumerate}

Wir werden deterministische Turingmaschinen nur verwenden um Sprachen zu definieren, sondern auch um (partielle) Funktionen vom Typ $\Sigma^*\rightarrow\Sigma^*$ zu definieren.
Das Funktionsargument ist dabei das was zu Beginn auf dem Band steht, das Resultat das was am Ende auf dem Band steht (ohne den Teil links des Kopfes).
Um Sicherzustellen dass die Resultate Element von $\Sigma^*$ sind projezieren wir den Bandinhalt nach dem Halten auf $\Sigma$.

Da die Maschine in endlich vielen Schritten nur endlich viele Zeichen schreiben kann ist klar dass das Resultat nach dem Halten ein endliches Wort ist.
Hält die Maschine nicht ist der Funktionswert für die entsprechende Eingabe undefiniert.




% \hide{
% Ein primitives Rechenmodell:
% 
% \vspace{-.5em}
% \begin{tabu}{>{\bfseries}X[.22]X[.72]}
% 	Turingband & \vspace{-1em}\begin{itemize}[leftmargin=1em,parsep=0pt,topsep=0pt]
% 	\item unendliches Band
% 	\item Jedes Feld enthält ein Symbol aus einem Bandalphabet $\Gamma$.
% 	\item uninitialisiert: $\blank\in\Gamma$ (Blank) ist ein spezielles Symbol welches ein Feld als ,,leer'' markiert
% 	\end{itemize}
% 	\\
% 	Kopf & \vspace{-1em}\begin{itemize}[leftmargin=1em,parsep=0pt,topsep=0pt]
% 	\item zeigt immer auf ein Feld
% 	\item nur am Kopf kann die \ac{TM} ein Zeichen lesen und schreiben
% 	\item kann nach rechts /links bewegt werden
% 	\end{itemize}\\
% 	Zustand & \vspace{-1em}\begin{itemize}[leftmargin=1em,parsep=0pt,topsep=0pt]
% 	\item kann verändert werden
% 	\item kann gelesen werden
% 	\item es gibt nur endlich viele Zustände
% 	\end{itemize}\\
% 	Turingtabelle\newline\normalfont
% 	\begin{tabular}{|*5{c|}}
% 		q & a & q' & a' & d \\\hline
% 		&&&&\\
% 		&&&&
% 	\end{tabular}
% 	& $\sim$ Programm $\sim$ Transitionsfunktion \newline
% 	$\rightarrow$ Wenn \ac{TM} in Zustand $q$ und Kopf liest
%         gerade Symbol $a\in\Gamma$ dann wechsle in Zustand $q'$,
%         schreibe $a'$ (über altes $a$) und bewege den Kopf gemäß
%         $d\in\{L,R,N\}$ 
% \end{tabu}\

\goodbreak

\begin{samepage}
	\begin{Bsp*}\ 
		\vspace{-2em}
		\begin{figure}[H]
                   \begin{center}
			\begin{tikzpicture}[every node/.style={block}]
				\node (1) {\blank};
				\node (2) [right=of 1] {\blank};
				\node (3) [right=of 2] {\cancel{b}};
				\node (4) [right=of 3] {a};
				\node (5) [right=of 4] {n};
				\node (6) [right=of 5] {a};
				\node (7) [right=of 6] {n};
				\node (8) [right=of 7] {e};
				\node (9) [right=of 8] {\cancel{\blank}};
				\node (10) [right=of 9] {\blank};
				\node (last) [right=of 10] {\blank};
				
				\node (q1) [draw=none, above=-2pt of 3] {\blank};
				\node (q3) [draw=none, above=-2pt of 9] {b};
				
				% Open begin and end.
				\draw (1.north west) -- ++(-1cm,0) (1.south west)
				-- ++ (-1cm,0) (last.north east)
				-- ++ ( 1cm,0) (last.south east)
				-- ++ ( 1cm,0);
			\end{tikzpicture}
			\\[2ex]
                        \begin{displaymath}
                          \begin{array}{|c|c||c|c|c||l}
                            \hline
			  	q_0 &  x & q_x & \blank & R  & x\ne\blank\\\hline
                                q_x & \blank & q_3 & x & L\\\hline
                                q_x & y & q_x & y & R & y \ne \blank \\\hline
                                q_3 & y & q_3 & y & L & y \ne \blank\\\hline
                                q_3 & \blank & q_4 & \blank & R\\\hline
                          \end{array}
                        \end{displaymath}
                        \end{center}
			\caption{Bsp.: Turingmaschine---Füge das erste
                        Zeichen am Ende der Eingabe an}
		\end{figure}
	\end{Bsp*}
\end{samepage}
%
% Was kann die \ac{TM} ausrechnen?
% \begin{enumerate}
% 	\item Die \ac{TM} kann eine Sprache $L\subseteq\Sigma^*$ erkennen.
% 	\begin{itemize}
% 		\item Wörter müssen auf Band repräsentierbar sein $\Sigma\subseteq\Gamma\setminus\{\blank\}$
% 	\end{itemize}
% 	Ein Wort $w$ wird von einer \ac{TM} erkannt, wenn
% 	\begin{itemize}
% 		\item zu Beginn steht nur $w$ auf dem Band, alle anderen Zellen $=\blank$
% 		\item Kopf auf erstem Zeichen von $w$
% 		\item Zustand ist Startzustand $q_0$
% 		\item Abarbeitung der \ac{TT}
% 		\item Falls \ac{TM} nicht terminiert: $w\notin L$
% 		\item Falls \ac{TM} terminiert betrachte den errechneten Zustand $q$.\\
% 		Falls $q\in F$ (akzeptierender Zustand), dann $w\in L$, anderenfalls $w\notin L$
% 	\end{itemize}
% 	
% 	\begin{Bsp*}
% 		\begin{flalign*}
% 			\Sigma &=\{0,1\} &\\
% 			L &=\left\{w\in\Sigma^* \mid \,w\text{ ist Palindrom}\right\}\\
% 			Q &= \{q_0,q_1,q_r^0, q_r^1, {q_r^0}', {q_r^1}', q_l^0, q_l^1 \} \quad F=\{q_1\}
% 		\end{flalign*}
% 		\begin{tabular}{@{}*6{M{l}}}
% 			q_0      & \blank & q_1      & \blank & N & q_1 \x q_1\x N\\
% 			q_0      & 0      & q_r^0    & \blank & R\\
% 			q_0      & 1      & q_r^1    & \blank & R
% 			\\ \cmidrule{1-5}
% 			q_r^0    & \blank & q_1      & \blank & N\\
% 			q_r^0    & 0      & {q_r^0}' & 0      & R\\
% 			q_r^0    & 1      & {q_r^0}' & 1      & R\\
% 			{q_r^0}' & \blank & q_l^0    & \blank & L & q_l\->\text{prüfe $0$, fahre zum linken Rand und weiter mit }q_0\\
% 			{q_r^0}' & 0      & {q_r^0}' & 0      & R & \multirow{2}{*}{\hspace{-1em}$\begin{rcases}\\[1em]\end{rcases}$ Rechtslauf} \\
% 			{q_r^0}' & 1      & {q_r^0}' & 1      & R
% 		\end{tabular}\\[.5em]
% 		\begin{tabular}{@{}*6{M{l}}|}
% 			\multicolumn{6}{@{}l|}{Alternative 1:}\\
% 			\multicolumn{6}{@{}l|}{\ac{TM} hält bei jeder Eingabe an.}\\[.5em]
% 			q_l^0 & \blank & ---  & --- & --- & \<-\text{Halt}\\
%                         q_l^0 & 0   & q_l    & \blank & L &\\
% 			q_l^0 & 1   & ---  & ---      & --- & \<-\text{Halt}
% 		\end{tabular}\quad\begin{tabular}{@{}*5{M{l}}@{ }l}
% 		\multicolumn{6}{@{}l}{Alternative 2:}\\
% 		\multicolumn{6}{@{}l}{\ac{TM} hält nur bei Palindrom an.}\\[.5em]
% 		q_l^0 & \blank & q_l^0 & 1      & N & \multirow{3}{*}{\scalebox{2.9}{\rotatebox[origin=rb]{-90}{$\curvearrowleftright$}}}\\
% 		q_l^0 & 0      & q_l   & \blank & L\\
% 		q_l^0 & 1      & q_l^0 & \blank & N
% 		\end{tabular}
% 	\end{Bsp*}
% 	
% 	\item Eine \ac{TM} berechnet eine \emph{partielle} Funktion $f: \Sigma^*\dashrightarrow\Sigma^*$\\
% 	Die Berechnung von $f(w),\ w\in\Sigma^*$
% 	\begin{itemize}
% 		\item $w$ auf leeres Band
% 		\item Kopf auf erstes Zeichen, Standardzustand $q_0$
% 		\item Abarbeitung der \ac{TT}
% 		\item Falls terminiert und Kopf steht auf dem ersten Symbol von $v\in\Sigma^*$\\
% 		Dann $f(w)=v$
% 	\end{itemize}
% \end{enumerate}
% \begin{alignat*}{3}
% 	\text{Schreibe}&\quad& A &\-->B &\quad& \text{totale Funktion von $A$ nach $B$}\\
% 	&& A&\dashrightarrow B && \text{partielle Funktion von $A$ nach $B$}
% \end{alignat*}
% \begin{Bsp} % 2.1
% 	$\Sigma=\{0,1\}$\\
% 	Gesucht eine \ac{TM}, die die Nachfolgerfunktion auf natürliche Zahlen in Binärdarstellung berechnet.\\
% 	Annahme: niederwertigste Stelle der Zahl am Anfang der Eingabe.\medskip\\
% 	\begin{tabular}{@{}M{l}@{ } *5{M{l}} @{ }l}
% 		\xrightarrow{\text{Start}} & q_0 & \blank & q_2 & 1 & L \\
% 		& q_0 & 0      & q_1 & 1 & L \\
% 		& q_1 & 1      & q_0 & 0 & R \\[.5em]
% 		& q_1 & \blank & q_1 & \blank & N & \<-Halt\\
% 		& q_1 & 0      & q_2 & 0 & L & \multirow{2}{*}{$\begin{rcases}\\[1em]\end{rcases}$ Linksmaschine}\\
% 		& q_1 & 1      & q_2 & 1 & L
% 	\end{tabular}
% \end{Bsp}
% }

{
\color{red}
Sie finden zwei weitere Beispiele für Turingmaschinen in den Folien die auf der Vorlesungswebsite verlinkt sind.

\url{https://swt.informatik.uni-freiburg.de/teaching/WS2017-18/info3}
}

\subsection{Formalisierung der \ac{TM}} % 2.2
\begin{Def}[name={[\acs*{TM}]}]\label{def:4.tm} % 2.1
	Eine \ac{TM} ist ein 7-Tupel
	\begin{equation*}
		\M=\left(\Sigma,Q,\Gamma,\delta,\qinit,\blank,F\right)\\
	\end{equation*}
Dabei ist
	\begin{itemize}
		\item $\Sigma$ ist Alphabet,
		\item $Q$ ist endliche Menge deren Elemente wir Zustände nennen,
		\item $\Gamma\supsetneq\Sigma$ ist Menge die wir \emph{Bandalphabet} nennen,
		\item $\delta: Q\times\Gamma\--> \Powerset(Q\times\Gamma\times\{R,L,N\})$ eine Funktion, die wir \emph{Transitionsfunktion} nennen,
		\item $\qinit\in Q$ ein Zustand, den wir \emph{Startzustand} nennen,
		\item $\blank\in\Gamma\setminus\Sigma$ ein Zeichen dass wir \emph{Blank} nennen und
		\item $F\subseteq Q$ eine Teilmenge der Zustände, deren Elemente wir \emph{akzeptierende} Zustände nennen.
	\end{itemize}
        Die TM $\M$ heisst \emph{deterministisch}
        (DTM), falls $\forall q\in Q, \forall a\in\Gamma, |\delta
        (q,a)| \le 1$.\\
        Ansonsten ist $\M$ \emph{nichtdeterministisch} (NTM).
\end{Def}
Im Folgenden sei $\M=(\Sigma, Q, \Gamma, \delta, \qinit, \blank, F)$ eine \ac{TM}.


Die Konfiguration einer Turingmaschine wird durch drei Dinge beschrieben:
\begin{enumerate}
 \item Der aktuelle Zustand
 \item Der Bandinhalt
 \item Die Position des Kopfes
\end{enumerate}
Die scheinbar ``natürlichste'' Beschreibung der Konfiguration wäre also ein Tripel aus
$Q\times(\mathbb{Z}\rightarrow\Gamma)\times\mathbb{Z}$.
Dies würde aber erfordern dass wir jeder Zelle des Bandes mit einer ganzen Zahl identifizieren
und wir müssten uns eine endliche Repräsentation für die Abbildung $\mathbb{Z}\rightarrow\Gamma$ überlegen.
Wir verwenden statt dessen den folgenden ``Trick'' und beschreiben eine Konfiguration durch ein Tripel $(v,q,w)$.
Dabei beschreibt
\begin{itemize}
	\item $v$ einen endlichen Suffix der Bandhälfte links des Kopfes der alle nicht-Blank Zeichen enthält,
	\item $q$ den aktuellen Zustand und
	\item $w$ einen endlichen Präfix der verbleibenden Bandhälfte (also unter Kopf und rechts davon) der alle nicht-Blank Zeichen enthält.
\end{itemize}
\begin{figure}[H]\centering
	\begin{tikzpicture}[every node/.style={block}, decoration={brace, amplitude=5pt}]
		\node (A) {$v$};
		\node (B) [right=of A] {$a$};
		\node (C) [right=of B] {$w'$};
		\node (D) [below=1em of B, draw=none] {Kopf; Zustand $q$}
		edge [->, shorten >=.5ex, semithick] (B);
		
		\draw [decorate, semithick] (B.north west) -- (C.north east)
		node [draw=none,above,midway] {$w$};
		\draw (A.north west) -- ++(-1cm,0) (A.south west)
		-- ++ (-1cm,0) (C.north east)
		-- ++ ( 1cm,0) (C.south east)
		-- ++ ( 1cm,0);
	\end{tikzpicture}
	\caption{Informelle graphische Darstellung einer Turingmaschinenkonfiguration}
\end{figure}

\begin{Def}[name={[Konfiguration einer \acs*{TM}]}] % 2.2
	Die Menge der Konfiguration einer \ac{TM} ist $\Konf(\M)=\Gamma^*\times Q\times\Gamma^+$
	Die \emph{Startkonfiguration} bei Eingabe $w$ ist: $(\Eps,\qinit,w)$.
	Eine \emph{Haltekonfiguration} ist eine Konfiguration $(v,q,aw)$ bei der $\delta(q,a)=\emptyset$ gilt.
\end{Def}
Notation: Wenn aus dem Kontext klar wird welcher Buchstabe zu welche Komponente des Tripels gehört, dürfen wir die Klammern und Kommata auch weglassen. Wir schreiben z.B. die Startkonfiguration als \ $\qinit w$\ .
%



\begin{Def}[name={[Rechenschrittrelation]}] % 2.3
	Die \emph{Rechenschrittrelation}
	\[ {\vdash} \subseteq \Konf(\M)\x\Konf(\M) \]
	ist definiert durch
	\footnote{Die beiden Spezialfälle dass $q$ ``ganz rechts'' oder ``ganz links'' steht wurde in der Vorlesung am 20.12.2017 vergessen. Die Definition wird am 10.1.2018 nochmal kurz besprochen werden.}
	\begin{alignat*}{3}
		1.&\ & v qaw &\vdash v q'a'w &\quad& \text{falls }\delta(q,a)=(q',a',N)\\
		2.&& v qaw &\vdash v a'q'w && \text{falls }\delta(q,a)=(q',a',R), w\neq \Eps\\
		&&&\phantom{{}\vdash{}} v a'q'\blank && \ruleplaceholder{\widthof{falls $\delta(q,a)=(q',a',R)$}}, w=\Eps\\
		3.&& qaw &\vdash q'\blank a'w && \delta(q,a)=(q',a',L)\\
		&& vbqaw &\vdash vq'ba'w && \ruleplaceholder{\widthof{$\delta(q,a)=(q',a',L)$}}\  b\in\Gamma
	\end{alignat*}
\end{Def}
Analog zu den vorigen Kapiteln schreiben wir
% $\vdash$ Einzelschritt, gesuchte Relation für endlich viele Schritte \smallskip\\
	${\vdash^*} \subseteq \Konf(\M) \x \Konf(\M)$ 
	für die reflexive transitive Hülle von $\vdash$.
\begin{itemize}
 \item Das Symbol $\vdash$ steht also für eine Relation auf $\Konf(\M)$ die einzelne Berechnungsschritte beschreibt und
 \item der Ausdruck $\vdash^*$ steht für eine Relation auf $\Konf(\M)$ die eine endliche Sequenz von  Berechnungsschritten beschreibt.
\end{itemize}

 
%
%
\begin{Def}[Die von \acs*{TM} $\M$ akzeptierte Sprache] % 2.5
	\begin{align*}
		L(\M)=\{ w\in\Sigma^* \mid{}
		& \qinit w \vdash^*uqv\\
		&uqv \text{ Haltekonfiguration}\\
		&q\in F\}
	\end{align*}
\end{Def}


\begin{Def}[name={[Terminierung]}]
 Eine \ac{DTM} $\M$ \emph{terminiert} auf Eingabe $w$ 
 falls es eine Haltekonfiguration $uq'v$ gibt sodass $\qinit w\vdash^*uq'v$ gilt.
\end{Def}

Falls eine \ac{DTM} $\M$ ein Wort $w$ nicht akzeptiert kann dies zwei Ursachen haben:
\begin{enumerate}
 \item $\M$ terminiert nicht auf $w$.
 \item Der Zustand der Haltekonfiguration von $\M$ auf $w$ ist nicht akzeptierend.
\end{enumerate}

% \emph{Beachte:}
% $w\notin L(\M)\begin{casesarrows}
% \M\text{ kann anhalten}          \\
% \M\text{ kann nicht terminieren}
% \end{casesarrows}$

\begin{Def}[Die von einer \ac{DTM} $\M$ berechnete Funktion] % 2.6
Sei $\M$ eine \ac{DTM} dann ist die von $\M$ \emph{berechnete partielle
\footnote{Im Gegensatz zu einer Funktion muss eine partielle Funktion nicht für alle Elemente des Urbildes definiert sein.
Ist $f$ keine Funktion, sondern nur eine partielle Funktion von $X$ nach $Y$ so schreiben wir $f: X\nrightarrow Y$ statt $f: X\rightarrow Y$. }
Funktion} $f_\M:\Sigma^*\nrightarrow\Sigma^*$ wie folgt definiert.
$$
f_\M(w)= 
\begin{cases}
 v &\text{falls } \exists (u,q,v')\in \Konf(\M) \text{ sodass }
 \begin{array}{ll}
  & \qinit w \vdash^*uqv'\\
  \text{und} & uqv'\text{ ist Haltekonfiguration}\\
  \text{und} & v=\out( v')
 \end{array}
\\
 \mathit{undef.} & \text{sonst}
\end{cases}
$$
Dabei ist $\out$ die Funktion, die ein Wort auf das Eingabealphabet $\Sigma$ projeziert. 
Wie definieren $\out:\Gamma^* \-> \Sigma^*$ formal wie folgt.
	\begin{alignat*}{3}
		&\out(\Eps) &&= \Eps\\
		&\out(au) &&= a\cdot\out(u) &\quad& a\in\Sigma\\
		&\out(bu) &&= \out(u) && b\in\Gamma\setminus\Sigma
	\end{alignat*}
\emph{Beobachtung:} Die Funktion $f_\M(w)$ ist genau für die Wörter definiert, 
für die $\M$ angesetzt auf $w$ terminiert.
\end{Def}

% 
% 	\begin{alignat*}{3}
% 		&&&f_\M:\Sigma^*\nrightarrow\Sigma^*\\
% 		&&&f_\M(w)=v\\
% 		&\text{ falls } &&\qinit w \vdash^*uqv'&\quad&\text{Haltekonf.}\\
% 		&\text{ und } &&v=\out( v')\\[.5em]
% 		&\out:\Gamma^* &&\-> \Sigma^*\\
% 		&\out(\Eps) &&= \Eps\\
% 		&\out(au) &&= a\cdot\out(u) &\quad& a\in\Sigma\\
% 		&\out(bu) &&= \out(u) && b\in\Gamma\setminus\Sigma
% 	\end{alignat*}
% \end{Def}
% \emph{Beachte:} Falls $\qinit w$ nicht terminiert, dann ist $f_\M(w)$ nicht definiert.
% 
% Eine \ac{TM} $\M$ terminiert nicht bei Eingabe $w$, falls für alle $uq'v$, so dass $\qinit w\vdash^*uq'v$\\
% $uq'v$ ist keine Haltekonfiguration.

\subsection{\ac{TM} Programmierung mit Hilfe von Flußdiagrammen}
\datenote{10.01.2018}
% 
Auch für einfache Turingmaschinen können die Turingtafeln sehr groß werden.
Wir führen deshalb einen Formalismus ein, der uns erlaubt mehrere kleine Turingmaschinen zu einer Großen zusammenzubauen.

Idee: Wir nehmen einen gerichteten Graphen und beschriften dessen Knoten mit Turingmaschinen und dessen Kanten mit einer Teilmenge des Bandalphabets.
Eine Kante $\M_1\stackrel{\{a,b,c\}}{\longrightarrow}\M_2$ bedeutet dann:
Wenn die TM $\M_1$ hält und $a$, $b$ oder $c$ unter dem Kopf steht, dann kann TM $\M_2$ beginnend mit ihren Startzustand auf dem aktuellen Bandinhalb weitermachen.
Voraussetzung ist dass alle TMs das gleiche Bandalphabet und das gleiche Blanksymbol verwenden.

\begin{Bsp} Wir definieren uns zunächst für beliebiges $\Sigma$ und $\Gamma$ drei sehr einfache Turingmaschinen. 
 \begin{itemize}
  \item 1-Schritt Rechtsmaschine $\M_r$
  
  Geht für beliebigen Bandinhalt einen Schritt nach rechts und hält dann.
  
  $\M_r=\left(\Sigma,\{q_0,q_1\},\Gamma,\delta,q_0,\blank,\{\}\right)$ mit
  $\delta(q,x)=\begin{cases}\{(q_1, x, R)\} & \text{ falls } q = q_0\\ \{\} & \text{ falls } q = q_1\end{cases}$
  
  \item 1-Schritt Linksmaschine $\M_l$
  
  Geht für beliebigen Bandinhalt einen Schritt nach links und hält dann.

  $\M_l=\left(\Sigma,\{q_0,q_1\},\Gamma,\delta,q_0,\blank,\{\}\right)$ mit
  $\delta(q,x)=\begin{cases}\{(q_1, x, L)\} & \text{ falls } q = q_0\\ \{\} & \text{ falls } q = q_1\end{cases}$
  
  \item Für jedes Zeichen $y\in\Gamma$ die Druckmaschine $\mathcal{D}_y$.
  
  Schreibt das Zeichen $y$ auf die Zelle über der sich der Kopf befindet und hält dann.

  $\mathcal{D}_y=\left(\Sigma,\{q_0,q_1\},\Gamma,\delta,q_0,\blank,\{\}\right)$ mit
  $\delta(q,x)=\begin{cases}\{(q_1, y, N)\} & \text{ falls } q = q_0\\ \{\} & \text{ falls } q = q_1\end{cases}$

 \end{itemize}
\end{Bsp}

\begin{Bsp}
 Mit Hilfe von Flußdiagrammen definieren wir uns nun zwei weitere Turingmaschinen
  \begin{itemize}
  \item Rechtsmaschine $\M_R$
  
  Geht für beliebigen Bandinhalt nach rechts bis ein Blanksymbol erreicht ist und hält dann.
  
  \begin{tikzpicture}
   \node (0) at (0,0) {$\M_r$};
   \node (init) at (-1.1,.2) {};
   \draw[->] (init) to (0);
   \draw[->,loop right] (0) to node {$\Gamma\backslash\{\blank\}$} (0);
  \end{tikzpicture}

  
  \item Linksmaschine $\M_L$
  
  Geht für beliebigen Bandinhalt nach links bis ein Blanksymbol erreicht ist und hält dann.

  \begin{tikzpicture}
   \node (0) at (0,0) {$\M_l$};
   \node (init) at (-1.1,.2) {};
   \draw[->] (init) to (0);
   \draw[->,loop right] (0) to node {$\Gamma\backslash\{\blank\}$} (0);
  \end{tikzpicture}
 \end{itemize}
\end{Bsp}

Wir definieren das Flußdiagram und nun formal wie folgt.


\begin{Def}%[Flußdiagram]
\draftnote{12.01.2018}
Ein \emph{Flußdiagram} ist ein 7-Tupel $G=(\Sigma, \Gamma, \blank,V,E,v_0,L_V,L_E)$, dabei ist
\begin{itemize}
 \item $\Sigma$ ein Alphabet das wir \emph{Eingabealphabet} nennen,
 \item $\Gamma\supsetneq \Sigma$ ein Alphabet das wir \emph{gemeinsames Bandalphabet} nennen,
 \item $\blank\in \Gamma$ ein Zeichen, das wir \emph{gemeinsames Blanksymbol} nennen,
 \item $(V,E)$ ein gerichteter Graph,
 \item $v^\mathsf{init}\in V$ ein Knoten den wir \emph{Startknoten} nennen,
 \item $L_V$ eine Abbildung, die jedem Knoten eine Turingmaschine zuordnet und
 \item $L_E$ eine Abbildung, die jeder Kante $(v,v')\in E$ eine Teilmenge des Alphabets zuordnet.
\end{itemize}
\end{Def}


Notation: Bis zum Ende des Kapitels schreiben wir
$\M_v=\left(\Sigma,Q_v,\Gamma_v,\delta,\qinit,\blank,F_v\right)$
für die Turingmaschine $L_V(v)$, also die Turingmaschine die dem Knoten $v$ zugeordnet ist.

\begin{Def}%[Flußdiagram]
Die von einem Flußdiagram $G=(\Sigma, \Gamma, \blank,V,E,v^\mathsf{init},L_V,L_E)$ definierte Turingmaschine ist 
$\M=\left(\Sigma,Q,\Gamma,\delta,\qinit,\blank,F\right)$ wobei
\begin{itemize}
 \item $Q=\overset{.}{\bigcup\limits_{v\in V}} Q_v$ \footnote{Um eine disjunkte Vereinigung zu erreichen müssen wir also ggf. Zustände umbenennen.}
 \item $\qinit = \qinit_{v^\mathsf{init}}$
 \item $\delta(q,x) = 
 \begin{cases}
 (q',x',d) & \text{ falls } q\in Q_v \text{ und } (q',x',d)\in\delta_v(q,x)\\
 (q'',x,N) & \text{ falls } q\in Q_v \text{ und } 
      \begin{array}{l}
       \delta_v(q,x)=\{\}\\
       (v,v')\in E\\
       x\in L_E((v,v'))\\
       q''=\qinit_{{v'}^\mathsf{init}}\\
      \end{array}\\
 \{\} & \text{ sonst }\\
 \end{cases}$
 \item $F=\overset{.}{\bigcup\limits_{v\in V}} F_v$ \qedhere
\end{itemize}
\end{Def}

Konvention: Analog zu Startzuständen von endlichen Automaten verwenden wir eingehende Pfeile um den Startknoten eines Flußdiagrams zu Kennzeichnen.

\begin{Bsp}
Wir wollen nun eine TM konstruieren, die die wie folgt definierte Subtraktion von natürlichen Zahlen implementiert.
% 
% mit Hilfe eines Flußdiagrams eine .
% 
$$
f: \N\times\N\rightarrow \N
\qquad \qquad
f(x,y)=
\begin{cases}
 x - y & \text{ falls } x\geq 0\\
 0  & \text{ sonst}
\end{cases}
$$

Wir nehmen dabei eine auf Strichsymbolen basierende Unärcodierung für die Operanden
und nehmen an dass das Zeichen $\#$ verwendet wird um die Operanden $x$ und $y$ in der Eingabe zu trennen.
Wir haben also $\Sigma=\{I,\#\}$ und für Minuend $x$ und Subtrahend $y$ hat die Eingabe hat die folgende Form. 
$$\underbrace{I\ldots I}_{x\text{-mal}}\#\underbrace{I\ldots I}_{y\text{-mal}}$$


Idee: Lösche für jeden $y$ Strich einen $x$ Strich. Lösche falls nötig (Fall $y>x$) verbleibendene $y$-Striche. Lösche anschließend das $\#$ Zeichen.
Wir verwenden die übliche Notation für das Blanksymbol, 
benötigen keine weiteren Zeichen für unsere Berechnung und verwenden daher das Bandalphabet $\Gamma:=\Sigma\cup\{\mblank\}$.

  \begin{tikzpicture}
   \node (0) at (0,0) {$\M_R$};
   \node (init) at (-1.1,.2) {};
   \node[right= of 0] (1) {$\M_l$};
   \node[right= of 1] (2) {$\D_\mblank$};
   \node[right= of 2] (3) {$\M_L$};
   \node[right= of 3] (4) {$\M_r$};
   \node[right= of 4] (5) {$\D_\mblank$};
   
   \node[below= of 4] (6) {$\D_\mblank$};
   \node[right= of 6] (7) {$\M_r$};
   \node[right= of 7] (8) {$\D_\mblank$};
   \node[below= of 1] (9) {$\D_\mblank$};
   
   \draw[->] (init) to (0);
   \draw[->] (0) to node[auto] {$\Gamma$} (1);
   \draw[->] (1) to node[auto] {$\{I\}$} (2);
   \draw[->] (2) to node[auto] {$\Gamma$} (3);
   \draw[->] (3) to node[auto] {$\Gamma$} (4);
   \draw[->] (4) to node[auto] {$\{I\}$} (5);
   \draw[->, in=90, out=90] (5) to node[auto] {$\Gamma$} (1);
   \draw[->] (4) to node[auto] {$\{\#\}$} (6);
   \draw[->] (6) to node[auto] {$\Gamma$} (7);
   \draw[->] (7) to node[auto] {$\{I\}$} (8);
   \draw[->, in=-90, out=-90] (8) to node[auto] {$\Gamma$} (6);
   \draw[->] (1) to node[auto] {$\{\#\}$} (9);
  \end{tikzpicture}
  
  Die obere Schleife beschreibt das simultane Enfernen von Strichen in $x$ und $y$.
  Die erste Abzweigung nach unten wird genommen wenn der Fall $y=0$ eintritt.
  Die untere Schleife wird ausgeführt wenn $y$ größer als $x$ war und dient dazu die verbleibenden Zeichen auf dem Band zu löschen 
  um die unärcodierte $0$ (also das leere Wort) zu erzeugen.
\end{Bsp}
  
\subsection{Varianten von \ac{TM}s}
Es gibt sehr viele Varianten von Turingmaschinen.
Diese sind äquivalent zu \autoref{def:4.tm} im Sinne dass sie nicht mehr Sprachen akzeptieren oder mehr Funktionen berechnen können.
Je nach Anwendung können diese Varianten aber bequemer sein. Wir stellen in diesem Unterabschnitt zwei weitere Varianten vor.

\begin{itemize}
% 	\item Endlicher Speicher\\
% 	Zum Abspeichern eines Elements aus endl. Menge $A$ verwende
% 	\[ Q'=Q\x A \]
	\item Mehrspurmachinen
	\begin{figure}[H]\centering
		{\renewcommand{\arraystretch}{0.8}
		\begin{tabu} to .5\textwidth {X[.35]|X[.65]}
			&\\\hline
			&\\\hline
			&\\\hline
			&\\\hline
			&
		\end{tabu}}
		\caption{Mehrspurmachine}
	\end{figure}
	
	Eine $k$-Spur \ac{TM} kann gleichzeitig $k\geq 1$ Symbole $\<- \Gamma$ unter dem Kopf lesen.\\
	Kann durch Standard \ac{TM} simuliert werden:
	\[ \Gamma' = \Sigma \overset{.}{\cup} \Gamma^k\text{ mit } \blank'=\blank^k \]
	\dots vereinfacht die Programmierung\\
	\begin{Bsp*}
		Schulalg. für binäre Addition, Multiplikation
	\end{Bsp*}
	\item \emph{Mehrbandmachinenen}\\
	Eine $k$-Band \ac{TM} besitzt $k\geq1$ Bänder und $k$ Köpfe, die bei jedem Schritt lesen, schreiben und sich unabhängig voneinander bewegen.
	\[ \delta_K:Q\x\Gamma^k \-> Q\x\Gamma^k\x \{R,L,N\}^k \]
	\item keine herkömmliche \ac{TM} (für $k>1$)
	\item kann durch $2k+1$ Spur \ac{TM} simuliert werden:\\
	\begin{tabular}{lll}
		Spur\\
		1 & \ruleplaceholder[ Band 1 ]{.5\linewidth}\\
		2 & \hspace{.23\linewidth}\# Kopf für Band 1\\
		3 & \ruleplaceholder[ Band 2 ]{.5\linewidth}\\
		4 & \multicolumn1r{\# Kopf\qquad\ }\\
		& \vdots\\
		$2k$ & \hspace{.09\linewidth}\dots\hspace{.09\linewidth} \# Kopf $k$\\
		$2k+1$ & \# &\#\#\\
		& linker Rand & rechter Rand
	\end{tabular}
\end{itemize}
\vspace{1em}

\begin{Satz}[name={[Simulation von $k$-Band \acs*{TM} durch 1-Band \acs*{TM}]}]
	Eine $k$-Band \ac{TM} kann durch eine 1-Band \ac{TM} simuliert werden.\quad $M=(Q\dots)$
\end{Satz}
\begin{proof}
	Zeige: ein Schritt der $k$-Band \ac{TM} wird durch endlich viele Schritte auf einer 1-Band \ac{TM} simuliert.
	\begin{enumerate}
		\item Schritt: Kodierung der Konfiguration der $k$-Band \ac{TM}\\
		Definiere $M'$ als \ac{TM} mit $2k+1$ Spuren und $\Gamma'=\Gamma\cup\{\#\}$
		\begin{itemize}
			\item Die Spuren $1,3,\dots,2k-1$ enthalten das entspr. Band von $M$: Band $i\<->$ Spur $2i-1$
			\item Die Spuren $2,4,\dots,2k$ sind leer bis auf eine Marke \#, die auf Spur $2i$ die Position des Kopfes auf Band $i$ markiert
			\item Spur $2k+1$ enthält\\
			\#\phantom{\#} Marke für linken Rand\\
			\#\# Marke für rechten Rand\\
			Zwischen den beiden Marken befindet sich der bearbeitete Bereich des Bands. D.h. die \ac{TM} arbeitet zwischen der linken und rechten Marke und schiebt die Marken bei Bedarf weiter.
		\end{itemize}
		
		\item Schritt: Herstellen der Start-Konfiguration.\\
		Annahme: Eingabe für $M$ auf Band 1\\
		Jetzt Eingabe (für $M'$) $w = a_1\dots a_n$
		\begin{enumerate}
			\item Kopiere $w$ auf Spur 1
			\item Kopf setzen auf Spur $2,\dots,2k$ an die Position des ersten Symbols von $w$
			\item auf Spur $2k+1$: \verb*!# ##!
	\end{enumerate}
	\begin{tabular}{*2{M{l}}}
		2k+1 & \#\blank\#\#\\
		2k & \#\\
		2k-1 & \blank\\
		\vdots\\
		4 & \#\\
		3 & \blank\\
		2 & \#\\
		\text{Spur }1 & a_1a_2\dots a_2
	\end{tabular} 
	
	Springe nach Sim($\qinit$), der Zustand in $M'$, an dem die Simulation des Zustands $q$ aus $M$ beginnt.
	
	\item Simulation eines Rechnerschritts im Zustand Sim($q$):\\
	Kopf auf linker Begrenzung, d.h. linker \# auf Spur $2k+1$
	\begin{itemize}
		\item Durchlauf bis rechter Rand, sammle dabei Symbole unter den Köpfen, speichern in endl. Zustand $\overrightarrow{\gamma} \in \Gamma^k$
		\item Berechne $\delta(q,\overrightarrow{\gamma})=(q',\overrightarrow{\gamma'},\overrightarrow{d})$\\
		neuer Zustand, für jeden Kopf ein neues Symbol $\overrightarrow{\gamma'}$ und Richtung $\overrightarrow{d}$.
		\item Rücklauf nach links, dabei Schreiben um $\overrightarrow{\gamma}'$ und Versetzen der Köpfe gem"a"s $\overrightarrow{d}$.
	\end{itemize}
	Falls eine Kopfbewegung den Rand auf Spur $2k+1$ überschreitet, dann verschiebe Randmarke entsprechend.
	
	Beim Rücklauf: Test auf Haltekonfiguration der $k$-Band \ac{TM}.\\
	Falls ja, dann Sprung in Haltekonf. von $M'$
	
	Weiter im Zustand Sim$(q')$.
	\end{enumerate}
\end{proof}

\begin{Korollar*}
	Beim Erkunden eines Worts der Länge $n$ benötige die $k$-Band Maschine $M\ T(n)$ Schritte und $S(n)$ Zellen auf den Bändern.
	\begin{itemize}
		\item $M'$ benötigt $O(S(n))$ Zellen
		\item $M'$ benötigt $O(S(n\cdot T(n)))$ Schritte $=O(T(n)^2)$
	\end{itemize}
	Weitere \ac{TM}-Booster
	\begin{itemize}
		\item Unbeschränkt großer Speicher
		\begin{itemize}[label=\->]
			\item für jede "`Variable"' ein neues Band
	\end{itemize}
	\item Datenstrukturen
	\begin{itemize}[label={\rotatebox[origin=c]{180}{$\Lsh$}}]
		\item ensprechend kodieren.
	\end{itemize}
	\end{itemize}
\end{Korollar*}

\subsection{Das Gesetz von Church-Turing (Churchsche These)} % 2.6
\begin{Satz}[name={[Intuitiv berechenbare Funktionen sind mit \acs*{TM} berechenbar]}]
	Jede intuitiv berechenbare Funktion ist mit \ac{TM} (in formalem Sinn) berechenbar.
	
	"`Intuitiv berechenbar"' $\equiv$ man kann Algorithmus hinschreiben
	\begin{itemize}
		\item endliche Beschreibung
		\item jeder Schritt effektiv durchführbar
		\item klare Vorschrift
	\end{itemize}
	Status wie Naturgesetz -- nicht beweisbar
	\begin{itemize}[label=\->]
		\item allgemein anerkannt
		\item weitere Versuche Berechenbarkeit zu formulieren, äquivalent zu \ac{TM}en erwiesen.
	\end{itemize}
\end{Satz}



\subsection{Universelle Turingmaschine}
Die bisher betrachteten Turingmaschinen waren (ebenso wie endliche Automaten oder Kellerautomaten) 
auf einen Einsatzzweck beschränkt und konnten nur eine bestimmte Sprache akzeptieren oder eine bestimmte Funktion berechnen.
Dies entspricht nicht unserer Vorstellung eines Computers, denn soch einer kann ja normalerweile beliebige Programme ausführen.

In diesem Kapitel konstruieren wir nun eine \ac{TM} $\M_U$ die als Eingabe sowohl
eine Codierung $\godel{\M}$ einer beliebigen \ac{TM} $\M$ als auch deren Eingabe $w$ nimmt, sodass die folgenden drei Eigenschaften gelten.
\begin{align}
	\label{Mu-eq-lang} w\in L(\M) &\Leftrightarrow \godel{\M}\#w) \in L(\M_U)\\ 
	\label{Mu-eq-term} \text{$\M$ terminiert bei Eingabe $w$} &\Leftrightarrow \text{$\M_U$ terminiert bei Eingabe $\godel{\M}\#w$}\\
	\label{Mu-eq-func} \text{$f_\M(w)$} &= \text{$f_{\M_U}(\godel{\M}\#w)$}
\end{align}

Wir nennen $\M_U$ die \emph{universelle Turingmaschine}.

\subsubsection{Codierung von Turingmaschinen}

Zunächst wollen wir uns eine Codierung $\godel{\M}$ für $\mathcal{\M}=(\Sigma, Q, \Gamma, \delta, \qinit, \blank, F)$ überlegen.
Die Effizienz unserer Codierung ist uns dabei nicht wichtig und wir versuchen der Einfachheit wegen mit Unärcodierungen zu arbeiten.

Einige Vorüberlegungen
\begin{itemize}
 \item Da $Q$ und $\Gamma$ endlich sind können wir jedem ihrer Elemente eine natürliche Zahl zuordnen.
 \item Wir müssen $\qinit$ und $\blank$ nicht explizit codieren, wir verwenden die Konvention, 
 dass $\qinit$ der Zustand mit der Nummer $1$ und $\blank$ das Zeichen mit der Nummer $1$ ist.
 \item Wir müssen $\Sigma$, $Q$ und $\Gamma$ nicht explizit in die Codierung mit aufnehmen.
 Für das Ausführen einer \ac{TM} sind nur die Elemente aus $\Sigma$, $Q$ und $\Gamma$ relevant die auch in der Transitionsfunktion $\delta$ vorkommen.
\end{itemize}

Idee: Codiere Elemente aus $Q$ und $\Gamma$ unär mit Hilfe des Zeichen~$0$.
Verwende das Zeichen~$1$ um unärcodierte Zahlen zu trennen.

\begin{itemize}
 \item Zustände
 
    Seien $q_1,\dots, q_{n}$ die Elemente von $Q$ sodass $q_1=\qinit$ 
    
    Definiere $\godel{q_i}:=0^i$
    
  \item Menge der akzeptierenden Zustände
  
    Seien $q_{k_1},\dots, q_{k_n}$ die Elemente von $F$
    
    Definiere $\godel{F}:=\godel{q_{k_1}} 1 \ldots 1 \godel{q_{k_n}}$
    
  \item Bandalphabet
  
    Seien $a_1,\dots, a_{n}$ die Elemente von $\Gamma$ sodass $a_1=\blank$
    
    Definiere $\godel{a_j}:=0^j$
    
  \item Richtung (in die der Schreib-Lesekopf bewegt wird)
  
    Definiere $\godel{L}:=0$, $\godel{N}:=00$ und $\godel{R}:=000$.
    
  \item Transitionsfunktion
  
   Seien $(q_{t_1}, a_{t_1}, q_{t_1}', a_{t_1}', d_{t_1}),\dots (q_{t_n}, a_{t_n}, q_{t_n}', a_{t_n}', d_{t_n})$ die Elemente von $\delta$.
  
   Definiere 
   
   $\godel{\delta}:= 11\godel{q_{t_1}}1\godel{a_{t_1}}1\godel{q_{t_1}'}1\godel{a_{t_1}'}1\godel{d_{t_1}}11\dots 11 \godel{q_{t_n}}1\godel{a_{t_n}}1\godel{q_{t_n}'}1 \godel{a_{t_n}'}1\godel{d_{t_n}}11$
   
   \item Turingmaschine
   
   $\godel{\M}:=111\godel{\delta}111\godel{F}111$
\end{itemize}


\subsubsection{Arbeitsweise der universellen Turingmaschine}
Da die universelle Turingmaschine $\M_U$ recht komplex ist wollen wir diese hier nicht formal definieren sondern geben nur eine informelle Beschreibung der Arbeitsweise.


Wie definiere $\M_U$ als 3-Band Maschine wobei die Bänder $B_1$, $B_2$ und $B_3$ wie folgt genutzt werden.
\begin{itemize}
\item[$B_1:$] Zu Beginn steht hier die Eingabe für $\M_U$.
Nach der Initialisierung von $\M_U$ verwenden wir $B_1$ um das Band der Eingabeturingmaschine $\M$ nachzuahmen.
\item[$B_2:$] Speichere die Codierung der Eingabeturingmaschine $\ulcorner \M \urcorner$
\item[$B_3:$] Speichere den aktuellen Zustand von $\M$ ($0^k$ für Zustand $q_k$)
\end{itemize}

Die universelle Turingmaschine $\M_U$ beginnt zunächst mit einer Initialisierung die aus den folgenden drei Schritten besteht.

\begin{enumerate}
 \item Prüfe ob der erste Teil der Eingabe $\ulcorner \M \urcorner$ eine gültige Turingmaschine codiert.
 
 Die Menge der gültigen Codierungen kann mit Hilfe einer kontextfreien Grammatik beschrieben werden (siehe Übungen).
 
 \item Verschiebe $\ulcorner \M \urcorner$ auf $B_2$. Schreibe dabei Blanksymbole auf die Bandzellen von $B_1$.
 Überschreibe anschließend das Symbol $\#$ das Eingabeturingmaschine und Eingabewort trennt durch ein Blanksymbole.
 
 \item Schreibe $0$ (also die Codierung des Startzustandes von $\M$) auf $B_3$.
\end{enumerate}

Nach der Initialisierung ist $\M_U$ also in der folgenden Konfiguration.
\begin{itemize}
\item[$B_1:$] $w$
\item[$B_2:$] $\ulcorner \M \urcorner$
\item[$B_3:$] $0$
\end{itemize}

Nach der Initialisierung wird $\M_U$ das Verhalten von $\M$ nachahmen.
Für jeden Schritt von $\M$ macht $\M_U$ dabei die folgenden Schritte.

 Suche ein Element $(q, a, q', a', d)$ der Transitionsfunktion $\delta$ von $\M$ das zum aktuellen Zustand $q$ und zum aktuellen Bandsymbol $a$ passt.
 
 Wir laufen hierfür einmal über das komplette Band $B_2$. 
 Zustände werden Zeichenweise mit dem Inhalt von $B_3$ verglichen.
 Die Zuordnung von Alphabetsymbolen $a$ zur Codierung $\godel{a}$ ist fest in $\M_U$ eingebaut.
 
 \begin{itemize}
 
 \item Falls solch ein Element der Transitionsfunktion existiert,
    schreibe auf $B_1$ das Zeichen $a'$ und Bewege den Kopf wie durch $d$ definiert.
    Ersetze außerdem die Codierung $\godel{q}$ auf $B_3$ durch $\godel{q'}$.
    Fahre anschließend mit dem Nachahmen des nächsten Schrittes von $\M$ fort.
    
 \item Falls kein solches Element der Transitionsfunktion existiert hat $\M$ eine Haltekonfiguration erreicht.
 Wir vergleichen nun den aktuellen Zustand (auf $B_3$) mit den auf $B_2$ gespeicherten akzeptierenden Zuständen und akzeptieren oder verwerfen die Eingabe entsprechend.

\end{itemize}

\begin{Satz}
	Die universelle \ac{TM} $\M_U$ erfüllt die Eigenschaften (\ref{Mu-eq-lang}), (\ref{Mu-eq-term}) und (\ref{Mu-eq-func}).
\end{Satz}


%%% Local Variables:
%%% mode: latex
%%% TeX-master: "Info_3_Skript_WS2016-17"
%%% End:

% \section[Berechenbarkeit]{Berechenbarkeit}



In diesem Kapitel wollen wir zeigen, 
dass es Sprachen gibt die von keiner TM akzeptiert werden können und
dass es Funktionen gibt die von keiner TM berechnet werden können.


\subsection{Das Gesetz von Church-Turing (Churchsche These)} % 2.6
\begin{These*}[name={[Intuitiv berechenbare Funktionen sind mit \acs*{TM} berechenbar]}]
	Jede intuitiv berechenbare Funktion ist mit einer \ac{TM} (in formalem Sinn) berechenbar.
	
	{\color{red}
	"`Intuitiv berechenbar"' $\equiv$ man kann einen Algorithmus hinschreiben
	\begin{itemize}
		\item endliche Beschreibung
		\item jeder Schritt effektiv durchführbar
		\item klare Vorschrift
	\end{itemize}
	Status wie Naturgesetz -- nicht beweisbar
	\begin{itemize}[label=\->]
		\item allgemein anerkannt
		\item Weitere Versuche um Berechenbarkeit zu formulieren haben sich als äquivalent zu einer \ac{TM} erwiesen.
	\qedhere
	\end{itemize}
	}
\end{These*}

\begin{Def}\label{def:berechenbar}
 Eine Funktion $f:\Sigma^*\rightarrow\Sigma^*$ heißt \emph{berechenbar} wenn es eine Turingmaschine $\M$ gibt,
 sodass $f$ die von $\M$ berechnete Funktion ist.
\end{Def}
\begin{Bemerkung}
 In der Literatur wird in \autoref{def:berechenbar} oft auch nicht der Begriff \emph{berechenbar} sondern der Begriff \emph{Turing-berechenbar} oder der Begriff \emph{rekursiv} verwendet.
 Ziel dabei ist den Begriff \emph{berechenbar} für ``intuitiv berechenbar'' aufzusparen.
 Wir wollen uns hier aber ganz auf die Churchsche These verlassen und zur Vereinfachung den Begriff auch formal besetzen.
\end{Bemerkung}




\subsection{Nicht berechenbare Funktionen}

Wir werden nun zeigen dass es Funktionen gibt die nicht berechenbar sind.
Unser Vorgehen wird dabei das folgende sein.
\begin{itemize}
 \item Wir zeigen, dass es so viele Turingmaschinen wie natürliche Zahlen gibt.
 \item Wir zeigen, dass es mehr Funktionen als natürliche Zahlen gibt.
\end{itemize}

Da es unendlich viele natürliche Zahlen, Turingmaschinen, und Funktionen gibt 
sind die aus der Schule bekannten formalismen zum Vergleichen von Mengen 
(Größe einer Menge ist Anzahl der Elemente)
nicht anwendbar und wir werden über die Existenz von bijektiven Abbildungen argumentieren.

Zur Vereinfachung werden wir uns in diesem Unterkapitel auf das Alphabet $\Sigma=\{0,1\}$ beschränken.

\newcommand{\tmach}{\textit{TMACH}}
\newcommand{\bintonat}{\textit{bin2nat}}
\newcommand{\stdnum}{\textit{stdnum}}
\newcommand{\nattotm}{\textit{nat2tm}}
\newcommand{\langtopower}{\textit{lang2power}}
\newcommand{\functopower}{\textit{func2power}}
\begin{lemma}\label{satz:stdnum}
 Die Standardnummerierung $\stdnum: \Sigma^*\rightarrow\N$ 
 $$ \stdnum(a_1\dots a_n)=\begin{cases}
0 & \text{falls } a_1\dots a_n=\Eps\\
2^{n-1}+\bintonat(a_1\dots a_n) & \text{sonst}
\end{cases}$$
ist eine bijektive Abbildung.
\end{lemma}

\textbf{Notation:}
Wir verwenden im folgenden \tmach für die Menge aller Turingmaschinen.

\begin{lemma}\label{satz:nattotm}
 Die Abbildung $\nattotm: \N\rightarrow \tmach$ 
 $$ \nattotm(n)=\begin{cases}
\M & \text{falls } \begin{array}{l}\stdnum^{-1}(n) \text{ Code einer TM ist}\\ \text{und } \stdnum^{-1}(n)=\godel{\M}\end{array}\\
? & \text{sonst}
\end{cases}$$
ist surjektiv.
\end{lemma}

\begin{lemma}\label{satz:functopower}
Die Abbildung $\functopower:(\Sigma^*\rightarrow\Sigma^*)\rightarrow \calP(\N)$
$$\functopower(f)=\{\stdnum(w)\mid f(w)=1\}$$
ist bijektiv.
\end{lemma}


\begin{Satz}[Cantor]\label{satz:Cantor}
Es gibt keine surjektive Abbildung einer Menge in ihre Potenzmenge.
% Sei $X$ eine beliebige Menge. Es gibt keine surjektive Abbildung 
\end{Satz}
Den Beweis dieses Satzes werden wir später diskutieren.

\begin{Satz}
 Es gibt Funktionen die nicht berechenbar sind.
\end{Satz}
\begin{proof}
 Angenommen jede Funktion $f:\Sigma^*\rightarrow\Sigma^*$ sei berechenbar.
 Dann gibt es für jede Funktion $f$ eine TM $\M_L$ mit $f_\M=f$ und somit eine surjektive Abbildung $g:\tmach\rightarrow(\Sigma^*\rightarrow\Sigma^*)$.
 Nach \autoref{satz:nattotm} und \autoref{satz:functopower} wäre dann aber auch die Abbildung $\functopower\circ g \circ\nattotm:\N\rightarrow\calP(\N)$ surjektiv.
 Dies steht im Wiederspruch zu \autoref{satz:Cantor}.
\end{proof}

Ein zentraler und nicht offensichtlicher Teile dieses Beweises war \autoref{satz:Cantor}.
Wir wollen das restliche Unterkapitel nutzen um diesen zu diskutieren.
Zunächst geben wir einen kurzen formalen Beweis.

\begin{proof}[Von \autoref{satz:Cantor}]
Sei $X$ beliebige Menge. 
Angenommen es gibt surjektive Abbildung $f:X\rightarrow\calP(X)$.
Betrachte die Menge $Z=\{x\in X\mid x\notin f(x)\}\in\calP(X)$.
Da $f$ surjektiv gibt es $z\in X$ sodass $f(z)=Z$.
Nach Definition der Menge $Z$ gilt nun aber $z\in f(z)\Leftrightarrow z\notin f(z)$. Widerspruch!
\end{proof}

Ein Beweis dieser Form wird auch \emph{Diagonalargument} genannt und ist nicht einfach zu verstehen.
Wir werden den Beweis für den Spezialfall dass $X$ eine abzählbare Menge ist mit der folgenden Matrix illustrieren.

Da $X$ abzählbar ist können wir dessen Elemente als undendliche Sequenz $x_0,x_1,\ldots$ schreiben.
Wir nehmen nun an es gäbe eine surjektive Abbildung $f:X\rightarrow\calP(X)$.
Dann können wir $f$ mit Hilfe der folgenden Matrix darstellen.
Zeile $i$ beschreibt dabei das Bild $f(x_i)$. 
Wir haben eine Spalte für jedes $x_j\in X$.
Gilt $x_j\in f(x_i)$ so steht in Eintrag $(i,j)$ ein Y. Ansonsten steht dort ein N.

Da $f$ surjektiv ist muss es wie im formalen Beweis erwähnt auch eine Zeile für die Menge $Z=\{x\in X\mid x\notin f(x)\}\in\calP(X)$ geben.
Diese Zeile, wir nehmen an es sei die Zeile für $f(x_k)$, enthält nun in genau den Spalten $j\in\N$ ein Y, in denen deren Diagonaleintrag $(j, j)$ kein Y enthält.

Wir erkennen den Widerspruch wenn wir überlegen was im Diagonaleintrag $(k, k)$ steht.
Angenommen bei $(k, k)$ steht ein N, dann müsste nach Definition von Zeile $k$ dort ein Y stehen.
Angenommen bei $(k, k)$ steht ein Y, dann ist die Definition von Zeile $k$ verletzt denn es darf nur dort ein Y stehen wo auch im Diagonaleintrag ein N steht.

\begin{tikzpicture}[scale=0.7]

\draw[fill=lightgray!20, draw=lightgray!20] (-1.5,-3.5) rectangle (7.7,-4.5);

\node at (0,1) {$x_0$};
\node at (1,1) {$\dots$};
\node at (2,1) {$x_i$};
\node at (3,1) {$\dots$};
\node at (4,1) {$x_k$};
\node at (5,1) {$\dots$};
\node at (6,1) {$x_j$};
\node at (7,1) {$\dots$};

\node at (-1,0) {$x_0$};
\node at (-1,-1) {$\vdots$};
\node at (-1,-2) {$x_i$};
\node at (-1,-3) {$\vdots$};
\node at (-1,-4) {$x_k$};
\node at (-1,-5) {$\vdots$};
\node at (-1,-6) {$x_j$};
\node at (-1,-7) {$\vdots$};

\draw (-1.5, 0.5) to (7.5,0.5);
\draw (-0.5, 1.5) to (-0.5,-7.5);

\node[circle, draw=lightgray!50, fill=lightgray!50, inner sep=0] (00) at (0,0) {N};
\node[gray] at (1,0) {$\ldots$};
\node[gray] at (2,0) {Y};
\node (k0) at (0,-4) {Y};
\draw[->] (00) to (k0);

\node[circle, draw=lightgray!50, fill=lightgray!50, inner sep=0] (ii) at (2,-2) {Y};
\node (ki) at (2,-4) {N};
\draw[->] (ii) to (ki);

\node[circle, draw=lightgray!50, fill=lightgray!50, inner sep=0] (jj) at (6,-6) {N};
\node (kj) at (6,-4) {Y};
\draw[->] (jj) to (kj);

\node[circle, draw=lightgray!50, fill=lightgray!50, inner sep=1] (kk) at (4,-4) {\textbf{?}};
\end{tikzpicture}




\subsection{Sprachakzeptanz und Berechenbarkeit}

\begin{Def}[name={[Akzeptanz, Entscheidbarkeit, Semi-Entscheidbarkeit]}]
	Sei $M$ \ac{TM}.
	\begin{itemize}
	\item $M$  \emph{akzeptiert} $w\in\Sigma^*$, falls $q_0w
          \vdash^* uq'v$ Haltekonfiguration und $q'\in F$
	\item $M$ \emph{akzeptiert} $L\subseteq\Sigma^*$, falls $M$ akzeptiert $w \<-> w\in L$
	\item $M$ \emph{entscheidet} $L\subseteq\Sigma^*$, falls $M$ akzeptiert $L$ und $M$ hält für jede Eingabe an.
	\item $L\subseteq\Sigma^*$ ist \emph{semi-entscheidbar} (rekursiv aufzählbar), falls $\exists M$, die $L$ akzeptiert.
	\item $L\subseteq\Sigma^*$ ist \emph{entscheidbar} (rekursiv), falls $\exists M$, die $L$ entscheidet.
	\end{itemize}
\end{Def}


\begin{Satz}
    Eine Sprache $L\subseteq\Sigma^*$ ist entscheidbar genau dann wenn die Funktion
    $f:\Sigma\rightarrow\{0,1\}$ mit 
    $$f(w)=\begin{cases}
       1 & \text{falls } w\in L\\
       0 & \text{sonst}
      \end{cases}$$
    berechenbar ist.
\end{Satz}


\begin{lemma}
Die Menge der Wörter
\end{lemma}

\begin{lemma}
Die Abbildung \textit{lang2power}
\end{lemma}



\begin{Satz}
Es gibt Sprachen, die von keiner Turingmaschine akzeptiert werden.
\end{Satz}


\begin{lemma}
Die Abbildung $\langtopower:\calP(\Sigma^*)\rightarrow \calP(\N)$
$$\langtopower(L)=\{\stdnum(w)\mid w\in L\}$$
ist bijektiv.
\end{lemma}


\begin{Satz}
 Es gibt Sprachen die nicht entscheidbar sind.
\end{Satz}
\begin{proof}
 Angenommen jede Sprache $L\subseteq\Sigma^*$ sei entscheidbar.
 Dann gibt es für alle $L\subseteq\Sigma^*$ eine TM $\M$ mit $f_\M=f$ und somit eine surjektive Abbildung $f:?\rightarrow\calP(\Sigma^*)$.
 Nach ... wäre dann aber auch die Abbildung $functopower\circ f \circ\nattotm:\N\rightarrow\calP(\N)$ surjektiv.
 Dies steht im Wiederspruch zu ...
\end{proof}



\hide{
\subsection{Typ-0 und Typ-1 Sprachen}
%\ptnote{checkpoint}
% \begin{Def}[name={[NTM]}]
% 	\acf{NTM} ist ein Tupel $(Q,\Sigma,\Gamma,\delta,q_0,\blank,F)$, wobei alles wie bei einer \ac{DTM} außer $\delta: Q\x\Gamma\-> \wp(Q\x\Gamma\x\{N,L,R\})$.\\
% 	Konfiguration und Berechnungsrelation wie gehabt.
% \end{Def}

\begin{Def}[name={[Akzeptanz, Entscheidbarkeit, Semi-Entscheidbarkeit]}]
	Sei $M$ \ac{TM}.
	\begin{itemize}
	\item $M$  \emph{akzeptiert} $w\in\Sigma^*$, falls $q_0w
          \vdash^* uq'v$ Haltekonfiguration und $q'\in F$
	\item $M$ \emph{akzeptiert} $L\subseteq\Sigma^*$, falls $M$ akzeptiert $w \<-> w\in L$
	\item $M$ \emph{entscheidet} $L\subseteq\Sigma^*$, falls $M$ akzeptiert $L$ und $M$ hält für jede Eingabe an.
	\item $L\subseteq\Sigma^*$ ist \emph{semi-entscheidbar} (rekursiv aufzählbar), falls $\exists M$, die $L$ akzeptiert.
	\item $L\subseteq\Sigma^*$ ist \emph{entscheidbar} (rekursiv), falls $\exists M$, die $L$ entscheidet.
	\end{itemize}
\end{Def}
\begin{Def}[name={[Laufzeit und Platzbedarf einer \acs*{TM}]}]
	Laufzeit und Platzbedarf einer \ac{TM} $M:$
	
	Laufzeit : $T_M(w) =
	\begin{cases}
		\parbox{.74\textwidth}{\raggedleft Anzahl der Schritte einer kürzesten Berechnung, die zur Akz. von $w$ führt (falls $\exists$)}\\[-1em]
		\text{1, sonst}
	\end{cases}
	$\\
	Platzbedarf : $S_M(w) =
	\begin{cases}
		\parbox{.7\textwidth}{\raggedleft geringster Platzbedarf (Länge einer Konf.) einer akz. Berechnung von $w$ (falls $\exists$)}\\[-1em]
		\text{1, sonst}
	\end{cases}
	$
\textbf{Zeitbeschränkt mit $t(n)$}: $\forall w\in\Sigma^*: |w|\leq n \=> T_M(w)\leq t(n)$,\\
platzbeschränkt analog.
\end{Def}
\begin{Satz}[name={[Zu jeder NTM gibt es \acs*{DTM}]}]\label{satz:6.1}
	Zu jeder NTM gibt es eine \ac{DTM} $M'$, so dass
	\begin{itemize}
	\item $M'$ akzeptiert $L(M)$
	\item $M'$ terminiert gdw. $M$ terminiert
	\item Falls $M$ zeit- und platzbeschränkt ist mit $t(n)$
          bzw. $s(n)$ ($n=$ Länge der Eingabe), dann ist $M'$
          zeitbeschränkt mit $2^{O(t(n))}$ und platzbeschränkt mit
          $O(s(n)\cdot t(n))$. 
	\end{itemize}
\end{Satz}\vspace{-2em}
\begin{proof}
	Die Konfigurationen von $M$ bilden einen Baum, dessen Kanten durch $\vdash$ gegeben sind. Er ist endlich verzweigt, hat aber ggf. unendlich lange Äste.
	
	Definiere eine (Mehrband-)\ac{DTM}, die den Konfigurationsbaum
        systematisch durchläuft und akzeptiert, sobald eine
        Haltekonfiguration erreicht ist, in der $M$ akzeptiert. 
	
	Die \ac{DTM} terminiert ebenfalls, wenn alle Blätter des Baumes besucht worden sind, ohne dass eine akzeptierende Konfiguration gefunden wurde.
	
	Baumsuche mit Kontrollinformation und bereits besuchten Konf. auf ein Extraband.
	\begin{itemize}
	\item Tiefensuche? Nicht geeignet, sie könnte in unendlichen Ast laufen.
	\item Breitensuche? OK, aber Platzbedarf $O(2^{t(n)}\cdot s(n))$
	\item iterative deepending\,: Tiefensuche mit vorgegebener Schranke, bei erfolgloser Suche Neustart mit erhöhter Schranke. \qedhere
	\end{itemize}
\end{proof}
Nächstes Ziel: Charakterisierung von Typ-1 Sprachen.
\begin{Def}[name={[$\DTAPE$ und $\NTAPE$]}]\
\begin{itemize}
\item $\DTAPE(s(n)):$ Menge der Sprachen, die von einer \ac{DTM} in Platz $s(n)$ akzeptiert werden können.
\item $\NTAPE(s(n)):$ Wie für $\DTAPE$, aber mit \ac{NTM}. \qedhere
\end{itemize}
\end{Def}
\begin{Bemerkung}\
	\newcommand{\underarrowset}[2]{%
		\underset{%
			\mathclap{%
				\overset{\displaystyle\uparrow}{\mathclap{#1}}%
			}%
		}{#2}%
	}
	\begin{enumerate}
	\item Für "`$s(n)\leq n$"' betrachte 2-Band \ac{TM}, bei denen die Eingabe read-only ist und nur das zweite Arbeitsband der Platzschranke unterliegt (so ist $s(n)$ sublinear möglich).
	\item Jede Platzbeschränkung impliziert Laufzeitschranke.\\
	Angenommen Platzschranke $s(n)$\\
	$\curvearrowright$ \ac{TM} hat nur endlich viele Konfigurationen
	\[ N := \underarrowset{%
			\parbox{\widthof{\scriptsize Eingabeband}}{\raggedright\scriptsize Kopfpos. im Eingabeband}\hspace{1cm}
		}{n\vphantom{|}}
		|Q| \quad\cdot\quad
		\underarrowset{\parbox{2.2cm}{\scriptsize\centering mögliche Inhalte des Arbeitsbands}}{|\Gamma|}^{s(n)}
		\ \cdot\
		\underarrowset{\hspace{1.7cm}\parbox{2cm}{\scriptsize Kopfpos. auf\\ Arbeitsband}}{s(n)}
		\in 2^{O(\log n + s(n))}
	\]
	\item \ac{DTM} mit Platzschranke\,: $M$ entscheidet,\\
	falls sie akzeptiert, dann in weniger als $N$ Schritten,\\
	falls nach $N$ Schritten keine Termination erfolgt\\
	\quad$\curvearrowright$ Endlosschleife -- Abbruch
	\item \ac{NTM}\,: nutze den \ac{ND} optimistisch aus\,:\\
	falls eine akzeptierende Berechnung existiert, dann muss es eine Berechnung ohne wiederholte Konfiguration geben.
	\end{enumerate}
\end{Bemerkung}
\begin{Satz}[name={[$L\in\DTAPE(n),\ L\in\NTAPE(n)$]}]\label{satz:6.2}\
	\begin{itemize}
	\item $L\in\DTAPE(n) \curvearrowright\ \exists$ \ac{DTM}, die $L$ in Zeit $2^{O(n)}$ entscheidet.
	\item $L\in\NTAPE(n)$ analog.
	\end{itemize}
\end{Satz}\vspace{-2em}
\begin{proof}
	siehe oben.
\end{proof}
\begin{Bemerkung}
	Die Klasse $\NTAPE (n)$ heißt auch \ac{LBA}.
\end{Bemerkung}
%
\begin{Satz}\label{satz:6.3}
	$\mathcal L_1=\mathrm{NTAPE}(n)$
\end{Satz}
\begin{proof}\
\begin{itemize}
	\item["`\=>"'\,:] Sei $G=(N,\Sigma,P,S)$ Typ-1 Grammatik für $L$.\\
		Konstruiere \ac{NTM} $M$ mit $L=L(M),\ \Gamma = \Sigma\cup N\cup\{\blank\}$
		\begin{enumerate}
		\item $M$ rät nicht deterministisch eine Position auf
                  dem Band und eine Produktion $\alpha\-> \beta$. Falls $\beta$ gefunden wird, ersetze durch $\alpha$, weiter bei 1.
		\item Falls Bandinhalt $=S$ \quad stop, akzeptiert.
		\end{enumerate}
		Dieses Verfahren terminiert.
	\item["`\<="'\,:] %
	Gegeben: \ac{NTM} $M$ linear beschränkt.\\
	Gesucht: Typ-1 Grammatik $\mathcal{G}$ mit $L(\mathcal{G})=L(M)$\\
	Idee: 
	\begin{align*}
		a_1\cdots a_n &\-->
			\pmqty{a_1\\a_1} \pmqty{a_2\\a_2} \pmqty{(q,a)\\a_3} \pmqty{a_4\\a_4} \pmqty{a_n\\a_n}
			&&\begin{matrix}\text{Spur 1}\\\text{Spur 2}\end{matrix}\\
	\shortintertext{ad\footnotemark\ Spur 1: Alphabet $\Gamma\cup(Q\x\Gamma) = \triangle$}
		P' &\begin{cases}
			(q,a) &\-> (q',a')   \quad q\in Q,a\in\Gamma\\
			(q,a)b &\-> a'(q',b) \quad b\in\Gamma\\
			b(q,a) &\-> (q',b)a'
		\end{cases}
		& &\begin{aligned}
			\delta(q,a) &\ni (q',a',N)\\
			\delta(q,a) &\ni (q',a',R)\\
			\delta(q,a) &\ni (q',a',L)
		\end{aligned}
	\end{align*}\footnotetext{ad $\approx$ zur}%
	Def. $\widetilde{uqav} = u(q,a)v\ ,\ u,v\in\Gamma^*,a\in\Gamma$\\
	Es gilt: $uqav\vdash^* k' \curvearrowleftright \widetilde{uqav} \overset{*}{\=>} \widetilde{k'}$ mit Produktion $P'$.
	
	Def. $\mathcal{G}$ durch $N = \{S\}\dotcup\triangle\x\Sigma$
	\begin{align*}
		\text{mit } P &= \\
		S &\-> \pmqty{(q_0,a)\\a} &&\forall a\in\Sigma\\
		S &\-> S\pmqty{a\\a} &&\forall a\in\Sigma\\
		\pmqty{\alpha\\a}
			&\-> \pmqty{\beta\\a}
			&&\begin{aligned}
				\forall \alpha\->\beta &\in P'\\
				\alpha,\beta &\in\triangle
			\end{aligned}\\
		\pmqty{\alpha_1\\a_1}\pmqty{\alpha_2\\a_2}
			&\-> \pmqty{\beta_1\\a_1}\pmqty{\beta_2\\a_2}
			&&\begin{aligned}
				\forall \alpha_1\alpha_2\->\beta_1\beta_2 &\in P'\\
				\alpha_i,\beta_i &\in\triangle
			\end{aligned}\\
		\pmqty{x\\a} &\-> a
			&&\begin{aligned}
				x&\in\Gamma\\
				a&\in\Sigma
			\end{aligned}\\
		\pmqty{(q',x)\\a} &\-> a
			&&\begin{aligned}
				x&\in\Gamma, q'\in F, \delta (q',x)=\emptyset\\
				a&\in\Sigma
			\end{aligned}
	\end{align*}
	\begin{align*}
		S &\xRightarrow{*} \pmqty{(q_0,a_1)\\a_1}\pmqty{a_2\\a_2}\dots\pmqty{a_n\\a_n}\\
		&\phantom{{}\xRightarrow{*}{}}\ \acs*{TM}\ \,\dots\\
		&\xRightarrow{*} \pmqty{x_1\\a_1}\dots\pmqty{(q',x_i)\\a_i}\dots\pmqty{x_n\\a_n}\\
		&\xRightarrow{*} a_1\dots a_i\dots a_n
	\end{align*}
	Damit gesehen $L(\mathcal{G})\subseteq L(M)$\\
	Rückrichtung: selbst \qedhere
	\end{itemize}
\end{proof}

\begin{Satz}
	Die Typ-1 Sprachen sind abgeschlossen unter {\thinmuskip=6mu$\cup,\cap,\cdot,{}^*$} und Komplement.
\end{Satz}
\begin{proof}
	Zu $\cup$ und $\cap$ betrachte \ac{NTM}.\\
	Für $\cdot$ und $^*$ konstruiere Grammatik.\\
	ad Komplement "`2. \acsu{LBA}-Problem\footnote{\acs*{LBA} = \acl*{LBA} -- 1964 Kuroda}"' bis 1987, dann gelöst durch Immerman und Szelepcsényi.
\end{proof}
\begin{Bemerkung}
  1. \ac{LBA}-Problem (1964): Ist $\mathrm{NTAPE}(n) = \mathrm{DTAPE}(n)$? Bisher ungelöst.
\end{Bemerkung}
\begin{Satz}
	Das Wortproblem für Typ-1 Sprachen ist entscheidbar.
\end{Satz}
\begin{proof}
	\begin{align*}
		L\in\mathcal{L}_1 &\curvearrowleftright L\in\mathrm{NTAPE}(n)\\
		&\curvearrowright \text{nach \autoref{satz:6.2}: $L$ entscheidbar}
	\end{align*}
	Nach \autoref{satz:6.1} sogar mit \ac{DTM}.
\end{proof}
Die Rückrichtung "`L entscheidbar. $\xcancel{\curvearrowright}\ L$ ist Typ-1 Sprache"' gilt nicht!

\begin{Satz}\label{satz:6.6}
	$\mathcal{L}_0 = \ac{NTM}$
\end{Satz}
\begin{proof}
	\begin{itemize}
	\item["'\=>"'] Kontruktion einer \ac{NTM} $M$ wie in \autoref{satz:6.3}, aber ohne Platzbeschränkung.
	\item["'\<="'] Konstruktion analog zu \autoref{satz:6.3} + Startsymbol $S'$
	\begin{align*}
		S' &\-> \pmqty{\blank\\\Eps} S' \pmqty{\blank\\\Eps} &&\text{Schaffe Platz für Berechnung von }M\\
		S' &\-> S\\
	\shortintertext{Erweitere $N$}
		&= \{S',S\}\cup\triangle\x(\Sigma\cup\{\Eps\})\\
	\shortintertext{Neue Löschregeln:}
		\pmqty{x\\ \Eps } &\-> \Eps &&\forall x\in\Gamma\\
		\rotatebox{90}{$\Rsh$}\ &\mathrlap{\text{die einzigen
                                          Regeln, die Typ-1 Bedingung verletzen.}} \tag*{\qedhere}
	\end{align*}
	\end{itemize}
\end{proof}

\begin{Satz}[name={[Abgeschlossenheit von Typ-0 Sprachen]}]\label{satz:Typ-0-abgeschl}
	Die Typ-0 Sprachen sind unter $\thinmuskip=6mu\cup,\cap,\cdot,{}^*$ abgeschlossen.
\end{Satz}
\begin{proof}
	Konstruiere \ac{NTM} für $\thinmuskip=6mu\cup,\cap$ ; Typ-0-Grammatiken für $\cdot$ und $^*$.
\end{proof}

\begin{Bem}
	Typ-0 Sprachen sind \emph{nicht} unter Komplement abgeschlossen!
\end{Bem}
}

\subsection{Universelle \acs*{TM} und das Halteproblem}


% \begin{Satz}
% 	Es gibt eine universelle \ac{TM} $\M_U$ mit $L(\M_U) = \{ \ulcorner \M \urcorner w\mid w \in L(\M)\}$
% \end{Satz}



Schreibe ab jetzt $M_w$ für die Maschine mit Gödelnummer $w$. Falls
$w$ kein gültiger Code für eine TM, dann sei $M_w$ eine beliebige fest
TM $M$ mit $L(M) = \varnothing$. 
\begin{Def}[name={[Spezielles Halteproblem]}]
  Das spezielle Halteproblem besteht aus allen Codes von Maschinen, die
  anhalten, falls sie auf den eigenen Code angesetzt werden.
  \begin{align*}
    K &= \{ w \in \{0,1\}^* \mid M_w \text{ angesetzt auf }w\text{
        terminiert} \}
  \end{align*}
\end{Def}
\begin{Satz}\label{satz:6.10}
  Das spezielle Halteproblem ist unentscheidbar.
\end{Satz}
\begin{proof}
  Angenommen $M$ ist eine TM, die $K$ entscheidet.

  Konstruiere Maschine $M'$, die zunächst $M$ auf ihre Eingabe
  anwendet. Falls $M$ akzeptiert, dann geht $M'$ in eine
  Endlosschleife. Falls $M$ nicht akzeptiert, dann hält $M'$ an.

  Sei $w' = \ulcorner M' \urcorner$ der Code von $M'$ und setze $M$
  auf $w'$ an. Es gilt:

  $M$ akzeptiert $w'$

  gdw.(nach Def von $K$) $M'$ angesetzt auf $w'$ terminiert

  gdw. $M$ akzeptiert $w'$ nicht.

  Ein Widerspruch. $\qquad\lightning$
\end{proof}
\begin{Korollar}\label{kor:6.11}
  $\overline{K} = \{w \in\{0,1\}^* \mid M_w\text{ h"alt nicht bei
    Eingabe }w \}$,   das Komplement von $K$, 
  ist nicht entscheidbar.
\end{Korollar}
\begin{proof}
	Angenommen $\overline{K}$ sei entscheidbar durch $M$. Dann
        entscheidet $M'$  $K$. $M'$ führt zuerst $M$ aus und negiert das Ergebnis. $\lightning$ \autoref{satz:6.10}
\end{proof}
\begin{lemma}[name={[$K$ ist semi-entscheidbar]}]
	$K$ ist semi-entscheidbar.
\end{lemma}
\begin{proof}
  Die Maschine $M$ kopiert die Eingabe $w$ und f''uhrt die universelle
  TM f''ur $M_w$ aus. Falls diese Simulation stoppt, geht $M$ in einen
  akzeptierenden Zustand und terminiert.
\end{proof}
% \begin{Def}[name={[length-lexicographic order]}]
% 	Die \emph{length-lexicographic order} auf $\{0, 1\}^*$ (mit $0<1$) ist definiert durch
% 	\begin{align*}
% 		v \leq w \<=> &\ |v| < |w|\\
% 		\vee &\ v = w\\
% 		\vee &\ |v| = |w| \text{ und } \exists u\in \{0, 1\}^*\\
% 		& \text{mit } v = u0v' \text{ und } w = u1w'
% 	\end{align*}
% \end{Def}
% \begin{Satz}[name={[$\leq:$ totale Ordnung]}]\
% 	\begin{itemize}
% 		\item $\leq$ ist totale Ordnung auf $\{0, 1\}^*$
% 		\item $\exists$ bijektive Abbildung $w: \N \to \{0, 1\}^*$ mit $i \leq j \to w(i) \leq w(j)$
% 	\end{itemize}
% \end{Satz}
% \begin{Def}[name={[Diagonalsprache]}]
% 	Sei $M_i$ die Turingmaschine mit Gödelnummer $\ulcorner M_i\urcorner = w(i)$.
% 	(falls $w(i)$ kein gültiger Code, dann sei $M_i$ eine beliebige \ac{TM} mit $L(M_i) = \varnothing$). Die \emph{Diagonalsprache}
% 	\[D = \{w(i) \mid w(i) \notin L(M_i)\}\]
% 	das heißt $M_i$ akzeptiert $w(i)$ nicht.
% \end{Def}
% \begin{Satz}[name={[$D$ ist nicht entscheidbar]}]\label{satz:6.10}
% 	$D$ ist nicht entscheidbar.
% \end{Satz}
% \begin{proof}
% 	Angenommen $\exists M$, die $D$ entscheidet.\\
% 	$M$ muss in Aufzählung vorkommen, das heißt $\exists j\in \N$, sodass $M = M_j$.
% 	Wende $M_j$ auf $w(j)$ an:
% 	\begin{itemize}
% 		\item $M_j$ stoppt/ja: $w(j) \in L(M_j) = D \qquad \lightning$ Def. von $D$
% 		\item $M_j$ stoppt/nein: $w(j) \notin L(M_j) = D \quad \lightning$ Def. von $D$ \qedhere
% 	\end{itemize}
% \end{proof}
% \begin{Korollar}\label{kor:6.11}
% 	$\overline{D} = \{w(i) \mid M_i\text{ akzepztiert }w(i) \}$ ist nicht entscheidbar.
% \end{Korollar}
% \begin{proof}
% 	Angenommen $\overline{D}$ sei entscheidbar durch $M$. Dann entscheidet $M'\ D$. $M'$ führt 
% 	zuerst $M$ aus und negiert das Ergebnis. $\lightning$ \autoref{satz:6.10}
% \end{proof}
% \begin{lemma}[name={[$\overline{D}$ ist semi-entscheidbar]}]
% 	$\overline{D}$ ist semi-entscheidbar.
% \end{lemma}
% \begin{proof}
% 	Bei Eingabe $w$
% 	\begin{itemize}
% 		\item Falls $w$ kein gültiger Code: stop mit Ergebnis nein.
% 		\item Falls $w$ gültiger Code, dann $\exists i$, sodass $ww = \left\ulcorner M_j
% 		\right\urcorner w(i) \qquad w = w(i)$
% 	\end{itemize}
% 	Wende $U$ auf $ww$ an.\\
% 	Insgesamt: \ac{TM}, die $\overline{D}$ akzeptiert.
% \end{proof}
\begin{Satz}[name={[{$L,\overline{L}$ semi-entscheidbar $\=> L$ entscheidbar}]},restate={[name=Wiederholung]repeatSatz613}]\label{satz:6.13}
	Falls $L$ semi-entscheidbar und $\overline{L}$ semi-entscheidbar, dann ist $L$ entscheidbar.
\end{Satz}
\begin{proof}
	Sei $M$ die \ac{TM} für $L$, $\overline{M}$ die \ac{TM} für $\overline{L}$.\\
	Führe $M$ und $\overline{M}$ "`parallel"' mit der gleichen Eingabe aus.\\
	Falls $M$ akzeptiert $\Rightarrow$ Ja\\
	Falls $\overline{M}$ akzeptiert $\Rightarrow$ Nein.\\
	Eine der Maschinen muss anhalten, wegen Voraussetzung.
\end{proof}
$K$ \quad nicht entscheidbar\\
$\overline{K}$ \quad nicht entscheidbar\\
$K$ \quad semi-entscheidbar (Typ-0)\\
$\overline{K}$ \quad nicht-semientscheidbar (keine Typ-0)
\begin{Korollar}
	$\mathcal{L}_0 \supsetneqq \mathcal{L}_1$
\end{Korollar}
\begin{proof}
	$K$ ist unentscheidbar (also $\notin \mathcal{L}_1$), aber semi-entscheidbar (also $\in \mathcal{L}_0$).
\end{proof}
\emph{\textbf{Fragen:}}
\begin{enumerate}
	\item Ist $\mathcal{L}_1$ = Menge der entscheidbaren Sprachen?
	\item[--] Nein:\\
	\begin{minipage}[t]{.7\linewidth}
	Konstruiere eine Kodierung von Typ-1 Grammatiken als Worte
        $w\in\{0,1\}^*$. Die Grammatik zum Wort $w$ sei $G_w$; falls
        $w$ kein sinnvoller Kode ist, setze $G_w = (\{S\}, \{0,1\},
        \{\}, S)$ die leere Grammatik.
        
	Die \emph{Diagonalsprache} $D = \{w \in\{\0,1\}^* \mid w\notin
        L(G_w)\}$ ist entscheidbar, weil das Wortproblem für Typ-1  
	Sprachen entscheidbar, aber es $\nexists w$, sodass $L(G_w) =
        D$. Beweis durch Widerspruch.
	\end{minipage}\quad
	\begin{tabular}[t]{M{c} | *4{M{c}@{ }}}
		\ &w_1&w_2&w_3&\cdots\\\hline
		G_1 &\\
		G_2 &\\
		G_3 &\\
		\vdots&
	\end{tabular}
	\item Ist $\mathcal{L}_0$ = Menge aller Sprachen?
	\item[--] Nein: $\overline{K} \notin \mathcal{L}_0$
\end{enumerate}
\begin{Def}[Reduktion]\ \\
  Seien $U, V \subseteq \Sigma^*$ Sprachen.\\
  \emph{$U$ ist auf $V$ reduzierbar ($U \preceq V$)}, falls eine totale berechenbare Funktion
  $f:\Sigma^* \to \Sigma^*$ existiert, so dass $\forall x \in \Sigma^*:x \in U \iff f(x) \in V$.
\end{Def}
\begin{lemma}
  Falls $U \preceq V$ und $V$ \mbox{(semi-)entscheidbar}, dann ist auch $U$
  \mbox{(semi-)entscheidbar}.
\end{lemma}\vspace{-1.5em}
\begin{proof}
  Wenn $M$ ein (Semi-)Entscheidungsverfahren für $V$ ist, dann konstruiere $M'$ wie folgt
  \begin{itemize}
  \item wende erst $f$ auf die Eingabe $x$ an ($f$ ist berechenbare
    Funktion gemäß Reduktion und kann daher programmiert werden)
  \item führe $M$ auf dem Ergebnis $f(x)$ aus
  \end{itemize}
  $\curvearrowright$ $M'$ ist (Semi-)Entscheidungsverfahren für
  $U$. (Weil $f$ total ist, terminiert der Code  f''ur $f$ immer und daher "andert das Terminationsverhalten nicht.)
\end{proof}
\emph{Anwendung:} $U \preceq V$ und $U$
unentscheidbar $\curvearrowright$ $V$ unentscheidbar.

\begin{Def}[name={[Halteproblem]}]
	Das \emph{Halteproblem} ist definiert durch
	\[H = \{\ulcorner M\urcorner\# w \mid M \text{ hält bei Eingabe } w \text{ an}\} \qedhere\]
\end{Def}
\begin{Satz}[name={[$H$ ist unentscheidbar]}]\label{satz:H ist unentscheidbar}
	$H$ ist unentscheidbar.
\end{Satz}
\begin{proof}
  Die Funktion $f (w) = w\#w$ ist total berechenbar und liefert eine
  Reduktion  $K \preceq H$.

  Denn: $w \in K$ gdw. $M_w$ hält bei Eingabe $w$ an gdw. $w\#w \in H$.
\end{proof}
% \begin{proof}
% 	Angenommen $M_0$ entscheidet $H$.\\
% 	Konstruiere $M'$ wie folgt:\\
% 	Bei Eingabe $w$ bestimme $i$, sodass $w = w(i)$\\
% 	Verwende $M_0$ um festzustellen, ob $\ulcorner M_i\urcorner w$ anhält.\\
% 	Antwort von $M_0:$\\
% 	nein: $\begin{aligned}[t]
% 		&\curvearrowright w(i) \notin L(M_i)\\
% 		&\curvearrowright M'\text{ akzeptiert }w\text{ nicht}.
% 	\end{aligned}$\\
% 	ja: Führe $\ulcorner M_i\urcorner w$ mit $U$ aus (muss ja terminieren) und akzeptiere entsprechend das Ergebnis von $U$.\\
% 	Insgesamt: $M'$ entscheidet $\overline{D} \qquad \lightning$ \autoref{kor:6.11}\\
% 	$\curvearrowright M_0$ existiert nicht.
% \end{proof}
\begin{Satz}[name={[$H$ ist semi-entscheidbar]}]
	$H$ ist semi-entscheidbar.
\end{Satz}
\begin{proof}
	Modifiziere $U$, sodass sie jede Eingabe akzeptiert, bei der sie anhält.
\end{proof}
\begin{Def}[name={[Halteproblem auf leerem Band $H_\varepsilon$]}]
  Das Halteproblem auf leerem Band $H_\varepsilon = \{\ulcorner M\urcorner \mid M 
	\text{ terminiert auf leeren Band}\}$
\end{Def}
\begin{Satz}[name={[$H_\varepsilon$ ist unentscheidbar]}]
	$H_\varepsilon$ ist unentscheidbar.
\end{Satz}
\begin{proof}
  Konstruiere eine Reduktion $H \preceq H_\varepsilon$ mit Hilfe der
  Funktion $f(w\#x) = w'$, wobei $w'$ der Code einer TM ist, die
  \begin{itemize}
  \item zuerst $x$ aufs leere Band schreibt und dann
  \item $M_w$ auf diese Eingabe anwendet.
  \end{itemize}
  Offenbar gilt $w\#x\in H$ gdw. $f (w\#x) \in H_\varepsilon$.
\end{proof}
% \begin{proof}
% 	Angenommen $M_\varepsilon$ entscheidet $H_\varepsilon$.\\
% 	Konstruiere $M'$ (mit Hilfe von $M_\varepsilon$), sodass $M'$ entscheidet $H$.\\
% 	\begin{tabular}{r@{ }l}
% 	$M':$ & Bei Eingabe $\ulcorner M\urcorner w$.\\
% 	& Konstruiere $M^*$, $M^*$ schreibt zuerst $w$ aufs (leere) Band und 
% 	startet dann $M$ auf $w$.\\
% 	& Wende $M_\varepsilon$ auf $\ulcorner M^*\urcorner$ an.
% 	\end{tabular}\\
% 	$\curvearrowright M'$ entscheidet $H \qquad \lightning$ \autoref{satz:H ist unentscheidbar}
% \end{proof}

 % Latex-Qelle von M. Geffken
Nun betrachten wir \ac{TM}s vom Blickwinkel der von ihnen berechneten
(partiellen) Funktionen. Sei $R$ die Menge der von \ac{TM}s berechneten
Funktionen.

\begin{Satz}[Satz von Rice]\ \\
  Sei $R$ die Menge aller partiellen \ac{TM}-berechenbaren Funktionen und
  $\varnothing \neq S \subsetneq R$ eine nichttriviale (nicht-leere,
  echte) Teilmenge davon.\\
  Dann ist $L(S)=\{\ulcorner M \urcorner \mid M \text{ berechnet Funkt. aus
  }S\}$ unentscheidbar.
\end{Satz}
\begin{proof}
  Angenommen $M_S$ entscheidet $L(S)$.\\
  Sei $\Omega \in R$ die überall
  undefinierte Funktion. Wir nehmen an, dass $\Omega \in S$ (anderenfalls betrachten wir $\overline{L(S)}$).
  
  Da $R \setminus S \neq \varnothing$ gibt es eine berechenbare Funktion $ f \in R \setminus
  S$ und $f$ werde von \ac{TM} $M_f$ berechnet.
  
  Definiere $M'=M'_{(M, f)}$ wie folgt: $M'$ führt zunächst $M$ (beliebige \ac{TM}) auf leerer Eingabe aus. Falls $M$ anhält, wendet $M'$ dann $M_f$ auf die tatsächliche Eingabe an.\\
  Die von $M'$ berechnete Funktion ist also
  $f_{M'}=\begin{cases}
    f & \text{falls } M \text{ auf leerem Band hält,}\\
    \Omega & \text{sonst.}
  \end{cases}$
  
  Definiere nun $M''$ wie folgt:
  \begin{itemize}
  \item Bei Eingabe $\ulcorner M \urcorner$ berechne die Gödelnummer von $M'$.
  \item Wende nun $M_S$ auf $\ulcorner M' \urcorner$ an.
    \begin{align*}
      M_s\text{ akzeptiert }\ulcorner M' \urcorner
      &\iff M' \text{ berechnet Funktion in }S\\
      &\iff M' \text{ berechnet }\Omega \text{ ($f_{M'} \in \{\Omega, f\}, \Omega \in S, f \notin S$ )}\\
      &\iff M \text{ hält \emph{nicht} auf leerem Band an}
    \end{align*}
  \end{itemize}
  Also entscheidet $M''$ $H_\varepsilon$.$\qquad\lightning$
\end{proof}

\subsection{Eigenschaften von entscheidbaren und semi-entscheidbaren Sprachen}
\begin{Satz}[name={[Eigenschaften von Entscheidbarkeit]}]
\label{thm:eigensch-von-entsch}
  Seien $L_1$ und $L_2$ entscheidbar. Dann sind $\overline{L_1}$, $\overline{L_2}$, $L_1 \cup L_2$ und $L_1 \cap L_2$ entscheidbar.
\end{Satz}\vspace{-1.5em}
\begin{proof}
  Übung oder selbst.
\end{proof}
\begin{Satz}[name={[Eigenschaften von Semi-Entscheidbarkeit]}]
  Seien $L_1$ und $L_2$ semi-entscheidbar. Dann sind $L_1 \cup L_2$ und $L_1 \cap L_2$ semi-entscheidbar.
\end{Satz}
\begin{proof}
  vgl. \autoref{satz:Typ-0-abgeschl}.
\end{proof}
\csname repeatSatz613\endcsname*
\begin{Satz}
  Die Menge der semi-entscheidbaren Sprachen ist \emph{nicht}
  unter Komplement abgeschlossen.
\end{Satz}
\begin{proof}
  Laut \autoref{satz:6.10} und \autoref{kor:6.11} sind das spezielle Halteproblem $K$ und $\overline{K}$ nicht entscheidbar.
  
  $K$ ist semi-entscheidbar, aber nicht $\overline{K}$.
\end{proof}


\subsection{Weitere unentscheidbare Probleme}
\draftnote{1.2.17}
\paragraph[\acf*{PCP}]{\acf{PCP}}\ \\
\emph{Gegeben:}\\
Endliche Folge von Wortpaaren $K=((x_1, y_1), \dots, (x_k, y_k))$ mit $x_i, y_i \in \Sigma^+$

\emph{Gesucht:}\\
Indexfolge $i_1, \dots, i_n \in \{1, \dots, k\}\ (n \geq 1)$, so dass $x_{i_1} \cdots x_{i_n}=y_{i_1} \cdots y_{i_n}$

Die Folge $i_1, \dots, i_n$ (falls diese existiert) heißt \emph{Lösung} des Korrespondenzproblems $K$.
\begin{Bsp*}
\begin{align*}
	K &=((\underbrace{1,101}_{x_1, y_1}), (\underbrace{10, 00}_{x_2, y_2}), (\underbrace{011, 11}_{x_3, y_3}))\\
\shortintertext{besitzt die Lösung $(1,3,2,3)$, denn}
	x_1 x_3 x_2 x_3 &= \underbrace{1 \cdot 01}_{y_1}\underbrace{1 \cdot 1}_{y_3}\underbrace{0 \cdot 0}_{y_2}\underbrace{11}_{y_3} = y_1 y_3 y_2 y_3
\end{align*}
\end{Bsp*}\vspace{-1em}
\emph{Frage:}
\begin{align*}
  x_1&=001 & x_2&=01 & x_3&=01 & x_2&=10\\
  y_1&=0  & y_2&=011 & y_3&=101  & y_2&=001\\
\end{align*}
Besitzt dieses \ac{PCP} eine Lösung? Ja, aber mit $66$ Indizes [Schöning, S.124]
\begin{Bemerkung}\ \\
  Offensichtlich ist das \ac{PCP} semi-entscheidbar: Systematisches
  Ausprobieren von Indexfolgen findet Lösung nach endlicher Zeit,
  \emph{sofern es eine gibt}.
\end{Bemerkung}
Ziel: \ac{PCP} ist unentscheidbar. Vorbereitung:
Es interessiert uns ab hier nur, ob das Problem eine Lösung hat oder nicht.
\paragraph[\acf*{MPCP}]{\acf{MPCP}}\ \\
\emph{Gegeben:} wie bei \ac{PCP}

\emph{Gesucht:} Lösung des \acsu{CP} mit $i_1=1$
\begin{lemma}[name={[MPCP $\preceq$ PCP]}]
    \ac{MPCP} $\preceq$ \ac{PCP}
\end{lemma}
\begin{proof}
	Betrachte \ac{MPCP} $K=((x_1,y_1),\dots(x_k,y_k))$ über $\Sigma$.\\
	Sei $\Sigma'=\Sigma\uplus\{\#,\$\}$ \\
	Für ein Wort $w=a_1\dots a_n\in\Sigma^+$ sei
	\begin{align*}
		\bar w &= \#a_1\#a_2\#\dots\#a_n\#\\
		\grave w &= \#a_1\#a_2\#\dots\#a_n &&\text{(am Ende kein \#)}\\
		\acute w &= \phantom{\#}a_1\#a_2\#\dots\#a_n\# &&(\text{am Anfang kein }\#)
	\end{align*}
	Definiere nun
	\[ f(K) = (\underbrace{(\bar x_1,\grave y_1)}_1, \underbrace{(\acute x_1, \grave y_1)}_2, \underbrace{(\acute x_2,\grave y_2)}_{2+1}, \dots, \underbrace{(\acute x_k,\grave y_k)}_{k+1}, \underbrace{(\$,\#\$)}_{k+2}) \]
	eine totale berechenbare Funktion.
	
	Zeige $K \in \ac{MPCP} \<==> f(K) \in \ac{PCP}$:
	
	"`$\==>$"': $1, i_2, \dots, i_n$ Lösung für $K$\\
	$\curvearrowright\ 1, i_{2}+1, \dots, i_{n}+1, k+2$ Lösung für $f(K)$
	
	"`$\<==$"':
  \begin{enumerate}
  \item \label{it:mpcp-pcp-only-if-1} Sei $i_1, \dots, i_n$ Lösung für $f(K)$, in der das Paar $k+2$ höchstens einmal vorkommt.\\
  $\curvearrowright$  Durch die Stuktur der Worte gilt für Lösungen immer: $ i_1=1,\ i_n=k+2 $ (ansonsten fehlt am Anfang oder am Ende das Symbol $\#$)\\
  Es gilt ferner für $1 < j < n$
  \begin{itemize}
  \item $i_j \neq 1$, da in der $x$-Konkatenation sonst $\#$ doppelt vorkommt, was in der $y$-Konkatenation nicht möglich ist.
  \item $i_j \neq k+2$, da $k+2$ per Annahme nur einmal vorkommt.
  \end{itemize}
  Also gilt für $1 < j < n$: $i_j \in \{2, \dots, k+1\}$.\\
	$\curvearrowright 1, i_{2}-1, \dots, i_{n-1}-1$ Lösung für $K$
\item Sei $i_1, \dots, i_n$ Lösung für $f(K)$, in der das Paar $k+2$ mehrmals vorkommt.
  Dann gibt es auch eine Lösung $i_m,\dots,i_{m+l}$ mit $1 \le m \le m+l \le n$ so dass $k+2$ nur einmal vorkommt (ohne Beweis).
  Weiter bei \ref{it:mpcp-pcp-only-if-1}.
  \end{enumerate}

\end{proof}
Es reicht nun zu zeigen, dass \ac{MPCP} unentscheidbar ist!

\begin{lemma}[name={[H $\preceq$ \ac{MPCP}]}] 
	H $\preceq$ \ac{MPCP}
\end{lemma}
\begin{proof}
	\ac{TM} $M=(Q,\Sigma,\Gamma,\delta,q_0,\blank,F)$ und Eingabewort $w\in\Sigma^*$.
	
	Gesucht: totale berechenbare Funktion, die $(\ulcorner M \urcorner, w) \mapsto \underbrace{(x_1,y_1),\dots,(x_k,y_k)}_k$, sodass\\
	$\ulcorner M \urcorner w\in H \text{ gdw } K$ eine Lösung als \ac{MPCP} besitzt.
	
	Idee: Definiere $K$ so, dass die Berechnung von $M$ simuliert wird.\\
	Alphabet für $K: \triangle = \Gamma\cup Q\cup\{\#\}$\\
	$(x_1,y_1) = (\#,\#q_0w\#)$
	\begin{enumerate}
	\item Kopieren
		\[ (a,a)\quad, a\in\Gamma\cup\{\#\} \]
	\item Transition ($a \in \Gamma$)
		\begin{alignat*}{2}
			(qa,q'a') &\quad& \forall q,a:\delta(q,a) &\ni(q',a',N)\\
			(qa,a'q') && \dots\quad &\ni(q',a',R)\\
			(bqa,q'ba') &&  &\ni(q',a',L),b\in\Gamma\\
			(q\#,q'a'\#) && \forall q: \delta(q,\blank)&\ni(q',a',N)\\
			(q\#,a'q'\#) && &\ni(q',a',R)\\
			(bq\#,q'ba'\#) && &\ni(q',a',L),b\in\Gamma\\
			(\#qa,\#q'\blank a')&& \forall q,a:\delta(q,a) &\ni(q',a',L)\\
			(\#q\#,\#q'\blank a'\#)&& \forall q:\delta(q,\blank) &\ni(q',a',L)
		\end{alignat*}
	\item Löschen
		\begin{align*}
      &(aqb,qb) \text{ für $a,b\in \Gamma$  und $\delta(q,b) = \emptyset$ } \\
      &(qba,qb) \text{ für $a,b\in \Gamma$ und $\delta(q,b) = \emptyset$ } \\
		\end{align*}
	\item Abschluss
    \begin{align*}
      (qb\#\#,\#) \text{ für $b \in \Gamma$ und $\delta(q,b) = \emptyset$}
      (q\#\#,\#) \text{ für $\delta(q,\blank) = \emptyset$}
  \end{align*}
	\end{enumerate}
	$\ulcorner M \urcorner w\in H$\\
  \<==> Folge von $\Konf$ von $M,\ k_0\dots k_t$ mit $k_0 =q_0w$ und $k_t = uq'bv$ mit $\delta(q',b) = \emptyset$
	mit $h_{i-1}\vdash k_i\quad \forall 1\leq i\leq t$\\
	\<==> Die Instanz $K$ von \ac{MPCP} besitzt Lösung und ein Lösungswort der Form
	\[ \#k_0\#k_1\# \dots \#h_t\#k_t^1\#k_t^2\# \dots \#q'b\#\# \]
  oder 
	\[ \#k_0\#k_1\# \dots \#h_t\#k_t^1\#k_t^2\# \dots \#q'\#\# \]
	wobei $k_t^0 = k_t$ und $k_t^j$ durch Streichen eines Bandsymbols rechts oder links von $q'$ bzw $q'b$ aus ihrem Vorgänger $k_t^{j-1}$entsteht.

  Intuition: ``Die Konkatenation der $x_i$s hinkt immer um eine Konfiguration der Konkatenation der $y_i$s hinterher''.
\end{proof}
\begin{Satz}[name={[\ac{PCP} ist unentscheidbar.]}]
	\ac{PCP} ist unentscheidbar.
\end{Satz}
\begin{proof}
	$\text{H}\leq\ac{MPCP}$ und $\ac{MPCP}\leq\ac{PCP}$
\end{proof}

\draftnote{3.2.17}

\begin{Satz}
	Das Schnittproblem "`$L(\mathcal{G}_1)\cap L(\mathcal{G}_2)\neq \varnothing$?"' für \ac{CFL} ist unentscheidbar.
\end{Satz}
\begin{proof} Durch Reduktion $\ac{PCP}\leq{}$Schnittproblem.\\
	Sei $K=\{(x_i,y_i) \mid 1\leq i\leq k\}$ Instanz von \ac{PCP} über $\Sigma$.\\
	Berechne aus $K$ zwei \ac{CFG} $\mathcal{G}_1$ und $\mathcal{G}_2$, so dass $K$ eine Lösung hat $\<==> L(\mathcal{G}_1)\cap L(\mathcal{G}_2)\neq \varnothing$
	\begin{align*}
		\mathcal{G}_1: S_1\-> &1x_1 |\dots |kx_k &&\text{Alphabet}: \Sigma\cup\{1\dots k\}\\
		& |1S_1x_1|\dots|kS_1x_k\\
		\mathcal{G}_2: S_2\-> &1y_1 |\dots |ky_k\\
		&|1S_2x_1|\dots|kS_2y_k
	\end{align*}
	\begin{alignat*}{2}
		&&w&\in L(\mathcal{G}_1)\cap L(\mathcal{G}_2)\\
		\<==>&\ &  w &= k_n\dots k_1,xk_1\dots xk_n\\
		&&&= k_n\dots k_1,yk_1\dots yk_n\\
		\<==>&& \mathrlap{\hspace{-1ex}(k_1\dots k_n)\text{ ist Indexfolge zur Lösung von \ac{PCP} } k} \tag*{\qedhere}
	\end{alignat*}
\end{proof}
\paragraph{Folgerung :}Schnittproblem für Typ 1 und Typ 0 Sprachen ist ebenfalls unentscheidbar.

\begin{Korollar}
	Das Schnittproblem ist auch für \ac{DCFL} unentscheidbar.
\end{Korollar}
\begin{proof}
	$L(G_1)$ ist auch \ac{DPDA} erkennbar.
\end{proof}

\begin{Satz}
	Das Äquivalenzproblem für \ac{CFL} ist unentscheidbar.
\end{Satz}
\begin{proof}
	Sei $A=\{\mcG_1,\mcG_2 \mid L(\mcG_1)=L(\mcG_2)\}$\\
	Angenommen $\mcG_1,\mcG_2$ sind Typ 2 Grammatiken für \ac{DCFL}.\\
	Dann ist $(\mcG_1,\mcG_2)\in{}$Schnittproblem.
	\begin{align*}
		\<==> L(\mcG_1) &\cap L(\mcG_2)=\varnothing\\
		\<==> L(\mcG_1) &\subseteq \overline{L(\mcG_2)}
	\end{align*}
	Da $\mcG_2$ eine \ac{DCFG} $\exists\,\mcG_2$ mit $L(\mcG'_2)=\overline{L(\mcG_2)}$ (Abschluss unter Komplement).
	\begin{align*}
		\<==>\quad & L(\mcG_1) \subseteq L(\mcG'_2) \quad\leadsto\text{Inklusionsproblem} \tag{$*$}\label{eq:Inklusionsproblem}\\
		\<==>\quad & L(\mcG_1) \cup L(\mcG'_2) = L(\mcG'_2)\\
	\shortintertext{Wegen Abschluss unter $\cup:\exists\,\mcG_3\in\ac{CFG}$ mit $L(\mcG_3)=L(\mcG_1)\cup L(\mcG'_2)$}
		\<==>\quad & L(\mcG_3)=L(\mcG'_2)\\
		\<==>\quad & (\mcG_3,\mcG'_2)\in A
	\end{align*}
	$\curvearrowright$ Äquivalenzproblem ist unentscheidbar.\\
	\eqref{eq:Inklusionsproblem} \-> (Inklusionsproblem ist ebenfalls unentscheidbar.)
\end{proof}

\begin{Satz}
	Das Leerheitsproblem für Typ 1 Sprachen ist unentscheidbar.
\end{Satz}
\begin{proof}
	Reduktion auf Schnittproblem für \ac{CFL}.\\
	Sei $(\mcG_1,\mcG_2)\in{}$Schnittproblem (Typ 1).\\
	Insbesondere $\mcG_1,\mcG_2$ Typ 1 Grammatiken.\\
	Typ 1 Sprachen sind unter $\cap$ abgeschlossen, also $\exists\, \mcG$ Typ 1 Gramatik mit $L(\mcG)=L(\mcG_1)\cap L(\mcG_2)$\\
	Also "`$L(\mcG)=\varnothing$"' unentscheidbar.
\end{proof}


%%% Local Variables:
%%% mode: latex
%%% TeX-master: "Info_3_Skript_WS2016-17"
%%% End:

% \section[Komplexitätstheorie]{Komplexitätstheorie}
\subsection{Komplexitätsklassen und P/NP}

{\color{red} TODO: Dieser rote Teil des Skriptes wir erst im Lauf des Wochenendes in einen akzeptablen Zustand gebracht.

Aus Info2 kennen Sie: Worst-case Laufzeit eines Algorithmus bestimmen.

Hier: Betrachte Problemstellung und bestime was Wort-case Laufzeit von bestmöglichem Algorithmus ist.

\begin{Def}
Sei $f:\N\rightarrow\N$ Funktion und $\M$ eine nichtdet. \ac{TM} mit Eingabealphabet $\Sigma$.
\begin{itemize}
 \item $\M$ hat \emph{Zeitkomplexität} $f(n)$, falls $\forall w\in\Sigma^*$ mit Länge $n$ gilt: $\M$ hält auf Eingabe $w$ für jede Berechnung in höchstens $f(n)$ Schritten.
 \item $\M$ hat \emph{Platzkomplexität} $f(n)$ wenn
\end{itemize}
\end{Def}


\begin{Def}[name={[$\NTIME$ Klasse]}]
	Sei $f:\N\->\N$ eine Funktion.
	\begin{itemize}
	 \item $\DTIME(f(n)) = \{L\mid \exists \text{det. Mehrband TM $\M$ sodass $L(\M)=L$ und $\M$ hat Zeitkomplexität $f(n)$} \} $
	 \item $\NTIME(f(n)) = \{L\mid \exists \text{nichtdet. Mehrband TM $\M$ sodass $L(\M)=L$ und $\M$ hat Zeitkomplexität $f(n)$} \} $
	 \item $\DSPACE(f(n)) = \{L\mid \exists \text{det. Mehrband TM $\M$ sodass $L(\M)=L$ und $\M$ hat Platzkomplexität $f(n)$} \} $
	 \item $\NSPACE(f(n)) = \{L\mid \exists \text{nichtdet. Mehrband TM $\M$ sodass $L(\M)=L$ und $\M$ hat Platzkomplexität $f(n)$} \} $
	\end{itemize}

% 	Die Klasse $\NTIME(f(n))$ besteht aus allen Sprachen, die von einer (Mehrkanal-)\ac{TM} $M$ in $T_M(w)\leq f(|w|)$ akzeptiert werden.\\
% 	Dabei $T_M(w) =
% 	\begin{cases}
% 		\mathrlap{\text{Anzahl der Schritte einer kürzesten akzeptierenden Berechnung von $M$ auf }w}\\
% 		1 & \text{falls }\nexists
% 	\end{cases}$\\
\end{Def}

Diagramm

\bigskip

Probleme als Sprachen.

Bsp: Erfüllbarkeitsproblem der Aussagenlogik.

\begin{Def}[name={[$SAT$: Erfüllbarkeitsproblem der \acs*{AL}]}]
	$SAT$, das Erfüllbarkeitsproblem der \acf{AL} ist  definiert durch
	\begin{description}
	\item[Eingaben:] Formal $F$ der \acl{AL}.
	\item[Frage:] Ist $F$ erfüllbar, d.h. existiert eine Belegung $\beta$ der Variablen mit $\{0,1\}$, so dass $F[\beta]=1$ ist.
	\end{description}
	$SAT=\{\code(F) \mid F\text{ist erfüllbare Formel der \acs{AL}}\}$
\end{Def}

Exkurs Aussagenlogische Formel.

Bekannt aus Logik. Beschreiben unsere Notation mit Hilfe folgender Grammatik.

\begin{Bsp}\label{bsp:3.sameNumber}
  $\mathcal{G} = (\Sigma, N, P, S)$ mit
	\begin{align*}
		\Sigma &= \{ 0, 1 \}\\
		N &= \{S\}\\
		P &= \begin{aligned}[t]
      \{ S & \to 1S0S \\
        S & \to 0S1S \\
        S & \to \Eps
      \}
        \end{aligned}
      \qedhere
	\end{align*}
\end{Bsp}


$\{F\in L(G_{AL})\mid F \text{erfüllbar}\}$

Betrachen zunächst Teilmenge.

DTM?

Laufzeitkomplexität?

Exkurs
\begin{Def} Sei $g:\N\rightarrow\N$
$$\mathcal{O}(g(n)) = \{f:\N\rightarrow\N \mid \exists n_0, k\in\N \forall n\geq n_0: f(n)\leq k\cdot g(n)\}$$
\end{Def}

Anschaulich: $f(n)\in \mathcal{O}(g(n))$ $f$ wächst garantiert nicht viel stärker als $g$.


\bigskip

\begin{Def}[name={[Polynom]}]
	Ein Polynom ist eine Funktion $p:\N\->\N$ mit $\exists k\in \N\ a_0,\dots,a_k\in\N$ und \mbox{$p(n)=\sum_i^k a_in^k$}
\end{Def}
\begin{Def}
 \[ P = \bigcup_{p\text{ Polynom}} DTIME(P(n)) \]
 \[ NP = \bigcup_{p\text{ Polynom}} \NTIME(p(n)) \]
 
\end{Def}



% \begin{Def}[name={[$NP$ Klasse]}]
% 	Die Klasse $NP$ besteht aus allen Sprachen, die von \ac{NTM} in polynomieller Zeit akzeptiert werden können.
% 	\[ NP = \cup_{p\text{ Polynom}} \NTIME(p(n)) \]
% \end{Def}
% Analog für deterministische TM:
% \begin{Def}[name={[$\DTIME$ Klasse]}]
% 	Sei $f:\N\->\N$ Funktion\\
% 	$\DTIME(f(n)) =$ Klasse der Sprachen, die von \ac{DTM} in $T_M(w)\leq f(|w|)$ Schritten akzeptiert wird.
% 	\[ P = \cup_{p\text{ Polynom}} DTIME(P(n)) \qedhere \]
% \end{Def}
Offenbar $P\leq NP$. Seit 1970 weiß man nicht, ob $P=NP$ oder $P\neq NP$
\begin{description}
\item[Praktische Relevanz:] Es existieren wichtige Probleme, die offensichtlich in $NP$ liegen, aber die besten bekannten Algorithmen sind exponentiell.\\
	Beispiel: Traveling Salesman ($O(2^n)$), Erfüllbarkeit der Aussagenlogik.
\item[Struktur:] Viele der $NP$-Probleme haben sich als gleichwertig erwiesen, in dem Sinn, dass eine $P$-Lösung für alle anderen liefert.\\
	$\leadsto NP$-Vollständigkeit.
\end{description}
}

\begin{Def}[Polynomielle Reduktion]\label{def:PolyReduktion}\ \\
  Seien $U, V \subseteq \Sigma^*$ Sprachen.
  \emph{$U$ ist auf $V$ polynomiell reduzierbar (Notation: $U \preceq_p V$)}, falls eine totale, berechenbare Funktion
  $f:\Sigma^* \to \Sigma^*$ existiert, sodass
  \begin{itemize}
   \item $\forall w \in \Sigma^*:w \in U \iff f(w) \in V$
   \item $f$ wird von det. TM berechnet deren Laufzeitkomplexität ein polynom ist.
  \end{itemize}
  Wir nennen $f$ \emph{Reduktionsfunktion}.
\end{Def}

\datenote{02.02.2018}
Eine aussagenlogische Formel $F$ ist in \emph{konjunktiver Normalform} (CNF) wenn $F$ eine Konjunktion von Disjunktionen von Literalen ist.
Wir verwenden $CNF$ als Notation für die Menge aller AL Formeln in CNF.
Analog schreiben wir $3CNF$ für die Menge aller Formeln in CNF bei denen jeder Konjukt aus höchstens drei Disjunkten besteht.



\begin{Def}[name={[$3SAT$]}]
Das Problem 3SAT ist wie folgt definiert.\footnote{
Typischerweise nennt man Sprachen die im kontext der Komplexitätstheorie definiert wernden \emph{Probleme}.
Es ist üblich von einem konkretes Alphabet $\Sigma$ und keine konkrete Codierung der Objekte (hier Formeln) zu abstrahieren und das Problem als (Gegeben,Frage)-Paar zu formulieren.
Es bleibt dem Leser überlassen sich selbst ein geeignete Alphabete und Codierungen zu überlegen.
Für den Fall von aussagenlogische Formeln haben wir dies in ... nocheinmal gemacht, werden aber in Zukunft darauf verzichten. }
\begin{center}
\framebox[\textwidth]{\parbox{.95\textwidth}{
\smallskip
\textit{Gegeben:} Eine aussagenlogische Formel $F\in$ 3CNF

\medskip

\textit{Frage:} Ist $F$ erfüllbar.
}}
\end{center}
	
\end{Def}
Alternativ könnten wir die Definition von $3SAT$ auch wie folgt aufschreiben.
$$3SAT = \{F\in L(\mathcal{G}_\mathsf{AL})\mid F \text{ ist in CNF und hat höchstens 3 Literale pro Klausel}\}$$

Offensichtlich gilt $3SAT \preceq_p SAT$.\footnote{
In der Vorlesung vom 02.02. wurde (fälschlicherweise) gesagt, die Identität sei eine geeignete Reduktionsfunktion.
Dies ist nicht korrekt, da z.B. 
$(A_1\lor A_2\lor A_3\lor A_4)\notin 3SAT$
(weil nicht in 3CNF) aber $(A_1\lor A_2\lor A_3\lor A_4)\in SAT$.
} Wir zeigen als nächstes auch die umgekehrte Richtung.

\begin{lemma}\label{lem:sat3sat}
	$SAT \preceq_p 3SAT$
\end{lemma}


\begin{proof}
Unser Zeil ist nun eine Transformation anzugeben die sich in polynomieller Zeit berechnen lässt, die jede Aussagenlogische Formel $F_\mathsf{AL}$ in eine 3CNF Formel $F_\mathsf{3CNF}$ überführt, sodass 
$F_\mathsf{AL}\in SAT \<==> F_\mathsf{3CNF}\in 3SAT$
gilt.

Sei $F_\mathsf{AL}$ eine beliebige Formel aus $L(\mathcal{G}_\mathsf{AL})$.

\begin{enumerate}
 \item Erzeuge aus $F_\mathsf{AL}$ eine äquivalente Formel $F_\mathsf{NNF}$ in Negationsnormalform\footnote{Eine Formel ist in Negationsnormalform wenn der Negationsoperator immer nur direkt vor einer Variablen vorkommt.}.
 Wir können jede Formel in Negationsnormalform bringen indem wir mit Hilfe der De Morganschen Regeln die Negationen ``nach innen'' ziehen.
 
 Beispiel: $\neg(\neg (A_1\lor \neg A_3)\land A_2) \quad\rightsquigarrow\quad ((A_1\lor \neg A_3)\land \neg A_2)$
 
 Ideen zur Implementierung: Erstelle auf zusätzlichem Band eine Kopie der Formel. 
 Lasse vor jedem nicht negierten Literal ein Feld Platz um später ggf. ein Negationssymbol zu plazieren.
 Wende De Morgans Regel von Aussen nach Innen an.
 Laufe für jede Anwendung ein mal über die Formel.
 
 
 \item Erzeuge aus $F_\mathsf{NNF}$ eine Formel $F_{\alpha\gamma}$ mit Biimplikationszeichen ``$\<->$'' mit Hilfe der folgenden induktiv definierten Abbildungen $\alpha$ und $\gamma$.\footnote{
 Die hier beschriebene Transformation ist auch als Tseytin-Transformation bekannt.}
 
 Idee: Die Abbildung $\gamma$ liefert für jede $\land$-Teilformel und jede $\lor$-Teilformel eine neue Hilfsvariable.
 Diese Hilfsvariable hat in der Resultierenden Formel den Wahrheitswert, den die entsprechende Teilformel in der Eingabeformel hätte.
 Die Abbildung $\alpha$ konstuiert für jede Teilformel die entsprechenden Bedingungen für die Hilfsvariable.
 Erzeugt $\gamma$ keine Hilfsvariable, erzeugt $\alpha$ die (nicht einschränkende) Bedingung $1$.
 
 $$\gamma(F)=
 \begin{cases}
   0 & \text{ falls } F=0\\
   1 & \text{ falls } F=1\\
   A & \text{ falls } F=A\\
   \neg F_1 & \text{ falls } F=\neg F_1\\
   B_F & \text{ falls } F=F_1\land F_2\\
   B_F & \text{ falls } F=F_1\lor F_2
  \end{cases}$$
 
 
 
 $$\alpha(F)=
 \begin{cases}
   1 & \text{ falls } F=0\\
   1 & \text{ falls } F=1\\
   1 & \text{ falls } F=A\\
   1 & \text{ falls } F=\neg F_1\\
   \big(\gamma(F)\<-> \gamma(F_1) \land \gamma(F_2)\big)\land\alpha(F_1)\land\alpha(F_2) & \text{ falls } F=F_1\land F_2\\
   \big(\gamma(F)\<-> \gamma(F_1) \lor \gamma(F_2)\big)\land\alpha(F_1)\land\alpha(F_2) & \text{ falls } F=F_1\lor F_2
  \end{cases}$$
  
  Wir definieren das Resultat $F_{\alpha\gamma}:=\gamma(F_\mathsf{NNF})\land \alpha(F_\mathsf{NNF})$
  
  Beispiel:
  \begin{align*}
   \alpha((A_1\lor \neg A_3)\land \neg A_2)) = & \;
  \big(B_{((A_1\lor \neg A_3)\land \neg A_2)} \leftrightarrow (B_{(A_1\lor \neg A_3)} \land \neg A_2)\big)\\
  & \land (B_{(A_1\lor \neg A_3)} \leftrightarrow (A_1\lor \neg A_3)) \land 1 \land 1\\
  & \land 1
  \end{align*}

  
  Ideen zur Implementierung:
  
  Wähle Bandalphabet so dass $B$ enthalten, verwende außerde eine zusätzlich Art von Klammern (z.B. eckige Klammern) um den Subscriptanteil der B-Variablen von anderem Bandinhalt zu unterscheiden.
  Laufe für jede $\{\land,\lor\}$-Teilformel einmal über die Eingabe.
  Zusätzliches Band zum Schreiben von Resultat.
  Zusätzliches Band für aktuell bearbeitete $\{\land,\lor\}$-Teilformel da diese immer zwei mal benötigt wird (Variablenname und $\alpha$)
  Bearbeite äußere Teilformeln vor inneren.
  Lösche Teile die nicht mehr benötigt werden.
\item Ersetze die Formelteile mit Biimplikationszeichen in $F_{\alpha\gamma}$ wie folgt durch logisch äquivalente Formeln in 3CNF.
	\begin{itemize}
	\item $F_1\<-> (F_2\land F_3)
% 		= \big(F_1\land (F_2\land F_3)\big)\lor \big(\neg F_1\land \neg(F_2\land F_3)\big)
% 		= \big(F_1\lor \neg(F_2\land F_3)\big)\land \big(\neg F_1\lor (F_2\land F_3)\big)
		\quad\rightsquigarrow\quad \big(F_1\lor \neg F_2\lor \neg F_3\big)\land \big(\neg F_1\lor F_2\big)\land \big(\neg F_1\lor F_3\big)
		$
	\item $F_1\<-> (F_2\lor F_3)
% 		= \big(F_1\lor \neg(F_2\lor F_3)\big)\land \big(\neg F_1\lor (F_2\lor F_3)\big)
		\quad\rightsquigarrow\quad \big(F_1\lor \neg F_2\big)\land\big(F_1\lor \neg F_3\big)\land \big(\neg F_1\lor F_2\lor F_3\big)
$
	\end{itemize}
	
	Ideen zur Implementierung:
	
	Zusätzliches Band zum Schreiben von Resultat.
    Zusätzliches Band für Operanden der Biimplikation da diese immer mehrfach benötigt werden.
  
\item Ersetzte alle Aussagenlogischen Variablen der Form $B_F$ durch aussagenlogische Variablen der Form $A_i$.\footnote{
Dieser Schritt ist nur nötig damit das Resultat in dem von uns definierten Alphabet $\Sigma_{AL}$ dargestellt werden kann.
Alternativ hätten wir auch zu Beginn ein Reichhaltigeres Alphabet wählen können.
}

	Ideen zur Implementierung:
	
    Finde zunächst höchsten Index $i_{max}$ von $A_i$ Variablen (speichere akutelles Maximum auf zusätzlichem Band).
    Verwende anschließend $i_{max}+1, i_{max}+2, \ldots$ für neue Variablen.
    Schreibe zunächst eine Übersetzungsvorschrift (z.B. $A_4:=B_{((A_1\lor \neg A_3)\land \neg A_2)}, A_4:=B_{(A_1\lor \neg A_3)}$) auf ein zusätzliches Band und ersetze erst dann.

\end{enumerate}

Sei $f:\Sigma_{AL}\rightarrow\Sigma_{AL}$ eine Funktion die alle Formeln aus $L(\mathcal{G}_\mathsf{AL})$ ensprechend obiger Konstuktion abbildet und alle anderen Wörter unverändert lässt, dann gilt:
\begin{itemize}
 \item $F_{AL}\in SAT \<==> f(F_{AL})\in 3SAT$ (hier ohne Beweis)
 \item $f$ ist total und lässt sich in polynomieller Zeit berechnen (hier ohne Beweis)
\end{itemize}



\end{proof}



\begin{lemma}\label{lem:A<B + BinP => AinP}
	Falls $A\preceq_p B$ und $B\in P$ (bzw. $NP$) dann auch $A\in P$ (bzw. $NP$).
\end{lemma}
\begin{proof}
	$B\in P: \exists M$, die $B$ in $p(n)$ Schritten akzeptiert.\\
	$\exists M_f$, die die Reduktion $A\preceq_p B$ implementiert. Die Laufzeit von $M_f$ sei durch $q$ Polynom beschränkt.\\
	Betrachte $M'=$"'erst $M_f$, dann $M$ auf dem Ergebnis"'
	$M'$ akzeptiert $A$.\\
	$w\in A\ M_f(w)$ liefert $f(w)$ in $\subseteq q(|w|)$ Schritten ohne $|f(m)|\subseteq q(|w|)$\\
	$M(f(w))$ benötigt $\leq p(|f(w)|)\leq p(q(|w|))$ Schritte zum akzeptieren.\\
	$\curvearrowright A\in \DTIME(q(w)+p(q(w))\subseteq P$
\end{proof}


% \draftnote{8.2.17}

\begin{Def}[name={[$NP$-schwierig und $NP$-vollständig]}]\
	\begin{itemize}
	\item Eine Sprache $U$ heißt \emph{$NP$-schwierig}, falls $\forall L\in NP : L\preceq_p U$.
	\item Eine Sprache $U$ heißt \emph{$NP$-vollständig}, wenn $U$ $NP$-schwierig und $U\in NP$. \qedhere
	\end{itemize}
\end{Def}

{\color{red}
TODO: Ausformulieren
\begin{itemize}
 \item $NP$-schwierig, sehr starke Forderung dass an alle(!) NP Probleme reduzieren kann
 \item zunächst unklar ob es überhaupt ein $NP$-schwieriges Problem gibt
 \item bedeutet wenn ich das $NP$-vollst. Problem gefunden habe dass ich effizient Lösen kann, kann ich alle NP Probleme effizient lösen (wie folgender Satz zeigt).
\end{itemize}
}
\begin{Satz}
	Wenn $A\ NP$-vollständig ist, dann gilt die folgende Äquivalenz.
	\[ A\in P \<==> P=NP \]
\end{Satz}
\begin{proof}\ \\
	"`\<=="' trivial.\\
	"`\==>"' $A\in P\subseteq NP\quad \forall L\in NP: L\preceq_p A$. Nach \autoref{lem:A<B + BinP => AinP} : $L\in P$
\end{proof}

\begin{lemma}[name={[$\preceq_p$ ist reflexiv und transitiv]}]
	$\preceq_p$ ist reflexiv und transitiv.
\end{lemma}
\begin{proof}
Übungsblatt~14, Aufgabe~1.
% 	Identität; ähnlich wie Beweis von \autoref{lem:A<B + BinP => AinP}.
\end{proof}

\begin{Bem}
    Aus der Transitivität folgt:
	Sobald eine $NP$-schwieriges Problem $U$ bekannt ist, reicht es $U\preceq_p V$ zu finden, um zu zeigen, dass $V$ ebenfalls $NP$-schwierig ist.
\end{Bem}
% Ein erstes $NP$-vollständiges Problem.

\begin{Satz}[Cook]\label{satz:cook}
	$SAT$ ist $NP$-vollständig.
\end{Satz}
Wir werden nur für $SAT$ die NP-Schwierigkeit direkt Beweisen und für alle anderen $NP$-vollständigen Probleme die $NP$-Schwierigkeit wie in obiger Bemerkung angedeutet mit Hilfe einer polynomiellen Reduktion zeigen.
Da der Beweis von \autoref{satz:cook} sehr Umfangreich ist wollen wir diesen etwas aufschieben und zunächt die $NP$-Vollständigkeit weiterer Probleme zeigen.


\begin{Satz}[name={[$3SAT$ ist $NP$-vollständig]}]
	$3SAT$ ist $NP$-vollständig.
\end{Satz}
\begin{proof}
 \begin{itemize}
  \item $3SAT\in NP$
  
  Wir können dies auf zwei Arten zeigen. 
  Entweder wir zeigen direkt, dass es eine nichtdeterministische \ac{TM} gibt die $3SAT$ entscheidet oder 
  wir Zeigen dies indirekt mit Hilfe einer polynomiellen Reduktion auf ein Problem in $NP$.
  Wir wählen letzteres Vorgehen da zum Beweis nur nochmal erwähnt werden muss dass $SAT\in NP$ und (offensichtlich) $3SAT\preceq_p SAT$ gilt.
  
  \item $3SAT$ ist NP-schwierig
  
  Folgt aus \autoref{lem:sat3sat}.\qedhere
 \end{itemize}
\end{proof}


Wir gehen davon aus, dass die Leser des Skiptes mit Graphen vertraut sind, machen aber zum fixieren von Notation und Terminologie die folgende Definition.
\begin{Def}
 Ein \emph{gerichteter Graph} ist ein Paar $\mathcal{G}=(V,E)$ bei dem 
 \begin{itemize}
  \item $V$ eine Menge ist, deren Elemente wir \emph{Knoten} nennen und
  \item $E\subseteq V\times V$ eine Menge von geordneten Paaren über $V$ ist. 
  Wir nennen diese geordneten Paare \emph{Kanten}.
 \end{itemize}
 Ein \emph{ungerichteter Graph} ist ein gerichteter Graph bei dem $E$ symmetrisch ist.\footnote{Für die Zwecke dieser Vorlesung ist es komfortabel den ungerichteten Graphen als Speziallfall des gerichteten Graphen zu definieren. Eine häufig verwendete Alternative ist, die Knotenmenge als Menge von zweielementigen Mengen zu definieren.}
\end{Def}



\begin{Def}[$\CLIQUE$]
    Das Problem $\CLIQUE$ ist wie folgt definiert.
    \begin{center}
    \framebox[\textwidth]{\parbox{.95\textwidth}{
    \smallskip
    \textit{Gegeben:} Ein ungerichteter Graph $\mathcal{G}=(V,E)$ und eine Zahl $k\in\N$.

    \medskip

    \textit{Frage:} Hat $\mathcal{G}$ eine $k$-Clique?
    
    Eine $k$-Clique ist eine $k$-elementige Menge von Knoten die paarweise durch Kanten verbunden sind.
    D.h. eine Menge $C$ soass $|C|=k$ und $\forall u, v\in C: u\neq v\rightarrow (u,v)\in E$.
    }}
    \end{center}
\end{Def}
\begin{Satz}[name={[$\CLIQUE$ ist $NP$-vollständig]}]
	$\CLIQUE$ ist $NP$-vollständig.
\end{Satz}
\begin{Bemerkung}
 Der folgende Beweis ist für ein Vorlesungsskript sehr knapp gehalten.
 Der Beweis enthält aber genau die Informationen die wir in Übungsaufgaben und Klausur zum erreichen der maximalen Punktzahl erwarten würde.
 
 Beweise der $NP$-Vollständigkeit folgen typischerweise dem hier verwendeten Schema.
 Der schwierige (und sehr viel Kreativität erfordernde) Teil ist dabei eine geeignete Konstruktion für die Reduktionsfunktion zu finden.
 Für den folgenden Beweis wird die Idee der Konstuktion in \autoref{fig:3sat-clique} mit Hilfe eines Beispieles illustriert.
 
 In unserem Übungs- und Klausuraufgeben wird die Aufgabenstellung meist solch ein Beispiel enthalten.
 Damit soll ein Hinweis auf eine geeignete Konstruktion gegeben werden.
 Eine Besonderheit des folgenden Beweises ist dass die Konstuktion aus zwei Schritten besteht (Erweiterung Eingabeformel, Konstuktion Graph) Beide Schritte werden in \autoref{fig:3sat-clique} illustriert.
\end{Bemerkung}

\begin{proof}
    \begin{itemize}
     \item Zeige $\CLIQUE\in NP$.
     
     Verfahren: 
     Wähle nichtdeterministisch eine $k$-Elementige Knotenmenge $C$.
     Prüfe ob $C$ Clique ist.
     Dies ist in polynomieller Laufzeit möglich, da wir für jedes Knotenpaar höchstens einmal die Menge der Kanten durchlaufen müssen.
     
     \item Zeige das $\CLIQUE$ $NP$-schwierig mit Hilfe der Reduktion  $3SAT \preceq_p \CLIQUE$.
     
     Ziel: Konstruiere für gegebene 3CNF Formel $F$ einen Graphen $\mathcal{G}=(V,E)$ sodass $F\in 3SAT$ genau dann wenn $\mathcal{G}$ eine $k$-Clique hat.
     
     Unsere Konstruktion besteht aus zwei Schritten.
     
     \begin{itemize}
      \item Schritt 1.
      
      Erweitere $F$ so dass jeder Konjunkt aus genau drei Disjunkten besteht.
      Wähle hierfür eine Äquivalenzumformung die einfach nur existiende Disjunkte wiederholt werden.
      Das resultierende $F$ hat also die folgende Form.
      
      $F = \bigwedge\limits_{i=1}^m (z_{i,1}\lor z_{i,2}\lor z_{i,3})$ \ wobei\ $z_{i,j}\in \{A_1,\dots,A_n\}\cup\{\neg A_1,\dots,\neg A_n\}$
      \item Schritt 2.
      	Definiere $\mathcal{G} = (V,E)$ und $k$ wie folgt:
	\begin{align*}
		V &= \{ (i,j) \mid 1\leq i\leq m, j\in\{1,2,3\} \}\\
		E &= \{\big((i,j),(p,q)\big) \mid i\neq p, z_{i,j}\neq\neg z_{p,q}\}\\
		k &= m
	\end{align*}
     \end{itemize}

    Nun gilt:
    
    \begin{align*}
    F \text{ ist erfüllbar } \<==> \;\; & \text{Es gibt Folge $z_{1,j_2},\dots,z_{m,j_m}$ sodass $F$ wahr unter einer}\\
    & \text{Belegung die jedem Folgeglied $1$ zuordnet.}\\
    \<==> \;\;   & \text{Es gibt Folge $z_{1,j_2},\dots,z_{m,j_m}$ sodass $\forall i\neq p: z_{i,j_i}\neq \neg z_{p,j_p}$}\\
    \<==> \;\;   & \text{$\mathcal{G}$ hat Menge von Knoten $\{(1,j_1),\dots,(m,j_m)\}$}\\
    & \text{die paarweise durch Kanten verbunden sind.}\\
    \<==> \;\;   & \text{$\mathcal{G}$ hat $k$-Clique für $k=m$}\\
    \end{align*}
    
    Es gibt eine Reduktionsfunktion die diese Konstruktion in polynomieller Zeit berechnet und somit gilt $3SAT \preceq_p \CLIQUE$.
    \end{itemize}
\end{proof}



 \begin{figure}[H]
 \framebox{
 \begin{minipage}{0.96\textwidth}
 Die Formel 
 $$ (X\lor\neg Y)\land (Y\lor \neg X)\land (X\lor Y) $$
 ist in 3CNF. Die folgende Formel ist äquivalent und jeder Konjunkt besteht aus drei Disjunkten.
 $$ \underbrace{(X\lor\neg Y\lor \neg Y)}_{1}\land \underbrace{(Y\lor \neg X\lor\neg X)}_{2}\land \underbrace{(X\lor X\lor Y)}_{3} $$
 Die Belegung $\beta$ definiert druch $\beta(X)=1, \beta(Y)=1$ ist eine erfüllende Velegung für diese Formel,
 da z.B. der erste Disjunkt im ersten Konjunkt, der erste Disjunkt im zweiten Konjunkt und der dritte Disjunkt im zweiten Konjunkt auf 1 gesetzt sind.
 
 Im folgenden Graphen bilden die Knoten $(1,1)$, $(2,1)$ und $(3,1)$ eine $3$-Clique.
 
 \centering
\begin{tikzpicture}[
% node distance=1mm
]
\node (11) at (-2,-1.5) {$(1,1)$};
\node (12) at (0,-1.5) {$(1,2)$};
\node (13) at (2,-1.5) {$(1,3)$};

\node (21) at (-2,3) {$(2,1)$};
\node (22) at (-3,2) {$(2,2)$};
\node (23) at (-4,1) {$(2,3)$};

\node (31) at (2,3) {$(3,1)$};
\node (32) at (3,2) {$(3,2)$};
\node (33) at (4,1) {$(3,3)$};

\node (11v) [node distance=1mm, below =of 11] {$X$};
\node (12v) [node distance=1mm, below =of 12] {$\neg Y$};
\node (13v) [node distance=1mm, below =of 13] {$\neg Y$};

\node (21v) [node distance=1mm, above left =of 21] {$Y$};
\node (22v) [node distance=1mm, above left =of 22] {$\neg X$};
\node (23v) [node distance=1mm, above left =of 23] {$\neg X$};

\node (31v) [node distance=1mm, above right =of 31] {$X$};
\node (32v) [node distance=1mm, above right =of 32] {$X$};
\node (33v) [node distance=1mm, above right =of 33] {$Y$};

\draw[-] (11) to (21);
\draw[-] (11) to (31);
\draw[-] (11) to (32);
\draw[-] (11) to (33);

\draw[-] (12) to (22);
\draw[-] (12) to (23);
\draw[-] (12) to (31);
\draw[-] (12) to (32);

\draw[-] (13) to (22);
\draw[-] (13) to (23);
\draw[-] (13) to (31);
\draw[-] (13) to (32);

\draw[-] (21) to (31);
\draw[-] (21) to (32);
\draw[-] (21) to (33);

\draw[-] (22) to (33);
\draw[-] (23) to (33);
\end{tikzpicture}
\end{minipage}
}
	\caption{Beispiel zu $3SAT \preceq_p \CLIQUE$}
	\label{fig:3sat-clique}
\end{figure}






\begin{Bsp*}
	\begin{align*}
	F &= (\underbrace{x\lor y\lor \overline{y}}_1) \land (\underbrace{z\lor \overline{y} \lor \overline x}_2)\\
	\mathcal{G} &: \tikz[baseline=(11.base),label distance=-.5em]\graph[math nodes, chain shift={(0,-1)}, group shift={(1,0)}]{
		{x[label={[name=11](1,1)}], y[label={(1,2)}], "\overline y"[label={(1,3)}]}
		--[complete bipartite] 
		{z[label={below:(2,1)}], 22/\overline y[label={below:(2,2)}], "\overline x"[label={below:(2,3)}]}
	};
	\end{align*}\rlnote{Grafik überprüfen}
\end{Bsp*}

\begin{proof}\
	\begin{enumerate}
	\item $SAT\in NP$\\
		Rate nicht deterministisch eine Belegung $\beta$\\
		Werte $F[\beta]$ aus\\
		$\curvearrowright$ in $\NTIME(n)$, polynomiell
	\item $SAT$ ist $NP$-hart.\\
		Zeige: $\forall L\in NP : L \preceq_p SAT$\\
		$L\in NP : \exists p$ Polynom, \ac{NTM} $M$ mit $L=L(M)$ mit Zeitschranke $T_M(w)\leq p(|w|)$.
		
		Sei $w = x_1\dots x_n\in\Sigma^*$ Eingabe für $M$.\\
		Definiere $F$, so dass $F$ erfüllbar $\<==> M$ akzeptiert $w$
		
		Sei $Q=Q(M)$ mit $\{q_1,\dots,q_k\}=Q$\\
		Sei $\Gamma = \Gamma(M)$ mit $\{a_i,\dots,a_l\} = \Gamma$\\
		Definiere folgende Variablen zur Ver. in $F$
		\begin{itemize}
		\item $\mathrm{state}(t,q) = 1$, genau dann wenn $M$ nach $T$ Schritten im Zustand $q$
		\item $\mathrm{pos}(t,i) = 1$, gdw. der Kopf von $M$ steht nach $t$ Schritten auf Position $i$.\\
		$t\in\{0,\dots P(n)\}$\\
		$i\in \{-p(n),\dots,0,1,\dots,p(n)\}$
		\item $\mathrm{tape}(t,i,a) = 1$, gdw. nach $t$ Schritten befindet sich $a$ an Position $i$ auf dem Band.\\
		$t\in\{0,\dots,p(n)\}$\\
		$i\in\{-p(n),\dots,p(n)\}$\\
		$a\in\Gamma$ \qedhere
		\end{itemize}
	\end{enumerate}
\end{proof}
\begin{lemma}
	Für jedes $k\in\N$ existiert eine Formel $G$, sodass $G(x_i,\dots,x_k)=1$ gdw. $\exists j: x_j = 1$ und $\forall i \neq j: x_i = 0$. Es gilt $|G| \in O(k^2)$.
\end{lemma}
\begin{proof}
	\[ G(x_i,\dots,x_k) = \bigvee_{i=1}^k x_i\land \bigwedge_{i\neg j}\neg (x_i\land x_j) \]
	$M=(Q,\Sigma,\Gamma,\delta,q_0,\blank,F)$ erkennt $L$ in $\NTIME(p),\ p$ Polynom.
	
	Ziel: Konstruiere aus $M,w$ eine Formel $F$, so dass
	\begin{align*}
		F\text{ erfüllbar} &\<-> M\text{ akzeptiert }w\\
		\state(t,q) &\phantom{\<->} t\in 0,\dots,p(n), q\in Q\\
		&\<=>\text{ nach $t$ Schritten ist $M$ in Zeile }q\\
		\pos(t,i) &\<=>\text{ nach $t$ Schritten ist Kopf amn Pos } i\quad,\quad -p(n)\leq i\leq p(n)\\
		\tape(t,i,a) &\<=>\text{ nach $t$ Schritten enthält Band}[i] = q\in\Gamma\\
		F &= R \land A\land T_1
	\end{align*}
	\begin{enumerate}
	\item Randbedingungen
		\begin{align*}
			R =&\phantom{\land}\, \bigwedge_t G(\state(t,q1),\dots,\state(t,q_k))\\
			& \land \bigwedge_t G(\pos(t,-p(n)),\dots,\pos(t,D),\dots,\pos(t,p(n)))\\
			& \land \bigwedge_{t,i} G(\tape(t,i,a_1),\dots,\tape(t,i,a_l)
		\end{align*}
	\item Anfangskonfiguration
		\begin{align*}
			A =&\phantom{\land}\; \state(0,q_1)\land \pos(0,1)\\
			&\land\tape(0,1,x_1)\land\dots\land\tape(0,n,x_n)\\
			&\land\bigwedge_{-p(n)\leq i\leq p(n)} \tape(0,i,\blank)
		\end{align*}
	\item Transitionsschritte
	\begin{align*}
		T_1 =& \bigwedge_{\substack{t\in 0,\dots,p(n)-1,\\i,n}} \state(t,q)\land \pos(t,i) \land \tape(t,i,a)\\
		&\-> \state(z+1,q')\land \pos(t+1,i+d) \land \tape(t+1,i,a')\\
		&\delta(q,a)\ni(q',a',d)\\
		&d\in\{-1,a,1\}\\
		T_2 =& \bigwedge_{\substack{t,i,q\\t<p(n)}} \neg \pos(t,i)\land \tape(t,i,a) \-> \tape(t+1,i,a)
	\end{align*}
	\item Endkonfiguration
		\[ E = \bigvee_{q\in F} \state(p(n),q) \]
		$|F|$ ist polznomiell beschränkt in $|M,w|$, also $L\preceq_p \mathrm{SAT}$\\
		$\curvearrowright\ \mathrm{SAT}$ ist $NP$-vollständig.
	\end{enumerate}
\end{proof}

\subsection{Weitere \textit{NP}-vollständige Probleme}



% \section[Rekursive Funktionen]{Rekursive Funktionen\datenote{10.02.16}}
\begin{Def}[Schema der primitiven Rekursion]
	\begin{align*}
		\text{Sei } \mathcal{G}&: \N^k \-> \N \qquad,\ k\geq 0\\
		h&: \N^{k+R}\->\N\\
		\text{Sei } f&: \N^{k+1}\->\N\text{ eine Funktion, die folgende Gleichungen erfüllt:}\\
		f(0,\overline{x}) &= g(\overline{x}) \qquad,\ \overline{x}\in\N^k\\
		\begin{split}
			f(n+1,\overline{x}) &= h(f(n,\overline{x}),n,\overline{x})
		\end{split} \numbereq\label{eq:rekursives f}
	\end{align*}
	Dann sei
	\[ f =: PR(g,h) \]
	aus $g$ und $h$ durch primitive Rekursion definiert.
\end{Def}
\begin{lemma}[name={[$PR(g,h)$ ist wohldefiniert.]}]
	$PR(g,h)$ ist wohldefiniert.
\end{lemma}
\begin{proof}
	\begin{align*}
		\text{Angenommen } f&=PR(g,h)\\
		\text{und } f' &= PR(g,h)\\
	\shortintertext{d.h. $f$ und $f'$ erfüllen \eqref{eq:rekursives f}.}
		\text{Zeige } \forall n\in\N,&\ \forall \overline{x}\in\N^k:\\
		f(n,\overline{x}) &= f'(n,\overline{x})\\
		\text{I.A. }n=0: f(0,\overline{x}) &\overset{\eqref{eq:rekursives f}}= g(\overline{x}) \overset{\eqref{eq:rekursives f}}= f'(0,\overline{x})\\
		\text{I.S. }n\->n+1: f(n+1,\overline{x}) &\overset{\eqref{eq:rekursives f}}= h(f(n,\overline{x}),n,\overline{x})\\
		&\overset{\text{I.V.}}= h(f'(n,\overline{x}),n,\overline{x})\\
		&\overset{\eqref{eq:rekursives f}}= f'(n+1,\overline{x}) \tag*{\qedhere}
	\end{align*}
\end{proof}
\begin{Def}[name={[$\PREC$ primitiv rekursiven Funktionen über $\N$]}]\label{def:PREC}
	Die $\PREC$ der \emph{primitiv rekursiven Funktionen} über $\N$, ist induktiv definiert:
	\begin{enumerate}
	\item\label{itm:Nullfunktion} $\forall k\in\N:
		\begin{aligned}[t]
			&\mathcal{O}^k : \N^k\->\N\\
			&\mathcal{O}^k(\overline{x}) = 0\\
		\end{aligned}$\\
		$\mathcal{O}^k\in\PREC$ (die Nullfunktion)
	\item\label{itm:Projektion} $\forall k\geq1: \forall 1\leq j\leq k:
		\begin{aligned}[t]
			&\pi^k_j : \N^k\->\N\\
			\pi^k_j(x_1,\dots,x_k) = x_j
		\end{aligned}$\\
		$\pi^k_j\in\PREC$ (Projektionen)
	\item\label{itm:Nachfolgerfunktion} $S:\N\->\N\ \in\PREC$ (Nachfolgerfunktion)
	\pagebreak[3]
	\item\label{itm:Komposition} Feld $g:\N^k\->\N\ \PREC$
		\begin{align*}
			\forall 1\leq i\leq k: h_i:\N^m\->\N \quad&\in\PREC\\
			\text{dann ist }g\circ[h_1,\dots,h_k]:\N^m\->\N \quad&\in\PREC\\
			(g\circ[h_1,\dots,h_k])(\overline{x}) = g(h_1(\overline{x}),\dots,h_k(\overline{x})) \quad\x&\in\N^m
		\end{align*}
		(Funktionskomposition)
	\item\label{itm:Rekursion} Falls $g:\N^k\->\N$ und $h:\N^{k+2}\->\N\quad\in\PREC$\\
		dann ist auch $PR(g,h)\in\PREC$ \quad(Rekursion) \qedhere
	\end{enumerate}
\end{Def}
\begin{Bsp*}\
	\begin{enumerate}
	\item Addition $\in\PREC$
		\begin{align*}
			\add(0,x) &=x\\
			\add(n+1,x) &= \add(n,x)+1\\
			&= h(\add(n,x),n,x)\\
			\add &= PR(g,h)\\
			\text{mit }g(x) &= x \quad ; g=\pi'_1\\
			h(y,n,x) &= y+1 \quad ; h= S\circ \pi^S_1\\
			\add &= PR(\pi'_1,S\circ\pi^3_1)
		\end{align*}
	\item Multiplikation:
		\begin{alignat*}{2}
			&&\mult&: \N^2\->\N\\
			&&\mult(0,x) &=0\\
			&&\mult(n+1,x) &= \add(x,\mult(n,x))\\
			&&\mult &= PR(g,h)\\
			\text{mit}&\ & g&= \mathcal{O}^1\\
			&&h(y,n,x) &= \add(x,y)\\
			\text{d.h.}&& h &=\add\circ[\pi^3_3,\pi^3_1]
		\end{alignat*}
	\end{enumerate}
\end{Bsp*}
\begin{Beobachtung}
	Alle primitiv rekursiven Funktionen sind total.\\
	Aber: $\exists$ totale Funktion, die nicht primitiv rekursiv ist.
\end{Beobachtung}
\begin{Def}[Minimierung]\label{def:Minimierung}
	Sei $f:\N^{k+1}-\->\N$\\
	dann ist $\mu f:= g\in\N^k-\->\N$\\
	definiert durch:
	\begin{align*}
		g(\overline{x}) &= \min\{n \mid f(n,\overline{x})=0\text{ und }\forall j<n: f(j,\overline{x})\text{ def. und }\neq 0\} \tag*{\qedhere}\\[.5em]
		\text{Falls }f(n,\overline{x}) &\neq 0\ \forall n: g(\overline{x})\text{ undef.}\\
		\text{Falls }f(n,\overline{x}) &=0\text{ aber }\exists j< n\ f(j,\overline{x})\text{ undef.}\\
		&\curvearrowright g(\overline{x})\text{ undef.}
	\end{align*}
\end{Def}
\begin{Def}[name={[Klasse der $\mu$-rekursiven Funktionen]}]
	Die Klasse der \emph{$\mu$-rekursiven Funktionen} über $\N$ ist die kleinste Klasse von Funktionen, die die Basisfunktionen (1,2,3 aus der \autoref{def:PREC}) enthalten und abgeschlossen ist unter \hyperref[itm:Komposition]{Komposition \ref*{itm:Komposition}} \hyperref[itm:Rekursion]{Rekursion \ref*{itm:Rekursion}} und \hyperref[def:Minimierung]{Minimierung}
\end{Def}
\begin{Satz}[name={[$\mu$-Funktionen-Klasse $\hat=$ Klasse der \acs*{TM}-berechenbaren Fkt.]}]
	Die Klasse der $\mu$-Funktionen stimmt mit der Klasse der \ac{TM}-berechenbaren Funktionen über $\N$ überein.
\end{Satz}
\begin{proof}\
	\begin{itemize}
	\item simuliere \rlerror*{URM = unbeschränkte Registermaschine?}{URM} mit $\mu$-Rekursion.
	\item simuliere $\mu$-Rekursion mit $\ac{TM}$ (analog zur universellen \ac{TM}) \qedhere
	\end{itemize}
\end{proof}
\begin{Satz}[Kleene]
	Für jede $\mu$-rekursive Funktion $F: \N^k-\->\N$ gibt es zwei primitiv rekursive Funktionen $p,q:\N^{k+1}\->\N$, sodass $f(\overline{x})=p(\overline{x},\mu q(\overline{x}))$
\end{Satz}
$\curvearrowright$ Jede berechenbare Funktion über $\N$ kann mit einer WHILE-Schleife programmiert werden.
\begin{Def}[name={[Ackermannfunktion]}]
	Die \emph{Ackermannfunktion}
	\begin{align*}
		&A:\N^2\->\N\\
		&A(0,y)=y+1\\
		x>0:\ &A(x,y)=A(x-1,1)\\
		y>0:\ &A(x,y)=A(x-1,A(x.y-1)) \tag*{\qedhere}
	\end{align*}
\end{Def}
\begin{Satz}
	Sei $\A$ die Menge der Funktionen, die durch $A$ majorisiert werden.
	\[ \text{Es gilt: }\PREC\subseteq \A \]\vspace{-2em}
\end{Satz}
\begin{proof}
	$A=\{f:\N^k\->\N \mid \exists n \forall\overline{x}:f(\overline{x})< A(m,\max\overline{x})\}$.\\
	Induktion über $\PREC$:
	\begin{enumerate}
	\item $\mathcal{O}(\overline{x})=0 < A(0,\max\overline{x})=\max \overline{x}+1,\quad \mathcal{O}\in\A$
	\item $S(x)=\A+1 < x+2 = A(\uuline{1},x)$, also $S\in \mathcal A$
	\item $\pi^k_j(x_1,\dots,x_k) = x_j\leq \max\overline{x} < \max \overline{x} +1 = A(\uuline{0},\max \overline{x})$, also $\pi^k_j\in\A$
	\item\ \vspace{-2em}
		\begin{align*}
			&\left.\begin{array}{rr<{{}}@{}l}
				\text{Sei } & g_1,\dots,g_m : &\N^k\;\->\N\\
				& h : &\N^m\->\N
			\end{array}\right\rbrace \in\A\\
			&\begin{array}{rr@{}>{{}}l@{\qquad}r@{\;}l}
				\text{also } &g_i(\overline{x}) &< A(r_i,\max\overline{x}) & r_i &\exists \text{ nach I.V.}\\
				&h(\overline{x}) &< A(s,\max\overline{y}) &s &\exists \text{ nach I.V.}
			\end{array}
		\end{align*}
		\newpage
		Betrachte $f= h\circ [g_i,\dots,g_m]$\\
		Wähle $j$ sodass $r_j=\max r_i$\\
		$\curvearrowright\ g_i(x) < A(r_j,\max\overline{x})$
		\begin{align*}
			\text{Dann }f(\overline{x}) &= h(g_1(\overline{x}),\dots,g_m(\overline{x})\\
			&< A(s,\max(g_1(\overline{x}),\dots,g_m(\overline{x}))\\
			&< A(s,\max(A(r_1,\max\overline{x}), \dots, A(r_m,\max\overline{x}))\\
			&= A(s,A(r_j,\max\overline{x}))\\
			&\overset!< A(\uuline{s+r_j+2},\max\overline{x})\\
			\text{Also }f\in\A
		\end{align*}
	\item Sei $g:\N^k\->\N,\ h:\N^[k+2]\->\N\qquad \in\A$
		\begin{align*}
			\curvearrowright \exists r,s :{}& g(\overline{x}< A(r,\max\overline{x}))\\
			& h(\overline{y}< A(s,\max\overline{y}))
		\end{align*}
		Sei $f= PR(g,h)$. Zeige $f\in\A$\\
		Zunächst $\forall n\in\N: f(n,\overline{x}) < A(q,n+\max\overline x)$, wobei $q$ nicht von $n$ oder $\overline{x}$ abhängt.\\
		Wähle $q=\max(r,s)+1$.
		
		Induktion über $n$:
		\begin{description}
		\item[$n=0:$]\qquad $f(0,\overline{x}) = g(\overline{x}) < A(r,\max\overline{x}) < A(g,\max x)$
		\item[$n\->n+1:$]\ \vspace{-2em}
			\begin{align*}
				f(n+1,\overline{x}) &= h(f(n,\overline{x}),n,\overline{x}) < A(s,z) &&,\ z=\max(f(n,\overline{x}),n,\overline{x})\\
				\text{nach I.V.\,:}\quad f(n,\overline{x}) &< A(g,n+\max\overline{x})\\
				\max(n,\overline{x}) \leq n+\max\overline{x} &< A(q,n+\max\overline{x})\\
				\curvearrowright z &< A(q,n+\max\overline{x})\\
				f(n+1,\overline{x}) &< A(s,z)\\
				&< A(s,A(q,n+\max\overline{x}))\\
				&\leq A(q-1,A(q,n+\max\overline{x}))\\
				&= A(q,A(q,n+1+\max\overline{x})) \tag*{\qedsymbol\ Ind. $n$}
			\end{align*}
		\end{description}
		\begin{align*}
			\text{Nun setze }w &=\max(n,\overline{x})\\
			f(n,\overline{x}) < A(q,n+\max\overline{x})\\
			&\leq A(q,2w)\\
			&\overset!< A(\uuline{q+2},w)\\
			\text{Also }f\in\A \tag*{\qedhere}
		\end{align*}
	\end{enumerate}
\end{proof}
\begin{Korollar}
	Die Ackermannfunktion $A$ ist nicht primitiv rekursiv.
\end{Korollar}
\begin{proof}
	$A\notin\A$
\end{proof}

\renewcommand{\listtheoremname}{Liste der Definitionen}
\listoftheorems[ignoreall,show={Def,Def*,subDef},numwidth=2.8em]
\renewcommand{\listtheoremname}{Liste der Sätze}
\listoftheorems[ignoreall,show={Satz,Satz*,lemma,lemma*,Korollar,Korollar*},numwidth=2.8em]
\listoffigures
\addsec{Abkürzungsverzeichnis}
\begin{acronym}[MPCP]
	\acro{AL}{Aussagenlogik}
	\acro{CFL}{Menge der kontextfreien Sprachen}
	\acro{CFG}{Menge der kontextfreien Grammatiken}
	\acro{CNF}{Chomsky Normalform}
	\acro{CP}{Korrespondenzproblem}
	\acro{CYK}{Cocke, Younger, Kasami}
	\acro{DAG}{gerichteter azyklischer Graph}
	\acro{DCFG}{deterministische \acs*{CFG}}
	\acro{DCFL}{deterministische \acs*{CFL}}
	\acro{DEA}{deterministischer endlicher Automat}
	\acro{DFA}{\acroextra{engl.: }deterministic finite automaton}
	\acro{DPDA}{deterministischer Kellerautomat}
	\acro{DTM}{deterministische \acs*{TM}}
	\acro{EA}{endlicher Automat}
	\acro{LBA}{Linear Bounded Automaton}
	\acro{MPCP}{Das modifizierte \acs*{PCP}}
	\acro{ND}{Nicht-Determinismus}
	\acro{NEA}{nichtdeterministischer endlicher Automat}
	\acro{NFA}{\acroextra{engl.: }nondeterministic finite automaton}
	\acro{NPDA}{nichtdeterministischer Kellerautomat}
	\acro{NT}{Nichtterminal}
	\acroplural{NT}[NT]{Nichtterminale}
	\acro{NTM}{Eine nichtdeterministische \acs*{TM}}
	\acro{PCP}{Das Postsche Korrespondenzproblem}
	\acro{PDA}{pushdown automaton\acroextra{ (Kellerautomat)}}
	\acro{PL}{Pumping Lemma}
	\acro{RE}[\textit{RE}]{Menge der regulären Ausdrücke}
	\acro{REG}[\textit{REG}]{Menge der regulären Sprachen}
	\acro{RM}{Registermaschine}
	\acro{TM}{Turing-Maschine}
	\acro{TT}{Turingtabelle}
\end{acronym}
\listoffixmes

\end{document}
