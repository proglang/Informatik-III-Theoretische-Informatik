\section[Vorspann: Sprachen]{Vorspann: Sprachen\datenote{19.10.16}}
\begin{Def}[name={[Alphabet $\Sigma$]}]
	Ein \emph{Alphabet} $\Sigma$ ist eine endliche Menge von \emph{Zeichen}.
\end{Def} % 1.1
Zeichen sind hier beliebige abstrakte Symbole.
Im Folgenden steht $\Sigma$ immer für ein Alphabet.
\begin{Bsp*} ~
  \begin{itemize}
  \item $\Sigma = \{a,\dots,z\}$
  \item $\Sigma = \{0, 1\}$
  \end{itemize}
\end{Bsp*}
\begin{Def}[name={[Wort $w$ über $\Sigma$]}]\label{def:1.2}
	Die Menge $\Sigma^*$ der \emph{Worte} über $\Sigma$ ist induktiv definiert durch:
  \begin{enumerate}
  \item $\Eps \in \Sigma^*$
  \item Wenn $a \in \Sigma$ und $w \in \Sigma^*$ dann auch $aw \in \Sigma^*$
  \end{enumerate}
  Die \emph{Länge} eines Wortes, $|\cdot| : \Sigma^* \to \N$, ist induktiv definiert durch
  \begin{enumerate}
  \item $|\Eps| = 0$
  \item $|aw| = 1 + |w|$
  \end{enumerate}
\end{Def}
Ein Wort ist also immer eine endliche Folge von Zeichen.
\begin{Bsp*} ~
  \begin{itemize}
  \item \verb/rambo/ (Länge $5$)
  \item \verb/pizza/, \verb/zipza/ (ungleich) 
  \item $\Eps$ (Länge $0$)
  \item \verb/rambopizza/
  \end{itemize}
\end{Bsp*}
Wörter lassen sich ,,verketten''/,,hintereinanderreihen''.
Die entsprechende Operation heißt \emph{Konkatenation}, geschrieben ,,$\cdot$'' (wie Multiplikation).
\begin{Bsp*} ~
  \begin{itemize}
  \item $\mathtt{rambo}\cdot\mathtt{pizza} = \mathtt{rambopizza}$
  \item $\mathtt{rambo} \cdot \Eps = \mathtt{rambo} = \Eps \cdot \mathtt{rambo}$ 
  \end{itemize}
\end{Bsp*}
\begin{Def}[Konkatenation von Wörtern]
  Die \emph{Konkatenation}, $\mathord{\cdot} : \Sigma^* \times \Sigma^* \to \Sigma^*$, ist induktiv definiert durch:
  \begin{enumerate}
  \item $\Eps\cdot v = v$
  \item $(aw)\cdot v = a(w\cdot v)$
  \end{enumerate}
\end{Def}
Eigenschaften von "`$\cdot$"':
\begin{itemize}
\item Assoziativität
\item $\Eps$ ist neutrales Element
\item \emph{nicht} kommutativ
\end{itemize}
\begin{lemma}
  Für alle $w \in \Sigma^*$ gilt
  \begin{displaymath}
    w \cdot \Eps = w
  \end{displaymath}
\end{lemma}
\begin{proof}
  Per struktureller Induktion über das Wort $w$:
  \begin{description}
  \item[IA] $w = \Eps$.
      Es folgt aus Fall 1 der Definition von ,,$\cdot$'': $\Eps \cdot \Eps = \Eps$.
  \item[IV] $w' \cdot \Eps = w'$.
  \item[IS] $w = aw'$.
    Mit Fall 2 von ,,$\cdot$'' folgt: $aw' \cdot \Eps = a (w' \cdot \Eps) \stackrel{\mathsf{IV}}{=} aw'$
  \end{description}
\end{proof}
\begin{proof}
  Alternativ, per Induktion über $n = |w|$:
  \begin{description}
  \item[IA] $n = 0$.
    Es folgt aus der Definition von ,,$|\cdot|$'' (und dem arithmetischen Fakt, dass $1 + x \not = 0$), dass $w = \Eps$.
    Es folgt aus Fall 1 der Definition von ,,$\cdot$'': $\Eps \cdot \Eps = \Eps$.
  \item[IV] Für alle $w'$ mit $|w'| = n'$ gilt $w' \cdot \Eps = w'$.
  \item[IS] $n = n' + 1$.
    Es folgt aus der Definition von ,,$|\cdot|$'' (und dem arithmetischen Fakt, dass $1 + x \not = 0$), dass $w = aw'$.
    Somit gilt $|aw'| = 1 + |w'| = 1 + n'$ und daher auch $|w'| = n'$.
    Mit Fall 2 von ,,$\cdot$'' folgt: $aw' \cdot \Eps = a (w' \cdot \Eps) \stackrel{\mathsf{IV}}{=} aw'$
  \end{description}
\end{proof}
\begin{lemma}
  Für $v,w \in \Sigma^*$ gilt
  \begin{displaymath}
    |v\cdot w| = |v| + |w|
  \end{displaymath}
\end{lemma}
\begin{proof}
  Per Induktion über $v$.
  \begin{description}
  \item[IA] $v = \Eps$.
    Nach Def.\ von ,,$\cdot$'' gilt $|\Eps \cdot w| = |w|$.
    Mit Def.\ von ,,$|\cdot|$'', Fall 1 folgt:
    \begin{displaymath}
     |w| = 0 + |w| = |\Eps| + |w|
   \end{displaymath}
 \item[IV] $|v' \cdot w| = |v'| + |w'|$.
 \item[IS] $v = av'$.
   Mit Def.\ von ,,$|\cdot|$'' und ,,$\cdot$'' und Assoziativität von ,,$\cdot$'' folgt:
   \begin{displaymath}
   |av'\cdot w| = |a(v'\cdot w)| = 1 + |v' \cdot w| \stackrel{\mathsf{IV}}{=} 1 + |v'| + |w| = (1 + |v'|) + |w| = |v| + |w|
   \end{displaymath}
  
  \end{description}
\end{proof}

Der Konkatenationsoperator ,,$\cdot$'' wird oft weggelassen (ähnlich wie der Mulitplikationsoperator in der Arithmetik).
Ebenso können durch die Assoziativität Klammern weggelassen werden:
\begin{displaymath}
  w_1w_2w_3 \quad \text{ heißt also } \quad w_1\cdot w_2\cdot w_3 \quad = (w_1 \cdot w_2) \cdot w_3 = w_1 \cdot (w_2 \cdot w_3)
\end{displaymath}

Wörter lassen sich außerdem \emph{potenzieren}:
\begin{Def}
  Die \emph{Potenzierung} von Worten, $\cdot ^ \cdot: \Sigma^*\times \N \to \Sigma^*$, ist induktiv definiert durch
  \begin{enumerate}
  \item $w^0 = \Eps$ 
  \item $w^{n+1} = w \cdot w^n$
  \end{enumerate}
\end{Def}
\begin{Bsp*} ~
  \begin{itemize}
  \item $\mathtt{rambo}^1 = \mathtt{rambo}$
  \item $\mathtt{rambo}^0 = \Eps$
  \item $\mathtt{rambo}^3 = \mathtt{ramboramborambo}$
  \end{itemize}
\end{Bsp*}

\begin{Def}[name={[Sprache über $\Sigma$]}]
	Eine \emph{Sprache} über $\Sigma$ ist eine Menge $L\subseteq\Sigma^*$.
\end{Def}
\begin{Bsp*}~ 
  \begin{itemize}
  \item 
	$\{\mathtt{banane}, \mathtt{aprikose},\mathtt{orange},\dots\}$
  \item
    $\{\mathtt{rot},\mathtt{gelb},\mathtt{grün}\}$
  \item
    $\{\mathtt{rambo}, \mathtt{pizza}, \Eps, \mathtt{blümchen}\}$
  \item $\{\}$ (die ,,leere Sprache'')
  \item $\{ \Eps \}$
  \item $\Sigma^*$
  \end{itemize}
\end{Bsp*}
Sämtliche Mengenoperationen sind auch Sprachoperationen, insbesondere Schnitt ($L_1 \cap L_2$), Vereinigung ($L_1 \cup L_2$), Differenz ($L_1 \setminus L_2$) und Komplement ($L_1^{-1} = \Sigma^* \setminus L_1$).

Weitere Operationen auf Sprachen sind Konkatenation und Potenzierung, sowie der \emph{Kleene Abschluss}.
\begin{Def}[Konkatenation und Potenzierung von Sprachen] % 1.5
	Sei $U,V\subseteq \Sigma^*$ dann ist die \emph{Konkatenation} $U$ und $V$ definiert durch
	\[ U\cdot V = \{uv \mid u\in U, v\in V \} \]
  und die \emph{Potenzierung} von $U$ induktiv definiert durch
  \begin{enumerate}
  \item $U^0 = \{\Eps\}$
  \item $U^{n_1} = U \cdot U^{n}$
  \end{enumerate}
\end{Def}
\begin{Bsp*}~
  \begin{itemize}
  \item $\{\mathtt{rambo}, \mathtt{pizza}\} \cdot \{\mathtt{rot}, \mathtt{gelb}\} = \{\mathtt{ramborot}, \mathtt{pizzarot}, \mathtt{rambogelb}, \mathtt{pizzagelb}\}$
  \item $\{\mathtt{rambo}, \mathtt{pizza}\} \cdot \{\Eps, \mathtt{gelb}\} = \{\mathtt{rambo}, \mathtt{pizza}, \mathtt{rambogelb}, \mathtt{pizzagelb}\}$
  \item $\{\mathtt{rambo, \Eps}\}^3 = \{\mathtt{\Eps}, \mathtt{rambo}, \mathtt{ramborambo}, \mathtt{ramboramborambo}\}$
  \end{itemize}
\end{Bsp*}
Wie bei der Konkatenation von Wörtern lässt man den Konkatenationsoperator oft weg.
%
\begin{Def}[Kleene-Abschluss, Kleene-Stern]
	Sei $U\subseteq\Sigma^*$ dann ist
	\begin{align*}
		U^* &= \bigcup_{n\in\N} U^n \quad [\ni\Eps] && \text{Leeres Wort ist darin }n=0\\
		U^+ &= \bigcup_{n\ge1} U^n && \text{wenn $\Eps\in U$ \=> auch in $U^*$} \qedhere
	\end{align*}
\end{Def}
\begin{Bemerkung}
	$\Sigma^*=\bigcup_{n\in\N}\Sigma^n \text{ ebenso } \Sigma^+ = \bigcup_{n\ge1}\Sigma^n$
\end{Bemerkung}