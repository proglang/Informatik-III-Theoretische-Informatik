\section[Vorspann: Sprachen]{Vorspann: Sprachen\datenote{19.10.2018}}
\begin{Def}[name={[Alphabet $\Sigma$]}]
	Ein \emph{Alphabet} ist eine nicht leere, endliche Menge von \emph{Zeichen}.
\end{Def} % 1.1
Zeichen sind hier beliebige abstrakte Symbole.

\newcommand{\aro}{\textit{rot}}
\newcommand{\age}{\textit{gelb}}
\newcommand{\agr}{\textit{grün}}

\begin{Bsp*} für Alphabete, die in dieser Vorlesung, im täglichem Umgang mit Computern oder in der Forschung an unserem Lehrstuhl eine Rolle spielen
  \begin{itemize}
  \item $\{a,\dots,z\}$
  \item $\{0, 1\}$
  \item $\{\aro, \age, \agr\}$ (Ampelfarben)
  \item Die Menge aller ASCII-Symbole
  \item Die Menge aller Statements eines Computerprogramms
  \qedhere
  \end{itemize}
\end{Bsp*}
Wir verwenden typischerweise den griechischen Buchstaben $\Sigma$ als Namen für ein Alphabet und die lateinischen Buchstaben $a,b,c$ als Namen für Zeichen.%
\footnote{Dies ist eine Konventionen analog zu den folgenden, die Sie möglicherweise in der Schule befolgten: Verwende $n,m$ für natürliche Zahlen. Verwende $\alpha, \beta$ für Winkel in Dreiecken. Verwende $A$ für Matrizen.}
Im Folgenden sei $\Sigma$ immer ein beliebiges Alphabet.%
\footnote{Dieser Satz dient dazu, dass die Autoren dieses Skripts nicht jede Definition mit ``Sei $\Sigma$ ein Alphabet...'' beginnen müssen.}

\begin{Def}[name={[Wort $w$ über $\Sigma$]}]\label{def:1.2}
  Wir nennen eine endliche Folge von Elementen aus $\Sigma$ ein \emph{Wort}
  und schreiben solch eine Folge immer ohne Trennsymbole wie z.B. Komma.\footnote{Wir schreiben also z.B. \texttt{einhorn} statt \texttt{e,i,n,h,o,r,n}.}
  Die leere Folge nennen wir das \emph{leere Wort}; als Konvention stellen wir das leere Wort mit dem griechischen Buchstaben $\Eps$ dar.\footnote{Eine analoge Konvention, die sie aus der Schule kennen: Verwende immer $\pi$ als Symbol für die Kreiszahl.}
  Wir bezeichnen die Menge aller Wörter mit $\Sigma^*$ und die Menge aller nicht leeren Wörter mit $\Sigma^+$.
  Die \emph{Länge} eines Wortes, $|\cdot| : \Sigma^* \to \N$, ist die Anzahl der Elemente der Folge.
\end{Def}
  Wir verwenden typischerweise $u,v,w$ als Namen für Wörter.

\goodbreak

\begin{Bsp*} für Wörter über $\Sigma=\{a,\dots,z\}$
  \begin{itemize}
  \item \verb/rambo/ (Länge $5$)
  \item \verb/eis/, \verb/ies/ (beide Länge 3 aber ungleich) 
  \item $\Eps$ (Länge $0$)
  \qedhere
  \end{itemize}
\end{Bsp*}
Wörter lassen sich "`verketten''/"`hintereinanderreihen''.
Die entsprechende Operation heißt \emph{Konkatenation}, geschrieben "`$\cdot$'' (wie Multiplikation).
\begin{Def}[Konkatenation von Wörtern]
  Die \emph{Konkatenation}, $\mathord{\cdot} : \Sigma^* \times \Sigma^* \to \Sigma^*$, ist für $u=u_1\ldots u_n\in\Sigma^*$ und $v=v_1\ldots v_m\in\Sigma^*$ definiert durch
  $u\cdot v = u_1\ldots u_nv_1\ldots v_m$
%   \begin{enumerate}
%   \item $\Eps\cdot v = v$ (leeres Wort ist linksneutrales Element)
%   \item $v\cdot \Eps = v$ (leeres Wort ist rechtsneutrales Element)
%   \item $u\cdot (v \cdot w) = (u\cdot v) \cdot w$ (Assoziativität)
%   \item $u(xw)\cdot v = ux(w\cdot v)$
%   \end{enumerate}
\end{Def}
% Punkt 2 (rechtsneutrales Element) folgt bereits aus 1. und 3. und könnte auch weggelassen werden.
\begin{Bsp*} ~
  \begin{itemize}
  \item $\mathtt{eis}\cdot\mathtt{rambo} =\mathtt{eisrambo}$
  \item $\mathtt{rambo} \cdot \Eps = \mathtt{rambo} = \Eps \cdot \mathtt{rambo}$ 
  \qedhere
  \end{itemize}
\end{Bsp*}
Eigenschaften von "`$\cdot$"':
\begin{itemize}
\item assoziativ
\item $\Eps$ ist neutrales Element
\item \emph{nicht} kommutativ
\end{itemize}


Der Konkatenationsoperator "`$\cdot$"' wird oft weggelassen (ähnlich wie der Multiplikationsoperator in der Arithmetik).
Ebenso können durch die Assoziativität Klammern weggelassen werden.
\begin{displaymath}
  w_1w_2w_3 \;\; \text{ steht also auch für } \;\; w_1\cdot w_2\cdot w_3,\;\;\text{ für }\;\;  (w_1 \cdot w_2) \cdot w_3 \;\;\text{ und für }\;\; w_1 \cdot (w_2 \cdot w_3)
\end{displaymath}

Bemerkung: Die Zeichenfolge \texttt{rambo}$\varepsilon$ ist \emph{kein} Wort.
Diese Zeichenfolge ist lediglich eine Notation für eine Konkatenationsoperation, die ein Wort der Länge 5 (nämlich \texttt{rambo}) beschreibt.

Wörter lassen sich außerdem \emph{potenzieren}:
\begin{Def}
  Die \emph{Potenzierung} von Wörtern, $\cdot ^ \cdot: \Sigma^*\times \N \to \Sigma^*$, ist induktiv definiert durch
  \begin{enumerate}
  \item $w^0 = \Eps$ 
  \item $w^{n+1} = w \cdot w^n$
  \qedhere
  \end{enumerate}
\end{Def}
\begin{Bsp*} $\mathtt{eis}^3 
\stackrel{(2.)}{=} \mathtt{eis}\cdot\mathtt{eis}^2 
\stackrel{\text{zweimal } (2.)}{=} \mathtt{eis}\cdot\mathtt{eis}\cdot\mathtt{eis}\cdot\mathtt{eis}^0
\stackrel{(1.)}{=} \mathtt{eis}\cdot\mathtt{eis}\cdot\mathtt{eis}\cdot\Eps
= \mathtt{eiseiseis}
$
\end{Bsp*}

\begin{Def}[name={[Sprache über $\Sigma$]}]
	Eine \emph{Sprache} über $\Sigma$ ist eine Menge $L\subseteq\Sigma^*$.
\end{Def}

\goodbreak

\begin{Bsp*}~ 
  \begin{itemize}
  \item 
	$\{\mathtt{eis}, \mathtt{rambo}\}$
  \item
    $\{w\in\{0,1\}^*\mid w \text{ ist Binärcodierung einer Primzahl}\}$
  \item $\{\}$ (die "`leere Sprache"')
  \item $\{ \Eps \}$ (ist verschieden von der leeren Sprache)
  \item $\Sigma^*$
  \qedhere
  \end{itemize}
\end{Bsp*}
Sämtliche Mengenoperationen sind auch Sprachoperationen, insbesondere Schnitt ($L_1 \cap L_2$), Vereinigung ($L_1 \cup L_2$), Differenz ($L_1 \setminus L_2$) und Komplement ($\overline{L_1} = \Sigma^* \setminus L_1$).

Weitere Operationen auf Sprachen sind Konkatenation und Potenzierung, sowie der \emph{Kleene-Abschluss}.
\begin{Def}[Konkatenation und Potenzierung von Sprachen] % 1.5
	Seien $U,V\subseteq \Sigma^*$. Dann ist die \emph{Konkatenation} von $U$ und $V$ definiert durch
	\[ U\cdot V = \{uv \mid u\in U, v\in V \} \]
  und die \emph{Potenzierung} von $U$ induktiv definiert durch
  \begin{enumerate}
  \item $U^0 = \{\Eps\}$
  \item $U^{n+1} = U \cdot U^{n}$
  \qedhere
  \end{enumerate}
\end{Def}
\begin{Bsp*}~
  \begin{itemize}
  \item $\{\mathtt{eis}, \Eps\} \cdot \{\mathtt{rambo}\} = \{\mathtt{eisrambo}, \mathtt{rambo}\}$
  \item $\{\mathtt{eis}, \Eps\} \cdot \{\} = \{\}$
  \item $\{\}^0 = \{\Eps\}$
  \item $\{\}^4 = \{\}$
  \item $\{\mathtt{eis}, \Eps\}^2 = \{\Eps, \mathtt{eis}, \mathtt{eiseis}\}$
  \qedhere
  \end{itemize}
\end{Bsp*}
Wie bei der Konkatenation von Wörtern dürfen wir den Konkatenationsoperator auch weglassen.
%
\begin{Def}[Kleene-Abschluss, Kleene-Stern, Kleene-Plus]
  Gegeben eine Sprache $U\subseteq\Sigma^*$, definieren wir den \emph{Kleene-Abschluss} $U^*$ wie folgt.
      $$U^* = \bigcup_{n\in\N} U^n$$
  Wie nennen den Operator $\cdot^*: \mathcal{P}(\Sigma^*)\rightarrow\mathcal{P}(\Sigma^*)$ der für eine gegebene Sprache den Kleene-Abschluss liefert den \emph{Kleene-Stern}. Analog definieren wir den \emph{Kleene-Plus} Operator $\cdot^+: \mathcal{P}(\Sigma^*)\rightarrow\mathcal{P}(\Sigma^*)$ als 
  $$U^+ = \bigcup_{n\ge1} U^n.$$
\end{Def}
Es gilt also $U^*=U^+\cup\{\Eps\}$.

%%% Local Variables:
%%% mode: latex
%%% TeX-master: "Info_3_Skript_WS2016-17"
%%% End:
