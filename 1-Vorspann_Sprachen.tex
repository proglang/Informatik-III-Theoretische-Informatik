\section[Vorspann: Sprachen]{Vorspann: Sprachen\datenote{23.10.15}}
Zeichen, Symbole: Buchstaben, Ziffern\\
hier: abstrakte Zeichen
\begin{Def}[name={[Alphabet $\Sigma$]}]
	Ein \underline{Alphabet} $\Sigma$ ist eine endl. Menge von Zeichen.
	
	Aus Zeichen \-> Wörter durch Hintereinanderschreiben.
\end{Def} % 1.1
\begin{Bsp*} $\Sigma = \{a,\dots,z\}$\\
	Worte: rambo (5 Zeichen), ist, hungrig, \qquad $\epsilon$ (0 Zeichen) = leeres Wort\\
	Wörter verketten = konkatenieren
\end{Bsp*}
\begin{Bsp*} rambo$\cdot$ist$\cdot$hungrig\\
	"`$\cdot$"' ist Konkat.-Operator
	
	Wörter potenzieren: $\begin{aligned}[t]
	(\text{la})^3 &= \text{la}\cdot\text{la}\cdot\text{la} = \text{lalala}\\
	(\text{rambo})^0 &= \epsilon
	\end{aligned}$
\end{Bsp*}

\begin{Def}[name={[Wort $w$ über $\Sigma$]}]\label{def:1.2}
	Ein Wort $w$ über $\Sigma$ ist eine endliche Folge von Zeichen $w=a_1a_2\dots a_n$ mit $n\in\N$ und $a_i\in\Sigma\ (1\leq i\leq n)$.\\
	Schreibe $\epsilon$ falls $n=0$\\
	$|w|=n$ ist die Länge des Wortes $w$.\\
	$\Sigma^*$ ist die Menge aller Wörter über $\Sigma$.
\end{Def}
\begin{Def}[Konkatenation von Wörtern]
	\begin{align*}
		\text{Sei }u &= a_1\dots a_n\in\Sigma^*\\
		v &= b_1\dots b_m\in\Sigma^*\\
		\shortintertext{dann ist $u\cdot v=c_1\cdots c_{n+m}\ ,\ c_i \in \Sigma$}
		c_i &= \begin{cases}
		a_i & 1\leq i\leq n\\
		b_{i-n} & n+1\leq i\leq n+m
		\end{cases}
	\end{align*}
	Eigenschaften von "`$\cdot$"':
	\begin{itemize}
		\item assoziativ
		\item $\epsilon$ ist neutrales Element
	\end{itemize}
	Die Potenz $v^k$ , $v\in \Sigma^*,k\in\N^*$ ist def. durch
	\[ v^0=\epsilon,\ v^{k+1}=v\cdot v^k \]
\end{Def}
%
Eine Sprache ist Menge von Wörtern über $\Sigma=\{a,\dots,z,"a,"o,"u\}$
\begin{align*}
	L_\text{obst} &= \{\text{banane,aprikose,orange,\dots}\}\\
	L_\text{farbe} &= \{\text{rot,gelb,grün}\}\\
	L_\text{krach} &= \{\text{ra}\cdot(\text{ta})^n\cdot\text{mm} \mid n\in\N \}\\
	L &= \{\} \quad\text{leere Sprache}\\
	L &= \Sigma^*
\end{align*}

\begin{Def}[name={[Sprache über $\Sigma$]}]
	Eine \underline{Sprache über $\Sigma$} ist Menge $L\subseteq\Sigma^*$.
	\[ L_\text{lala} = \{(\text{la})^n \mid n\in\N\} \ni\Sigma
	\qedhere \]
\end{Def}
sämtliche Mengenoperationen sind auch Sprachoperationen, insbesondere:\medskip\\
\begin{tabular}{lllll}
	$L_1\cap L_2$ &,& $L_1\cup L_2$ &,& $\Sigma^*\setminus L$\\
	Schnitt && Verein. && Komplement
\end{tabular}\medskip\\
$\rightsquigarrow$ Weitere Operationen auf Sprachen: Konkatenation
\begin{Bsp*}
	\begin{align*}
		L_\text{farbe}\cdot L_\text{obst} &\subseteq \{\text{rotbanane, rotaprikose, gelbbanane, gelbaprikose,}\\
		&\phantom{{}\subseteq \{}\text{gelborange, grünbanane, grünaprikose, grünorange}\}\\
		L_\text{farbe}\cdot\{\epsilon\}\cdot L_\text{obst} &\subseteq \{ \text{rot\textbf{e}banane, rot\textbf{e}aprikose, rot\textbf{e}orange, gelb\textbf{e}banane, \dots} \}
	\end{align*}
\end{Bsp*}
\begin{Def}[Konkatenation von Sprachen] % 1.5
	Sei $U,V\subseteq \Sigma^*$ dann ist
	\[ U\cdot V = \{uv \mid u\in U, v\in V \} \qedhere \]
\end{Def}
Potenzieren
\begin{align*}
	L_\text{farbe}^2 &= L_\text{farbe}\cdot L_\text{farbe}\\
	&= \{ \text{rotorot, rotgelb, rotgrün, gelbrot, gelbgelb, \dots} \}\\
	L^0 &=\{\epsilon\} \qquad \{\epsilon\}\cdot L = \{\epsilon\cdot w \mid w\in L\} = L \qedhere
\end{align*}
%
\begin{Def}[Potenzierung von Sprachen] Sei $U\subseteq\Sigma^*$
	\begin{align*}
		U^0 &= \{\epsilon\} & U^{n+1}= U \cdot U^n
	\end{align*}
\end{Def}
Gegeben $L$, sind sämtliche Kombinationen von Worten aus $L$ gesucht.
\begin{Bsp*}
	$\{(la)^n \mid n\in\N\}\cdot\{la\}^*$
\end{Bsp*}
\vspace{.5em}
\begin{Def}[Stern-Operator, Abschluss, Kleene-Stern]
	Sei $U\in\Sigma^*$ dann ist
	\begin{align*}
		U^* &= \bigcup_{n\in\N} U^n \quad [\ni\epsilon] && \text{Leeres Wort ist darin }n=0\\
		U^+ &= \bigcup_{n\ge1} U^n && \text{wenn $\epsilon\in U$ \=> auch in $U^*$} \qedhere
	\end{align*}
\end{Def}
\begin{Bemerkung}
	$\Sigma^*=\bigcup_{n\in\N}\Sigma^n \text{ ebenso } \Sigma^+ = \bigcup_{n\ge1}\Sigma^n$
\end{Bemerkung}
\begin{Def}[Alternative Definition von $\Sigma^*$]\label{def:1.8}\ \\
	Die Menge $\Sigma^*$ ist kleinste Menge, so dass
	\begin{enumerate}[label={(\arabic*)}]
		\item $\epsilon\in\Sigma^*$
		\item\label{def:1.8.2} $a\in\Sigma\ ,w\in\Sigma^*\ \=> aw\in\Sigma^*$\qedhere
	\end{enumerate}
\end{Def}
Die Definitionen \ref{def:1.8} und \ref{def:1.2} sind gleichwertig.
\begin{proof}\ 
	\begin{itemize}
		\item "`\autoref{def:1.2}"' \=> "`\autoref{def:1.8}"':\\
		$\forall n\in\N\ w=a_1\dots a_n \quad, a_i\in\Sigma$
		
		Zeige $w\in\Sigma^*$ (gemäß \autoref{def:1.8})
		
		\underline{I.A.: $n=0$} $\=> w = \epsilon \in \Sigma^*$ \ (\autoref{def:1.8})\\
		\underline{$n\->n+1$} $w = a_1\dots a_{n+1}$ nach \autoref{def:1.2} $w'=\overbrace{a_2\dots a_{n+1}}^{\smash{n\text{ Buchstaben}}} \in \Sigma^*$\\
		\-> Nach \autoref{def:1.8} \ref{def:1.8.2} $a_1a_2\dots a_{n+1} \in\Sigma^*$ (\autoref{def:1.8})
		\item "`\autoref{def:1.8}"' \=> "`\autoref{def:1.2}"':\\
		Induktion über $w\in\Sigma^*$ \ (\ref{def:1.8})
		Zeige: f"ur jedes $w\in\Sigma$ gibt es ein $n\in\N$ und $a_1, \dots, a_n \in \Sigma$, so dass $w=a_1\dots a_n$.
		\begin{itemize}
			\item $\epsilon\in\Sigma^*$ Wähle $n=0$ dann $w=\epsilon$.
			\item $aw':\ a\in\Sigma,\ w'\in\Sigma^*$ (\ref{def:1.8})\\
			Nach Induktionsbehauptung $w'\in\Sigma^*$ (\autoref{def:1.2})\\
			Also $\exists n: w'=a_1\dots a_n\in\Sigma^*$ (\autoref{def:1.2})\\
			W"ahle (als neues $n$) $m = n+1$ und $b_1, \dots b_{n+1} \in\Sigma$
			mit $b_1 = a$ und $b_{i+1} = a_i$ f"ur $1\le i\le n$.\\
			Dann ist $w = b_1b_2\dots b_{n+1}\in\Sigma^*$ (\ref{def:1.2}) \qedhere
	\end{itemize}
	\end{itemize}
\end{proof}
Alternative Definition von Konkatenation:
\begin{align*}
	\epsilon\cdot v &= v\\
	(aw)\cdot v &= a(w\cdot v)
\end{align*}
%
\begin{Bsp*} Für Eigenschaft von Sprachen:
	\begin{align*}
		U^* &= \{\epsilon\}\cup U\cdot U^*
	\end{align*}
	Beweis durch Kalkulation:
	$U^*=\{\epsilon\}\cup U\cdot U^*\subseteq U^*\cup U\cdot\bigcup_n U^n = U^*\cup\bigcup_n U^{n+1} \subseteq U^*\cup U^* = U^*$
\end{Bsp*}
Elementarer Beweis:
\begin{enumerate}[label={Zeige (\arabic*)},leftmargin=*,itemindent=*]
	\item $U^*\subseteq \{\epsilon\}\cup U\cdot U^*$\\
	Sei $w\in U^*=\bigcup_{n\in\N} U^n$\\
	$\curvearrowright \exists n: w\subset U^n$
	\begin{itemize}
		\item Falls $n=0: w=\epsilon\subset\{\epsilon\}\cup U\cdot U^*$
		\item Falls $n= n'+1: w\in U\cdot U^n\subseteq U\cdot U^*\cup \{\epsilon\}$
	\end{itemize}
	\item $\{\epsilon\}\cup U\cdot U^*\subseteq U^*$\\
	Sei $w\in\{\epsilon\} \cup U\cdot U^*$
	\begin{itemize}
		\item Falls $w=\epsilon: w=\epsilon\in U^*$
		\item Falls $w\in U\cdot U^n: \exists n\ w\in U\cdot U^n = U^{n+1}\subseteq U^*$
	\end{itemize}
\end{enumerate}